\section{stm32f4xx\+\_\+tim.\+c}
\label{stm32f4xx__tim_8c_source}\index{C\+:/\+Users/\+Md. Istiaq Mahbub/\+Desktop/\+I\+M\+U/\+M\+P\+U6050\+\_\+\+Motion\+Driver/\+S\+T\+M32\+F4xx\+\_\+\+Std\+Periph\+\_\+\+Driver/src/stm32f4xx\+\_\+tim.\+c@{C\+:/\+Users/\+Md. Istiaq Mahbub/\+Desktop/\+I\+M\+U/\+M\+P\+U6050\+\_\+\+Motion\+Driver/\+S\+T\+M32\+F4xx\+\_\+\+Std\+Periph\+\_\+\+Driver/src/stm32f4xx\+\_\+tim.\+c}}

\begin{DoxyCode}
00001 \textcolor{comment}{/**}
00002 \textcolor{comment}{  ******************************************************************************}
00003 \textcolor{comment}{  * @file    stm32f4xx\_tim.c}
00004 \textcolor{comment}{  * @author  MCD Application Team}
00005 \textcolor{comment}{  * @version V1.0.0}
00006 \textcolor{comment}{  * @date    30-September-2011}
00007 \textcolor{comment}{  * @brief   This file provides firmware functions to manage the following }
00008 \textcolor{comment}{  *          functionalities of the TIM peripheral:}
00009 \textcolor{comment}{  *            - TimeBase management}
00010 \textcolor{comment}{  *            - Output Compare management}
00011 \textcolor{comment}{  *            - Input Capture management}
00012 \textcolor{comment}{  *            - Advanced-control timers (TIM1 and TIM8) specific features  }
00013 \textcolor{comment}{  *            - Interrupts, DMA and flags management}
00014 \textcolor{comment}{  *            - Clocks management}
00015 \textcolor{comment}{  *            - Synchronization management}
00016 \textcolor{comment}{  *            - Specific interface management}
00017 \textcolor{comment}{  *            - Specific remapping management      }
00018 \textcolor{comment}{  *              }
00019 \textcolor{comment}{  *  @verbatim}
00020 \textcolor{comment}{  *  }
00021 \textcolor{comment}{  *          ===================================================================}
00022 \textcolor{comment}{  *                                 How to use this driver}
00023 \textcolor{comment}{  *          ===================================================================}
00024 \textcolor{comment}{  *          This driver provides functions to configure and program the TIM }
00025 \textcolor{comment}{  *          of all STM32F4xx devices.}
00026 \textcolor{comment}{  *          These functions are split in 9 groups: }
00027 \textcolor{comment}{  *   }
00028 \textcolor{comment}{  *          1. TIM TimeBase management: this group includes all needed functions }
00029 \textcolor{comment}{  *             to configure the TM Timebase unit:}
00030 \textcolor{comment}{  *                   - Set/Get Prescaler}
00031 \textcolor{comment}{  *                   - Set/Get Autoreload  }
00032 \textcolor{comment}{  *                   - Counter modes configuration}
00033 \textcolor{comment}{  *                   - Set Clock division  }
00034 \textcolor{comment}{  *                   - Select the One Pulse mode}
00035 \textcolor{comment}{  *                   - Update Request Configuration}
00036 \textcolor{comment}{  *                   - Update Disable Configuration}
00037 \textcolor{comment}{  *                   - Auto-Preload Configuration }
00038 \textcolor{comment}{  *                   - Enable/Disable the counter     }
00039 \textcolor{comment}{  *                 }
00040 \textcolor{comment}{  *          2. TIM Output Compare management: this group includes all needed }
00041 \textcolor{comment}{  *             functions to configure the Capture/Compare unit used in Output }
00042 \textcolor{comment}{  *             compare mode: }
00043 \textcolor{comment}{  *                   - Configure each channel, independently, in Output Compare mode}
00044 \textcolor{comment}{  *                   - Select the output compare modes}
00045 \textcolor{comment}{  *                   - Select the Polarities of each channel}
00046 \textcolor{comment}{  *                   - Set/Get the Capture/Compare register values}
00047 \textcolor{comment}{  *                   - Select the Output Compare Fast mode }
00048 \textcolor{comment}{  *                   - Select the Output Compare Forced mode  }
00049 \textcolor{comment}{  *                   - Output Compare-Preload Configuration }
00050 \textcolor{comment}{  *                   - Clear Output Compare Reference}
00051 \textcolor{comment}{  *                   - Select the OCREF Clear signal}
00052 \textcolor{comment}{  *                   - Enable/Disable the Capture/Compare Channels    }
00053 \textcolor{comment}{  *                   }
00054 \textcolor{comment}{  *          3. TIM Input Capture management: this group includes all needed }
00055 \textcolor{comment}{  *             functions to configure the Capture/Compare unit used in }
00056 \textcolor{comment}{  *             Input Capture mode:}
00057 \textcolor{comment}{  *                   - Configure each channel in input capture mode}
00058 \textcolor{comment}{  *                   - Configure Channel1/2 in PWM Input mode}
00059 \textcolor{comment}{  *                   - Set the Input Capture Prescaler}
00060 \textcolor{comment}{  *                   - Get the Capture/Compare values      }
00061 \textcolor{comment}{  *                   }
00062 \textcolor{comment}{  *          4. Advanced-control timers (TIM1 and TIM8) specific features}
00063 \textcolor{comment}{  *                   - Configures the Break input, dead time, Lock level, the OSSI,}
00064 \textcolor{comment}{  *                      the OSSR State and the AOE(automatic output enable)}
00065 \textcolor{comment}{  *                   - Enable/Disable the TIM peripheral Main Outputs}
00066 \textcolor{comment}{  *                   - Select the Commutation event}
00067 \textcolor{comment}{  *                   - Set/Reset the Capture Compare Preload Control bit}
00068 \textcolor{comment}{  *                              }
00069 \textcolor{comment}{  *          5. TIM interrupts, DMA and flags management}
00070 \textcolor{comment}{  *                   - Enable/Disable interrupt sources}
00071 \textcolor{comment}{  *                   - Get flags status}
00072 \textcolor{comment}{  *                   - Clear flags/ Pending bits}
00073 \textcolor{comment}{  *                   - Enable/Disable DMA requests }
00074 \textcolor{comment}{  *                   - Configure DMA burst mode}
00075 \textcolor{comment}{  *                   - Select CaptureCompare DMA request  }
00076 \textcolor{comment}{  *              }
00077 \textcolor{comment}{  *          6. TIM clocks management: this group includes all needed functions }
00078 \textcolor{comment}{  *             to configure the clock controller unit:}
00079 \textcolor{comment}{  *                   - Select internal/External clock}
00080 \textcolor{comment}{  *                   - Select the external clock mode: ETR(Mode1/Mode2), TIx or ITRx}
00081 \textcolor{comment}{  *         }
00082 \textcolor{comment}{  *          7. TIM synchronization management: this group includes all needed }
00083 \textcolor{comment}{  *             functions to configure the Synchronization unit:}
00084 \textcolor{comment}{  *                   - Select Input Trigger  }
00085 \textcolor{comment}{  *                   - Select Output Trigger  }
00086 \textcolor{comment}{  *                   - Select Master Slave Mode }
00087 \textcolor{comment}{  *                   - ETR Configuration when used as external trigger   }
00088 \textcolor{comment}{  *     }
00089 \textcolor{comment}{  *          8. TIM specific interface management, this group includes all }
00090 \textcolor{comment}{  *             needed functions to use the specific TIM interface:}
00091 \textcolor{comment}{  *                   - Encoder Interface Configuration}
00092 \textcolor{comment}{  *                   - Select Hall Sensor   }
00093 \textcolor{comment}{  *         }
00094 \textcolor{comment}{  *          9. TIM specific remapping management includes the Remapping }
00095 \textcolor{comment}{  *             configuration of specific timers               }
00096 \textcolor{comment}{  *   }
00097 \textcolor{comment}{  *  @endverbatim}
00098 \textcolor{comment}{  *    }
00099 \textcolor{comment}{  ******************************************************************************}
00100 \textcolor{comment}{  * @attention}
00101 \textcolor{comment}{  *}
00102 \textcolor{comment}{  * THE PRESENT FIRMWARE WHICH IS FOR GUIDANCE ONLY AIMS AT PROVIDING CUSTOMERS}
00103 \textcolor{comment}{  * WITH CODING INFORMATION REGARDING THEIR PRODUCTS IN ORDER FOR THEM TO SAVE}
00104 \textcolor{comment}{  * TIME. AS A RESULT, STMICROELECTRONICS SHALL NOT BE HELD LIABLE FOR ANY}
00105 \textcolor{comment}{  * DIRECT, INDIRECT OR CONSEQUENTIAL DAMAGES WITH RESPECT TO ANY CLAIMS ARISING}
00106 \textcolor{comment}{  * FROM THE CONTENT OF SUCH FIRMWARE AND/OR THE USE MADE BY CUSTOMERS OF THE}
00107 \textcolor{comment}{  * CODING INFORMATION CONTAINED HEREIN IN CONNECTION WITH THEIR PRODUCTS.}
00108 \textcolor{comment}{  *}
00109 \textcolor{comment}{  * <h2><center>&copy; COPYRIGHT 2011 STMicroelectronics</center></h2>}
00110 \textcolor{comment}{  ******************************************************************************}
00111 \textcolor{comment}{  */}
00112 
00113 \textcolor{comment}{/* Includes ------------------------------------------------------------------*/}
00114 \textcolor{preprocessor}{#}\textcolor{preprocessor}{include} "stm32f4xx_tim.h"
00115 \textcolor{preprocessor}{#}\textcolor{preprocessor}{include} "stm32f4xx_rcc.h"
00116 
00117 \textcolor{comment}{/** @addtogroup STM32F4xx\_StdPeriph\_Driver}
00118 \textcolor{comment}{  * @\{}
00119 \textcolor{comment}{  */}
00120 
00121 \textcolor{comment}{/** @defgroup TIM }
00122 \textcolor{comment}{  * @brief TIM driver modules}
00123 \textcolor{comment}{  * @\{}
00124 \textcolor{comment}{  */}
00125 
00126 \textcolor{comment}{/* Private typedef -----------------------------------------------------------*/}
00127 \textcolor{comment}{/* Private define ------------------------------------------------------------*/}
00128 
00129 \textcolor{comment}{/* ---------------------- TIM registers bit mask ------------------------ */}
00130 \textcolor{preprocessor}{#}\textcolor{preprocessor}{define} \textcolor{preprocessor}{SMCR\_ETR\_MASK}      \textcolor{preprocessor}{(}\textcolor{preprocessor}{(}\textcolor{preprocessor}{uint16\_t}\textcolor{preprocessor}{)}0x00FF\textcolor{preprocessor}{)}
00131 \textcolor{preprocessor}{#}\textcolor{preprocessor}{define} \textcolor{preprocessor}{CCMR\_OFFSET}        \textcolor{preprocessor}{(}\textcolor{preprocessor}{(}\textcolor{preprocessor}{uint16\_t}\textcolor{preprocessor}{)}0x0018\textcolor{preprocessor}{)}
00132 \textcolor{preprocessor}{#}\textcolor{preprocessor}{define} \textcolor{preprocessor}{CCER\_CCE\_SET}       \textcolor{preprocessor}{(}\textcolor{preprocessor}{(}\textcolor{preprocessor}{uint16\_t}\textcolor{preprocessor}{)}0x0001\textcolor{preprocessor}{)}
00133 \textcolor{preprocessor}{#}\textcolor{preprocessor}{define} \textcolor{preprocessor}{CCER\_CCNE\_SET}      \textcolor{preprocessor}{(}\textcolor{preprocessor}{(}\textcolor{preprocessor}{uint16\_t}\textcolor{preprocessor}{)}0x0004\textcolor{preprocessor}{)}
00134 \textcolor{preprocessor}{#}\textcolor{preprocessor}{define} \textcolor{preprocessor}{CCMR\_OC13M\_MASK}    \textcolor{preprocessor}{(}\textcolor{preprocessor}{(}\textcolor{preprocessor}{uint16\_t}\textcolor{preprocessor}{)}0xFF8F\textcolor{preprocessor}{)}
00135 \textcolor{preprocessor}{#}\textcolor{preprocessor}{define} \textcolor{preprocessor}{CCMR\_OC24M\_MASK}    \textcolor{preprocessor}{(}\textcolor{preprocessor}{(}\textcolor{preprocessor}{uint16\_t}\textcolor{preprocessor}{)}0x8FFF\textcolor{preprocessor}{)}
00136 
00137 \textcolor{comment}{/* Private macro -------------------------------------------------------------*/}
00138 \textcolor{comment}{/* Private variables ---------------------------------------------------------*/}
00139 \textcolor{comment}{/* Private function prototypes -----------------------------------------------*/}
00140 \textcolor{keyword}{static} \textcolor{keywordtype}{void} TI1_Config(TIM\_TypeDef* TIMx, uint16\_t TIM\_ICPolarity, uint16\_t TIM\_ICSelection,
00141                        uint16\_t TIM\_ICFilter);
00142 \textcolor{keyword}{static} \textcolor{keywordtype}{void} TI2_Config(TIM\_TypeDef* TIMx, uint16\_t TIM\_ICPolarity, uint16\_t TIM\_ICSelection,
00143                        uint16\_t TIM\_ICFilter);
00144 \textcolor{keyword}{static} \textcolor{keywordtype}{void} TI3_Config(TIM\_TypeDef* TIMx, uint16\_t TIM\_ICPolarity, uint16\_t TIM\_ICSelection,
00145                        uint16\_t TIM\_ICFilter);
00146 \textcolor{keyword}{static} \textcolor{keywordtype}{void} TI4_Config(TIM\_TypeDef* TIMx, uint16\_t TIM\_ICPolarity, uint16\_t TIM\_ICSelection,
00147                        uint16\_t TIM\_ICFilter);
00148 
00149 \textcolor{comment}{/* Private functions ---------------------------------------------------------*/}
00150 
00151 \textcolor{comment}{/** @defgroup TIM\_Private\_Functions}
00152 \textcolor{comment}{  * @\{}
00153 \textcolor{comment}{  */}
00154 
00155 \textcolor{comment}{/** @defgroup TIM\_Group1 TimeBase management functions}
00156 \textcolor{comment}{ *  @brief   TimeBase management functions }
00157 \textcolor{comment}{ *}
00158 \textcolor{comment}{@verbatim   }
00159 \textcolor{comment}{ ===============================================================================}
00160 \textcolor{comment}{                       TimeBase management functions}
00161 \textcolor{comment}{ ===============================================================================  }
00162 \textcolor{comment}{  }
00163 \textcolor{comment}{       ===================================================================      }
00164 \textcolor{comment}{              TIM Driver: how to use it in Timing(Time base) Mode}
00165 \textcolor{comment}{       =================================================================== }
00166 \textcolor{comment}{       To use the Timer in Timing(Time base) mode, the following steps are mandatory:}
00167 \textcolor{comment}{       }
00168 \textcolor{comment}{       1. Enable TIM clock using RCC\_APBxPeriphClockCmd(RCC\_APBxPeriph\_TIMx, ENABLE) function}
00169 \textcolor{comment}{                    }
00170 \textcolor{comment}{       2. Fill the TIM\_TimeBaseInitStruct with the desired parameters.}
00171 \textcolor{comment}{       }
00172 \textcolor{comment}{       3. Call TIM\_TimeBaseInit(TIMx, &TIM\_TimeBaseInitStruct) to configure the Time Base unit}
00173 \textcolor{comment}{          with the corresponding configuration}
00174 \textcolor{comment}{          }
00175 \textcolor{comment}{       4. Enable the NVIC if you need to generate the update interrupt. }
00176 \textcolor{comment}{          }
00177 \textcolor{comment}{       5. Enable the corresponding interrupt using the function TIM\_ITConfig(TIMx, TIM\_IT\_Update) }
00178 \textcolor{comment}{       }
00179 \textcolor{comment}{       6. Call the TIM\_Cmd(ENABLE) function to enable the TIM counter.}
00180 \textcolor{comment}{             }
00181 \textcolor{comment}{       Note1: All other functions can be used separately to modify, if needed,}
00182 \textcolor{comment}{          a specific feature of the Timer. }
00183 \textcolor{comment}{}
00184 \textcolor{comment}{@endverbatim}
00185 \textcolor{comment}{  * @\{}
00186 \textcolor{comment}{  */}
00187 
00188 \textcolor{comment}{/**}
00189 \textcolor{comment}{  * @brief  Deinitializes the TIMx peripheral registers to their default reset values.}
00190 \textcolor{comment}{  * @param  TIMx: where x can be 1 to 14 to select the TIM peripheral.}
00191 \textcolor{comment}{  * @retval None}
00192 \textcolor{comment}{}
00193 \textcolor{comment}{  */}
00194 \textcolor{keywordtype}{void} TIM_DeInit(TIM\_TypeDef* TIMx)
00195 \{
00196   \textcolor{comment}{/* Check the parameters */}
00197   assert_param(IS\_TIM\_ALL\_PERIPH(TIMx));
00198 
00199   \textcolor{keywordflow}{if} (TIMx == TIM1)
00200   \{
00201     RCC_APB2PeriphResetCmd(RCC_APB2Periph_TIM1, ENABLE);
00202     RCC_APB2PeriphResetCmd(RCC_APB2Periph_TIM1, DISABLE);
00203   \}
00204   \textcolor{keywordflow}{else} \textcolor{keywordflow}{if} (TIMx == TIM2)
00205   \{
00206     RCC_APB1PeriphResetCmd(RCC_APB1Periph_TIM2, ENABLE);
00207     RCC_APB1PeriphResetCmd(RCC_APB1Periph_TIM2, DISABLE);
00208   \}
00209   \textcolor{keywordflow}{else} \textcolor{keywordflow}{if} (TIMx == TIM3)
00210   \{
00211     RCC_APB1PeriphResetCmd(RCC_APB1Periph_TIM3, ENABLE);
00212     RCC_APB1PeriphResetCmd(RCC_APB1Periph_TIM3, DISABLE);
00213   \}
00214   \textcolor{keywordflow}{else} \textcolor{keywordflow}{if} (TIMx == TIM4)
00215   \{
00216     RCC_APB1PeriphResetCmd(RCC_APB1Periph_TIM4, ENABLE);
00217     RCC_APB1PeriphResetCmd(RCC_APB1Periph_TIM4, DISABLE);
00218   \}
00219   \textcolor{keywordflow}{else} \textcolor{keywordflow}{if} (TIMx == TIM5)
00220   \{
00221     RCC_APB1PeriphResetCmd(RCC_APB1Periph_TIM5, ENABLE);
00222     RCC_APB1PeriphResetCmd(RCC_APB1Periph_TIM5, DISABLE);
00223   \}
00224   \textcolor{keywordflow}{else} \textcolor{keywordflow}{if} (TIMx == TIM6)
00225   \{
00226     RCC_APB1PeriphResetCmd(RCC_APB1Periph_TIM6, ENABLE);
00227     RCC_APB1PeriphResetCmd(RCC_APB1Periph_TIM6, DISABLE);
00228   \}
00229   \textcolor{keywordflow}{else} \textcolor{keywordflow}{if} (TIMx == TIM7)
00230   \{
00231     RCC_APB1PeriphResetCmd(RCC_APB1Periph_TIM7, ENABLE);
00232     RCC_APB1PeriphResetCmd(RCC_APB1Periph_TIM7, DISABLE);
00233   \}
00234   \textcolor{keywordflow}{else} \textcolor{keywordflow}{if} (TIMx == TIM8)
00235   \{
00236     RCC_APB2PeriphResetCmd(RCC_APB2Periph_TIM8, ENABLE);
00237     RCC_APB2PeriphResetCmd(RCC_APB2Periph_TIM8, DISABLE);
00238   \}
00239   \textcolor{keywordflow}{else} \textcolor{keywordflow}{if} (TIMx == TIM9)
00240   \{
00241     RCC_APB2PeriphResetCmd(RCC_APB2Periph_TIM9, ENABLE);
00242     RCC_APB2PeriphResetCmd(RCC_APB2Periph_TIM9, DISABLE);
00243    \}
00244   \textcolor{keywordflow}{else} \textcolor{keywordflow}{if} (TIMx == TIM10)
00245   \{
00246     RCC_APB2PeriphResetCmd(RCC_APB2Periph_TIM10, ENABLE);
00247     RCC_APB2PeriphResetCmd(RCC_APB2Periph_TIM10, DISABLE);
00248   \}
00249   \textcolor{keywordflow}{else} \textcolor{keywordflow}{if} (TIMx == TIM11)
00250   \{
00251     RCC_APB2PeriphResetCmd(RCC_APB2Periph_TIM11, ENABLE);
00252     RCC_APB2PeriphResetCmd(RCC_APB2Periph_TIM11, DISABLE);
00253   \}
00254   \textcolor{keywordflow}{else} \textcolor{keywordflow}{if} (TIMx == TIM12)
00255   \{
00256     RCC_APB1PeriphResetCmd(RCC_APB1Periph_TIM12, ENABLE);
00257     RCC_APB1PeriphResetCmd(RCC_APB1Periph_TIM12, DISABLE);
00258   \}
00259   \textcolor{keywordflow}{else} \textcolor{keywordflow}{if} (TIMx == TIM13)
00260   \{
00261     RCC_APB1PeriphResetCmd(RCC_APB1Periph_TIM13, ENABLE);
00262     RCC_APB1PeriphResetCmd(RCC_APB1Periph_TIM13, DISABLE);
00263   \}
00264   \textcolor{keywordflow}{else}
00265   \{
00266     \textcolor{keywordflow}{if} (TIMx == TIM14)
00267     \{
00268       RCC_APB1PeriphResetCmd(RCC_APB1Periph_TIM14, ENABLE);
00269       RCC_APB1PeriphResetCmd(RCC_APB1Periph_TIM14, DISABLE);
00270     \}
00271   \}
00272 \}
00273 
00274 \textcolor{comment}{/**}
00275 \textcolor{comment}{  * @brief  Initializes the TIMx Time Base Unit peripheral according to }
00276 \textcolor{comment}{  *         the specified parameters in the TIM\_TimeBaseInitStruct.}
00277 \textcolor{comment}{  * @param  TIMx: where x can be  1 to 14 to select the TIM peripheral.}
00278 \textcolor{comment}{  * @param  TIM\_TimeBaseInitStruct: pointer to a TIM\_TimeBaseInitTypeDef structure}
00279 \textcolor{comment}{  *         that contains the configuration information for the specified TIM peripheral.}
00280 \textcolor{comment}{  * @retval None}
00281 \textcolor{comment}{  */}
00282 \textcolor{keywordtype}{void} TIM_TimeBaseInit(TIM\_TypeDef* TIMx, TIM\_TimeBaseInitTypeDef* TIM\_TimeBaseInitStruct)
00283 \{
00284   uint16\_t tmpcr1 = 0;
00285 
00286   \textcolor{comment}{/* Check the parameters */}
00287   assert_param(IS\_TIM\_ALL\_PERIPH(TIMx));
00288   assert_param(IS\_TIM\_COUNTER\_MODE(TIM\_TimeBaseInitStruct->TIM\_CounterMode));
00289   assert_param(IS\_TIM\_CKD\_DIV(TIM\_TimeBaseInitStruct->TIM\_ClockDivision));
00290 
00291   tmpcr1 = TIMx->CR1;
00292 
00293   \textcolor{keywordflow}{if}((TIMx == TIM1) || (TIMx == TIM8)||
00294      (TIMx == TIM2) || (TIMx == TIM3)||
00295      (TIMx == TIM4) || (TIMx == TIM5))
00296   \{
00297     \textcolor{comment}{/* Select the Counter Mode */}
00298     tmpcr1 &= (uint16\_t)(~(TIM_CR1_DIR | TIM_CR1_CMS));
00299     tmpcr1 |= (uint32\_t)TIM\_TimeBaseInitStruct->TIM_CounterMode;
00300   \}
00301 
00302   \textcolor{keywordflow}{if}((TIMx != TIM6) && (TIMx != TIM7))
00303   \{
00304     \textcolor{comment}{/* Set the clock division */}
00305     tmpcr1 &=  (uint16\_t)(~TIM_CR1_CKD);
00306     tmpcr1 |= (uint32\_t)TIM\_TimeBaseInitStruct->TIM_ClockDivision;
00307   \}
00308 
00309   TIMx->CR1 = tmpcr1;
00310 
00311   \textcolor{comment}{/* Set the Autoreload value */}
00312   TIMx->ARR = TIM\_TimeBaseInitStruct->TIM\_Period ;
00313 
00314   \textcolor{comment}{/* Set the Prescaler value */}
00315   TIMx->PSC = TIM\_TimeBaseInitStruct->TIM\_Prescaler;
00316 
00317   \textcolor{keywordflow}{if} ((TIMx == TIM1) || (TIMx == TIM8))
00318   \{
00319     \textcolor{comment}{/* Set the Repetition Counter value */}
00320     TIMx->RCR = TIM\_TimeBaseInitStruct->TIM\_RepetitionCounter;
00321   \}
00322 
00323   \textcolor{comment}{/* Generate an update event to reload the Prescaler }
00324 \textcolor{comment}{     and the repetition counter(only for TIM1 and TIM8) value immediatly */}
00325   TIMx->EGR = TIM_PSCReloadMode_Immediate;
00326 \}
00327 
00328 \textcolor{comment}{/**}
00329 \textcolor{comment}{  * @brief  Fills each TIM\_TimeBaseInitStruct member with its default value.}
00330 \textcolor{comment}{  * @param  TIM\_TimeBaseInitStruct : pointer to a TIM\_TimeBaseInitTypeDef}
00331 \textcolor{comment}{  *         structure which will be initialized.}
00332 \textcolor{comment}{  * @retval None}
00333 \textcolor{comment}{  */}
00334 \textcolor{keywordtype}{void} TIM_TimeBaseStructInit(TIM\_TimeBaseInitTypeDef* TIM\_TimeBaseInitStruct)
00335 \{
00336   \textcolor{comment}{/* Set the default configuration */}
00337   TIM\_TimeBaseInitStruct->TIM_Period = 0xFFFFFFFF;
00338   TIM\_TimeBaseInitStruct->TIM_Prescaler = 0x0000;
00339   TIM\_TimeBaseInitStruct->TIM_ClockDivision = TIM_CKD_DIV1;
00340   TIM\_TimeBaseInitStruct->TIM_CounterMode = TIM_CounterMode_Up;
00341   TIM\_TimeBaseInitStruct->TIM_RepetitionCounter = 0x0000;
00342 \}
00343 
00344 \textcolor{comment}{/**}
00345 \textcolor{comment}{  * @brief  Configures the TIMx Prescaler.}
00346 \textcolor{comment}{  * @param  TIMx: where x can be  1 to 14 to select the TIM peripheral.}
00347 \textcolor{comment}{  * @param  Prescaler: specifies the Prescaler Register value}
00348 \textcolor{comment}{  * @param  TIM\_PSCReloadMode: specifies the TIM Prescaler Reload mode}
00349 \textcolor{comment}{  *          This parameter can be one of the following values:}
00350 \textcolor{comment}{  *            @arg TIM\_PSCReloadMode\_Update: The Prescaler is loaded at the update event.}
00351 \textcolor{comment}{  *            @arg TIM\_PSCReloadMode\_Immediate: The Prescaler is loaded immediatly.}
00352 \textcolor{comment}{  * @retval None}
00353 \textcolor{comment}{  */}
00354 \textcolor{keywordtype}{void} TIM_PrescalerConfig(TIM\_TypeDef* TIMx, uint16\_t Prescaler, uint16\_t TIM\_PSCReloadMode)
00355 \{
00356   \textcolor{comment}{/* Check the parameters */}
00357   assert_param(IS\_TIM\_ALL\_PERIPH(TIMx));
00358   assert_param(IS\_TIM\_PRESCALER\_RELOAD(TIM\_PSCReloadMode));
00359   \textcolor{comment}{/* Set the Prescaler value */}
00360   TIMx->PSC = Prescaler;
00361   \textcolor{comment}{/* Set or reset the UG Bit */}
00362   TIMx->EGR = TIM\_PSCReloadMode;
00363 \}
00364 
00365 \textcolor{comment}{/**}
00366 \textcolor{comment}{  * @brief  Specifies the TIMx Counter Mode to be used.}
00367 \textcolor{comment}{  * @param  TIMx: where x can be  1, 2, 3, 4, 5 or 8 to select the TIM peripheral.}
00368 \textcolor{comment}{  * @param  TIM\_CounterMode: specifies the Counter Mode to be used}
00369 \textcolor{comment}{  *          This parameter can be one of the following values:}
00370 \textcolor{comment}{  *            @arg TIM\_CounterMode\_Up: TIM Up Counting Mode}
00371 \textcolor{comment}{  *            @arg TIM\_CounterMode\_Down: TIM Down Counting Mode}
00372 \textcolor{comment}{  *            @arg TIM\_CounterMode\_CenterAligned1: TIM Center Aligned Mode1}
00373 \textcolor{comment}{  *            @arg TIM\_CounterMode\_CenterAligned2: TIM Center Aligned Mode2}
00374 \textcolor{comment}{  *            @arg TIM\_CounterMode\_CenterAligned3: TIM Center Aligned Mode3}
00375 \textcolor{comment}{  * @retval None}
00376 \textcolor{comment}{  */}
00377 \textcolor{keywordtype}{void} TIM_CounterModeConfig(TIM\_TypeDef* TIMx, uint16\_t TIM\_CounterMode)
00378 \{
00379   uint16\_t tmpcr1 = 0;
00380 
00381   \textcolor{comment}{/* Check the parameters */}
00382   assert_param(IS\_TIM\_LIST3\_PERIPH(TIMx));
00383   assert_param(IS\_TIM\_COUNTER\_MODE(TIM\_CounterMode));
00384 
00385   tmpcr1 = TIMx->CR1;
00386 
00387   \textcolor{comment}{/* Reset the CMS and DIR Bits */}
00388   tmpcr1 &= (uint16\_t)~(TIM_CR1_DIR | TIM_CR1_CMS);
00389 
00390   \textcolor{comment}{/* Set the Counter Mode */}
00391   tmpcr1 |= TIM\_CounterMode;
00392 
00393   \textcolor{comment}{/* Write to TIMx CR1 register */}
00394   TIMx->CR1 = tmpcr1;
00395 \}
00396 
00397 \textcolor{comment}{/**}
00398 \textcolor{comment}{  * @brief  Sets the TIMx Counter Register value}
00399 \textcolor{comment}{  * @param  TIMx: where x can be 1 to 14 to select the TIM peripheral.}
00400 \textcolor{comment}{  * @param  Counter: specifies the Counter register new value.}
00401 \textcolor{comment}{  * @retval None}
00402 \textcolor{comment}{  */}
00403 \textcolor{keywordtype}{void} TIM_SetCounter(TIM\_TypeDef* TIMx, uint32\_t Counter)
00404 \{
00405   \textcolor{comment}{/* Check the parameters */}
00406    assert_param(IS\_TIM\_ALL\_PERIPH(TIMx));
00407 
00408   \textcolor{comment}{/* Set the Counter Register value */}
00409   TIMx->CNT = Counter;
00410 \}
00411 
00412 \textcolor{comment}{/**}
00413 \textcolor{comment}{  * @brief  Sets the TIMx Autoreload Register value}
00414 \textcolor{comment}{  * @param  TIMx: where x can be 1 to 14 to select the TIM peripheral.}
00415 \textcolor{comment}{  * @param  Autoreload: specifies the Autoreload register new value.}
00416 \textcolor{comment}{  * @retval None}
00417 \textcolor{comment}{  */}
00418 \textcolor{keywordtype}{void} TIM_SetAutoreload(TIM\_TypeDef* TIMx, uint32\_t Autoreload)
00419 \{
00420   \textcolor{comment}{/* Check the parameters */}
00421   assert_param(IS\_TIM\_ALL\_PERIPH(TIMx));
00422 
00423   \textcolor{comment}{/* Set the Autoreload Register value */}
00424   TIMx->ARR = Autoreload;
00425 \}
00426 
00427 \textcolor{comment}{/**}
00428 \textcolor{comment}{  * @brief  Gets the TIMx Counter value.}
00429 \textcolor{comment}{  * @param  TIMx: where x can be 1 to 14 to select the TIM peripheral.}
00430 \textcolor{comment}{  * @retval Counter Register value}
00431 \textcolor{comment}{  */}
00432 uint32\_t TIM_GetCounter(TIM\_TypeDef* TIMx)
00433 \{
00434   \textcolor{comment}{/* Check the parameters */}
00435   assert_param(IS\_TIM\_ALL\_PERIPH(TIMx));
00436 
00437   \textcolor{comment}{/* Get the Counter Register value */}
00438   \textcolor{keywordflow}{return} TIMx->CNT;
00439 \}
00440 
00441 \textcolor{comment}{/**}
00442 \textcolor{comment}{  * @brief  Gets the TIMx Prescaler value.}
00443 \textcolor{comment}{  * @param  TIMx: where x can be 1 to 14 to select the TIM peripheral.}
00444 \textcolor{comment}{  * @retval Prescaler Register value.}
00445 \textcolor{comment}{  */}
00446 uint16\_t TIM_GetPrescaler(TIM\_TypeDef* TIMx)
00447 \{
00448   \textcolor{comment}{/* Check the parameters */}
00449   assert_param(IS\_TIM\_ALL\_PERIPH(TIMx));
00450 
00451   \textcolor{comment}{/* Get the Prescaler Register value */}
00452   \textcolor{keywordflow}{return} TIMx->PSC;
00453 \}
00454 
00455 \textcolor{comment}{/**}
00456 \textcolor{comment}{  * @brief  Enables or Disables the TIMx Update event.}
00457 \textcolor{comment}{  * @param  TIMx: where x can be 1 to 14 to select the TIM peripheral.}
00458 \textcolor{comment}{  * @param  NewState: new state of the TIMx UDIS bit}
00459 \textcolor{comment}{  *          This parameter can be: ENABLE or DISABLE.}
00460 \textcolor{comment}{  * @retval None}
00461 \textcolor{comment}{  */}
00462 \textcolor{keywordtype}{void} TIM_UpdateDisableConfig(TIM\_TypeDef* TIMx, FunctionalState NewState)
00463 \{
00464   \textcolor{comment}{/* Check the parameters */}
00465   assert_param(IS\_TIM\_ALL\_PERIPH(TIMx));
00466   assert_param(IS\_FUNCTIONAL\_STATE(NewState));
00467 
00468   \textcolor{keywordflow}{if} (NewState != DISABLE)
00469   \{
00470     \textcolor{comment}{/* Set the Update Disable Bit */}
00471     TIMx->CR1 |= TIM_CR1_UDIS;
00472   \}
00473   \textcolor{keywordflow}{else}
00474   \{
00475     \textcolor{comment}{/* Reset the Update Disable Bit */}
00476     TIMx->CR1 &= (uint16\_t)~TIM_CR1_UDIS;
00477   \}
00478 \}
00479 
00480 \textcolor{comment}{/**}
00481 \textcolor{comment}{  * @brief  Configures the TIMx Update Request Interrupt source.}
00482 \textcolor{comment}{  * @param  TIMx: where x can be 1 to 14 to select the TIM peripheral.}
00483 \textcolor{comment}{  * @param  TIM\_UpdateSource: specifies the Update source.}
00484 \textcolor{comment}{  *          This parameter can be one of the following values:}
00485 \textcolor{comment}{  *            @arg TIM\_UpdateSource\_Global: Source of update is the counter}
00486 \textcolor{comment}{  *                 overflow/underflow or the setting of UG bit, or an update}
00487 \textcolor{comment}{  *                 generation through the slave mode controller.}
00488 \textcolor{comment}{  *            @arg TIM\_UpdateSource\_Regular: Source of update is counter overflow/underflow.}
00489 \textcolor{comment}{  * @retval None}
00490 \textcolor{comment}{  */}
00491 \textcolor{keywordtype}{void} TIM_UpdateRequestConfig(TIM\_TypeDef* TIMx, uint16\_t TIM\_UpdateSource)
00492 \{
00493   \textcolor{comment}{/* Check the parameters */}
00494   assert_param(IS\_TIM\_ALL\_PERIPH(TIMx));
00495   assert_param(IS\_TIM\_UPDATE\_SOURCE(TIM\_UpdateSource));
00496 
00497   \textcolor{keywordflow}{if} (TIM\_UpdateSource != TIM_UpdateSource_Global)
00498   \{
00499     \textcolor{comment}{/* Set the URS Bit */}
00500     TIMx->CR1 |= TIM_CR1_URS;
00501   \}
00502   \textcolor{keywordflow}{else}
00503   \{
00504     \textcolor{comment}{/* Reset the URS Bit */}
00505     TIMx->CR1 &= (uint16\_t)~TIM_CR1_URS;
00506   \}
00507 \}
00508 
00509 \textcolor{comment}{/**}
00510 \textcolor{comment}{  * @brief  Enables or disables TIMx peripheral Preload register on ARR.}
00511 \textcolor{comment}{  * @param  TIMx: where x can be 1 to 14 to select the TIM peripheral.}
00512 \textcolor{comment}{  * @param  NewState: new state of the TIMx peripheral Preload register}
00513 \textcolor{comment}{  *          This parameter can be: ENABLE or DISABLE.}
00514 \textcolor{comment}{  * @retval None}
00515 \textcolor{comment}{  */}
00516 \textcolor{keywordtype}{void} TIM_ARRPreloadConfig(TIM\_TypeDef* TIMx, FunctionalState NewState)
00517 \{
00518   \textcolor{comment}{/* Check the parameters */}
00519   assert_param(IS\_TIM\_ALL\_PERIPH(TIMx));
00520   assert_param(IS\_FUNCTIONAL\_STATE(NewState));
00521 
00522   \textcolor{keywordflow}{if} (NewState != DISABLE)
00523   \{
00524     \textcolor{comment}{/* Set the ARR Preload Bit */}
00525     TIMx->CR1 |= TIM_CR1_ARPE;
00526   \}
00527   \textcolor{keywordflow}{else}
00528   \{
00529     \textcolor{comment}{/* Reset the ARR Preload Bit */}
00530     TIMx->CR1 &= (uint16\_t)~TIM_CR1_ARPE;
00531   \}
00532 \}
00533 
00534 \textcolor{comment}{/**}
00535 \textcolor{comment}{  * @brief  Selects the TIMx's One Pulse Mode.}
00536 \textcolor{comment}{  * @param  TIMx: where x can be 1 to 14 to select the TIM peripheral.}
00537 \textcolor{comment}{  * @param  TIM\_OPMode: specifies the OPM Mode to be used.}
00538 \textcolor{comment}{  *          This parameter can be one of the following values:}
00539 \textcolor{comment}{  *            @arg TIM\_OPMode\_Single}
00540 \textcolor{comment}{  *            @arg TIM\_OPMode\_Repetitive}
00541 \textcolor{comment}{  * @retval None}
00542 \textcolor{comment}{  */}
00543 \textcolor{keywordtype}{void} TIM_SelectOnePulseMode(TIM\_TypeDef* TIMx, uint16\_t TIM\_OPMode)
00544 \{
00545   \textcolor{comment}{/* Check the parameters */}
00546   assert_param(IS\_TIM\_ALL\_PERIPH(TIMx));
00547   assert_param(IS\_TIM\_OPM\_MODE(TIM\_OPMode));
00548 
00549   \textcolor{comment}{/* Reset the OPM Bit */}
00550   TIMx->CR1 &= (uint16\_t)~TIM_CR1_OPM;
00551 
00552   \textcolor{comment}{/* Configure the OPM Mode */}
00553   TIMx->CR1 |= TIM\_OPMode;
00554 \}
00555 
00556 \textcolor{comment}{/**}
00557 \textcolor{comment}{  * @brief  Sets the TIMx Clock Division value.}
00558 \textcolor{comment}{  * @param  TIMx: where x can be 1 to 14 except 6 and 7, to select the TIM peripheral.}
00559 \textcolor{comment}{  * @param  TIM\_CKD: specifies the clock division value.}
00560 \textcolor{comment}{  *          This parameter can be one of the following value:}
00561 \textcolor{comment}{  *            @arg TIM\_CKD\_DIV1: TDTS = Tck\_tim}
00562 \textcolor{comment}{  *            @arg TIM\_CKD\_DIV2: TDTS = 2*Tck\_tim}
00563 \textcolor{comment}{  *            @arg TIM\_CKD\_DIV4: TDTS = 4*Tck\_tim}
00564 \textcolor{comment}{  * @retval None}
00565 \textcolor{comment}{  */}
00566 \textcolor{keywordtype}{void} TIM_SetClockDivision(TIM\_TypeDef* TIMx, uint16\_t TIM\_CKD)
00567 \{
00568   \textcolor{comment}{/* Check the parameters */}
00569   assert_param(IS\_TIM\_LIST1\_PERIPH(TIMx));
00570   assert_param(IS\_TIM\_CKD\_DIV(TIM\_CKD));
00571 
00572   \textcolor{comment}{/* Reset the CKD Bits */}
00573   TIMx->CR1 &= (uint16\_t)(~TIM_CR1_CKD);
00574 
00575   \textcolor{comment}{/* Set the CKD value */}
00576   TIMx->CR1 |= TIM\_CKD;
00577 \}
00578 
00579 \textcolor{comment}{/**}
00580 \textcolor{comment}{  * @brief  Enables or disables the specified TIM peripheral.}
00581 \textcolor{comment}{  * @param  TIMx: where x can be 1 to 14 to select the TIMx peripheral.}
00582 \textcolor{comment}{  * @param  NewState: new state of the TIMx peripheral.}
00583 \textcolor{comment}{  *          This parameter can be: ENABLE or DISABLE.}
00584 \textcolor{comment}{  * @retval None}
00585 \textcolor{comment}{  */}
00586 \textcolor{keywordtype}{void} TIM_Cmd(TIM\_TypeDef* TIMx, FunctionalState NewState)
00587 \{
00588   \textcolor{comment}{/* Check the parameters */}
00589   assert_param(IS\_TIM\_ALL\_PERIPH(TIMx));
00590   assert_param(IS\_FUNCTIONAL\_STATE(NewState));
00591 
00592   \textcolor{keywordflow}{if} (NewState != DISABLE)
00593   \{
00594     \textcolor{comment}{/* Enable the TIM Counter */}
00595     TIMx->CR1 |= TIM_CR1_CEN;
00596   \}
00597   \textcolor{keywordflow}{else}
00598   \{
00599     \textcolor{comment}{/* Disable the TIM Counter */}
00600     TIMx->CR1 &= (uint16\_t)~TIM_CR1_CEN;
00601   \}
00602 \}
00603 \textcolor{comment}{/**}
00604 \textcolor{comment}{  * @\}}
00605 \textcolor{comment}{  */}
00606 
00607 \textcolor{comment}{/** @defgroup TIM\_Group2 Output Compare management functions}
00608 \textcolor{comment}{ *  @brief    Output Compare management functions }
00609 \textcolor{comment}{ *}
00610 \textcolor{comment}{@verbatim   }
00611 \textcolor{comment}{ ===============================================================================}
00612 \textcolor{comment}{                        Output Compare management functions}
00613 \textcolor{comment}{ ===============================================================================  }
00614 \textcolor{comment}{   }
00615 \textcolor{comment}{       ===================================================================      }
00616 \textcolor{comment}{              TIM Driver: how to use it in Output Compare Mode}
00617 \textcolor{comment}{       =================================================================== }
00618 \textcolor{comment}{       To use the Timer in Output Compare mode, the following steps are mandatory:}
00619 \textcolor{comment}{       }
00620 \textcolor{comment}{       1. Enable TIM clock using RCC\_APBxPeriphClockCmd(RCC\_APBxPeriph\_TIMx, ENABLE) function}
00621 \textcolor{comment}{       }
00622 \textcolor{comment}{       2. Configure the TIM pins by configuring the corresponding GPIO pins}
00623 \textcolor{comment}{       }
00624 \textcolor{comment}{       2. Configure the Time base unit as described in the first part of this driver, }
00625 \textcolor{comment}{          if needed, else the Timer will run with the default configuration:}
00626 \textcolor{comment}{          - Autoreload value = 0xFFFF}
00627 \textcolor{comment}{          - Prescaler value = 0x0000}
00628 \textcolor{comment}{          - Counter mode = Up counting}
00629 \textcolor{comment}{          - Clock Division = TIM\_CKD\_DIV1}
00630 \textcolor{comment}{          }
00631 \textcolor{comment}{       3. Fill the TIM\_OCInitStruct with the desired parameters including:}
00632 \textcolor{comment}{          - The TIM Output Compare mode: TIM\_OCMode}
00633 \textcolor{comment}{          - TIM Output State: TIM\_OutputState}
00634 \textcolor{comment}{          - TIM Pulse value: TIM\_Pulse}
00635 \textcolor{comment}{          - TIM Output Compare Polarity : TIM\_OCPolarity}
00636 \textcolor{comment}{       }
00637 \textcolor{comment}{       4. Call TIM\_OCxInit(TIMx, &TIM\_OCInitStruct) to configure the desired channel with the }
00638 \textcolor{comment}{          corresponding configuration}
00639 \textcolor{comment}{       }
00640 \textcolor{comment}{       5. Call the TIM\_Cmd(ENABLE) function to enable the TIM counter.}
00641 \textcolor{comment}{       }
00642 \textcolor{comment}{       Note1: All other functions can be used separately to modify, if needed,}
00643 \textcolor{comment}{              a specific feature of the Timer. }
00644 \textcolor{comment}{          }
00645 \textcolor{comment}{       Note2: In case of PWM mode, this function is mandatory:}
00646 \textcolor{comment}{              TIM\_OCxPreloadConfig(TIMx, TIM\_OCPreload\_ENABLE); }
00647 \textcolor{comment}{              }
00648 \textcolor{comment}{       Note3: If the corresponding interrupt or DMA request are needed, the user should:}
00649 \textcolor{comment}{                1. Enable the NVIC (or the DMA) to use the TIM interrupts (or DMA requests). }
00650 \textcolor{comment}{                2. Enable the corresponding interrupt (or DMA request) using the function }
00651 \textcolor{comment}{                   TIM\_ITConfig(TIMx, TIM\_IT\_CCx) (or TIM\_DMA\_Cmd(TIMx, TIM\_DMA\_CCx))   }
00652 \textcolor{comment}{}
00653 \textcolor{comment}{@endverbatim}
00654 \textcolor{comment}{  * @\{}
00655 \textcolor{comment}{  */}
00656 
00657 \textcolor{comment}{/**}
00658 \textcolor{comment}{  * @brief  Initializes the TIMx Channel1 according to the specified parameters in}
00659 \textcolor{comment}{  *         the TIM\_OCInitStruct.}
00660 \textcolor{comment}{  * @param  TIMx: where x can be 1 to 14 except 6 and 7, to select the TIM peripheral.}
00661 \textcolor{comment}{  * @param  TIM\_OCInitStruct: pointer to a TIM\_OCInitTypeDef structure that contains}
00662 \textcolor{comment}{  *         the configuration information for the specified TIM peripheral.}
00663 \textcolor{comment}{  * @retval None}
00664 \textcolor{comment}{  */}
00665 \textcolor{keywordtype}{void} TIM_OC1Init(TIM\_TypeDef* TIMx, TIM\_OCInitTypeDef* TIM\_OCInitStruct)
00666 \{
00667   uint16\_t tmpccmrx = 0, tmpccer = 0, tmpcr2 = 0;
00668 
00669   \textcolor{comment}{/* Check the parameters */}
00670   assert_param(IS\_TIM\_LIST1\_PERIPH(TIMx));
00671   assert_param(IS\_TIM\_OC\_MODE(TIM\_OCInitStruct->TIM\_OCMode));
00672   assert_param(IS\_TIM\_OUTPUT\_STATE(TIM\_OCInitStruct->TIM\_OutputState));
00673   assert_param(IS\_TIM\_OC\_POLARITY(TIM\_OCInitStruct->TIM\_OCPolarity));
00674 
00675   \textcolor{comment}{/* Disable the Channel 1: Reset the CC1E Bit */}
00676   TIMx->CCER &= (uint16\_t)~TIM_CCER_CC1E;
00677 
00678   \textcolor{comment}{/* Get the TIMx CCER register value */}
00679   tmpccer = TIMx->CCER;
00680   \textcolor{comment}{/* Get the TIMx CR2 register value */}
00681   tmpcr2 =  TIMx->CR2;
00682 
00683   \textcolor{comment}{/* Get the TIMx CCMR1 register value */}
00684   tmpccmrx = TIMx->CCMR1;
00685 
00686   \textcolor{comment}{/* Reset the Output Compare Mode Bits */}
00687   tmpccmrx &= (uint16\_t)~TIM_CCMR1_OC1M;
00688   tmpccmrx &= (uint16\_t)~TIM_CCMR1_CC1S;
00689   \textcolor{comment}{/* Select the Output Compare Mode */}
00690   tmpccmrx |= TIM\_OCInitStruct->TIM_OCMode;
00691 
00692   \textcolor{comment}{/* Reset the Output Polarity level */}
00693   tmpccer &= (uint16\_t)~TIM_CCER_CC1P;
00694   \textcolor{comment}{/* Set the Output Compare Polarity */}
00695   tmpccer |= TIM\_OCInitStruct->TIM_OCPolarity;
00696 
00697   \textcolor{comment}{/* Set the Output State */}
00698   tmpccer |= TIM\_OCInitStruct->TIM_OutputState;
00699 
00700   \textcolor{keywordflow}{if}((TIMx == TIM1) || (TIMx == TIM8))
00701   \{
00702     assert_param(IS\_TIM\_OUTPUTN\_STATE(TIM\_OCInitStruct->TIM\_OutputNState));
00703     assert_param(IS\_TIM\_OCN\_POLARITY(TIM\_OCInitStruct->TIM\_OCNPolarity));
00704     assert_param(IS\_TIM\_OCNIDLE\_STATE(TIM\_OCInitStruct->TIM\_OCNIdleState));
00705     assert_param(IS\_TIM\_OCIDLE\_STATE(TIM\_OCInitStruct->TIM\_OCIdleState));
00706 
00707     \textcolor{comment}{/* Reset the Output N Polarity level */}
00708     tmpccer &= (uint16\_t)~TIM_CCER_CC1NP;
00709     \textcolor{comment}{/* Set the Output N Polarity */}
00710     tmpccer |= TIM\_OCInitStruct->TIM_OCNPolarity;
00711     \textcolor{comment}{/* Reset the Output N State */}
00712     tmpccer &= (uint16\_t)~TIM_CCER_CC1NE;
00713 
00714     \textcolor{comment}{/* Set the Output N State */}
00715     tmpccer |= TIM\_OCInitStruct->TIM_OutputNState;
00716     \textcolor{comment}{/* Reset the Output Compare and Output Compare N IDLE State */}
00717     tmpcr2 &= (uint16\_t)~TIM_CR2_OIS1;
00718     tmpcr2 &= (uint16\_t)~TIM_CR2_OIS1N;
00719     \textcolor{comment}{/* Set the Output Idle state */}
00720     tmpcr2 |= TIM\_OCInitStruct->TIM_OCIdleState;
00721     \textcolor{comment}{/* Set the Output N Idle state */}
00722     tmpcr2 |= TIM\_OCInitStruct->TIM_OCNIdleState;
00723   \}
00724   \textcolor{comment}{/* Write to TIMx CR2 */}
00725   TIMx->CR2 = tmpcr2;
00726 
00727   \textcolor{comment}{/* Write to TIMx CCMR1 */}
00728   TIMx->CCMR1 = tmpccmrx;
00729 
00730   \textcolor{comment}{/* Set the Capture Compare Register value */}
00731   TIMx->CCR1 = TIM\_OCInitStruct->TIM\_Pulse;
00732 
00733   \textcolor{comment}{/* Write to TIMx CCER */}
00734   TIMx->CCER = tmpccer;
00735 \}
00736 
00737 \textcolor{comment}{/**}
00738 \textcolor{comment}{  * @brief  Initializes the TIMx Channel2 according to the specified parameters }
00739 \textcolor{comment}{  *         in the TIM\_OCInitStruct.}
00740 \textcolor{comment}{  * @param  TIMx: where x can be 1, 2, 3, 4, 5, 8, 9 or 12 to select the TIM }
00741 \textcolor{comment}{  *         peripheral.}
00742 \textcolor{comment}{  * @param  TIM\_OCInitStruct: pointer to a TIM\_OCInitTypeDef structure that contains}
00743 \textcolor{comment}{  *         the configuration information for the specified TIM peripheral.}
00744 \textcolor{comment}{  * @retval None}
00745 \textcolor{comment}{  */}
00746 \textcolor{keywordtype}{void} TIM_OC2Init(TIM\_TypeDef* TIMx, TIM\_OCInitTypeDef* TIM\_OCInitStruct)
00747 \{
00748   uint16\_t tmpccmrx = 0, tmpccer = 0, tmpcr2 = 0;
00749 
00750   \textcolor{comment}{/* Check the parameters */}
00751   assert_param(IS\_TIM\_LIST2\_PERIPH(TIMx));
00752   assert_param(IS\_TIM\_OC\_MODE(TIM\_OCInitStruct->TIM\_OCMode));
00753   assert_param(IS\_TIM\_OUTPUT\_STATE(TIM\_OCInitStruct->TIM\_OutputState));
00754   assert_param(IS\_TIM\_OC\_POLARITY(TIM\_OCInitStruct->TIM\_OCPolarity));
00755 
00756   \textcolor{comment}{/* Disable the Channel 2: Reset the CC2E Bit */}
00757   TIMx->CCER &= (uint16\_t)~TIM_CCER_CC2E;
00758 
00759   \textcolor{comment}{/* Get the TIMx CCER register value */}
00760   tmpccer = TIMx->CCER;
00761   \textcolor{comment}{/* Get the TIMx CR2 register value */}
00762   tmpcr2 =  TIMx->CR2;
00763 
00764   \textcolor{comment}{/* Get the TIMx CCMR1 register value */}
00765   tmpccmrx = TIMx->CCMR1;
00766 
00767   \textcolor{comment}{/* Reset the Output Compare mode and Capture/Compare selection Bits */}
00768   tmpccmrx &= (uint16\_t)~TIM_CCMR1_OC2M;
00769   tmpccmrx &= (uint16\_t)~TIM_CCMR1_CC2S;
00770 
00771   \textcolor{comment}{/* Select the Output Compare Mode */}
00772   tmpccmrx |= (uint16\_t)(TIM\_OCInitStruct->TIM_OCMode << 8);
00773 
00774   \textcolor{comment}{/* Reset the Output Polarity level */}
00775   tmpccer &= (uint16\_t)~TIM_CCER_CC2P;
00776   \textcolor{comment}{/* Set the Output Compare Polarity */}
00777   tmpccer |= (uint16\_t)(TIM\_OCInitStruct->TIM_OCPolarity << 4);
00778 
00779   \textcolor{comment}{/* Set the Output State */}
00780   tmpccer |= (uint16\_t)(TIM\_OCInitStruct->TIM_OutputState << 4);
00781 
00782   \textcolor{keywordflow}{if}((TIMx == TIM1) || (TIMx == TIM8))
00783   \{
00784     assert_param(IS\_TIM\_OUTPUTN\_STATE(TIM\_OCInitStruct->TIM\_OutputNState));
00785     assert_param(IS\_TIM\_OCN\_POLARITY(TIM\_OCInitStruct->TIM\_OCNPolarity));
00786     assert_param(IS\_TIM\_OCNIDLE\_STATE(TIM\_OCInitStruct->TIM\_OCNIdleState));
00787     assert_param(IS\_TIM\_OCIDLE\_STATE(TIM\_OCInitStruct->TIM\_OCIdleState));
00788 
00789     \textcolor{comment}{/* Reset the Output N Polarity level */}
00790     tmpccer &= (uint16\_t)~TIM_CCER_CC2NP;
00791     \textcolor{comment}{/* Set the Output N Polarity */}
00792     tmpccer |= (uint16\_t)(TIM\_OCInitStruct->TIM_OCNPolarity << 4);
00793     \textcolor{comment}{/* Reset the Output N State */}
00794     tmpccer &= (uint16\_t)~TIM_CCER_CC2NE;
00795 
00796     \textcolor{comment}{/* Set the Output N State */}
00797     tmpccer |= (uint16\_t)(TIM\_OCInitStruct->TIM_OutputNState << 4);
00798     \textcolor{comment}{/* Reset the Output Compare and Output Compare N IDLE State */}
00799     tmpcr2 &= (uint16\_t)~TIM_CR2_OIS2;
00800     tmpcr2 &= (uint16\_t)~TIM_CR2_OIS2N;
00801     \textcolor{comment}{/* Set the Output Idle state */}
00802     tmpcr2 |= (uint16\_t)(TIM\_OCInitStruct->TIM_OCIdleState << 2);
00803     \textcolor{comment}{/* Set the Output N Idle state */}
00804     tmpcr2 |= (uint16\_t)(TIM\_OCInitStruct->TIM_OCNIdleState << 2);
00805   \}
00806   \textcolor{comment}{/* Write to TIMx CR2 */}
00807   TIMx->CR2 = tmpcr2;
00808 
00809   \textcolor{comment}{/* Write to TIMx CCMR1 */}
00810   TIMx->CCMR1 = tmpccmrx;
00811 
00812   \textcolor{comment}{/* Set the Capture Compare Register value */}
00813   TIMx->CCR2 = TIM\_OCInitStruct->TIM\_Pulse;
00814 
00815   \textcolor{comment}{/* Write to TIMx CCER */}
00816   TIMx->CCER = tmpccer;
00817 \}
00818 
00819 \textcolor{comment}{/**}
00820 \textcolor{comment}{  * @brief  Initializes the TIMx Channel3 according to the specified parameters}
00821 \textcolor{comment}{  *         in the TIM\_OCInitStruct.}
00822 \textcolor{comment}{  * @param  TIMx: where x can be 1, 2, 3, 4, 5 or 8 to select the TIM peripheral.}
00823 \textcolor{comment}{  * @param  TIM\_OCInitStruct: pointer to a TIM\_OCInitTypeDef structure that contains}
00824 \textcolor{comment}{  *         the configuration information for the specified TIM peripheral.}
00825 \textcolor{comment}{  * @retval None}
00826 \textcolor{comment}{  */}
00827 \textcolor{keywordtype}{void} TIM_OC3Init(TIM\_TypeDef* TIMx, TIM\_OCInitTypeDef* TIM\_OCInitStruct)
00828 \{
00829   uint16\_t tmpccmrx = 0, tmpccer = 0, tmpcr2 = 0;
00830 
00831   \textcolor{comment}{/* Check the parameters */}
00832   assert_param(IS\_TIM\_LIST3\_PERIPH(TIMx));
00833   assert_param(IS\_TIM\_OC\_MODE(TIM\_OCInitStruct->TIM\_OCMode));
00834   assert_param(IS\_TIM\_OUTPUT\_STATE(TIM\_OCInitStruct->TIM\_OutputState));
00835   assert_param(IS\_TIM\_OC\_POLARITY(TIM\_OCInitStruct->TIM\_OCPolarity));
00836 
00837   \textcolor{comment}{/* Disable the Channel 3: Reset the CC2E Bit */}
00838   TIMx->CCER &= (uint16\_t)~TIM_CCER_CC3E;
00839 
00840   \textcolor{comment}{/* Get the TIMx CCER register value */}
00841   tmpccer = TIMx->CCER;
00842   \textcolor{comment}{/* Get the TIMx CR2 register value */}
00843   tmpcr2 =  TIMx->CR2;
00844 
00845   \textcolor{comment}{/* Get the TIMx CCMR2 register value */}
00846   tmpccmrx = TIMx->CCMR2;
00847 
00848   \textcolor{comment}{/* Reset the Output Compare mode and Capture/Compare selection Bits */}
00849   tmpccmrx &= (uint16\_t)~TIM_CCMR2_OC3M;
00850   tmpccmrx &= (uint16\_t)~TIM_CCMR2_CC3S;
00851   \textcolor{comment}{/* Select the Output Compare Mode */}
00852   tmpccmrx |= TIM\_OCInitStruct->TIM_OCMode;
00853 
00854   \textcolor{comment}{/* Reset the Output Polarity level */}
00855   tmpccer &= (uint16\_t)~TIM_CCER_CC3P;
00856   \textcolor{comment}{/* Set the Output Compare Polarity */}
00857   tmpccer |= (uint16\_t)(TIM\_OCInitStruct->TIM_OCPolarity << 8);
00858 
00859   \textcolor{comment}{/* Set the Output State */}
00860   tmpccer |= (uint16\_t)(TIM\_OCInitStruct->TIM_OutputState << 8);
00861 
00862   \textcolor{keywordflow}{if}((TIMx == TIM1) || (TIMx == TIM8))
00863   \{
00864     assert_param(IS\_TIM\_OUTPUTN\_STATE(TIM\_OCInitStruct->TIM\_OutputNState));
00865     assert_param(IS\_TIM\_OCN\_POLARITY(TIM\_OCInitStruct->TIM\_OCNPolarity));
00866     assert_param(IS\_TIM\_OCNIDLE\_STATE(TIM\_OCInitStruct->TIM\_OCNIdleState));
00867     assert_param(IS\_TIM\_OCIDLE\_STATE(TIM\_OCInitStruct->TIM\_OCIdleState));
00868 
00869     \textcolor{comment}{/* Reset the Output N Polarity level */}
00870     tmpccer &= (uint16\_t)~TIM_CCER_CC3NP;
00871     \textcolor{comment}{/* Set the Output N Polarity */}
00872     tmpccer |= (uint16\_t)(TIM\_OCInitStruct->TIM_OCNPolarity << 8);
00873     \textcolor{comment}{/* Reset the Output N State */}
00874     tmpccer &= (uint16\_t)~TIM_CCER_CC3NE;
00875 
00876     \textcolor{comment}{/* Set the Output N State */}
00877     tmpccer |= (uint16\_t)(TIM\_OCInitStruct->TIM_OutputNState << 8);
00878     \textcolor{comment}{/* Reset the Output Compare and Output Compare N IDLE State */}
00879     tmpcr2 &= (uint16\_t)~TIM_CR2_OIS3;
00880     tmpcr2 &= (uint16\_t)~TIM_CR2_OIS3N;
00881     \textcolor{comment}{/* Set the Output Idle state */}
00882     tmpcr2 |= (uint16\_t)(TIM\_OCInitStruct->TIM_OCIdleState << 4);
00883     \textcolor{comment}{/* Set the Output N Idle state */}
00884     tmpcr2 |= (uint16\_t)(TIM\_OCInitStruct->TIM_OCNIdleState << 4);
00885   \}
00886   \textcolor{comment}{/* Write to TIMx CR2 */}
00887   TIMx->CR2 = tmpcr2;
00888 
00889   \textcolor{comment}{/* Write to TIMx CCMR2 */}
00890   TIMx->CCMR2 = tmpccmrx;
00891 
00892   \textcolor{comment}{/* Set the Capture Compare Register value */}
00893   TIMx->CCR3 = TIM\_OCInitStruct->TIM\_Pulse;
00894 
00895   \textcolor{comment}{/* Write to TIMx CCER */}
00896   TIMx->CCER = tmpccer;
00897 \}
00898 
00899 \textcolor{comment}{/**}
00900 \textcolor{comment}{  * @brief  Initializes the TIMx Channel4 according to the specified parameters}
00901 \textcolor{comment}{  *         in the TIM\_OCInitStruct.}
00902 \textcolor{comment}{  * @param  TIMx: where x can be 1, 2, 3, 4, 5 or 8 to select the TIM peripheral.}
00903 \textcolor{comment}{  * @param  TIM\_OCInitStruct: pointer to a TIM\_OCInitTypeDef structure that contains}
00904 \textcolor{comment}{  *         the configuration information for the specified TIM peripheral.}
00905 \textcolor{comment}{  * @retval None}
00906 \textcolor{comment}{  */}
00907 \textcolor{keywordtype}{void} TIM_OC4Init(TIM\_TypeDef* TIMx, TIM\_OCInitTypeDef* TIM\_OCInitStruct)
00908 \{
00909   uint16\_t tmpccmrx = 0, tmpccer = 0, tmpcr2 = 0;
00910 
00911   \textcolor{comment}{/* Check the parameters */}
00912   assert_param(IS\_TIM\_LIST3\_PERIPH(TIMx));
00913   assert_param(IS\_TIM\_OC\_MODE(TIM\_OCInitStruct->TIM\_OCMode));
00914   assert_param(IS\_TIM\_OUTPUT\_STATE(TIM\_OCInitStruct->TIM\_OutputState));
00915   assert_param(IS\_TIM\_OC\_POLARITY(TIM\_OCInitStruct->TIM\_OCPolarity));
00916 
00917   \textcolor{comment}{/* Disable the Channel 4: Reset the CC4E Bit */}
00918   TIMx->CCER &= (uint16\_t)~TIM_CCER_CC4E;
00919 
00920   \textcolor{comment}{/* Get the TIMx CCER register value */}
00921   tmpccer = TIMx->CCER;
00922   \textcolor{comment}{/* Get the TIMx CR2 register value */}
00923   tmpcr2 =  TIMx->CR2;
00924 
00925   \textcolor{comment}{/* Get the TIMx CCMR2 register value */}
00926   tmpccmrx = TIMx->CCMR2;
00927 
00928   \textcolor{comment}{/* Reset the Output Compare mode and Capture/Compare selection Bits */}
00929   tmpccmrx &= (uint16\_t)~TIM_CCMR2_OC4M;
00930   tmpccmrx &= (uint16\_t)~TIM_CCMR2_CC4S;
00931 
00932   \textcolor{comment}{/* Select the Output Compare Mode */}
00933   tmpccmrx |= (uint16\_t)(TIM\_OCInitStruct->TIM_OCMode << 8);
00934 
00935   \textcolor{comment}{/* Reset the Output Polarity level */}
00936   tmpccer &= (uint16\_t)~TIM_CCER_CC4P;
00937   \textcolor{comment}{/* Set the Output Compare Polarity */}
00938   tmpccer |= (uint16\_t)(TIM\_OCInitStruct->TIM_OCPolarity << 12);
00939 
00940   \textcolor{comment}{/* Set the Output State */}
00941   tmpccer |= (uint16\_t)(TIM\_OCInitStruct->TIM_OutputState << 12);
00942 
00943   \textcolor{keywordflow}{if}((TIMx == TIM1) || (TIMx == TIM8))
00944   \{
00945     assert_param(IS\_TIM\_OCIDLE\_STATE(TIM\_OCInitStruct->TIM\_OCIdleState));
00946     \textcolor{comment}{/* Reset the Output Compare IDLE State */}
00947     tmpcr2 &=(uint16\_t) ~TIM_CR2_OIS4;
00948     \textcolor{comment}{/* Set the Output Idle state */}
00949     tmpcr2 |= (uint16\_t)(TIM\_OCInitStruct->TIM_OCIdleState << 6);
00950   \}
00951   \textcolor{comment}{/* Write to TIMx CR2 */}
00952   TIMx->CR2 = tmpcr2;
00953 
00954   \textcolor{comment}{/* Write to TIMx CCMR2 */}
00955   TIMx->CCMR2 = tmpccmrx;
00956 
00957   \textcolor{comment}{/* Set the Capture Compare Register value */}
00958   TIMx->CCR4 = TIM\_OCInitStruct->TIM\_Pulse;
00959 
00960   \textcolor{comment}{/* Write to TIMx CCER */}
00961   TIMx->CCER = tmpccer;
00962 \}
00963 
00964 \textcolor{comment}{/**}
00965 \textcolor{comment}{  * @brief  Fills each TIM\_OCInitStruct member with its default value.}
00966 \textcolor{comment}{  * @param  TIM\_OCInitStruct: pointer to a TIM\_OCInitTypeDef structure which will}
00967 \textcolor{comment}{  *         be initialized.}
00968 \textcolor{comment}{  * @retval None}
00969 \textcolor{comment}{  */}
00970 \textcolor{keywordtype}{void} TIM_OCStructInit(TIM\_OCInitTypeDef* TIM\_OCInitStruct)
00971 \{
00972   \textcolor{comment}{/* Set the default configuration */}
00973   TIM\_OCInitStruct->TIM_OCMode = TIM_OCMode_Timing;
00974   TIM\_OCInitStruct->TIM_OutputState = TIM_OutputState_Disable;
00975   TIM\_OCInitStruct->TIM_OutputNState = TIM_OutputNState_Disable;
00976   TIM\_OCInitStruct->TIM_Pulse = 0x00000000;
00977   TIM\_OCInitStruct->TIM_OCPolarity = TIM_OCPolarity_High;
00978   TIM\_OCInitStruct->TIM_OCNPolarity = TIM_OCPolarity_High;
00979   TIM\_OCInitStruct->TIM_OCIdleState = TIM_OCIdleState_Reset;
00980   TIM\_OCInitStruct->TIM_OCNIdleState = TIM_OCNIdleState_Reset;
00981 \}
00982 
00983 \textcolor{comment}{/**}
00984 \textcolor{comment}{  * @brief  Selects the TIM Output Compare Mode.}
00985 \textcolor{comment}{  * @note   This function disables the selected channel before changing the Output}
00986 \textcolor{comment}{  *         Compare Mode. If needed, user has to enable this channel using}
00987 \textcolor{comment}{  *         TIM\_CCxCmd() and TIM\_CCxNCmd() functions.}
00988 \textcolor{comment}{  * @param  TIMx: where x can be 1 to 14 except 6 and 7, to select the TIM peripheral.}
00989 \textcolor{comment}{  * @param  TIM\_Channel: specifies the TIM Channel}
00990 \textcolor{comment}{  *          This parameter can be one of the following values:}
00991 \textcolor{comment}{  *            @arg TIM\_Channel\_1: TIM Channel 1}
00992 \textcolor{comment}{  *            @arg TIM\_Channel\_2: TIM Channel 2}
00993 \textcolor{comment}{  *            @arg TIM\_Channel\_3: TIM Channel 3}
00994 \textcolor{comment}{  *            @arg TIM\_Channel\_4: TIM Channel 4}
00995 \textcolor{comment}{  * @param  TIM\_OCMode: specifies the TIM Output Compare Mode.}
00996 \textcolor{comment}{  *           This parameter can be one of the following values:}
00997 \textcolor{comment}{  *            @arg TIM\_OCMode\_Timing}
00998 \textcolor{comment}{  *            @arg TIM\_OCMode\_Active}
00999 \textcolor{comment}{  *            @arg TIM\_OCMode\_Toggle}
01000 \textcolor{comment}{  *            @arg TIM\_OCMode\_PWM1}
01001 \textcolor{comment}{  *            @arg TIM\_OCMode\_PWM2}
01002 \textcolor{comment}{  *            @arg TIM\_ForcedAction\_Active}
01003 \textcolor{comment}{  *            @arg TIM\_ForcedAction\_InActive}
01004 \textcolor{comment}{  * @retval None}
01005 \textcolor{comment}{  */}
01006 \textcolor{keywordtype}{void} TIM_SelectOCxM(TIM\_TypeDef* TIMx, uint16\_t TIM\_Channel, uint16\_t TIM\_OCMode)
01007 \{
01008   uint32\_t tmp = 0;
01009   uint16\_t tmp1 = 0;
01010 
01011   \textcolor{comment}{/* Check the parameters */}
01012   assert_param(IS\_TIM\_LIST1\_PERIPH(TIMx));
01013   assert_param(IS\_TIM\_CHANNEL(TIM\_Channel));
01014   assert_param(IS\_TIM\_OCM(TIM\_OCMode));
01015 
01016   tmp = (uint32\_t) TIMx;
01017   tmp += CCMR_OFFSET;
01018 
01019   tmp1 = CCER_CCE_SET << (uint16\_t)TIM\_Channel;
01020 
01021   \textcolor{comment}{/* Disable the Channel: Reset the CCxE Bit */}
01022   TIMx->CCER &= (uint16\_t) ~tmp1;
01023 
01024   \textcolor{keywordflow}{if}((TIM\_Channel == TIM_Channel_1) ||(TIM\_Channel == TIM_Channel_3))
01025   \{
01026     tmp += (TIM\_Channel>>1);
01027 
01028     \textcolor{comment}{/* Reset the OCxM bits in the CCMRx register */}
01029     *(\_\_IO uint32\_t *) tmp &= CCMR_OC13M_MASK;
01030 
01031     \textcolor{comment}{/* Configure the OCxM bits in the CCMRx register */}
01032     *(\_\_IO uint32\_t *) tmp |= TIM\_OCMode;
01033   \}
01034   \textcolor{keywordflow}{else}
01035   \{
01036     tmp += (uint16\_t)(TIM\_Channel - (uint16\_t)4)>> (uint16\_t)1;
01037 
01038     \textcolor{comment}{/* Reset the OCxM bits in the CCMRx register */}
01039     *(\_\_IO uint32\_t *) tmp &= CCMR_OC24M_MASK;
01040 
01041     \textcolor{comment}{/* Configure the OCxM bits in the CCMRx register */}
01042     *(\_\_IO uint32\_t *) tmp |= (uint16\_t)(TIM\_OCMode << 8);
01043   \}
01044 \}
01045 
01046 \textcolor{comment}{/**}
01047 \textcolor{comment}{  * @brief  Sets the TIMx Capture Compare1 Register value}
01048 \textcolor{comment}{  * @param  TIMx: where x can be 1 to 14 except 6 and 7, to select the TIM peripheral.}
01049 \textcolor{comment}{  * @param  Compare1: specifies the Capture Compare1 register new value.}
01050 \textcolor{comment}{  * @retval None}
01051 \textcolor{comment}{  */}
01052 \textcolor{keywordtype}{void} TIM_SetCompare1(TIM\_TypeDef* TIMx, uint32\_t Compare1)
01053 \{
01054   \textcolor{comment}{/* Check the parameters */}
01055   assert_param(IS\_TIM\_LIST1\_PERIPH(TIMx));
01056 
01057   \textcolor{comment}{/* Set the Capture Compare1 Register value */}
01058   TIMx->CCR1 = Compare1;
01059 \}
01060 
01061 \textcolor{comment}{/**}
01062 \textcolor{comment}{  * @brief  Sets the TIMx Capture Compare2 Register value}
01063 \textcolor{comment}{  * @param  TIMx: where x can be 1, 2, 3, 4, 5, 8, 9 or 12 to select the TIM }
01064 \textcolor{comment}{  *         peripheral.}
01065 \textcolor{comment}{  * @param  Compare2: specifies the Capture Compare2 register new value.}
01066 \textcolor{comment}{  * @retval None}
01067 \textcolor{comment}{  */}
01068 \textcolor{keywordtype}{void} TIM_SetCompare2(TIM\_TypeDef* TIMx, uint32\_t Compare2)
01069 \{
01070   \textcolor{comment}{/* Check the parameters */}
01071   assert_param(IS\_TIM\_LIST2\_PERIPH(TIMx));
01072 
01073   \textcolor{comment}{/* Set the Capture Compare2 Register value */}
01074   TIMx->CCR2 = Compare2;
01075 \}
01076 
01077 \textcolor{comment}{/**}
01078 \textcolor{comment}{  * @brief  Sets the TIMx Capture Compare3 Register value}
01079 \textcolor{comment}{  * @param  TIMx: where x can be 1, 2, 3, 4, 5 or 8 to select the TIM peripheral.}
01080 \textcolor{comment}{  * @param  Compare3: specifies the Capture Compare3 register new value.}
01081 \textcolor{comment}{  * @retval None}
01082 \textcolor{comment}{  */}
01083 \textcolor{keywordtype}{void} TIM_SetCompare3(TIM\_TypeDef* TIMx, uint32\_t Compare3)
01084 \{
01085   \textcolor{comment}{/* Check the parameters */}
01086   assert_param(IS\_TIM\_LIST3\_PERIPH(TIMx));
01087 
01088   \textcolor{comment}{/* Set the Capture Compare3 Register value */}
01089   TIMx->CCR3 = Compare3;
01090 \}
01091 
01092 \textcolor{comment}{/**}
01093 \textcolor{comment}{  * @brief  Sets the TIMx Capture Compare4 Register value}
01094 \textcolor{comment}{  * @param  TIMx: where x can be 1, 2, 3, 4, 5 or 8 to select the TIM peripheral.}
01095 \textcolor{comment}{  * @param  Compare4: specifies the Capture Compare4 register new value.}
01096 \textcolor{comment}{  * @retval None}
01097 \textcolor{comment}{  */}
01098 \textcolor{keywordtype}{void} TIM_SetCompare4(TIM\_TypeDef* TIMx, uint32\_t Compare4)
01099 \{
01100   \textcolor{comment}{/* Check the parameters */}
01101   assert_param(IS\_TIM\_LIST3\_PERIPH(TIMx));
01102 
01103   \textcolor{comment}{/* Set the Capture Compare4 Register value */}
01104   TIMx->CCR4 = Compare4;
01105 \}
01106 
01107 \textcolor{comment}{/**}
01108 \textcolor{comment}{  * @brief  Forces the TIMx output 1 waveform to active or inactive level.}
01109 \textcolor{comment}{  * @param  TIMx: where x can be 1 to 14 except 6 and 7, to select the TIM peripheral.}
01110 \textcolor{comment}{  * @param  TIM\_ForcedAction: specifies the forced Action to be set to the output waveform.}
01111 \textcolor{comment}{  *          This parameter can be one of the following values:}
01112 \textcolor{comment}{  *            @arg TIM\_ForcedAction\_Active: Force active level on OC1REF}
01113 \textcolor{comment}{  *            @arg TIM\_ForcedAction\_InActive: Force inactive level on OC1REF.}
01114 \textcolor{comment}{  * @retval None}
01115 \textcolor{comment}{  */}
01116 \textcolor{keywordtype}{void} TIM_ForcedOC1Config(TIM\_TypeDef* TIMx, uint16\_t TIM\_ForcedAction)
01117 \{
01118   uint16\_t tmpccmr1 = 0;
01119 
01120   \textcolor{comment}{/* Check the parameters */}
01121   assert_param(IS\_TIM\_LIST1\_PERIPH(TIMx));
01122   assert_param(IS\_TIM\_FORCED\_ACTION(TIM\_ForcedAction));
01123   tmpccmr1 = TIMx->CCMR1;
01124 
01125   \textcolor{comment}{/* Reset the OC1M Bits */}
01126   tmpccmr1 &= (uint16\_t)~TIM_CCMR1_OC1M;
01127 
01128   \textcolor{comment}{/* Configure The Forced output Mode */}
01129   tmpccmr1 |= TIM\_ForcedAction;
01130 
01131   \textcolor{comment}{/* Write to TIMx CCMR1 register */}
01132   TIMx->CCMR1 = tmpccmr1;
01133 \}
01134 
01135 \textcolor{comment}{/**}
01136 \textcolor{comment}{  * @brief  Forces the TIMx output 2 waveform to active or inactive level.}
01137 \textcolor{comment}{  * @param  TIMx: where x can be  1, 2, 3, 4, 5, 8, 9 or 12 to select the TIM }
01138 \textcolor{comment}{  *         peripheral.}
01139 \textcolor{comment}{  * @param  TIM\_ForcedAction: specifies the forced Action to be set to the output waveform.}
01140 \textcolor{comment}{  *          This parameter can be one of the following values:}
01141 \textcolor{comment}{  *            @arg TIM\_ForcedAction\_Active: Force active level on OC2REF}
01142 \textcolor{comment}{  *            @arg TIM\_ForcedAction\_InActive: Force inactive level on OC2REF.}
01143 \textcolor{comment}{  * @retval None}
01144 \textcolor{comment}{  */}
01145 \textcolor{keywordtype}{void} TIM_ForcedOC2Config(TIM\_TypeDef* TIMx, uint16\_t TIM\_ForcedAction)
01146 \{
01147   uint16\_t tmpccmr1 = 0;
01148 
01149   \textcolor{comment}{/* Check the parameters */}
01150   assert_param(IS\_TIM\_LIST2\_PERIPH(TIMx));
01151   assert_param(IS\_TIM\_FORCED\_ACTION(TIM\_ForcedAction));
01152   tmpccmr1 = TIMx->CCMR1;
01153 
01154   \textcolor{comment}{/* Reset the OC2M Bits */}
01155   tmpccmr1 &= (uint16\_t)~TIM_CCMR1_OC2M;
01156 
01157   \textcolor{comment}{/* Configure The Forced output Mode */}
01158   tmpccmr1 |= (uint16\_t)(TIM\_ForcedAction << 8);
01159 
01160   \textcolor{comment}{/* Write to TIMx CCMR1 register */}
01161   TIMx->CCMR1 = tmpccmr1;
01162 \}
01163 
01164 \textcolor{comment}{/**}
01165 \textcolor{comment}{  * @brief  Forces the TIMx output 3 waveform to active or inactive level.}
01166 \textcolor{comment}{  * @param  TIMx: where x can be  1, 2, 3, 4, 5 or 8 to select the TIM peripheral.}
01167 \textcolor{comment}{  * @param  TIM\_ForcedAction: specifies the forced Action to be set to the output waveform.}
01168 \textcolor{comment}{  *          This parameter can be one of the following values:}
01169 \textcolor{comment}{  *            @arg TIM\_ForcedAction\_Active: Force active level on OC3REF}
01170 \textcolor{comment}{  *            @arg TIM\_ForcedAction\_InActive: Force inactive level on OC3REF.}
01171 \textcolor{comment}{  * @retval None}
01172 \textcolor{comment}{  */}
01173 \textcolor{keywordtype}{void} TIM_ForcedOC3Config(TIM\_TypeDef* TIMx, uint16\_t TIM\_ForcedAction)
01174 \{
01175   uint16\_t tmpccmr2 = 0;
01176 
01177   \textcolor{comment}{/* Check the parameters */}
01178   assert_param(IS\_TIM\_LIST3\_PERIPH(TIMx));
01179   assert_param(IS\_TIM\_FORCED\_ACTION(TIM\_ForcedAction));
01180 
01181   tmpccmr2 = TIMx->CCMR2;
01182 
01183   \textcolor{comment}{/* Reset the OC1M Bits */}
01184   tmpccmr2 &= (uint16\_t)~TIM_CCMR2_OC3M;
01185 
01186   \textcolor{comment}{/* Configure The Forced output Mode */}
01187   tmpccmr2 |= TIM\_ForcedAction;
01188 
01189   \textcolor{comment}{/* Write to TIMx CCMR2 register */}
01190   TIMx->CCMR2 = tmpccmr2;
01191 \}
01192 
01193 \textcolor{comment}{/**}
01194 \textcolor{comment}{  * @brief  Forces the TIMx output 4 waveform to active or inactive level.}
01195 \textcolor{comment}{  * @param  TIMx: where x can be  1, 2, 3, 4, 5 or 8 to select the TIM peripheral.}
01196 \textcolor{comment}{  * @param  TIM\_ForcedAction: specifies the forced Action to be set to the output waveform.}
01197 \textcolor{comment}{  *          This parameter can be one of the following values:}
01198 \textcolor{comment}{  *            @arg TIM\_ForcedAction\_Active: Force active level on OC4REF}
01199 \textcolor{comment}{  *            @arg TIM\_ForcedAction\_InActive: Force inactive level on OC4REF.}
01200 \textcolor{comment}{  * @retval None}
01201 \textcolor{comment}{  */}
01202 \textcolor{keywordtype}{void} TIM_ForcedOC4Config(TIM\_TypeDef* TIMx, uint16\_t TIM\_ForcedAction)
01203 \{
01204   uint16\_t tmpccmr2 = 0;
01205 
01206   \textcolor{comment}{/* Check the parameters */}
01207   assert_param(IS\_TIM\_LIST3\_PERIPH(TIMx));
01208   assert_param(IS\_TIM\_FORCED\_ACTION(TIM\_ForcedAction));
01209   tmpccmr2 = TIMx->CCMR2;
01210 
01211   \textcolor{comment}{/* Reset the OC2M Bits */}
01212   tmpccmr2 &= (uint16\_t)~TIM_CCMR2_OC4M;
01213 
01214   \textcolor{comment}{/* Configure The Forced output Mode */}
01215   tmpccmr2 |= (uint16\_t)(TIM\_ForcedAction << 8);
01216 
01217   \textcolor{comment}{/* Write to TIMx CCMR2 register */}
01218   TIMx->CCMR2 = tmpccmr2;
01219 \}
01220 
01221 \textcolor{comment}{/**}
01222 \textcolor{comment}{  * @brief  Enables or disables the TIMx peripheral Preload register on CCR1.}
01223 \textcolor{comment}{  * @param  TIMx: where x can be 1 to 14 except 6 and 7, to select the TIM peripheral.}
01224 \textcolor{comment}{  * @param  TIM\_OCPreload: new state of the TIMx peripheral Preload register}
01225 \textcolor{comment}{  *          This parameter can be one of the following values:}
01226 \textcolor{comment}{  *            @arg TIM\_OCPreload\_Enable}
01227 \textcolor{comment}{  *            @arg TIM\_OCPreload\_Disable}
01228 \textcolor{comment}{  * @retval None}
01229 \textcolor{comment}{  */}
01230 \textcolor{keywordtype}{void} TIM_OC1PreloadConfig(TIM\_TypeDef* TIMx, uint16\_t TIM\_OCPreload)
01231 \{
01232   uint16\_t tmpccmr1 = 0;
01233 
01234   \textcolor{comment}{/* Check the parameters */}
01235   assert_param(IS\_TIM\_LIST1\_PERIPH(TIMx));
01236   assert_param(IS\_TIM\_OCPRELOAD\_STATE(TIM\_OCPreload));
01237 
01238   tmpccmr1 = TIMx->CCMR1;
01239 
01240   \textcolor{comment}{/* Reset the OC1PE Bit */}
01241   tmpccmr1 &= (uint16\_t)(~TIM_CCMR1_OC1PE);
01242 
01243   \textcolor{comment}{/* Enable or Disable the Output Compare Preload feature */}
01244   tmpccmr1 |= TIM\_OCPreload;
01245 
01246   \textcolor{comment}{/* Write to TIMx CCMR1 register */}
01247   TIMx->CCMR1 = tmpccmr1;
01248 \}
01249 
01250 \textcolor{comment}{/**}
01251 \textcolor{comment}{  * @brief  Enables or disables the TIMx peripheral Preload register on CCR2.}
01252 \textcolor{comment}{  * @param  TIMx: where x can be  1, 2, 3, 4, 5, 8, 9 or 12 to select the TIM }
01253 \textcolor{comment}{  *         peripheral.}
01254 \textcolor{comment}{  * @param  TIM\_OCPreload: new state of the TIMx peripheral Preload register}
01255 \textcolor{comment}{  *          This parameter can be one of the following values:}
01256 \textcolor{comment}{  *            @arg TIM\_OCPreload\_Enable}
01257 \textcolor{comment}{  *            @arg TIM\_OCPreload\_Disable}
01258 \textcolor{comment}{  * @retval None}
01259 \textcolor{comment}{  */}
01260 \textcolor{keywordtype}{void} TIM_OC2PreloadConfig(TIM\_TypeDef* TIMx, uint16\_t TIM\_OCPreload)
01261 \{
01262   uint16\_t tmpccmr1 = 0;
01263 
01264   \textcolor{comment}{/* Check the parameters */}
01265   assert_param(IS\_TIM\_LIST2\_PERIPH(TIMx));
01266   assert_param(IS\_TIM\_OCPRELOAD\_STATE(TIM\_OCPreload));
01267 
01268   tmpccmr1 = TIMx->CCMR1;
01269 
01270   \textcolor{comment}{/* Reset the OC2PE Bit */}
01271   tmpccmr1 &= (uint16\_t)(~TIM_CCMR1_OC2PE);
01272 
01273   \textcolor{comment}{/* Enable or Disable the Output Compare Preload feature */}
01274   tmpccmr1 |= (uint16\_t)(TIM\_OCPreload << 8);
01275 
01276   \textcolor{comment}{/* Write to TIMx CCMR1 register */}
01277   TIMx->CCMR1 = tmpccmr1;
01278 \}
01279 
01280 \textcolor{comment}{/**}
01281 \textcolor{comment}{  * @brief  Enables or disables the TIMx peripheral Preload register on CCR3.}
01282 \textcolor{comment}{  * @param  TIMx: where x can be  1, 2, 3, 4, 5 or 8 to select the TIM peripheral.}
01283 \textcolor{comment}{  * @param  TIM\_OCPreload: new state of the TIMx peripheral Preload register}
01284 \textcolor{comment}{  *          This parameter can be one of the following values:}
01285 \textcolor{comment}{  *            @arg TIM\_OCPreload\_Enable}
01286 \textcolor{comment}{  *            @arg TIM\_OCPreload\_Disable}
01287 \textcolor{comment}{  * @retval None}
01288 \textcolor{comment}{  */}
01289 \textcolor{keywordtype}{void} TIM_OC3PreloadConfig(TIM\_TypeDef* TIMx, uint16\_t TIM\_OCPreload)
01290 \{
01291   uint16\_t tmpccmr2 = 0;
01292 
01293   \textcolor{comment}{/* Check the parameters */}
01294   assert_param(IS\_TIM\_LIST3\_PERIPH(TIMx));
01295   assert_param(IS\_TIM\_OCPRELOAD\_STATE(TIM\_OCPreload));
01296 
01297   tmpccmr2 = TIMx->CCMR2;
01298 
01299   \textcolor{comment}{/* Reset the OC3PE Bit */}
01300   tmpccmr2 &= (uint16\_t)(~TIM_CCMR2_OC3PE);
01301 
01302   \textcolor{comment}{/* Enable or Disable the Output Compare Preload feature */}
01303   tmpccmr2 |= TIM\_OCPreload;
01304 
01305   \textcolor{comment}{/* Write to TIMx CCMR2 register */}
01306   TIMx->CCMR2 = tmpccmr2;
01307 \}
01308 
01309 \textcolor{comment}{/**}
01310 \textcolor{comment}{  * @brief  Enables or disables the TIMx peripheral Preload register on CCR4.}
01311 \textcolor{comment}{  * @param  TIMx: where x can be  1, 2, 3, 4, 5 or 8 to select the TIM peripheral.}
01312 \textcolor{comment}{  * @param  TIM\_OCPreload: new state of the TIMx peripheral Preload register}
01313 \textcolor{comment}{  *          This parameter can be one of the following values:}
01314 \textcolor{comment}{  *            @arg TIM\_OCPreload\_Enable}
01315 \textcolor{comment}{  *            @arg TIM\_OCPreload\_Disable}
01316 \textcolor{comment}{  * @retval None}
01317 \textcolor{comment}{  */}
01318 \textcolor{keywordtype}{void} TIM_OC4PreloadConfig(TIM\_TypeDef* TIMx, uint16\_t TIM\_OCPreload)
01319 \{
01320   uint16\_t tmpccmr2 = 0;
01321 
01322   \textcolor{comment}{/* Check the parameters */}
01323   assert_param(IS\_TIM\_LIST3\_PERIPH(TIMx));
01324   assert_param(IS\_TIM\_OCPRELOAD\_STATE(TIM\_OCPreload));
01325 
01326   tmpccmr2 = TIMx->CCMR2;
01327 
01328   \textcolor{comment}{/* Reset the OC4PE Bit */}
01329   tmpccmr2 &= (uint16\_t)(~TIM_CCMR2_OC4PE);
01330 
01331   \textcolor{comment}{/* Enable or Disable the Output Compare Preload feature */}
01332   tmpccmr2 |= (uint16\_t)(TIM\_OCPreload << 8);
01333 
01334   \textcolor{comment}{/* Write to TIMx CCMR2 register */}
01335   TIMx->CCMR2 = tmpccmr2;
01336 \}
01337 
01338 \textcolor{comment}{/**}
01339 \textcolor{comment}{  * @brief  Configures the TIMx Output Compare 1 Fast feature.}
01340 \textcolor{comment}{  * @param  TIMx: where x can be 1 to 14 except 6 and 7, to select the TIM peripheral.}
01341 \textcolor{comment}{  * @param  TIM\_OCFast: new state of the Output Compare Fast Enable Bit.}
01342 \textcolor{comment}{  *          This parameter can be one of the following values:}
01343 \textcolor{comment}{  *            @arg TIM\_OCFast\_Enable: TIM output compare fast enable}
01344 \textcolor{comment}{  *            @arg TIM\_OCFast\_Disable: TIM output compare fast disable}
01345 \textcolor{comment}{  * @retval None}
01346 \textcolor{comment}{  */}
01347 \textcolor{keywordtype}{void} TIM_OC1FastConfig(TIM\_TypeDef* TIMx, uint16\_t TIM\_OCFast)
01348 \{
01349   uint16\_t tmpccmr1 = 0;
01350 
01351   \textcolor{comment}{/* Check the parameters */}
01352   assert_param(IS\_TIM\_LIST1\_PERIPH(TIMx));
01353   assert_param(IS\_TIM\_OCFAST\_STATE(TIM\_OCFast));
01354 
01355   \textcolor{comment}{/* Get the TIMx CCMR1 register value */}
01356   tmpccmr1 = TIMx->CCMR1;
01357 
01358   \textcolor{comment}{/* Reset the OC1FE Bit */}
01359   tmpccmr1 &= (uint16\_t)~TIM_CCMR1_OC1FE;
01360 
01361   \textcolor{comment}{/* Enable or Disable the Output Compare Fast Bit */}
01362   tmpccmr1 |= TIM\_OCFast;
01363 
01364   \textcolor{comment}{/* Write to TIMx CCMR1 */}
01365   TIMx->CCMR1 = tmpccmr1;
01366 \}
01367 
01368 \textcolor{comment}{/**}
01369 \textcolor{comment}{  * @brief  Configures the TIMx Output Compare 2 Fast feature.}
01370 \textcolor{comment}{  * @param  TIMx: where x can be  1, 2, 3, 4, 5, 8, 9 or 12 to select the TIM }
01371 \textcolor{comment}{  *         peripheral.}
01372 \textcolor{comment}{  * @param  TIM\_OCFast: new state of the Output Compare Fast Enable Bit.}
01373 \textcolor{comment}{  *          This parameter can be one of the following values:}
01374 \textcolor{comment}{  *            @arg TIM\_OCFast\_Enable: TIM output compare fast enable}
01375 \textcolor{comment}{  *            @arg TIM\_OCFast\_Disable: TIM output compare fast disable}
01376 \textcolor{comment}{  * @retval None}
01377 \textcolor{comment}{  */}
01378 \textcolor{keywordtype}{void} TIM_OC2FastConfig(TIM\_TypeDef* TIMx, uint16\_t TIM\_OCFast)
01379 \{
01380   uint16\_t tmpccmr1 = 0;
01381 
01382   \textcolor{comment}{/* Check the parameters */}
01383   assert_param(IS\_TIM\_LIST2\_PERIPH(TIMx));
01384   assert_param(IS\_TIM\_OCFAST\_STATE(TIM\_OCFast));
01385 
01386   \textcolor{comment}{/* Get the TIMx CCMR1 register value */}
01387   tmpccmr1 = TIMx->CCMR1;
01388 
01389   \textcolor{comment}{/* Reset the OC2FE Bit */}
01390   tmpccmr1 &= (uint16\_t)(~TIM_CCMR1_OC2FE);
01391 
01392   \textcolor{comment}{/* Enable or Disable the Output Compare Fast Bit */}
01393   tmpccmr1 |= (uint16\_t)(TIM\_OCFast << 8);
01394 
01395   \textcolor{comment}{/* Write to TIMx CCMR1 */}
01396   TIMx->CCMR1 = tmpccmr1;
01397 \}
01398 
01399 \textcolor{comment}{/**}
01400 \textcolor{comment}{  * @brief  Configures the TIMx Output Compare 3 Fast feature.}
01401 \textcolor{comment}{  * @param  TIMx: where x can be  1, 2, 3, 4, 5 or 8 to select the TIM peripheral.}
01402 \textcolor{comment}{  * @param  TIM\_OCFast: new state of the Output Compare Fast Enable Bit.}
01403 \textcolor{comment}{  *          This parameter can be one of the following values:}
01404 \textcolor{comment}{  *            @arg TIM\_OCFast\_Enable: TIM output compare fast enable}
01405 \textcolor{comment}{  *            @arg TIM\_OCFast\_Disable: TIM output compare fast disable}
01406 \textcolor{comment}{  * @retval None}
01407 \textcolor{comment}{  */}
01408 \textcolor{keywordtype}{void} TIM_OC3FastConfig(TIM\_TypeDef* TIMx, uint16\_t TIM\_OCFast)
01409 \{
01410   uint16\_t tmpccmr2 = 0;
01411 
01412   \textcolor{comment}{/* Check the parameters */}
01413   assert_param(IS\_TIM\_LIST3\_PERIPH(TIMx));
01414   assert_param(IS\_TIM\_OCFAST\_STATE(TIM\_OCFast));
01415 
01416   \textcolor{comment}{/* Get the TIMx CCMR2 register value */}
01417   tmpccmr2 = TIMx->CCMR2;
01418 
01419   \textcolor{comment}{/* Reset the OC3FE Bit */}
01420   tmpccmr2 &= (uint16\_t)~TIM_CCMR2_OC3FE;
01421 
01422   \textcolor{comment}{/* Enable or Disable the Output Compare Fast Bit */}
01423   tmpccmr2 |= TIM\_OCFast;
01424 
01425   \textcolor{comment}{/* Write to TIMx CCMR2 */}
01426   TIMx->CCMR2 = tmpccmr2;
01427 \}
01428 
01429 \textcolor{comment}{/**}
01430 \textcolor{comment}{  * @brief  Configures the TIMx Output Compare 4 Fast feature.}
01431 \textcolor{comment}{  * @param  TIMx: where x can be  1, 2, 3, 4, 5 or 8 to select the TIM peripheral.}
01432 \textcolor{comment}{  * @param  TIM\_OCFast: new state of the Output Compare Fast Enable Bit.}
01433 \textcolor{comment}{  *          This parameter can be one of the following values:}
01434 \textcolor{comment}{  *            @arg TIM\_OCFast\_Enable: TIM output compare fast enable}
01435 \textcolor{comment}{  *            @arg TIM\_OCFast\_Disable: TIM output compare fast disable}
01436 \textcolor{comment}{  * @retval None}
01437 \textcolor{comment}{  */}
01438 \textcolor{keywordtype}{void} TIM_OC4FastConfig(TIM\_TypeDef* TIMx, uint16\_t TIM\_OCFast)
01439 \{
01440   uint16\_t tmpccmr2 = 0;
01441 
01442   \textcolor{comment}{/* Check the parameters */}
01443   assert_param(IS\_TIM\_LIST3\_PERIPH(TIMx));
01444   assert_param(IS\_TIM\_OCFAST\_STATE(TIM\_OCFast));
01445 
01446   \textcolor{comment}{/* Get the TIMx CCMR2 register value */}
01447   tmpccmr2 = TIMx->CCMR2;
01448 
01449   \textcolor{comment}{/* Reset the OC4FE Bit */}
01450   tmpccmr2 &= (uint16\_t)(~TIM_CCMR2_OC4FE);
01451 
01452   \textcolor{comment}{/* Enable or Disable the Output Compare Fast Bit */}
01453   tmpccmr2 |= (uint16\_t)(TIM\_OCFast << 8);
01454 
01455   \textcolor{comment}{/* Write to TIMx CCMR2 */}
01456   TIMx->CCMR2 = tmpccmr2;
01457 \}
01458 
01459 \textcolor{comment}{/**}
01460 \textcolor{comment}{  * @brief  Clears or safeguards the OCREF1 signal on an external event}
01461 \textcolor{comment}{  * @param  TIMx: where x can be 1 to 14 except 6 and 7, to select the TIM peripheral.}
01462 \textcolor{comment}{  * @param  TIM\_OCClear: new state of the Output Compare Clear Enable Bit.}
01463 \textcolor{comment}{  *          This parameter can be one of the following values:}
01464 \textcolor{comment}{  *            @arg TIM\_OCClear\_Enable: TIM Output clear enable}
01465 \textcolor{comment}{  *            @arg TIM\_OCClear\_Disable: TIM Output clear disable}
01466 \textcolor{comment}{  * @retval None}
01467 \textcolor{comment}{  */}
01468 \textcolor{keywordtype}{void} TIM_ClearOC1Ref(TIM\_TypeDef* TIMx, uint16\_t TIM\_OCClear)
01469 \{
01470   uint16\_t tmpccmr1 = 0;
01471 
01472   \textcolor{comment}{/* Check the parameters */}
01473   assert_param(IS\_TIM\_LIST1\_PERIPH(TIMx));
01474   assert_param(IS\_TIM\_OCCLEAR\_STATE(TIM\_OCClear));
01475 
01476   tmpccmr1 = TIMx->CCMR1;
01477 
01478   \textcolor{comment}{/* Reset the OC1CE Bit */}
01479   tmpccmr1 &= (uint16\_t)~TIM_CCMR1_OC1CE;
01480 
01481   \textcolor{comment}{/* Enable or Disable the Output Compare Clear Bit */}
01482   tmpccmr1 |= TIM\_OCClear;
01483 
01484   \textcolor{comment}{/* Write to TIMx CCMR1 register */}
01485   TIMx->CCMR1 = tmpccmr1;
01486 \}
01487 
01488 \textcolor{comment}{/**}
01489 \textcolor{comment}{  * @brief  Clears or safeguards the OCREF2 signal on an external event}
01490 \textcolor{comment}{  * @param  TIMx: where x can be  1, 2, 3, 4, 5, 8, 9 or 12 to select the TIM }
01491 \textcolor{comment}{  *         peripheral.}
01492 \textcolor{comment}{  * @param  TIM\_OCClear: new state of the Output Compare Clear Enable Bit.}
01493 \textcolor{comment}{  *          This parameter can be one of the following values:}
01494 \textcolor{comment}{  *            @arg TIM\_OCClear\_Enable: TIM Output clear enable}
01495 \textcolor{comment}{  *            @arg TIM\_OCClear\_Disable: TIM Output clear disable}
01496 \textcolor{comment}{  * @retval None}
01497 \textcolor{comment}{  */}
01498 \textcolor{keywordtype}{void} TIM_ClearOC2Ref(TIM\_TypeDef* TIMx, uint16\_t TIM\_OCClear)
01499 \{
01500   uint16\_t tmpccmr1 = 0;
01501 
01502   \textcolor{comment}{/* Check the parameters */}
01503   assert_param(IS\_TIM\_LIST2\_PERIPH(TIMx));
01504   assert_param(IS\_TIM\_OCCLEAR\_STATE(TIM\_OCClear));
01505 
01506   tmpccmr1 = TIMx->CCMR1;
01507 
01508   \textcolor{comment}{/* Reset the OC2CE Bit */}
01509   tmpccmr1 &= (uint16\_t)~TIM_CCMR1_OC2CE;
01510 
01511   \textcolor{comment}{/* Enable or Disable the Output Compare Clear Bit */}
01512   tmpccmr1 |= (uint16\_t)(TIM\_OCClear << 8);
01513 
01514   \textcolor{comment}{/* Write to TIMx CCMR1 register */}
01515   TIMx->CCMR1 = tmpccmr1;
01516 \}
01517 
01518 \textcolor{comment}{/**}
01519 \textcolor{comment}{  * @brief  Clears or safeguards the OCREF3 signal on an external event}
01520 \textcolor{comment}{  * @param  TIMx: where x can be  1, 2, 3, 4, 5 or 8 to select the TIM peripheral.}
01521 \textcolor{comment}{  * @param  TIM\_OCClear: new state of the Output Compare Clear Enable Bit.}
01522 \textcolor{comment}{  *          This parameter can be one of the following values:}
01523 \textcolor{comment}{  *            @arg TIM\_OCClear\_Enable: TIM Output clear enable}
01524 \textcolor{comment}{  *            @arg TIM\_OCClear\_Disable: TIM Output clear disable}
01525 \textcolor{comment}{  * @retval None}
01526 \textcolor{comment}{  */}
01527 \textcolor{keywordtype}{void} TIM_ClearOC3Ref(TIM\_TypeDef* TIMx, uint16\_t TIM\_OCClear)
01528 \{
01529   uint16\_t tmpccmr2 = 0;
01530 
01531   \textcolor{comment}{/* Check the parameters */}
01532   assert_param(IS\_TIM\_LIST3\_PERIPH(TIMx));
01533   assert_param(IS\_TIM\_OCCLEAR\_STATE(TIM\_OCClear));
01534 
01535   tmpccmr2 = TIMx->CCMR2;
01536 
01537   \textcolor{comment}{/* Reset the OC3CE Bit */}
01538   tmpccmr2 &= (uint16\_t)~TIM_CCMR2_OC3CE;
01539 
01540   \textcolor{comment}{/* Enable or Disable the Output Compare Clear Bit */}
01541   tmpccmr2 |= TIM\_OCClear;
01542 
01543   \textcolor{comment}{/* Write to TIMx CCMR2 register */}
01544   TIMx->CCMR2 = tmpccmr2;
01545 \}
01546 
01547 \textcolor{comment}{/**}
01548 \textcolor{comment}{  * @brief  Clears or safeguards the OCREF4 signal on an external event}
01549 \textcolor{comment}{  * @param  TIMx: where x can be  1, 2, 3, 4, 5 or 8 to select the TIM peripheral.}
01550 \textcolor{comment}{  * @param  TIM\_OCClear: new state of the Output Compare Clear Enable Bit.}
01551 \textcolor{comment}{  *          This parameter can be one of the following values:}
01552 \textcolor{comment}{  *            @arg TIM\_OCClear\_Enable: TIM Output clear enable}
01553 \textcolor{comment}{  *            @arg TIM\_OCClear\_Disable: TIM Output clear disable}
01554 \textcolor{comment}{  * @retval None}
01555 \textcolor{comment}{  */}
01556 \textcolor{keywordtype}{void} TIM_ClearOC4Ref(TIM\_TypeDef* TIMx, uint16\_t TIM\_OCClear)
01557 \{
01558   uint16\_t tmpccmr2 = 0;
01559 
01560   \textcolor{comment}{/* Check the parameters */}
01561   assert_param(IS\_TIM\_LIST3\_PERIPH(TIMx));
01562   assert_param(IS\_TIM\_OCCLEAR\_STATE(TIM\_OCClear));
01563 
01564   tmpccmr2 = TIMx->CCMR2;
01565 
01566   \textcolor{comment}{/* Reset the OC4CE Bit */}
01567   tmpccmr2 &= (uint16\_t)~TIM_CCMR2_OC4CE;
01568 
01569   \textcolor{comment}{/* Enable or Disable the Output Compare Clear Bit */}
01570   tmpccmr2 |= (uint16\_t)(TIM\_OCClear << 8);
01571 
01572   \textcolor{comment}{/* Write to TIMx CCMR2 register */}
01573   TIMx->CCMR2 = tmpccmr2;
01574 \}
01575 
01576 \textcolor{comment}{/**}
01577 \textcolor{comment}{  * @brief  Configures the TIMx channel 1 polarity.}
01578 \textcolor{comment}{  * @param  TIMx: where x can be 1 to 14 except 6 and 7, to select the TIM peripheral.}
01579 \textcolor{comment}{  * @param  TIM\_OCPolarity: specifies the OC1 Polarity}
01580 \textcolor{comment}{  *          This parameter can be one of the following values:}
01581 \textcolor{comment}{  *            @arg TIM\_OCPolarity\_High: Output Compare active high}
01582 \textcolor{comment}{  *            @arg TIM\_OCPolarity\_Low: Output Compare active low}
01583 \textcolor{comment}{  * @retval None}
01584 \textcolor{comment}{  */}
01585 \textcolor{keywordtype}{void} TIM_OC1PolarityConfig(TIM\_TypeDef* TIMx, uint16\_t TIM\_OCPolarity)
01586 \{
01587   uint16\_t tmpccer = 0;
01588 
01589   \textcolor{comment}{/* Check the parameters */}
01590   assert_param(IS\_TIM\_LIST1\_PERIPH(TIMx));
01591   assert_param(IS\_TIM\_OC\_POLARITY(TIM\_OCPolarity));
01592 
01593   tmpccer = TIMx->CCER;
01594 
01595   \textcolor{comment}{/* Set or Reset the CC1P Bit */}
01596   tmpccer &= (uint16\_t)(~TIM_CCER_CC1P);
01597   tmpccer |= TIM\_OCPolarity;
01598 
01599   \textcolor{comment}{/* Write to TIMx CCER register */}
01600   TIMx->CCER = tmpccer;
01601 \}
01602 
01603 \textcolor{comment}{/**}
01604 \textcolor{comment}{  * @brief  Configures the TIMx Channel 1N polarity.}
01605 \textcolor{comment}{  * @param  TIMx: where x can be 1 or 8 to select the TIM peripheral.}
01606 \textcolor{comment}{  * @param  TIM\_OCNPolarity: specifies the OC1N Polarity}
01607 \textcolor{comment}{  *          This parameter can be one of the following values:}
01608 \textcolor{comment}{  *            @arg TIM\_OCNPolarity\_High: Output Compare active high}
01609 \textcolor{comment}{  *            @arg TIM\_OCNPolarity\_Low: Output Compare active low}
01610 \textcolor{comment}{  * @retval None}
01611 \textcolor{comment}{  */}
01612 \textcolor{keywordtype}{void} TIM_OC1NPolarityConfig(TIM\_TypeDef* TIMx, uint16\_t TIM\_OCNPolarity)
01613 \{
01614   uint16\_t tmpccer = 0;
01615   \textcolor{comment}{/* Check the parameters */}
01616   assert_param(IS\_TIM\_LIST4\_PERIPH(TIMx));
01617   assert_param(IS\_TIM\_OCN\_POLARITY(TIM\_OCNPolarity));
01618 
01619   tmpccer = TIMx->CCER;
01620 
01621   \textcolor{comment}{/* Set or Reset the CC1NP Bit */}
01622   tmpccer &= (uint16\_t)~TIM_CCER_CC1NP;
01623   tmpccer |= TIM\_OCNPolarity;
01624 
01625   \textcolor{comment}{/* Write to TIMx CCER register */}
01626   TIMx->CCER = tmpccer;
01627 \}
01628 
01629 \textcolor{comment}{/**}
01630 \textcolor{comment}{  * @brief  Configures the TIMx channel 2 polarity.}
01631 \textcolor{comment}{  * @param  TIMx: where x can be 1, 2, 3, 4, 5, 8, 9 or 12 to select the TIM }
01632 \textcolor{comment}{  *         peripheral.}
01633 \textcolor{comment}{  * @param  TIM\_OCPolarity: specifies the OC2 Polarity}
01634 \textcolor{comment}{  *          This parameter can be one of the following values:}
01635 \textcolor{comment}{  *            @arg TIM\_OCPolarity\_High: Output Compare active high}
01636 \textcolor{comment}{  *            @arg TIM\_OCPolarity\_Low: Output Compare active low}
01637 \textcolor{comment}{  * @retval None}
01638 \textcolor{comment}{  */}
01639 \textcolor{keywordtype}{void} TIM_OC2PolarityConfig(TIM\_TypeDef* TIMx, uint16\_t TIM\_OCPolarity)
01640 \{
01641   uint16\_t tmpccer = 0;
01642 
01643   \textcolor{comment}{/* Check the parameters */}
01644   assert_param(IS\_TIM\_LIST2\_PERIPH(TIMx));
01645   assert_param(IS\_TIM\_OC\_POLARITY(TIM\_OCPolarity));
01646 
01647   tmpccer = TIMx->CCER;
01648 
01649   \textcolor{comment}{/* Set or Reset the CC2P Bit */}
01650   tmpccer &= (uint16\_t)(~TIM_CCER_CC2P);
01651   tmpccer |= (uint16\_t)(TIM\_OCPolarity << 4);
01652 
01653   \textcolor{comment}{/* Write to TIMx CCER register */}
01654   TIMx->CCER = tmpccer;
01655 \}
01656 
01657 \textcolor{comment}{/**}
01658 \textcolor{comment}{  * @brief  Configures the TIMx Channel 2N polarity.}
01659 \textcolor{comment}{  * @param  TIMx: where x can be 1 or 8 to select the TIM peripheral.}
01660 \textcolor{comment}{  * @param  TIM\_OCNPolarity: specifies the OC2N Polarity}
01661 \textcolor{comment}{  *          This parameter can be one of the following values:}
01662 \textcolor{comment}{  *            @arg TIM\_OCNPolarity\_High: Output Compare active high}
01663 \textcolor{comment}{  *            @arg TIM\_OCNPolarity\_Low: Output Compare active low}
01664 \textcolor{comment}{  * @retval None}
01665 \textcolor{comment}{  */}
01666 \textcolor{keywordtype}{void} TIM_OC2NPolarityConfig(TIM\_TypeDef* TIMx, uint16\_t TIM\_OCNPolarity)
01667 \{
01668   uint16\_t tmpccer = 0;
01669 
01670   \textcolor{comment}{/* Check the parameters */}
01671   assert_param(IS\_TIM\_LIST4\_PERIPH(TIMx));
01672   assert_param(IS\_TIM\_OCN\_POLARITY(TIM\_OCNPolarity));
01673 
01674   tmpccer = TIMx->CCER;
01675 
01676   \textcolor{comment}{/* Set or Reset the CC2NP Bit */}
01677   tmpccer &= (uint16\_t)~TIM_CCER_CC2NP;
01678   tmpccer |= (uint16\_t)(TIM\_OCNPolarity << 4);
01679 
01680   \textcolor{comment}{/* Write to TIMx CCER register */}
01681   TIMx->CCER = tmpccer;
01682 \}
01683 
01684 \textcolor{comment}{/**}
01685 \textcolor{comment}{  * @brief  Configures the TIMx channel 3 polarity.}
01686 \textcolor{comment}{  * @param  TIMx: where x can be 1, 2, 3, 4, 5 or 8 to select the TIM peripheral.}
01687 \textcolor{comment}{  * @param  TIM\_OCPolarity: specifies the OC3 Polarity}
01688 \textcolor{comment}{  *          This parameter can be one of the following values:}
01689 \textcolor{comment}{  *            @arg TIM\_OCPolarity\_High: Output Compare active high}
01690 \textcolor{comment}{  *            @arg TIM\_OCPolarity\_Low: Output Compare active low}
01691 \textcolor{comment}{  * @retval None}
01692 \textcolor{comment}{  */}
01693 \textcolor{keywordtype}{void} TIM_OC3PolarityConfig(TIM\_TypeDef* TIMx, uint16\_t TIM\_OCPolarity)
01694 \{
01695   uint16\_t tmpccer = 0;
01696 
01697   \textcolor{comment}{/* Check the parameters */}
01698   assert_param(IS\_TIM\_LIST3\_PERIPH(TIMx));
01699   assert_param(IS\_TIM\_OC\_POLARITY(TIM\_OCPolarity));
01700 
01701   tmpccer = TIMx->CCER;
01702 
01703   \textcolor{comment}{/* Set or Reset the CC3P Bit */}
01704   tmpccer &= (uint16\_t)~TIM_CCER_CC3P;
01705   tmpccer |= (uint16\_t)(TIM\_OCPolarity << 8);
01706 
01707   \textcolor{comment}{/* Write to TIMx CCER register */}
01708   TIMx->CCER = tmpccer;
01709 \}
01710 
01711 \textcolor{comment}{/**}
01712 \textcolor{comment}{  * @brief  Configures the TIMx Channel 3N polarity.}
01713 \textcolor{comment}{  * @param  TIMx: where x can be 1 or 8 to select the TIM peripheral.}
01714 \textcolor{comment}{  * @param  TIM\_OCNPolarity: specifies the OC3N Polarity}
01715 \textcolor{comment}{  *          This parameter can be one of the following values:}
01716 \textcolor{comment}{  *            @arg TIM\_OCNPolarity\_High: Output Compare active high}
01717 \textcolor{comment}{  *            @arg TIM\_OCNPolarity\_Low: Output Compare active low}
01718 \textcolor{comment}{  * @retval None}
01719 \textcolor{comment}{  */}
01720 \textcolor{keywordtype}{void} TIM_OC3NPolarityConfig(TIM\_TypeDef* TIMx, uint16\_t TIM\_OCNPolarity)
01721 \{
01722   uint16\_t tmpccer = 0;
01723 
01724   \textcolor{comment}{/* Check the parameters */}
01725   assert_param(IS\_TIM\_LIST4\_PERIPH(TIMx));
01726   assert_param(IS\_TIM\_OCN\_POLARITY(TIM\_OCNPolarity));
01727 
01728   tmpccer = TIMx->CCER;
01729 
01730   \textcolor{comment}{/* Set or Reset the CC3NP Bit */}
01731   tmpccer &= (uint16\_t)~TIM_CCER_CC3NP;
01732   tmpccer |= (uint16\_t)(TIM\_OCNPolarity << 8);
01733 
01734   \textcolor{comment}{/* Write to TIMx CCER register */}
01735   TIMx->CCER = tmpccer;
01736 \}
01737 
01738 \textcolor{comment}{/**}
01739 \textcolor{comment}{  * @brief  Configures the TIMx channel 4 polarity.}
01740 \textcolor{comment}{  * @param  TIMx: where x can be 1, 2, 3, 4, 5 or 8 to select the TIM peripheral.}
01741 \textcolor{comment}{  * @param  TIM\_OCPolarity: specifies the OC4 Polarity}
01742 \textcolor{comment}{  *          This parameter can be one of the following values:}
01743 \textcolor{comment}{  *            @arg TIM\_OCPolarity\_High: Output Compare active high}
01744 \textcolor{comment}{  *            @arg TIM\_OCPolarity\_Low: Output Compare active low}
01745 \textcolor{comment}{  * @retval None}
01746 \textcolor{comment}{  */}
01747 \textcolor{keywordtype}{void} TIM_OC4PolarityConfig(TIM\_TypeDef* TIMx, uint16\_t TIM\_OCPolarity)
01748 \{
01749   uint16\_t tmpccer = 0;
01750 
01751   \textcolor{comment}{/* Check the parameters */}
01752   assert_param(IS\_TIM\_LIST3\_PERIPH(TIMx));
01753   assert_param(IS\_TIM\_OC\_POLARITY(TIM\_OCPolarity));
01754 
01755   tmpccer = TIMx->CCER;
01756 
01757   \textcolor{comment}{/* Set or Reset the CC4P Bit */}
01758   tmpccer &= (uint16\_t)~TIM_CCER_CC4P;
01759   tmpccer |= (uint16\_t)(TIM\_OCPolarity << 12);
01760 
01761   \textcolor{comment}{/* Write to TIMx CCER register */}
01762   TIMx->CCER = tmpccer;
01763 \}
01764 
01765 \textcolor{comment}{/**}
01766 \textcolor{comment}{  * @brief  Enables or disables the TIM Capture Compare Channel x.}
01767 \textcolor{comment}{  * @param  TIMx: where x can be 1 to 14 except 6 and 7, to select the TIM peripheral.}
01768 \textcolor{comment}{  * @param  TIM\_Channel: specifies the TIM Channel}
01769 \textcolor{comment}{  *          This parameter can be one of the following values:}
01770 \textcolor{comment}{  *            @arg TIM\_Channel\_1: TIM Channel 1}
01771 \textcolor{comment}{  *            @arg TIM\_Channel\_2: TIM Channel 2}
01772 \textcolor{comment}{  *            @arg TIM\_Channel\_3: TIM Channel 3}
01773 \textcolor{comment}{  *            @arg TIM\_Channel\_4: TIM Channel 4}
01774 \textcolor{comment}{  * @param  TIM\_CCx: specifies the TIM Channel CCxE bit new state.}
01775 \textcolor{comment}{  *          This parameter can be: TIM\_CCx\_Enable or TIM\_CCx\_Disable. }
01776 \textcolor{comment}{  * @retval None}
01777 \textcolor{comment}{  */}
01778 \textcolor{keywordtype}{void} TIM_CCxCmd(TIM\_TypeDef* TIMx, uint16\_t TIM\_Channel, uint16\_t TIM\_CCx)
01779 \{
01780   uint16\_t tmp = 0;
01781 
01782   \textcolor{comment}{/* Check the parameters */}
01783   assert_param(IS\_TIM\_LIST1\_PERIPH(TIMx));
01784   assert_param(IS\_TIM\_CHANNEL(TIM\_Channel));
01785   assert_param(IS\_TIM\_CCX(TIM\_CCx));
01786 
01787   tmp = CCER_CCE_SET << TIM\_Channel;
01788 
01789   \textcolor{comment}{/* Reset the CCxE Bit */}
01790   TIMx->CCER &= (uint16\_t)~ tmp;
01791 
01792   \textcolor{comment}{/* Set or reset the CCxE Bit */}
01793   TIMx->CCER |=  (uint16\_t)(TIM\_CCx << TIM\_Channel);
01794 \}
01795 
01796 \textcolor{comment}{/**}
01797 \textcolor{comment}{  * @brief  Enables or disables the TIM Capture Compare Channel xN.}
01798 \textcolor{comment}{  * @param  TIMx: where x can be 1 or 8 to select the TIM peripheral.}
01799 \textcolor{comment}{  * @param  TIM\_Channel: specifies the TIM Channel}
01800 \textcolor{comment}{  *          This parameter can be one of the following values:}
01801 \textcolor{comment}{  *            @arg TIM\_Channel\_1: TIM Channel 1}
01802 \textcolor{comment}{  *            @arg TIM\_Channel\_2: TIM Channel 2}
01803 \textcolor{comment}{  *            @arg TIM\_Channel\_3: TIM Channel 3}
01804 \textcolor{comment}{  * @param  TIM\_CCxN: specifies the TIM Channel CCxNE bit new state.}
01805 \textcolor{comment}{  *          This parameter can be: TIM\_CCxN\_Enable or TIM\_CCxN\_Disable. }
01806 \textcolor{comment}{  * @retval None}
01807 \textcolor{comment}{  */}
01808 \textcolor{keywordtype}{void} TIM_CCxNCmd(TIM\_TypeDef* TIMx, uint16\_t TIM\_Channel, uint16\_t TIM\_CCxN)
01809 \{
01810   uint16\_t tmp = 0;
01811 
01812   \textcolor{comment}{/* Check the parameters */}
01813   assert_param(IS\_TIM\_LIST4\_PERIPH(TIMx));
01814   assert_param(IS\_TIM\_COMPLEMENTARY\_CHANNEL(TIM\_Channel));
01815   assert_param(IS\_TIM\_CCXN(TIM\_CCxN));
01816 
01817   tmp = CCER_CCNE_SET << TIM\_Channel;
01818 
01819   \textcolor{comment}{/* Reset the CCxNE Bit */}
01820   TIMx->CCER &= (uint16\_t) ~tmp;
01821 
01822   \textcolor{comment}{/* Set or reset the CCxNE Bit */}
01823   TIMx->CCER |=  (uint16\_t)(TIM\_CCxN << TIM\_Channel);
01824 \}
01825 \textcolor{comment}{/**}
01826 \textcolor{comment}{  * @\}}
01827 \textcolor{comment}{  */}
01828 
01829 \textcolor{comment}{/** @defgroup TIM\_Group3 Input Capture management functions}
01830 \textcolor{comment}{ *  @brief    Input Capture management functions }
01831 \textcolor{comment}{ *}
01832 \textcolor{comment}{@verbatim   }
01833 \textcolor{comment}{ ===============================================================================}
01834 \textcolor{comment}{                      Input Capture management functions}
01835 \textcolor{comment}{ ===============================================================================  }
01836 \textcolor{comment}{   }
01837 \textcolor{comment}{       ===================================================================      }
01838 \textcolor{comment}{              TIM Driver: how to use it in Input Capture Mode}
01839 \textcolor{comment}{       =================================================================== }
01840 \textcolor{comment}{       To use the Timer in Input Capture mode, the following steps are mandatory:}
01841 \textcolor{comment}{       }
01842 \textcolor{comment}{       1. Enable TIM clock using RCC\_APBxPeriphClockCmd(RCC\_APBxPeriph\_TIMx, ENABLE) function}
01843 \textcolor{comment}{       }
01844 \textcolor{comment}{       2. Configure the TIM pins by configuring the corresponding GPIO pins}
01845 \textcolor{comment}{       }
01846 \textcolor{comment}{       2. Configure the Time base unit as described in the first part of this driver,}
01847 \textcolor{comment}{          if needed, else the Timer will run with the default configuration:}
01848 \textcolor{comment}{          - Autoreload value = 0xFFFF}
01849 \textcolor{comment}{          - Prescaler value = 0x0000}
01850 \textcolor{comment}{          - Counter mode = Up counting}
01851 \textcolor{comment}{          - Clock Division = TIM\_CKD\_DIV1}
01852 \textcolor{comment}{          }
01853 \textcolor{comment}{       3. Fill the TIM\_ICInitStruct with the desired parameters including:}
01854 \textcolor{comment}{          - TIM Channel: TIM\_Channel}
01855 \textcolor{comment}{          - TIM Input Capture polarity: TIM\_ICPolarity}
01856 \textcolor{comment}{          - TIM Input Capture selection: TIM\_ICSelection}
01857 \textcolor{comment}{          - TIM Input Capture Prescaler: TIM\_ICPrescaler}
01858 \textcolor{comment}{          - TIM Input CApture filter value: TIM\_ICFilter}
01859 \textcolor{comment}{       }
01860 \textcolor{comment}{       4. Call TIM\_ICInit(TIMx, &TIM\_ICInitStruct) to configure the desired channel with the }
01861 \textcolor{comment}{          corresponding configuration and to measure only frequency or duty cycle of the input signal,}
01862 \textcolor{comment}{          or,}
01863 \textcolor{comment}{          Call TIM\_PWMIConfig(TIMx, &TIM\_ICInitStruct) to configure the desired channels with the }
01864 \textcolor{comment}{          corresponding configuration and to measure the frequency and the duty cycle of the input
       signal}
01865 \textcolor{comment}{          }
01866 \textcolor{comment}{       5. Enable the NVIC or the DMA to read the measured frequency. }
01867 \textcolor{comment}{          }
01868 \textcolor{comment}{       6. Enable the corresponding interrupt (or DMA request) to read the Captured value,}
01869 \textcolor{comment}{          using the function TIM\_ITConfig(TIMx, TIM\_IT\_CCx) (or TIM\_DMA\_Cmd(TIMx, TIM\_DMA\_CCx)) }
01870 \textcolor{comment}{       }
01871 \textcolor{comment}{       7. Call the TIM\_Cmd(ENABLE) function to enable the TIM counter.}
01872 \textcolor{comment}{       }
01873 \textcolor{comment}{       8. Use TIM\_GetCapturex(TIMx); to read the captured value.}
01874 \textcolor{comment}{       }
01875 \textcolor{comment}{       Note1: All other functions can be used separately to modify, if needed,}
01876 \textcolor{comment}{              a specific feature of the Timer. }
01877 \textcolor{comment}{}
01878 \textcolor{comment}{@endverbatim}
01879 \textcolor{comment}{  * @\{}
01880 \textcolor{comment}{  */}
01881 
01882 \textcolor{comment}{/**}
01883 \textcolor{comment}{  * @brief  Initializes the TIM peripheral according to the specified parameters}
01884 \textcolor{comment}{  *         in the TIM\_ICInitStruct.}
01885 \textcolor{comment}{  * @param  TIMx: where x can be 1 to 14 except 6 and 7, to select the TIM peripheral.}
01886 \textcolor{comment}{  * @param  TIM\_ICInitStruct: pointer to a TIM\_ICInitTypeDef structure that contains}
01887 \textcolor{comment}{  *         the configuration information for the specified TIM peripheral.}
01888 \textcolor{comment}{  * @retval None}
01889 \textcolor{comment}{  */}
01890 \textcolor{keywordtype}{void} TIM_ICInit(TIM\_TypeDef* TIMx, TIM\_ICInitTypeDef* TIM\_ICInitStruct)
01891 \{
01892   \textcolor{comment}{/* Check the parameters */}
01893   assert_param(IS\_TIM\_LIST1\_PERIPH(TIMx));
01894   assert_param(IS\_TIM\_IC\_POLARITY(TIM\_ICInitStruct->TIM\_ICPolarity));
01895   assert_param(IS\_TIM\_IC\_SELECTION(TIM\_ICInitStruct->TIM\_ICSelection));
01896   assert_param(IS\_TIM\_IC\_PRESCALER(TIM\_ICInitStruct->TIM\_ICPrescaler));
01897   assert_param(IS\_TIM\_IC\_FILTER(TIM\_ICInitStruct->TIM\_ICFilter));
01898 
01899   \textcolor{keywordflow}{if} (TIM\_ICInitStruct->TIM_Channel == TIM_Channel_1)
01900   \{
01901     \textcolor{comment}{/* TI1 Configuration */}
01902     TI1_Config(TIMx, TIM\_ICInitStruct->TIM_ICPolarity,
01903                TIM\_ICInitStruct->TIM_ICSelection,
01904                TIM\_ICInitStruct->TIM_ICFilter);
01905     \textcolor{comment}{/* Set the Input Capture Prescaler value */}
01906     TIM_SetIC1Prescaler(TIMx, TIM\_ICInitStruct->TIM_ICPrescaler);
01907   \}
01908   \textcolor{keywordflow}{else} \textcolor{keywordflow}{if} (TIM\_ICInitStruct->TIM_Channel == TIM_Channel_2)
01909   \{
01910     \textcolor{comment}{/* TI2 Configuration */}
01911     assert_param(IS\_TIM\_LIST2\_PERIPH(TIMx));
01912     TI2_Config(TIMx, TIM\_ICInitStruct->TIM_ICPolarity,
01913                TIM\_ICInitStruct->TIM_ICSelection,
01914                TIM\_ICInitStruct->TIM_ICFilter);
01915     \textcolor{comment}{/* Set the Input Capture Prescaler value */}
01916     TIM_SetIC2Prescaler(TIMx, TIM\_ICInitStruct->TIM_ICPrescaler);
01917   \}
01918   \textcolor{keywordflow}{else} \textcolor{keywordflow}{if} (TIM\_ICInitStruct->TIM_Channel == TIM_Channel_3)
01919   \{
01920     \textcolor{comment}{/* TI3 Configuration */}
01921     assert_param(IS\_TIM\_LIST3\_PERIPH(TIMx));
01922     TI3_Config(TIMx,  TIM\_ICInitStruct->TIM_ICPolarity,
01923                TIM\_ICInitStruct->TIM_ICSelection,
01924                TIM\_ICInitStruct->TIM_ICFilter);
01925     \textcolor{comment}{/* Set the Input Capture Prescaler value */}
01926     TIM_SetIC3Prescaler(TIMx, TIM\_ICInitStruct->TIM_ICPrescaler);
01927   \}
01928   \textcolor{keywordflow}{else}
01929   \{
01930     \textcolor{comment}{/* TI4 Configuration */}
01931     assert_param(IS\_TIM\_LIST3\_PERIPH(TIMx));
01932     TI4_Config(TIMx, TIM\_ICInitStruct->TIM_ICPolarity,
01933                TIM\_ICInitStruct->TIM_ICSelection,
01934                TIM\_ICInitStruct->TIM_ICFilter);
01935     \textcolor{comment}{/* Set the Input Capture Prescaler value */}
01936     TIM_SetIC4Prescaler(TIMx, TIM\_ICInitStruct->TIM_ICPrescaler);
01937   \}
01938 \}
01939 
01940 \textcolor{comment}{/**}
01941 \textcolor{comment}{  * @brief  Fills each TIM\_ICInitStruct member with its default value.}
01942 \textcolor{comment}{  * @param  TIM\_ICInitStruct: pointer to a TIM\_ICInitTypeDef structure which will}
01943 \textcolor{comment}{  *         be initialized.}
01944 \textcolor{comment}{  * @retval None}
01945 \textcolor{comment}{  */}
01946 \textcolor{keywordtype}{void} TIM_ICStructInit(TIM\_ICInitTypeDef* TIM\_ICInitStruct)
01947 \{
01948   \textcolor{comment}{/* Set the default configuration */}
01949   TIM\_ICInitStruct->TIM_Channel = TIM_Channel_1;
01950   TIM\_ICInitStruct->TIM_ICPolarity = TIM_ICPolarity_Rising;
01951   TIM\_ICInitStruct->TIM_ICSelection = TIM_ICSelection_DirectTI;
01952   TIM\_ICInitStruct->TIM_ICPrescaler = TIM_ICPSC_DIV1;
01953   TIM\_ICInitStruct->TIM_ICFilter = 0x00;
01954 \}
01955 
01956 \textcolor{comment}{/**}
01957 \textcolor{comment}{  * @brief  Configures the TIM peripheral according to the specified parameters}
01958 \textcolor{comment}{  *         in the TIM\_ICInitStruct to measure an external PWM signal.}
01959 \textcolor{comment}{  * @param  TIMx: where x can be  1, 2, 3, 4, 5,8, 9 or 12 to select the TIM }
01960 \textcolor{comment}{  *         peripheral.}
01961 \textcolor{comment}{  * @param  TIM\_ICInitStruct: pointer to a TIM\_ICInitTypeDef structure that contains}
01962 \textcolor{comment}{  *         the configuration information for the specified TIM peripheral.}
01963 \textcolor{comment}{  * @retval None}
01964 \textcolor{comment}{  */}
01965 \textcolor{keywordtype}{void} TIM_PWMIConfig(TIM\_TypeDef* TIMx, TIM\_ICInitTypeDef* TIM\_ICInitStruct)
01966 \{
01967   uint16\_t icoppositepolarity = TIM_ICPolarity_Rising;
01968   uint16\_t icoppositeselection = TIM_ICSelection_DirectTI;
01969 
01970   \textcolor{comment}{/* Check the parameters */}
01971   assert_param(IS\_TIM\_LIST2\_PERIPH(TIMx));
01972 
01973   \textcolor{comment}{/* Select the Opposite Input Polarity */}
01974   \textcolor{keywordflow}{if} (TIM\_ICInitStruct->TIM_ICPolarity == TIM_ICPolarity_Rising)
01975   \{
01976     icoppositepolarity = TIM_ICPolarity_Falling;
01977   \}
01978   \textcolor{keywordflow}{else}
01979   \{
01980     icoppositepolarity = TIM_ICPolarity_Rising;
01981   \}
01982   \textcolor{comment}{/* Select the Opposite Input */}
01983   \textcolor{keywordflow}{if} (TIM\_ICInitStruct->TIM_ICSelection == TIM_ICSelection_DirectTI)
01984   \{
01985     icoppositeselection = TIM_ICSelection_IndirectTI;
01986   \}
01987   \textcolor{keywordflow}{else}
01988   \{
01989     icoppositeselection = TIM_ICSelection_DirectTI;
01990   \}
01991   \textcolor{keywordflow}{if} (TIM\_ICInitStruct->TIM_Channel == TIM_Channel_1)
01992   \{
01993     \textcolor{comment}{/* TI1 Configuration */}
01994     TI1_Config(TIMx, TIM\_ICInitStruct->TIM_ICPolarity, TIM\_ICInitStruct->
      TIM_ICSelection,
01995                TIM\_ICInitStruct->TIM_ICFilter);
01996     \textcolor{comment}{/* Set the Input Capture Prescaler value */}
01997     TIM_SetIC1Prescaler(TIMx, TIM\_ICInitStruct->TIM_ICPrescaler);
01998     \textcolor{comment}{/* TI2 Configuration */}
01999     TI2_Config(TIMx, icoppositepolarity, icoppositeselection, TIM\_ICInitStruct
      ->TIM_ICFilter);
02000     \textcolor{comment}{/* Set the Input Capture Prescaler value */}
02001     TIM_SetIC2Prescaler(TIMx, TIM\_ICInitStruct->TIM_ICPrescaler);
02002   \}
02003   \textcolor{keywordflow}{else}
02004   \{
02005     \textcolor{comment}{/* TI2 Configuration */}
02006     TI2_Config(TIMx, TIM\_ICInitStruct->TIM_ICPolarity, TIM\_ICInitStruct->
      TIM_ICSelection,
02007                TIM\_ICInitStruct->TIM_ICFilter);
02008     \textcolor{comment}{/* Set the Input Capture Prescaler value */}
02009     TIM_SetIC2Prescaler(TIMx, TIM\_ICInitStruct->TIM_ICPrescaler);
02010     \textcolor{comment}{/* TI1 Configuration */}
02011     TI1_Config(TIMx, icoppositepolarity, icoppositeselection, TIM\_ICInitStruct
      ->TIM_ICFilter);
02012     \textcolor{comment}{/* Set the Input Capture Prescaler value */}
02013     TIM_SetIC1Prescaler(TIMx, TIM\_ICInitStruct->TIM_ICPrescaler);
02014   \}
02015 \}
02016 
02017 \textcolor{comment}{/**}
02018 \textcolor{comment}{  * @brief  Gets the TIMx Input Capture 1 value.}
02019 \textcolor{comment}{  * @param  TIMx: where x can be 1 to 14 except 6 and 7, to select the TIM peripheral.}
02020 \textcolor{comment}{  * @retval Capture Compare 1 Register value.}
02021 \textcolor{comment}{  */}
02022 uint32\_t TIM_GetCapture1(TIM\_TypeDef* TIMx)
02023 \{
02024   \textcolor{comment}{/* Check the parameters */}
02025   assert_param(IS\_TIM\_LIST1\_PERIPH(TIMx));
02026 
02027   \textcolor{comment}{/* Get the Capture 1 Register value */}
02028   \textcolor{keywordflow}{return} TIMx->CCR1;
02029 \}
02030 
02031 \textcolor{comment}{/**}
02032 \textcolor{comment}{  * @brief  Gets the TIMx Input Capture 2 value.}
02033 \textcolor{comment}{  * @param  TIMx: where x can be 1, 2, 3, 4, 5, 8, 9 or 12 to select the TIM }
02034 \textcolor{comment}{  *         peripheral.}
02035 \textcolor{comment}{  * @retval Capture Compare 2 Register value.}
02036 \textcolor{comment}{  */}
02037 uint32\_t TIM_GetCapture2(TIM\_TypeDef* TIMx)
02038 \{
02039   \textcolor{comment}{/* Check the parameters */}
02040   assert_param(IS\_TIM\_LIST2\_PERIPH(TIMx));
02041 
02042   \textcolor{comment}{/* Get the Capture 2 Register value */}
02043   \textcolor{keywordflow}{return} TIMx->CCR2;
02044 \}
02045 
02046 \textcolor{comment}{/**}
02047 \textcolor{comment}{  * @brief  Gets the TIMx Input Capture 3 value.}
02048 \textcolor{comment}{  * @param  TIMx: where x can be 1, 2, 3, 4, 5 or 8 to select the TIM peripheral.}
02049 \textcolor{comment}{  * @retval Capture Compare 3 Register value.}
02050 \textcolor{comment}{  */}
02051 uint32\_t TIM_GetCapture3(TIM\_TypeDef* TIMx)
02052 \{
02053   \textcolor{comment}{/* Check the parameters */}
02054   assert_param(IS\_TIM\_LIST3\_PERIPH(TIMx));
02055 
02056   \textcolor{comment}{/* Get the Capture 3 Register value */}
02057   \textcolor{keywordflow}{return} TIMx->CCR3;
02058 \}
02059 
02060 \textcolor{comment}{/**}
02061 \textcolor{comment}{  * @brief  Gets the TIMx Input Capture 4 value.}
02062 \textcolor{comment}{  * @param  TIMx: where x can be 1, 2, 3, 4, 5 or 8 to select the TIM peripheral.}
02063 \textcolor{comment}{  * @retval Capture Compare 4 Register value.}
02064 \textcolor{comment}{  */}
02065 uint32\_t TIM_GetCapture4(TIM\_TypeDef* TIMx)
02066 \{
02067   \textcolor{comment}{/* Check the parameters */}
02068   assert_param(IS\_TIM\_LIST3\_PERIPH(TIMx));
02069 
02070   \textcolor{comment}{/* Get the Capture 4 Register value */}
02071   \textcolor{keywordflow}{return} TIMx->CCR4;
02072 \}
02073 
02074 \textcolor{comment}{/**}
02075 \textcolor{comment}{  * @brief  Sets the TIMx Input Capture 1 prescaler.}
02076 \textcolor{comment}{  * @param  TIMx: where x can be 1 to 14 except 6 and 7, to select the TIM peripheral.}
02077 \textcolor{comment}{  * @param  TIM\_ICPSC: specifies the Input Capture1 prescaler new value.}
02078 \textcolor{comment}{  *          This parameter can be one of the following values:}
02079 \textcolor{comment}{  *            @arg TIM\_ICPSC\_DIV1: no prescaler}
02080 \textcolor{comment}{  *            @arg TIM\_ICPSC\_DIV2: capture is done once every 2 events}
02081 \textcolor{comment}{  *            @arg TIM\_ICPSC\_DIV4: capture is done once every 4 events}
02082 \textcolor{comment}{  *            @arg TIM\_ICPSC\_DIV8: capture is done once every 8 events}
02083 \textcolor{comment}{  * @retval None}
02084 \textcolor{comment}{  */}
02085 \textcolor{keywordtype}{void} TIM_SetIC1Prescaler(TIM\_TypeDef* TIMx, uint16\_t TIM\_ICPSC)
02086 \{
02087   \textcolor{comment}{/* Check the parameters */}
02088   assert_param(IS\_TIM\_LIST1\_PERIPH(TIMx));
02089   assert_param(IS\_TIM\_IC\_PRESCALER(TIM\_ICPSC));
02090 
02091   \textcolor{comment}{/* Reset the IC1PSC Bits */}
02092   TIMx->CCMR1 &= (uint16\_t)~TIM_CCMR1_IC1PSC;
02093 
02094   \textcolor{comment}{/* Set the IC1PSC value */}
02095   TIMx->CCMR1 |= TIM\_ICPSC;
02096 \}
02097 
02098 \textcolor{comment}{/**}
02099 \textcolor{comment}{  * @brief  Sets the TIMx Input Capture 2 prescaler.}
02100 \textcolor{comment}{  * @param  TIMx: where x can be 1, 2, 3, 4, 5, 8, 9 or 12 to select the TIM }
02101 \textcolor{comment}{  *         peripheral.}
02102 \textcolor{comment}{  * @param  TIM\_ICPSC: specifies the Input Capture2 prescaler new value.}
02103 \textcolor{comment}{  *          This parameter can be one of the following values:}
02104 \textcolor{comment}{  *            @arg TIM\_ICPSC\_DIV1: no prescaler}
02105 \textcolor{comment}{  *            @arg TIM\_ICPSC\_DIV2: capture is done once every 2 events}
02106 \textcolor{comment}{  *            @arg TIM\_ICPSC\_DIV4: capture is done once every 4 events}
02107 \textcolor{comment}{  *            @arg TIM\_ICPSC\_DIV8: capture is done once every 8 events}
02108 \textcolor{comment}{  * @retval None}
02109 \textcolor{comment}{  */}
02110 \textcolor{keywordtype}{void} TIM_SetIC2Prescaler(TIM\_TypeDef* TIMx, uint16\_t TIM\_ICPSC)
02111 \{
02112   \textcolor{comment}{/* Check the parameters */}
02113   assert_param(IS\_TIM\_LIST2\_PERIPH(TIMx));
02114   assert_param(IS\_TIM\_IC\_PRESCALER(TIM\_ICPSC));
02115 
02116   \textcolor{comment}{/* Reset the IC2PSC Bits */}
02117   TIMx->CCMR1 &= (uint16\_t)~TIM_CCMR1_IC2PSC;
02118 
02119   \textcolor{comment}{/* Set the IC2PSC value */}
02120   TIMx->CCMR1 |= (uint16\_t)(TIM\_ICPSC << 8);
02121 \}
02122 
02123 \textcolor{comment}{/**}
02124 \textcolor{comment}{  * @brief  Sets the TIMx Input Capture 3 prescaler.}
02125 \textcolor{comment}{  * @param  TIMx: where x can be 1, 2, 3, 4, 5 or 8 to select the TIM peripheral.}
02126 \textcolor{comment}{  * @param  TIM\_ICPSC: specifies the Input Capture3 prescaler new value.}
02127 \textcolor{comment}{  *          This parameter can be one of the following values:}
02128 \textcolor{comment}{  *            @arg TIM\_ICPSC\_DIV1: no prescaler}
02129 \textcolor{comment}{  *            @arg TIM\_ICPSC\_DIV2: capture is done once every 2 events}
02130 \textcolor{comment}{  *            @arg TIM\_ICPSC\_DIV4: capture is done once every 4 events}
02131 \textcolor{comment}{  *            @arg TIM\_ICPSC\_DIV8: capture is done once every 8 events}
02132 \textcolor{comment}{  * @retval None}
02133 \textcolor{comment}{  */}
02134 \textcolor{keywordtype}{void} TIM_SetIC3Prescaler(TIM\_TypeDef* TIMx, uint16\_t TIM\_ICPSC)
02135 \{
02136   \textcolor{comment}{/* Check the parameters */}
02137   assert_param(IS\_TIM\_LIST3\_PERIPH(TIMx));
02138   assert_param(IS\_TIM\_IC\_PRESCALER(TIM\_ICPSC));
02139 
02140   \textcolor{comment}{/* Reset the IC3PSC Bits */}
02141   TIMx->CCMR2 &= (uint16\_t)~TIM_CCMR2_IC3PSC;
02142 
02143   \textcolor{comment}{/* Set the IC3PSC value */}
02144   TIMx->CCMR2 |= TIM\_ICPSC;
02145 \}
02146 
02147 \textcolor{comment}{/**}
02148 \textcolor{comment}{  * @brief  Sets the TIMx Input Capture 4 prescaler.}
02149 \textcolor{comment}{  * @param  TIMx: where x can be 1, 2, 3, 4, 5 or 8 to select the TIM peripheral.}
02150 \textcolor{comment}{  * @param  TIM\_ICPSC: specifies the Input Capture4 prescaler new value.}
02151 \textcolor{comment}{  *          This parameter can be one of the following values:}
02152 \textcolor{comment}{  *            @arg TIM\_ICPSC\_DIV1: no prescaler}
02153 \textcolor{comment}{  *            @arg TIM\_ICPSC\_DIV2: capture is done once every 2 events}
02154 \textcolor{comment}{  *            @arg TIM\_ICPSC\_DIV4: capture is done once every 4 events}
02155 \textcolor{comment}{  *            @arg TIM\_ICPSC\_DIV8: capture is done once every 8 events}
02156 \textcolor{comment}{  * @retval None}
02157 \textcolor{comment}{  */}
02158 \textcolor{keywordtype}{void} TIM_SetIC4Prescaler(TIM\_TypeDef* TIMx, uint16\_t TIM\_ICPSC)
02159 \{
02160   \textcolor{comment}{/* Check the parameters */}
02161   assert_param(IS\_TIM\_LIST3\_PERIPH(TIMx));
02162   assert_param(IS\_TIM\_IC\_PRESCALER(TIM\_ICPSC));
02163 
02164   \textcolor{comment}{/* Reset the IC4PSC Bits */}
02165   TIMx->CCMR2 &= (uint16\_t)~TIM_CCMR2_IC4PSC;
02166 
02167   \textcolor{comment}{/* Set the IC4PSC value */}
02168   TIMx->CCMR2 |= (uint16\_t)(TIM\_ICPSC << 8);
02169 \}
02170 \textcolor{comment}{/**}
02171 \textcolor{comment}{  * @\}}
02172 \textcolor{comment}{  */}
02173 
02174 \textcolor{comment}{/** @defgroup TIM\_Group4 Advanced-control timers (TIM1 and TIM8) specific features}
02175 \textcolor{comment}{ *  @brief   Advanced-control timers (TIM1 and TIM8) specific features}
02176 \textcolor{comment}{ *}
02177 \textcolor{comment}{@verbatim   }
02178 \textcolor{comment}{ ===============================================================================}
02179 \textcolor{comment}{          Advanced-control timers (TIM1 and TIM8) specific features}
02180 \textcolor{comment}{ ===============================================================================  }
02181 \textcolor{comment}{  }
02182 \textcolor{comment}{       ===================================================================      }
02183 \textcolor{comment}{              TIM Driver: how to use the Break feature}
02184 \textcolor{comment}{       =================================================================== }
02185 \textcolor{comment}{       After configuring the Timer channel(s) in the appropriate Output Compare mode: }
02186 \textcolor{comment}{                         }
02187 \textcolor{comment}{       1. Fill the TIM\_BDTRInitStruct with the desired parameters for the Timer}
02188 \textcolor{comment}{          Break Polarity, dead time, Lock level, the OSSI/OSSR State and the }
02189 \textcolor{comment}{          AOE(automatic output enable).}
02190 \textcolor{comment}{               }
02191 \textcolor{comment}{       2. Call TIM\_BDTRConfig(TIMx, &TIM\_BDTRInitStruct) to configure the Timer}
02192 \textcolor{comment}{          }
02193 \textcolor{comment}{       3. Enable the Main Output using TIM\_CtrlPWMOutputs(TIM1, ENABLE) }
02194 \textcolor{comment}{          }
02195 \textcolor{comment}{       4. Once the break even occurs, the Timer's output signals are put in reset}
02196 \textcolor{comment}{          state or in a known state (according to the configuration made in}
02197 \textcolor{comment}{          TIM\_BDTRConfig() function).}
02198 \textcolor{comment}{}
02199 \textcolor{comment}{@endverbatim}
02200 \textcolor{comment}{  * @\{}
02201 \textcolor{comment}{  */}
02202 
02203 \textcolor{comment}{/**}
02204 \textcolor{comment}{  * @brief  Configures the Break feature, dead time, Lock level, OSSI/OSSR State}
02205 \textcolor{comment}{  *         and the AOE(automatic output enable).}
02206 \textcolor{comment}{  * @param  TIMx: where x can be  1 or 8 to select the TIM }
02207 \textcolor{comment}{  * @param  TIM\_BDTRInitStruct: pointer to a TIM\_BDTRInitTypeDef structure that}
02208 \textcolor{comment}{  *         contains the BDTR Register configuration  information for the TIM peripheral.}
02209 \textcolor{comment}{  * @retval None}
02210 \textcolor{comment}{  */}
02211 \textcolor{keywordtype}{void} TIM_BDTRConfig(TIM\_TypeDef* TIMx, TIM\_BDTRInitTypeDef *TIM\_BDTRInitStruct)
02212 \{
02213   \textcolor{comment}{/* Check the parameters */}
02214   assert_param(IS\_TIM\_LIST4\_PERIPH(TIMx));
02215   assert_param(IS\_TIM\_OSSR\_STATE(TIM\_BDTRInitStruct->TIM\_OSSRState));
02216   assert_param(IS\_TIM\_OSSI\_STATE(TIM\_BDTRInitStruct->TIM\_OSSIState));
02217   assert_param(IS\_TIM\_LOCK\_LEVEL(TIM\_BDTRInitStruct->TIM\_LOCKLevel));
02218   assert_param(IS\_TIM\_BREAK\_STATE(TIM\_BDTRInitStruct->TIM\_Break));
02219   assert_param(IS\_TIM\_BREAK\_POLARITY(TIM\_BDTRInitStruct->TIM\_BreakPolarity));
02220   assert_param(IS\_TIM\_AUTOMATIC\_OUTPUT\_STATE(TIM\_BDTRInitStruct->TIM\_AutomaticOutput));
02221 
02222   \textcolor{comment}{/* Set the Lock level, the Break enable Bit and the Polarity, the OSSR State,}
02223 \textcolor{comment}{     the OSSI State, the dead time value and the Automatic Output Enable Bit */}
02224   TIMx->BDTR = (uint32\_t)TIM\_BDTRInitStruct->TIM\_OSSRState | TIM\_BDTRInitStruct->TIM\_OSSIState |
02225              TIM\_BDTRInitStruct->TIM\_LOCKLevel | TIM\_BDTRInitStruct->TIM\_DeadTime |
02226              TIM\_BDTRInitStruct->TIM\_Break | TIM\_BDTRInitStruct->TIM\_BreakPolarity |
02227              TIM\_BDTRInitStruct->TIM\_AutomaticOutput;
02228 \}
02229 
02230 \textcolor{comment}{/**}
02231 \textcolor{comment}{  * @brief  Fills each TIM\_BDTRInitStruct member with its default value.}
02232 \textcolor{comment}{  * @param  TIM\_BDTRInitStruct: pointer to a TIM\_BDTRInitTypeDef structure which}
02233 \textcolor{comment}{  *         will be initialized.}
02234 \textcolor{comment}{  * @retval None}
02235 \textcolor{comment}{  */}
02236 \textcolor{keywordtype}{void} TIM_BDTRStructInit(TIM\_BDTRInitTypeDef* TIM\_BDTRInitStruct)
02237 \{
02238   \textcolor{comment}{/* Set the default configuration */}
02239   TIM\_BDTRInitStruct->TIM_OSSRState = TIM_OSSRState_Disable;
02240   TIM\_BDTRInitStruct->TIM_OSSIState = TIM_OSSIState_Disable;
02241   TIM\_BDTRInitStruct->TIM_LOCKLevel = TIM_LOCKLevel_OFF;
02242   TIM\_BDTRInitStruct->TIM_DeadTime = 0x00;
02243   TIM\_BDTRInitStruct->TIM_Break = TIM_Break_Disable;
02244   TIM\_BDTRInitStruct->TIM_BreakPolarity = TIM_BreakPolarity_Low;
02245   TIM\_BDTRInitStruct->TIM_AutomaticOutput = TIM_AutomaticOutput_Disable;
02246 \}
02247 
02248 \textcolor{comment}{/**}
02249 \textcolor{comment}{  * @brief  Enables or disables the TIM peripheral Main Outputs.}
02250 \textcolor{comment}{  * @param  TIMx: where x can be 1 or 8 to select the TIMx peripheral.}
02251 \textcolor{comment}{  * @param  NewState: new state of the TIM peripheral Main Outputs.}
02252 \textcolor{comment}{  *          This parameter can be: ENABLE or DISABLE.}
02253 \textcolor{comment}{  * @retval None}
02254 \textcolor{comment}{  */}
02255 \textcolor{keywordtype}{void} TIM_CtrlPWMOutputs(TIM\_TypeDef* TIMx, FunctionalState NewState)
02256 \{
02257   \textcolor{comment}{/* Check the parameters */}
02258   assert_param(IS\_TIM\_LIST4\_PERIPH(TIMx));
02259   assert_param(IS\_FUNCTIONAL\_STATE(NewState));
02260 
02261   \textcolor{keywordflow}{if} (NewState != DISABLE)
02262   \{
02263     \textcolor{comment}{/* Enable the TIM Main Output */}
02264     TIMx->BDTR |= TIM_BDTR_MOE;
02265   \}
02266   \textcolor{keywordflow}{else}
02267   \{
02268     \textcolor{comment}{/* Disable the TIM Main Output */}
02269     TIMx->BDTR &= (uint16\_t)~TIM_BDTR_MOE;
02270   \}
02271 \}
02272 
02273 \textcolor{comment}{/**}
02274 \textcolor{comment}{  * @brief  Selects the TIM peripheral Commutation event.}
02275 \textcolor{comment}{  * @param  TIMx: where x can be  1 or 8 to select the TIMx peripheral}
02276 \textcolor{comment}{  * @param  NewState: new state of the Commutation event.}
02277 \textcolor{comment}{  *          This parameter can be: ENABLE or DISABLE.}
02278 \textcolor{comment}{  * @retval None}
02279 \textcolor{comment}{  */}
02280 \textcolor{keywordtype}{void} TIM_SelectCOM(TIM\_TypeDef* TIMx, FunctionalState NewState)
02281 \{
02282   \textcolor{comment}{/* Check the parameters */}
02283   assert_param(IS\_TIM\_LIST4\_PERIPH(TIMx));
02284   assert_param(IS\_FUNCTIONAL\_STATE(NewState));
02285 
02286   \textcolor{keywordflow}{if} (NewState != DISABLE)
02287   \{
02288     \textcolor{comment}{/* Set the COM Bit */}
02289     TIMx->CR2 |= TIM_CR2_CCUS;
02290   \}
02291   \textcolor{keywordflow}{else}
02292   \{
02293     \textcolor{comment}{/* Reset the COM Bit */}
02294     TIMx->CR2 &= (uint16\_t)~TIM_CR2_CCUS;
02295   \}
02296 \}
02297 
02298 \textcolor{comment}{/**}
02299 \textcolor{comment}{  * @brief  Sets or Resets the TIM peripheral Capture Compare Preload Control bit.}
02300 \textcolor{comment}{  * @param  TIMx: where x can be  1 or 8 to select the TIMx peripheral}
02301 \textcolor{comment}{  * @param  NewState: new state of the Capture Compare Preload Control bit}
02302 \textcolor{comment}{  *          This parameter can be: ENABLE or DISABLE.}
02303 \textcolor{comment}{  * @retval None}
02304 \textcolor{comment}{  */}
02305 \textcolor{keywordtype}{void} TIM_CCPreloadControl(TIM\_TypeDef* TIMx, FunctionalState NewState)
02306 \{
02307   \textcolor{comment}{/* Check the parameters */}
02308   assert_param(IS\_TIM\_LIST4\_PERIPH(TIMx));
02309   assert_param(IS\_FUNCTIONAL\_STATE(NewState));
02310   \textcolor{keywordflow}{if} (NewState != DISABLE)
02311   \{
02312     \textcolor{comment}{/* Set the CCPC Bit */}
02313     TIMx->CR2 |= TIM_CR2_CCPC;
02314   \}
02315   \textcolor{keywordflow}{else}
02316   \{
02317     \textcolor{comment}{/* Reset the CCPC Bit */}
02318     TIMx->CR2 &= (uint16\_t)~TIM_CR2_CCPC;
02319   \}
02320 \}
02321 \textcolor{comment}{/**}
02322 \textcolor{comment}{  * @\}}
02323 \textcolor{comment}{  */}
02324 
02325 \textcolor{comment}{/** @defgroup TIM\_Group5 Interrupts DMA and flags management functions}
02326 \textcolor{comment}{ *  @brief    Interrupts, DMA and flags management functions }
02327 \textcolor{comment}{ *}
02328 \textcolor{comment}{@verbatim   }
02329 \textcolor{comment}{ ===============================================================================}
02330 \textcolor{comment}{                 Interrupts, DMA and flags management functions}
02331 \textcolor{comment}{ ===============================================================================  }
02332 \textcolor{comment}{}
02333 \textcolor{comment}{@endverbatim}
02334 \textcolor{comment}{  * @\{}
02335 \textcolor{comment}{  */}
02336 
02337 \textcolor{comment}{/**}
02338 \textcolor{comment}{  * @brief  Enables or disables the specified TIM interrupts.}
02339 \textcolor{comment}{  * @param  TIMx: where x can be 1 to 14 to select the TIMx peripheral.}
02340 \textcolor{comment}{  * @param  TIM\_IT: specifies the TIM interrupts sources to be enabled or disabled.}
02341 \textcolor{comment}{  *          This parameter can be any combination of the following values:}
02342 \textcolor{comment}{  *            @arg TIM\_IT\_Update: TIM update Interrupt source}
02343 \textcolor{comment}{  *            @arg TIM\_IT\_CC1: TIM Capture Compare 1 Interrupt source}
02344 \textcolor{comment}{  *            @arg TIM\_IT\_CC2: TIM Capture Compare 2 Interrupt source}
02345 \textcolor{comment}{  *            @arg TIM\_IT\_CC3: TIM Capture Compare 3 Interrupt source}
02346 \textcolor{comment}{  *            @arg TIM\_IT\_CC4: TIM Capture Compare 4 Interrupt source}
02347 \textcolor{comment}{  *            @arg TIM\_IT\_COM: TIM Commutation Interrupt source}
02348 \textcolor{comment}{  *            @arg TIM\_IT\_Trigger: TIM Trigger Interrupt source}
02349 \textcolor{comment}{  *            @arg TIM\_IT\_Break: TIM Break Interrupt source}
02350 \textcolor{comment}{  *  }
02351 \textcolor{comment}{  * @note   For TIM6 and TIM7 only the parameter TIM\_IT\_Update can be used}
02352 \textcolor{comment}{  * @note   For TIM9 and TIM12 only one of the following parameters can be used: TIM\_IT\_Update,}
02353 \textcolor{comment}{  *          TIM\_IT\_CC1, TIM\_IT\_CC2 or TIM\_IT\_Trigger. }
02354 \textcolor{comment}{  * @note   For TIM10, TIM11, TIM13 and TIM14 only one of the following parameters can}
02355 \textcolor{comment}{  *          be used: TIM\_IT\_Update or TIM\_IT\_CC1   }
02356 \textcolor{comment}{  * @note   TIM\_IT\_COM and TIM\_IT\_Break can be used only with TIM1 and TIM8 }
02357 \textcolor{comment}{  *        }
02358 \textcolor{comment}{  * @param  NewState: new state of the TIM interrupts.}
02359 \textcolor{comment}{  *          This parameter can be: ENABLE or DISABLE.}
02360 \textcolor{comment}{  * @retval None}
02361 \textcolor{comment}{  */}
02362 \textcolor{keywordtype}{void} TIM_ITConfig(TIM\_TypeDef* TIMx, uint16\_t TIM\_IT, FunctionalState NewState)
02363 \{
02364   \textcolor{comment}{/* Check the parameters */}
02365   assert_param(IS\_TIM\_ALL\_PERIPH(TIMx));
02366   assert_param(IS\_TIM\_IT(TIM\_IT));
02367   assert_param(IS\_FUNCTIONAL\_STATE(NewState));
02368 
02369   \textcolor{keywordflow}{if} (NewState != DISABLE)
02370   \{
02371     \textcolor{comment}{/* Enable the Interrupt sources */}
02372     TIMx->DIER |= TIM\_IT;
02373   \}
02374   \textcolor{keywordflow}{else}
02375   \{
02376     \textcolor{comment}{/* Disable the Interrupt sources */}
02377     TIMx->DIER &= (uint16\_t)~TIM\_IT;
02378   \}
02379 \}
02380 
02381 \textcolor{comment}{/**}
02382 \textcolor{comment}{  * @brief  Configures the TIMx event to be generate by software.}
02383 \textcolor{comment}{  * @param  TIMx: where x can be 1 to 14 to select the TIM peripheral.}
02384 \textcolor{comment}{  * @param  TIM\_EventSource: specifies the event source.}
02385 \textcolor{comment}{  *          This parameter can be one or more of the following values:    }
02386 \textcolor{comment}{  *            @arg TIM\_EventSource\_Update: Timer update Event source}
02387 \textcolor{comment}{  *            @arg TIM\_EventSource\_CC1: Timer Capture Compare 1 Event source}
02388 \textcolor{comment}{  *            @arg TIM\_EventSource\_CC2: Timer Capture Compare 2 Event source}
02389 \textcolor{comment}{  *            @arg TIM\_EventSource\_CC3: Timer Capture Compare 3 Event source}
02390 \textcolor{comment}{  *            @arg TIM\_EventSource\_CC4: Timer Capture Compare 4 Event source}
02391 \textcolor{comment}{  *            @arg TIM\_EventSource\_COM: Timer COM event source  }
02392 \textcolor{comment}{  *            @arg TIM\_EventSource\_Trigger: Timer Trigger Event source}
02393 \textcolor{comment}{  *            @arg TIM\_EventSource\_Break: Timer Break event source}
02394 \textcolor{comment}{  * }
02395 \textcolor{comment}{  * @note   TIM6 and TIM7 can only generate an update event. }
02396 \textcolor{comment}{  * @note   TIM\_EventSource\_COM and TIM\_EventSource\_Break are used only with TIM1 and TIM8.}
02397 \textcolor{comment}{  *        }
02398 \textcolor{comment}{  * @retval None}
02399 \textcolor{comment}{  */}
02400 \textcolor{keywordtype}{void} TIM_GenerateEvent(TIM\_TypeDef* TIMx, uint16\_t TIM\_EventSource)
02401 \{
02402   \textcolor{comment}{/* Check the parameters */}
02403   assert_param(IS\_TIM\_ALL\_PERIPH(TIMx));
02404   assert_param(IS\_TIM\_EVENT\_SOURCE(TIM\_EventSource));
02405 
02406   \textcolor{comment}{/* Set the event sources */}
02407   TIMx->EGR = TIM\_EventSource;
02408 \}
02409 
02410 \textcolor{comment}{/**}
02411 \textcolor{comment}{  * @brief  Checks whether the specified TIM flag is set or not.}
02412 \textcolor{comment}{  * @param  TIMx: where x can be 1 to 14 to select the TIM peripheral.}
02413 \textcolor{comment}{  * @param  TIM\_FLAG: specifies the flag to check.}
02414 \textcolor{comment}{  *          This parameter can be one of the following values:}
02415 \textcolor{comment}{  *            @arg TIM\_FLAG\_Update: TIM update Flag}
02416 \textcolor{comment}{  *            @arg TIM\_FLAG\_CC1: TIM Capture Compare 1 Flag}
02417 \textcolor{comment}{  *            @arg TIM\_FLAG\_CC2: TIM Capture Compare 2 Flag}
02418 \textcolor{comment}{  *            @arg TIM\_FLAG\_CC3: TIM Capture Compare 3 Flag}
02419 \textcolor{comment}{  *            @arg TIM\_FLAG\_CC4: TIM Capture Compare 4 Flag}
02420 \textcolor{comment}{  *            @arg TIM\_FLAG\_COM: TIM Commutation Flag}
02421 \textcolor{comment}{  *            @arg TIM\_FLAG\_Trigger: TIM Trigger Flag}
02422 \textcolor{comment}{  *            @arg TIM\_FLAG\_Break: TIM Break Flag}
02423 \textcolor{comment}{  *            @arg TIM\_FLAG\_CC1OF: TIM Capture Compare 1 over capture Flag}
02424 \textcolor{comment}{  *            @arg TIM\_FLAG\_CC2OF: TIM Capture Compare 2 over capture Flag}
02425 \textcolor{comment}{  *            @arg TIM\_FLAG\_CC3OF: TIM Capture Compare 3 over capture Flag}
02426 \textcolor{comment}{  *            @arg TIM\_FLAG\_CC4OF: TIM Capture Compare 4 over capture Flag}
02427 \textcolor{comment}{  *}
02428 \textcolor{comment}{  * @note   TIM6 and TIM7 can have only one update flag. }
02429 \textcolor{comment}{  * @note   TIM\_FLAG\_COM and TIM\_FLAG\_Break are used only with TIM1 and TIM8.    }
02430 \textcolor{comment}{  *}
02431 \textcolor{comment}{  * @retval The new state of TIM\_FLAG (SET or RESET).}
02432 \textcolor{comment}{  */}
02433 FlagStatus TIM_GetFlagStatus(TIM\_TypeDef* TIMx, uint16\_t TIM\_FLAG)
02434 \{
02435   ITStatus bitstatus = RESET;
02436   \textcolor{comment}{/* Check the parameters */}
02437   assert_param(IS\_TIM\_ALL\_PERIPH(TIMx));
02438   assert_param(IS\_TIM\_GET\_FLAG(TIM\_FLAG));
02439 
02440 
02441   \textcolor{keywordflow}{if} ((TIMx->SR & TIM\_FLAG) != (uint16\_t)RESET)
02442   \{
02443     bitstatus = SET;
02444   \}
02445   \textcolor{keywordflow}{else}
02446   \{
02447     bitstatus = RESET;
02448   \}
02449   \textcolor{keywordflow}{return} bitstatus;
02450 \}
02451 
02452 \textcolor{comment}{/**}
02453 \textcolor{comment}{  * @brief  Clears the TIMx's pending flags.}
02454 \textcolor{comment}{  * @param  TIMx: where x can be 1 to 14 to select the TIM peripheral.}
02455 \textcolor{comment}{  * @param  TIM\_FLAG: specifies the flag bit to clear.}
02456 \textcolor{comment}{  *          This parameter can be any combination of the following values:}
02457 \textcolor{comment}{  *            @arg TIM\_FLAG\_Update: TIM update Flag}
02458 \textcolor{comment}{  *            @arg TIM\_FLAG\_CC1: TIM Capture Compare 1 Flag}
02459 \textcolor{comment}{  *            @arg TIM\_FLAG\_CC2: TIM Capture Compare 2 Flag}
02460 \textcolor{comment}{  *            @arg TIM\_FLAG\_CC3: TIM Capture Compare 3 Flag}
02461 \textcolor{comment}{  *            @arg TIM\_FLAG\_CC4: TIM Capture Compare 4 Flag}
02462 \textcolor{comment}{  *            @arg TIM\_FLAG\_COM: TIM Commutation Flag}
02463 \textcolor{comment}{  *            @arg TIM\_FLAG\_Trigger: TIM Trigger Flag}
02464 \textcolor{comment}{  *            @arg TIM\_FLAG\_Break: TIM Break Flag}
02465 \textcolor{comment}{  *            @arg TIM\_FLAG\_CC1OF: TIM Capture Compare 1 over capture Flag}
02466 \textcolor{comment}{  *            @arg TIM\_FLAG\_CC2OF: TIM Capture Compare 2 over capture Flag}
02467 \textcolor{comment}{  *            @arg TIM\_FLAG\_CC3OF: TIM Capture Compare 3 over capture Flag}
02468 \textcolor{comment}{  *            @arg TIM\_FLAG\_CC4OF: TIM Capture Compare 4 over capture Flag}
02469 \textcolor{comment}{  *}
02470 \textcolor{comment}{  * @note   TIM6 and TIM7 can have only one update flag. }
02471 \textcolor{comment}{  * @note   TIM\_FLAG\_COM and TIM\_FLAG\_Break are used only with TIM1 and TIM8.}
02472 \textcolor{comment}{  *    }
02473 \textcolor{comment}{  * @retval None}
02474 \textcolor{comment}{  */}
02475 \textcolor{keywordtype}{void} TIM_ClearFlag(TIM\_TypeDef* TIMx, uint16\_t TIM\_FLAG)
02476 \{
02477   \textcolor{comment}{/* Check the parameters */}
02478   assert_param(IS\_TIM\_ALL\_PERIPH(TIMx));
02479 
02480   \textcolor{comment}{/* Clear the flags */}
02481   TIMx->SR = (uint16\_t)~TIM\_FLAG;
02482 \}
02483 
02484 \textcolor{comment}{/**}
02485 \textcolor{comment}{  * @brief  Checks whether the TIM interrupt has occurred or not.}
02486 \textcolor{comment}{  * @param  TIMx: where x can be 1 to 14 to select the TIM peripheral.}
02487 \textcolor{comment}{  * @param  TIM\_IT: specifies the TIM interrupt source to check.}
02488 \textcolor{comment}{  *          This parameter can be one of the following values:}
02489 \textcolor{comment}{  *            @arg TIM\_IT\_Update: TIM update Interrupt source}
02490 \textcolor{comment}{  *            @arg TIM\_IT\_CC1: TIM Capture Compare 1 Interrupt source}
02491 \textcolor{comment}{  *            @arg TIM\_IT\_CC2: TIM Capture Compare 2 Interrupt source}
02492 \textcolor{comment}{  *            @arg TIM\_IT\_CC3: TIM Capture Compare 3 Interrupt source}
02493 \textcolor{comment}{  *            @arg TIM\_IT\_CC4: TIM Capture Compare 4 Interrupt source}
02494 \textcolor{comment}{  *            @arg TIM\_IT\_COM: TIM Commutation Interrupt source}
02495 \textcolor{comment}{  *            @arg TIM\_IT\_Trigger: TIM Trigger Interrupt source}
02496 \textcolor{comment}{  *            @arg TIM\_IT\_Break: TIM Break Interrupt source}
02497 \textcolor{comment}{  *}
02498 \textcolor{comment}{  * @note   TIM6 and TIM7 can generate only an update interrupt.}
02499 \textcolor{comment}{  * @note   TIM\_IT\_COM and TIM\_IT\_Break are used only with TIM1 and TIM8.}
02500 \textcolor{comment}{  *     }
02501 \textcolor{comment}{  * @retval The new state of the TIM\_IT(SET or RESET).}
02502 \textcolor{comment}{  */}
02503 ITStatus TIM_GetITStatus(TIM\_TypeDef* TIMx, uint16\_t TIM\_IT)
02504 \{
02505   ITStatus bitstatus = RESET;
02506   uint16\_t itstatus = 0x0, itenable = 0x0;
02507   \textcolor{comment}{/* Check the parameters */}
02508   assert_param(IS\_TIM\_ALL\_PERIPH(TIMx));
02509   assert_param(IS\_TIM\_GET\_IT(TIM\_IT));
02510 
02511   itstatus = TIMx->SR & TIM\_IT;
02512 
02513   itenable = TIMx->DIER & TIM\_IT;
02514   \textcolor{keywordflow}{if} ((itstatus != (uint16\_t)RESET) && (itenable != (uint16\_t)RESET))
02515   \{
02516     bitstatus = SET;
02517   \}
02518   \textcolor{keywordflow}{else}
02519   \{
02520     bitstatus = RESET;
02521   \}
02522   \textcolor{keywordflow}{return} bitstatus;
02523 \}
02524 
02525 \textcolor{comment}{/**}
02526 \textcolor{comment}{  * @brief  Clears the TIMx's interrupt pending bits.}
02527 \textcolor{comment}{  * @param  TIMx: where x can be 1 to 14 to select the TIM peripheral.}
02528 \textcolor{comment}{  * @param  TIM\_IT: specifies the pending bit to clear.}
02529 \textcolor{comment}{  *          This parameter can be any combination of the following values:}
02530 \textcolor{comment}{  *            @arg TIM\_IT\_Update: TIM1 update Interrupt source}
02531 \textcolor{comment}{  *            @arg TIM\_IT\_CC1: TIM Capture Compare 1 Interrupt source}
02532 \textcolor{comment}{  *            @arg TIM\_IT\_CC2: TIM Capture Compare 2 Interrupt source}
02533 \textcolor{comment}{  *            @arg TIM\_IT\_CC3: TIM Capture Compare 3 Interrupt source}
02534 \textcolor{comment}{  *            @arg TIM\_IT\_CC4: TIM Capture Compare 4 Interrupt source}
02535 \textcolor{comment}{  *            @arg TIM\_IT\_COM: TIM Commutation Interrupt source}
02536 \textcolor{comment}{  *            @arg TIM\_IT\_Trigger: TIM Trigger Interrupt source}
02537 \textcolor{comment}{  *            @arg TIM\_IT\_Break: TIM Break Interrupt source}
02538 \textcolor{comment}{  *}
02539 \textcolor{comment}{  * @note   TIM6 and TIM7 can generate only an update interrupt.}
02540 \textcolor{comment}{  * @note   TIM\_IT\_COM and TIM\_IT\_Break are used only with TIM1 and TIM8.}
02541 \textcolor{comment}{  *      }
02542 \textcolor{comment}{  * @retval None}
02543 \textcolor{comment}{  */}
02544 \textcolor{keywordtype}{void} TIM_ClearITPendingBit(TIM\_TypeDef* TIMx, uint16\_t TIM\_IT)
02545 \{
02546   \textcolor{comment}{/* Check the parameters */}
02547   assert_param(IS\_TIM\_ALL\_PERIPH(TIMx));
02548 
02549   \textcolor{comment}{/* Clear the IT pending Bit */}
02550   TIMx->SR = (uint16\_t)~TIM\_IT;
02551 \}
02552 
02553 \textcolor{comment}{/**}
02554 \textcolor{comment}{  * @brief  Configures the TIMx's DMA interface.}
02555 \textcolor{comment}{  * @param  TIMx: where x can be 1, 2, 3, 4, 5 or 8 to select the TIM peripheral.}
02556 \textcolor{comment}{  * @param  TIM\_DMABase: DMA Base address.}
02557 \textcolor{comment}{  *          This parameter can be one of the following values:}
02558 \textcolor{comment}{  *            @arg TIM\_DMABase\_CR1  }
02559 \textcolor{comment}{  *            @arg TIM\_DMABase\_CR2}
02560 \textcolor{comment}{  *            @arg TIM\_DMABase\_SMCR}
02561 \textcolor{comment}{  *            @arg TIM\_DMABase\_DIER}
02562 \textcolor{comment}{  *            @arg TIM1\_DMABase\_SR}
02563 \textcolor{comment}{  *            @arg TIM\_DMABase\_EGR}
02564 \textcolor{comment}{  *            @arg TIM\_DMABase\_CCMR1}
02565 \textcolor{comment}{  *            @arg TIM\_DMABase\_CCMR2}
02566 \textcolor{comment}{  *            @arg TIM\_DMABase\_CCER}
02567 \textcolor{comment}{  *            @arg TIM\_DMABase\_CNT   }
02568 \textcolor{comment}{  *            @arg TIM\_DMABase\_PSC   }
02569 \textcolor{comment}{  *            @arg TIM\_DMABase\_ARR}
02570 \textcolor{comment}{  *            @arg TIM\_DMABase\_RCR}
02571 \textcolor{comment}{  *            @arg TIM\_DMABase\_CCR1}
02572 \textcolor{comment}{  *            @arg TIM\_DMABase\_CCR2}
02573 \textcolor{comment}{  *            @arg TIM\_DMABase\_CCR3  }
02574 \textcolor{comment}{  *            @arg TIM\_DMABase\_CCR4}
02575 \textcolor{comment}{  *            @arg TIM\_DMABase\_BDTR}
02576 \textcolor{comment}{  *            @arg TIM\_DMABase\_DCR}
02577 \textcolor{comment}{  * @param  TIM\_DMABurstLength: DMA Burst length. This parameter can be one value}
02578 \textcolor{comment}{  *         between: TIM\_DMABurstLength\_1Transfer and TIM\_DMABurstLength\_18Transfers.}
02579 \textcolor{comment}{  * @retval None}
02580 \textcolor{comment}{  */}
02581 \textcolor{keywordtype}{void} TIM_DMAConfig(TIM\_TypeDef* TIMx, uint16\_t TIM\_DMABase, uint16\_t TIM\_DMABurstLength)
02582 \{
02583   \textcolor{comment}{/* Check the parameters */}
02584   assert_param(IS\_TIM\_LIST3\_PERIPH(TIMx));
02585   assert_param(IS\_TIM\_DMA\_BASE(TIM\_DMABase));
02586   assert_param(IS\_TIM\_DMA\_LENGTH(TIM\_DMABurstLength));
02587 
02588   \textcolor{comment}{/* Set the DMA Base and the DMA Burst Length */}
02589   TIMx->DCR = TIM\_DMABase | TIM\_DMABurstLength;
02590 \}
02591 
02592 \textcolor{comment}{/**}
02593 \textcolor{comment}{  * @brief  Enables or disables the TIMx's DMA Requests.}
02594 \textcolor{comment}{  * @param  TIMx: where x can be 1, 2, 3, 4, 5, 6, 7 or 8 to select the TIM peripheral.}
02595 \textcolor{comment}{  * @param  TIM\_DMASource: specifies the DMA Request sources.}
02596 \textcolor{comment}{  *          This parameter can be any combination of the following values:}
02597 \textcolor{comment}{  *            @arg TIM\_DMA\_Update: TIM update Interrupt source}
02598 \textcolor{comment}{  *            @arg TIM\_DMA\_CC1: TIM Capture Compare 1 DMA source}
02599 \textcolor{comment}{  *            @arg TIM\_DMA\_CC2: TIM Capture Compare 2 DMA source}
02600 \textcolor{comment}{  *            @arg TIM\_DMA\_CC3: TIM Capture Compare 3 DMA source}
02601 \textcolor{comment}{  *            @arg TIM\_DMA\_CC4: TIM Capture Compare 4 DMA source}
02602 \textcolor{comment}{  *            @arg TIM\_DMA\_COM: TIM Commutation DMA source}
02603 \textcolor{comment}{  *            @arg TIM\_DMA\_Trigger: TIM Trigger DMA source}
02604 \textcolor{comment}{  * @param  NewState: new state of the DMA Request sources.}
02605 \textcolor{comment}{  *          This parameter can be: ENABLE or DISABLE.}
02606 \textcolor{comment}{  * @retval None}
02607 \textcolor{comment}{  */}
02608 \textcolor{keywordtype}{void} TIM_DMACmd(TIM\_TypeDef* TIMx, uint16\_t TIM\_DMASource, FunctionalState NewState)
02609 \{
02610   \textcolor{comment}{/* Check the parameters */}
02611   assert_param(IS\_TIM\_LIST5\_PERIPH(TIMx));
02612   assert_param(IS\_TIM\_DMA\_SOURCE(TIM\_DMASource));
02613   assert_param(IS\_FUNCTIONAL\_STATE(NewState));
02614 
02615   \textcolor{keywordflow}{if} (NewState != DISABLE)
02616   \{
02617     \textcolor{comment}{/* Enable the DMA sources */}
02618     TIMx->DIER |= TIM\_DMASource;
02619   \}
02620   \textcolor{keywordflow}{else}
02621   \{
02622     \textcolor{comment}{/* Disable the DMA sources */}
02623     TIMx->DIER &= (uint16\_t)~TIM\_DMASource;
02624   \}
02625 \}
02626 
02627 \textcolor{comment}{/**}
02628 \textcolor{comment}{  * @brief  Selects the TIMx peripheral Capture Compare DMA source.}
02629 \textcolor{comment}{  * @param  TIMx: where x can be  1, 2, 3, 4, 5 or 8 to select the TIM peripheral.}
02630 \textcolor{comment}{  * @param  NewState: new state of the Capture Compare DMA source}
02631 \textcolor{comment}{  *          This parameter can be: ENABLE or DISABLE.}
02632 \textcolor{comment}{  * @retval None}
02633 \textcolor{comment}{  */}
02634 \textcolor{keywordtype}{void} TIM_SelectCCDMA(TIM\_TypeDef* TIMx, FunctionalState NewState)
02635 \{
02636   \textcolor{comment}{/* Check the parameters */}
02637   assert_param(IS\_TIM\_LIST3\_PERIPH(TIMx));
02638   assert_param(IS\_FUNCTIONAL\_STATE(NewState));
02639 
02640   \textcolor{keywordflow}{if} (NewState != DISABLE)
02641   \{
02642     \textcolor{comment}{/* Set the CCDS Bit */}
02643     TIMx->CR2 |= TIM_CR2_CCDS;
02644   \}
02645   \textcolor{keywordflow}{else}
02646   \{
02647     \textcolor{comment}{/* Reset the CCDS Bit */}
02648     TIMx->CR2 &= (uint16\_t)~TIM_CR2_CCDS;
02649   \}
02650 \}
02651 \textcolor{comment}{/**}
02652 \textcolor{comment}{  * @\}}
02653 \textcolor{comment}{  */}
02654 
02655 \textcolor{comment}{/** @defgroup TIM\_Group6 Clocks management functions}
02656 \textcolor{comment}{ *  @brief    Clocks management functions}
02657 \textcolor{comment}{ *}
02658 \textcolor{comment}{@verbatim   }
02659 \textcolor{comment}{ ===============================================================================}
02660 \textcolor{comment}{                         Clocks management functions}
02661 \textcolor{comment}{ ===============================================================================  }
02662 \textcolor{comment}{}
02663 \textcolor{comment}{@endverbatim}
02664 \textcolor{comment}{  * @\{}
02665 \textcolor{comment}{  */}
02666 
02667 \textcolor{comment}{/**}
02668 \textcolor{comment}{  * @brief  Configures the TIMx internal Clock}
02669 \textcolor{comment}{  * @param  TIMx: where x can be 1, 2, 3, 4, 5, 8, 9 or 12 to select the TIM }
02670 \textcolor{comment}{  *         peripheral.}
02671 \textcolor{comment}{  * @retval None}
02672 \textcolor{comment}{  */}
02673 \textcolor{keywordtype}{void} TIM_InternalClockConfig(TIM\_TypeDef* TIMx)
02674 \{
02675   \textcolor{comment}{/* Check the parameters */}
02676   assert_param(IS\_TIM\_LIST2\_PERIPH(TIMx));
02677 
02678   \textcolor{comment}{/* Disable slave mode to clock the prescaler directly with the internal clock */}
02679   TIMx->SMCR &=  (uint16\_t)~TIM_SMCR_SMS;
02680 \}
02681 
02682 \textcolor{comment}{/**}
02683 \textcolor{comment}{  * @brief  Configures the TIMx Internal Trigger as External Clock}
02684 \textcolor{comment}{  * @param  TIMx: where x can be 1, 2, 3, 4, 5, 8, 9 or 12 to select the TIM }
02685 \textcolor{comment}{  *         peripheral.}
02686 \textcolor{comment}{  * @param  TIM\_InputTriggerSource: Trigger source.}
02687 \textcolor{comment}{  *          This parameter can be one of the following values:}
02688 \textcolor{comment}{  *            @arg TIM\_TS\_ITR0: Internal Trigger 0}
02689 \textcolor{comment}{  *            @arg TIM\_TS\_ITR1: Internal Trigger 1}
02690 \textcolor{comment}{  *            @arg TIM\_TS\_ITR2: Internal Trigger 2}
02691 \textcolor{comment}{  *            @arg TIM\_TS\_ITR3: Internal Trigger 3}
02692 \textcolor{comment}{  * @retval None}
02693 \textcolor{comment}{  */}
02694 \textcolor{keywordtype}{void} TIM_ITRxExternalClockConfig(TIM\_TypeDef* TIMx, uint16\_t TIM\_InputTriggerSource)
02695 \{
02696   \textcolor{comment}{/* Check the parameters */}
02697   assert_param(IS\_TIM\_LIST2\_PERIPH(TIMx));
02698   assert_param(IS\_TIM\_INTERNAL\_TRIGGER\_SELECTION(TIM\_InputTriggerSource));
02699 
02700   \textcolor{comment}{/* Select the Internal Trigger */}
02701   TIM_SelectInputTrigger(TIMx, TIM\_InputTriggerSource);
02702 
02703   \textcolor{comment}{/* Select the External clock mode1 */}
02704   TIMx->SMCR |= TIM_SlaveMode_External1;
02705 \}
02706 
02707 \textcolor{comment}{/**}
02708 \textcolor{comment}{  * @brief  Configures the TIMx Trigger as External Clock}
02709 \textcolor{comment}{  * @param  TIMx: where x can be 1, 2, 3, 4, 5, 8, 9, 10, 11, 12, 13 or 14  }
02710 \textcolor{comment}{  *         to select the TIM peripheral.}
02711 \textcolor{comment}{  * @param  TIM\_TIxExternalCLKSource: Trigger source.}
02712 \textcolor{comment}{  *          This parameter can be one of the following values:}
02713 \textcolor{comment}{  *            @arg TIM\_TIxExternalCLK1Source\_TI1ED: TI1 Edge Detector}
02714 \textcolor{comment}{  *            @arg TIM\_TIxExternalCLK1Source\_TI1: Filtered Timer Input 1}
02715 \textcolor{comment}{  *            @arg TIM\_TIxExternalCLK1Source\_TI2: Filtered Timer Input 2}
02716 \textcolor{comment}{  * @param  TIM\_ICPolarity: specifies the TIx Polarity.}
02717 \textcolor{comment}{  *          This parameter can be one of the following values:}
02718 \textcolor{comment}{  *            @arg TIM\_ICPolarity\_Rising}
02719 \textcolor{comment}{  *            @arg TIM\_ICPolarity\_Falling}
02720 \textcolor{comment}{  * @param  ICFilter: specifies the filter value.}
02721 \textcolor{comment}{  *          This parameter must be a value between 0x0 and 0xF.}
02722 \textcolor{comment}{  * @retval None}
02723 \textcolor{comment}{  */}
02724 \textcolor{keywordtype}{void} TIM_TIxExternalClockConfig(TIM\_TypeDef* TIMx, uint16\_t TIM\_TIxExternalCLKSource,
02725                                 uint16\_t TIM\_ICPolarity, uint16\_t ICFilter)
02726 \{
02727   \textcolor{comment}{/* Check the parameters */}
02728   assert_param(IS\_TIM\_LIST1\_PERIPH(TIMx));
02729   assert_param(IS\_TIM\_IC\_POLARITY(TIM\_ICPolarity));
02730   assert_param(IS\_TIM\_IC\_FILTER(ICFilter));
02731 
02732   \textcolor{comment}{/* Configure the Timer Input Clock Source */}
02733   \textcolor{keywordflow}{if} (TIM\_TIxExternalCLKSource == TIM_TIxExternalCLK1Source_TI2)
02734   \{
02735     TI2_Config(TIMx, TIM\_ICPolarity, TIM_ICSelection_DirectTI, ICFilter);
02736   \}
02737   \textcolor{keywordflow}{else}
02738   \{
02739     TI1_Config(TIMx, TIM\_ICPolarity, TIM_ICSelection_DirectTI, ICFilter);
02740   \}
02741   \textcolor{comment}{/* Select the Trigger source */}
02742   TIM_SelectInputTrigger(TIMx, TIM\_TIxExternalCLKSource);
02743   \textcolor{comment}{/* Select the External clock mode1 */}
02744   TIMx->SMCR |= TIM_SlaveMode_External1;
02745 \}
02746 
02747 \textcolor{comment}{/**}
02748 \textcolor{comment}{  * @brief  Configures the External clock Mode1}
02749 \textcolor{comment}{  * @param  TIMx: where x can be  1, 2, 3, 4, 5 or 8 to select the TIM peripheral.}
02750 \textcolor{comment}{  * @param  TIM\_ExtTRGPrescaler: The external Trigger Prescaler.}
02751 \textcolor{comment}{  *          This parameter can be one of the following values:}
02752 \textcolor{comment}{  *            @arg TIM\_ExtTRGPSC\_OFF: ETRP Prescaler OFF.}
02753 \textcolor{comment}{  *            @arg TIM\_ExtTRGPSC\_DIV2: ETRP frequency divided by 2.}
02754 \textcolor{comment}{  *            @arg TIM\_ExtTRGPSC\_DIV4: ETRP frequency divided by 4.}
02755 \textcolor{comment}{  *            @arg TIM\_ExtTRGPSC\_DIV8: ETRP frequency divided by 8.}
02756 \textcolor{comment}{  * @param  TIM\_ExtTRGPolarity: The external Trigger Polarity.}
02757 \textcolor{comment}{  *          This parameter can be one of the following values:}
02758 \textcolor{comment}{  *            @arg TIM\_ExtTRGPolarity\_Inverted: active low or falling edge active.}
02759 \textcolor{comment}{  *            @arg TIM\_ExtTRGPolarity\_NonInverted: active high or rising edge active.}
02760 \textcolor{comment}{  * @param  ExtTRGFilter: External Trigger Filter.}
02761 \textcolor{comment}{  *          This parameter must be a value between 0x00 and 0x0F}
02762 \textcolor{comment}{  * @retval None}
02763 \textcolor{comment}{  */}
02764 \textcolor{keywordtype}{void} TIM_ETRClockMode1Config(TIM\_TypeDef* TIMx, uint16\_t TIM\_ExtTRGPrescaler,
02765                             uint16\_t TIM\_ExtTRGPolarity, uint16\_t ExtTRGFilter)
02766 \{
02767   uint16\_t tmpsmcr = 0;
02768 
02769   \textcolor{comment}{/* Check the parameters */}
02770   assert_param(IS\_TIM\_LIST3\_PERIPH(TIMx));
02771   assert_param(IS\_TIM\_EXT\_PRESCALER(TIM\_ExtTRGPrescaler));
02772   assert_param(IS\_TIM\_EXT\_POLARITY(TIM\_ExtTRGPolarity));
02773   assert_param(IS\_TIM\_EXT\_FILTER(ExtTRGFilter));
02774   \textcolor{comment}{/* Configure the ETR Clock source */}
02775   TIM_ETRConfig(TIMx, TIM\_ExtTRGPrescaler, TIM\_ExtTRGPolarity, ExtTRGFilter
      );
02776 
02777   \textcolor{comment}{/* Get the TIMx SMCR register value */}
02778   tmpsmcr = TIMx->SMCR;
02779 
02780   \textcolor{comment}{/* Reset the SMS Bits */}
02781   tmpsmcr &= (uint16\_t)~TIM_SMCR_SMS;
02782 
02783   \textcolor{comment}{/* Select the External clock mode1 */}
02784   tmpsmcr |= TIM_SlaveMode_External1;
02785 
02786   \textcolor{comment}{/* Select the Trigger selection : ETRF */}
02787   tmpsmcr &= (uint16\_t)~TIM_SMCR_TS;
02788   tmpsmcr |= TIM_TS_ETRF;
02789 
02790   \textcolor{comment}{/* Write to TIMx SMCR */}
02791   TIMx->SMCR = tmpsmcr;
02792 \}
02793 
02794 \textcolor{comment}{/**}
02795 \textcolor{comment}{  * @brief  Configures the External clock Mode2}
02796 \textcolor{comment}{  * @param  TIMx: where x can be  1, 2, 3, 4, 5 or 8 to select the TIM peripheral.}
02797 \textcolor{comment}{  * @param  TIM\_ExtTRGPrescaler: The external Trigger Prescaler.}
02798 \textcolor{comment}{  *          This parameter can be one of the following values:}
02799 \textcolor{comment}{  *            @arg TIM\_ExtTRGPSC\_OFF: ETRP Prescaler OFF.}
02800 \textcolor{comment}{  *            @arg TIM\_ExtTRGPSC\_DIV2: ETRP frequency divided by 2.}
02801 \textcolor{comment}{  *            @arg TIM\_ExtTRGPSC\_DIV4: ETRP frequency divided by 4.}
02802 \textcolor{comment}{  *            @arg TIM\_ExtTRGPSC\_DIV8: ETRP frequency divided by 8.}
02803 \textcolor{comment}{  * @param  TIM\_ExtTRGPolarity: The external Trigger Polarity.}
02804 \textcolor{comment}{  *          This parameter can be one of the following values:}
02805 \textcolor{comment}{  *            @arg TIM\_ExtTRGPolarity\_Inverted: active low or falling edge active.}
02806 \textcolor{comment}{  *            @arg TIM\_ExtTRGPolarity\_NonInverted: active high or rising edge active.}
02807 \textcolor{comment}{  * @param  ExtTRGFilter: External Trigger Filter.}
02808 \textcolor{comment}{  *          This parameter must be a value between 0x00 and 0x0F}
02809 \textcolor{comment}{  * @retval None}
02810 \textcolor{comment}{  */}
02811 \textcolor{keywordtype}{void} TIM_ETRClockMode2Config(TIM\_TypeDef* TIMx, uint16\_t TIM\_ExtTRGPrescaler,
02812                              uint16\_t TIM\_ExtTRGPolarity, uint16\_t ExtTRGFilter)
02813 \{
02814   \textcolor{comment}{/* Check the parameters */}
02815   assert_param(IS\_TIM\_LIST3\_PERIPH(TIMx));
02816   assert_param(IS\_TIM\_EXT\_PRESCALER(TIM\_ExtTRGPrescaler));
02817   assert_param(IS\_TIM\_EXT\_POLARITY(TIM\_ExtTRGPolarity));
02818   assert_param(IS\_TIM\_EXT\_FILTER(ExtTRGFilter));
02819 
02820   \textcolor{comment}{/* Configure the ETR Clock source */}
02821   TIM_ETRConfig(TIMx, TIM\_ExtTRGPrescaler, TIM\_ExtTRGPolarity, ExtTRGFilter
      );
02822 
02823   \textcolor{comment}{/* Enable the External clock mode2 */}
02824   TIMx->SMCR |= TIM_SMCR_ECE;
02825 \}
02826 \textcolor{comment}{/**}
02827 \textcolor{comment}{  * @\}}
02828 \textcolor{comment}{  */}
02829 
02830 \textcolor{comment}{/** @defgroup TIM\_Group7 Synchronization management functions}
02831 \textcolor{comment}{ *  @brief    Synchronization management functions }
02832 \textcolor{comment}{ *}
02833 \textcolor{comment}{@verbatim   }
02834 \textcolor{comment}{ ===============================================================================}
02835 \textcolor{comment}{                       Synchronization management functions}
02836 \textcolor{comment}{ ===============================================================================  }
02837 \textcolor{comment}{                   }
02838 \textcolor{comment}{       ===================================================================      }
02839 \textcolor{comment}{              TIM Driver: how to use it in synchronization Mode}
02840 \textcolor{comment}{       =================================================================== }
02841 \textcolor{comment}{       Case of two/several Timers}
02842 \textcolor{comment}{       **************************}
02843 \textcolor{comment}{       1. Configure the Master Timers using the following functions:}
02844 \textcolor{comment}{          - void TIM\_SelectOutputTrigger(TIM\_TypeDef* TIMx, uint16\_t TIM\_TRGOSource); }
02845 \textcolor{comment}{          - void TIM\_SelectMasterSlaveMode(TIM\_TypeDef* TIMx, uint16\_t TIM\_MasterSlaveMode);  }
02846 \textcolor{comment}{       2. Configure the Slave Timers using the following functions: }
02847 \textcolor{comment}{          - void TIM\_SelectInputTrigger(TIM\_TypeDef* TIMx, uint16\_t TIM\_InputTriggerSource);  }
02848 \textcolor{comment}{          - void TIM\_SelectSlaveMode(TIM\_TypeDef* TIMx, uint16\_t TIM\_SlaveMode); }
02849 \textcolor{comment}{          }
02850 \textcolor{comment}{       Case of Timers and external trigger(ETR pin)}
02851 \textcolor{comment}{       ********************************************       }
02852 \textcolor{comment}{       1. Configure the External trigger using this function:}
02853 \textcolor{comment}{          - void TIM\_ETRConfig(TIM\_TypeDef* TIMx, uint16\_t TIM\_ExtTRGPrescaler, uint16\_t
       TIM\_ExtTRGPolarity,}
02854 \textcolor{comment}{                               uint16\_t ExtTRGFilter);}
02855 \textcolor{comment}{       2. Configure the Slave Timers using the following functions: }
02856 \textcolor{comment}{          - void TIM\_SelectInputTrigger(TIM\_TypeDef* TIMx, uint16\_t TIM\_InputTriggerSource);  }
02857 \textcolor{comment}{          - void TIM\_SelectSlaveMode(TIM\_TypeDef* TIMx, uint16\_t TIM\_SlaveMode); }
02858 \textcolor{comment}{}
02859 \textcolor{comment}{@endverbatim}
02860 \textcolor{comment}{  * @\{}
02861 \textcolor{comment}{  */}
02862 
02863 \textcolor{comment}{/**}
02864 \textcolor{comment}{  * @brief  Selects the Input Trigger source}
02865 \textcolor{comment}{  * @param  TIMx: where x can be  1, 2, 3, 4, 5, 8, 9, 10, 11, 12, 13 or 14  }
02866 \textcolor{comment}{  *         to select the TIM peripheral.}
02867 \textcolor{comment}{  * @param  TIM\_InputTriggerSource: The Input Trigger source.}
02868 \textcolor{comment}{  *          This parameter can be one of the following values:}
02869 \textcolor{comment}{  *            @arg TIM\_TS\_ITR0: Internal Trigger 0}
02870 \textcolor{comment}{  *            @arg TIM\_TS\_ITR1: Internal Trigger 1}
02871 \textcolor{comment}{  *            @arg TIM\_TS\_ITR2: Internal Trigger 2}
02872 \textcolor{comment}{  *            @arg TIM\_TS\_ITR3: Internal Trigger 3}
02873 \textcolor{comment}{  *            @arg TIM\_TS\_TI1F\_ED: TI1 Edge Detector}
02874 \textcolor{comment}{  *            @arg TIM\_TS\_TI1FP1: Filtered Timer Input 1}
02875 \textcolor{comment}{  *            @arg TIM\_TS\_TI2FP2: Filtered Timer Input 2}
02876 \textcolor{comment}{  *            @arg TIM\_TS\_ETRF: External Trigger input}
02877 \textcolor{comment}{  * @retval None}
02878 \textcolor{comment}{  */}
02879 \textcolor{keywordtype}{void} TIM_SelectInputTrigger(TIM\_TypeDef* TIMx, uint16\_t TIM\_InputTriggerSource)
02880 \{
02881   uint16\_t tmpsmcr = 0;
02882 
02883   \textcolor{comment}{/* Check the parameters */}
02884   assert_param(IS\_TIM\_LIST1\_PERIPH(TIMx));
02885   assert_param(IS\_TIM\_TRIGGER\_SELECTION(TIM\_InputTriggerSource));
02886 
02887   \textcolor{comment}{/* Get the TIMx SMCR register value */}
02888   tmpsmcr = TIMx->SMCR;
02889 
02890   \textcolor{comment}{/* Reset the TS Bits */}
02891   tmpsmcr &= (uint16\_t)~TIM_SMCR_TS;
02892 
02893   \textcolor{comment}{/* Set the Input Trigger source */}
02894   tmpsmcr |= TIM\_InputTriggerSource;
02895 
02896   \textcolor{comment}{/* Write to TIMx SMCR */}
02897   TIMx->SMCR = tmpsmcr;
02898 \}
02899 
02900 \textcolor{comment}{/**}
02901 \textcolor{comment}{  * @brief  Selects the TIMx Trigger Output Mode.}
02902 \textcolor{comment}{  * @param  TIMx: where x can be 1, 2, 3, 4, 5, 6, 7 or 8 to select the TIM peripheral.}
02903 \textcolor{comment}{  *     }
02904 \textcolor{comment}{  * @param  TIM\_TRGOSource: specifies the Trigger Output source.}
02905 \textcolor{comment}{  *   This parameter can be one of the following values:}
02906 \textcolor{comment}{  *}
02907 \textcolor{comment}{  *  - For all TIMx}
02908 \textcolor{comment}{  *            @arg TIM\_TRGOSource\_Reset:  The UG bit in the TIM\_EGR register is used as the trigger
       output(TRGO)}
02909 \textcolor{comment}{  *            @arg TIM\_TRGOSource\_Enable: The Counter Enable CEN is used as the trigger output(TRGO)}
02910 \textcolor{comment}{  *            @arg TIM\_TRGOSource\_Update: The update event is selected as the trigger output(TRGO)}
02911 \textcolor{comment}{  *}
02912 \textcolor{comment}{  *  - For all TIMx except TIM6 and TIM7}
02913 \textcolor{comment}{  *            @arg TIM\_TRGOSource\_OC1: The trigger output sends a positive pulse when the CC1IF flag}
02914 \textcolor{comment}{  *                                     is to be set, as soon as a capture or compare match
       occurs(TRGO)}
02915 \textcolor{comment}{  *            @arg TIM\_TRGOSource\_OC1Ref: OC1REF signal is used as the trigger output(TRGO)}
02916 \textcolor{comment}{  *            @arg TIM\_TRGOSource\_OC2Ref: OC2REF signal is used as the trigger output(TRGO)}
02917 \textcolor{comment}{  *            @arg TIM\_TRGOSource\_OC3Ref: OC3REF signal is used as the trigger output(TRGO)}
02918 \textcolor{comment}{  *            @arg TIM\_TRGOSource\_OC4Ref: OC4REF signal is used as the trigger output(TRGO)}
02919 \textcolor{comment}{  *}
02920 \textcolor{comment}{  * @retval None}
02921 \textcolor{comment}{  */}
02922 \textcolor{keywordtype}{void} TIM_SelectOutputTrigger(TIM\_TypeDef* TIMx, uint16\_t TIM\_TRGOSource)
02923 \{
02924   \textcolor{comment}{/* Check the parameters */}
02925   assert_param(IS\_TIM\_LIST5\_PERIPH(TIMx));
02926   assert_param(IS\_TIM\_TRGO\_SOURCE(TIM\_TRGOSource));
02927 
02928   \textcolor{comment}{/* Reset the MMS Bits */}
02929   TIMx->CR2 &= (uint16\_t)~TIM_CR2_MMS;
02930   \textcolor{comment}{/* Select the TRGO source */}
02931   TIMx->CR2 |=  TIM\_TRGOSource;
02932 \}
02933 
02934 \textcolor{comment}{/**}
02935 \textcolor{comment}{  * @brief  Selects the TIMx Slave Mode.}
02936 \textcolor{comment}{  * @param  TIMx: where x can be 1, 2, 3, 4, 5, 8, 9 or 12 to select the TIM peripheral.}
02937 \textcolor{comment}{  * @param  TIM\_SlaveMode: specifies the Timer Slave Mode.}
02938 \textcolor{comment}{  *          This parameter can be one of the following values:}
02939 \textcolor{comment}{  *            @arg TIM\_SlaveMode\_Reset: Rising edge of the selected trigger signal(TRGI) reinitialize
       }
02940 \textcolor{comment}{  *                                      the counter and triggers an update of the registers}
02941 \textcolor{comment}{  *            @arg TIM\_SlaveMode\_Gated:     The counter clock is enabled when the trigger signal
       (TRGI) is high}
02942 \textcolor{comment}{  *            @arg TIM\_SlaveMode\_Trigger:   The counter starts at a rising edge of the trigger TRGI}
02943 \textcolor{comment}{  *            @arg TIM\_SlaveMode\_External1: Rising edges of the selected trigger (TRGI) clock the
       counter}
02944 \textcolor{comment}{  * @retval None}
02945 \textcolor{comment}{  */}
02946 \textcolor{keywordtype}{void} TIM_SelectSlaveMode(TIM\_TypeDef* TIMx, uint16\_t TIM\_SlaveMode)
02947 \{
02948   \textcolor{comment}{/* Check the parameters */}
02949   assert_param(IS\_TIM\_LIST2\_PERIPH(TIMx));
02950   assert_param(IS\_TIM\_SLAVE\_MODE(TIM\_SlaveMode));
02951 
02952   \textcolor{comment}{/* Reset the SMS Bits */}
02953   TIMx->SMCR &= (uint16\_t)~TIM_SMCR_SMS;
02954 
02955   \textcolor{comment}{/* Select the Slave Mode */}
02956   TIMx->SMCR |= TIM\_SlaveMode;
02957 \}
02958 
02959 \textcolor{comment}{/**}
02960 \textcolor{comment}{  * @brief  Sets or Resets the TIMx Master/Slave Mode.}
02961 \textcolor{comment}{  * @param  TIMx: where x can be 1, 2, 3, 4, 5, 8, 9 or 12 to select the TIM peripheral.}
02962 \textcolor{comment}{  * @param  TIM\_MasterSlaveMode: specifies the Timer Master Slave Mode.}
02963 \textcolor{comment}{  *          This parameter can be one of the following values:}
02964 \textcolor{comment}{  *            @arg TIM\_MasterSlaveMode\_Enable: synchronization between the current timer}
02965 \textcolor{comment}{  *                                             and its slaves (through TRGO)}
02966 \textcolor{comment}{  *            @arg TIM\_MasterSlaveMode\_Disable: No action}
02967 \textcolor{comment}{  * @retval None}
02968 \textcolor{comment}{  */}
02969 \textcolor{keywordtype}{void} TIM_SelectMasterSlaveMode(TIM\_TypeDef* TIMx, uint16\_t TIM\_MasterSlaveMode)
02970 \{
02971   \textcolor{comment}{/* Check the parameters */}
02972   assert_param(IS\_TIM\_LIST2\_PERIPH(TIMx));
02973   assert_param(IS\_TIM\_MSM\_STATE(TIM\_MasterSlaveMode));
02974 
02975   \textcolor{comment}{/* Reset the MSM Bit */}
02976   TIMx->SMCR &= (uint16\_t)~TIM_SMCR_MSM;
02977 
02978   \textcolor{comment}{/* Set or Reset the MSM Bit */}
02979   TIMx->SMCR |= TIM\_MasterSlaveMode;
02980 \}
02981 
02982 \textcolor{comment}{/**}
02983 \textcolor{comment}{  * @brief  Configures the TIMx External Trigger (ETR).}
02984 \textcolor{comment}{  * @param  TIMx: where x can be  1, 2, 3, 4, 5 or 8 to select the TIM peripheral.}
02985 \textcolor{comment}{  * @param  TIM\_ExtTRGPrescaler: The external Trigger Prescaler.}
02986 \textcolor{comment}{  *          This parameter can be one of the following values:}
02987 \textcolor{comment}{  *            @arg TIM\_ExtTRGPSC\_OFF: ETRP Prescaler OFF.}
02988 \textcolor{comment}{  *            @arg TIM\_ExtTRGPSC\_DIV2: ETRP frequency divided by 2.}
02989 \textcolor{comment}{  *            @arg TIM\_ExtTRGPSC\_DIV4: ETRP frequency divided by 4.}
02990 \textcolor{comment}{  *            @arg TIM\_ExtTRGPSC\_DIV8: ETRP frequency divided by 8.}
02991 \textcolor{comment}{  * @param  TIM\_ExtTRGPolarity: The external Trigger Polarity.}
02992 \textcolor{comment}{  *          This parameter can be one of the following values:}
02993 \textcolor{comment}{  *            @arg TIM\_ExtTRGPolarity\_Inverted: active low or falling edge active.}
02994 \textcolor{comment}{  *            @arg TIM\_ExtTRGPolarity\_NonInverted: active high or rising edge active.}
02995 \textcolor{comment}{  * @param  ExtTRGFilter: External Trigger Filter.}
02996 \textcolor{comment}{  *          This parameter must be a value between 0x00 and 0x0F}
02997 \textcolor{comment}{  * @retval None}
02998 \textcolor{comment}{  */}
02999 \textcolor{keywordtype}{void} TIM_ETRConfig(TIM\_TypeDef* TIMx, uint16\_t TIM\_ExtTRGPrescaler,
03000                    uint16\_t TIM\_ExtTRGPolarity, uint16\_t ExtTRGFilter)
03001 \{
03002   uint16\_t tmpsmcr = 0;
03003 
03004   \textcolor{comment}{/* Check the parameters */}
03005   assert_param(IS\_TIM\_LIST3\_PERIPH(TIMx));
03006   assert_param(IS\_TIM\_EXT\_PRESCALER(TIM\_ExtTRGPrescaler));
03007   assert_param(IS\_TIM\_EXT\_POLARITY(TIM\_ExtTRGPolarity));
03008   assert_param(IS\_TIM\_EXT\_FILTER(ExtTRGFilter));
03009 
03010   tmpsmcr = TIMx->SMCR;
03011 
03012   \textcolor{comment}{/* Reset the ETR Bits */}
03013   tmpsmcr &= SMCR_ETR_MASK;
03014 
03015   \textcolor{comment}{/* Set the Prescaler, the Filter value and the Polarity */}
03016   tmpsmcr |= (uint16\_t)(TIM\_ExtTRGPrescaler | (uint16\_t)(TIM\_ExtTRGPolarity | (uint16\_t)(ExtTRGFilter 
      << (uint16\_t)8)));
03017 
03018   \textcolor{comment}{/* Write to TIMx SMCR */}
03019   TIMx->SMCR = tmpsmcr;
03020 \}
03021 \textcolor{comment}{/**}
03022 \textcolor{comment}{  * @\}}
03023 \textcolor{comment}{  */}
03024 
03025 \textcolor{comment}{/** @defgroup TIM\_Group8 Specific interface management functions}
03026 \textcolor{comment}{ *  @brief    Specific interface management functions }
03027 \textcolor{comment}{ *}
03028 \textcolor{comment}{@verbatim   }
03029 \textcolor{comment}{ ===============================================================================}
03030 \textcolor{comment}{                    Specific interface management functions}
03031 \textcolor{comment}{ ===============================================================================  }
03032 \textcolor{comment}{}
03033 \textcolor{comment}{@endverbatim}
03034 \textcolor{comment}{  * @\{}
03035 \textcolor{comment}{  */}
03036 
03037 \textcolor{comment}{/**}
03038 \textcolor{comment}{  * @brief  Configures the TIMx Encoder Interface.}
03039 \textcolor{comment}{  * @param  TIMx: where x can be 1, 2, 3, 4, 5, 8, 9 or 12 to select the TIM }
03040 \textcolor{comment}{  *         peripheral.}
03041 \textcolor{comment}{  * @param  TIM\_EncoderMode: specifies the TIMx Encoder Mode.}
03042 \textcolor{comment}{  *          This parameter can be one of the following values:}
03043 \textcolor{comment}{  *            @arg TIM\_EncoderMode\_TI1: Counter counts on TI1FP1 edge depending on TI2FP2 level.}
03044 \textcolor{comment}{  *            @arg TIM\_EncoderMode\_TI2: Counter counts on TI2FP2 edge depending on TI1FP1 level.}
03045 \textcolor{comment}{  *            @arg TIM\_EncoderMode\_TI12: Counter counts on both TI1FP1 and TI2FP2 edges depending}
03046 \textcolor{comment}{  *                                       on the level of the other input.}
03047 \textcolor{comment}{  * @param  TIM\_IC1Polarity: specifies the IC1 Polarity}
03048 \textcolor{comment}{  *          This parameter can be one of the following values:}
03049 \textcolor{comment}{  *            @arg TIM\_ICPolarity\_Falling: IC Falling edge.}
03050 \textcolor{comment}{  *            @arg TIM\_ICPolarity\_Rising: IC Rising edge.}
03051 \textcolor{comment}{  * @param  TIM\_IC2Polarity: specifies the IC2 Polarity}
03052 \textcolor{comment}{  *          This parameter can be one of the following values:}
03053 \textcolor{comment}{  *            @arg TIM\_ICPolarity\_Falling: IC Falling edge.}
03054 \textcolor{comment}{  *            @arg TIM\_ICPolarity\_Rising: IC Rising edge.}
03055 \textcolor{comment}{  * @retval None}
03056 \textcolor{comment}{  */}
03057 \textcolor{keywordtype}{void} TIM_EncoderInterfaceConfig(TIM\_TypeDef* TIMx, uint16\_t TIM\_EncoderMode,
03058                                 uint16\_t TIM\_IC1Polarity, uint16\_t TIM\_IC2Polarity)
03059 \{
03060   uint16\_t tmpsmcr = 0;
03061   uint16\_t tmpccmr1 = 0;
03062   uint16\_t tmpccer = 0;
03063 
03064   \textcolor{comment}{/* Check the parameters */}
03065   assert_param(IS\_TIM\_LIST2\_PERIPH(TIMx));
03066   assert_param(IS\_TIM\_ENCODER\_MODE(TIM\_EncoderMode));
03067   assert_param(IS\_TIM\_IC\_POLARITY(TIM\_IC1Polarity));
03068   assert_param(IS\_TIM\_IC\_POLARITY(TIM\_IC2Polarity));
03069 
03070   \textcolor{comment}{/* Get the TIMx SMCR register value */}
03071   tmpsmcr = TIMx->SMCR;
03072 
03073   \textcolor{comment}{/* Get the TIMx CCMR1 register value */}
03074   tmpccmr1 = TIMx->CCMR1;
03075 
03076   \textcolor{comment}{/* Get the TIMx CCER register value */}
03077   tmpccer = TIMx->CCER;
03078 
03079   \textcolor{comment}{/* Set the encoder Mode */}
03080   tmpsmcr &= (uint16\_t)~TIM_SMCR_SMS;
03081   tmpsmcr |= TIM\_EncoderMode;
03082 
03083   \textcolor{comment}{/* Select the Capture Compare 1 and the Capture Compare 2 as input */}
03084   tmpccmr1 &= ((uint16\_t)~TIM_CCMR1_CC1S) & ((uint16\_t)~TIM_CCMR1_CC2S);
03085   tmpccmr1 |= TIM_CCMR1_CC1S_0 | TIM_CCMR1_CC2S_0;
03086 
03087   \textcolor{comment}{/* Set the TI1 and the TI2 Polarities */}
03088   tmpccer &= ((uint16\_t)~TIM_CCER_CC1P) & ((uint16\_t)~TIM_CCER_CC2P);
03089   tmpccer |= (uint16\_t)(TIM\_IC1Polarity | (uint16\_t)(TIM\_IC2Polarity << (uint16\_t)4));
03090 
03091   \textcolor{comment}{/* Write to TIMx SMCR */}
03092   TIMx->SMCR = tmpsmcr;
03093 
03094   \textcolor{comment}{/* Write to TIMx CCMR1 */}
03095   TIMx->CCMR1 = tmpccmr1;
03096 
03097   \textcolor{comment}{/* Write to TIMx CCER */}
03098   TIMx->CCER = tmpccer;
03099 \}
03100 
03101 \textcolor{comment}{/**}
03102 \textcolor{comment}{  * @brief  Enables or disables the TIMx's Hall sensor interface.}
03103 \textcolor{comment}{  * @param  TIMx: where x can be 1, 2, 3, 4, 5, 8, 9 or 12 to select the TIM }
03104 \textcolor{comment}{  *         peripheral.}
03105 \textcolor{comment}{  * @param  NewState: new state of the TIMx Hall sensor interface.}
03106 \textcolor{comment}{  *          This parameter can be: ENABLE or DISABLE.}
03107 \textcolor{comment}{  * @retval None}
03108 \textcolor{comment}{  */}
03109 \textcolor{keywordtype}{void} TIM_SelectHallSensor(TIM\_TypeDef* TIMx, FunctionalState NewState)
03110 \{
03111   \textcolor{comment}{/* Check the parameters */}
03112   assert_param(IS\_TIM\_LIST2\_PERIPH(TIMx));
03113   assert_param(IS\_FUNCTIONAL\_STATE(NewState));
03114 
03115   \textcolor{keywordflow}{if} (NewState != DISABLE)
03116   \{
03117     \textcolor{comment}{/* Set the TI1S Bit */}
03118     TIMx->CR2 |= TIM_CR2_TI1S;
03119   \}
03120   \textcolor{keywordflow}{else}
03121   \{
03122     \textcolor{comment}{/* Reset the TI1S Bit */}
03123     TIMx->CR2 &= (uint16\_t)~TIM_CR2_TI1S;
03124   \}
03125 \}
03126 \textcolor{comment}{/**}
03127 \textcolor{comment}{  * @\}}
03128 \textcolor{comment}{  */}
03129 
03130 \textcolor{comment}{/** @defgroup TIM\_Group9 Specific remapping management function}
03131 \textcolor{comment}{ *  @brief   Specific remapping management function}
03132 \textcolor{comment}{ *}
03133 \textcolor{comment}{@verbatim   }
03134 \textcolor{comment}{ ===============================================================================}
03135 \textcolor{comment}{                     Specific remapping management function}
03136 \textcolor{comment}{ ===============================================================================  }
03137 \textcolor{comment}{}
03138 \textcolor{comment}{@endverbatim}
03139 \textcolor{comment}{  * @\{}
03140 \textcolor{comment}{  */}
03141 
03142 \textcolor{comment}{/**}
03143 \textcolor{comment}{  * @brief  Configures the TIM2, TIM5 and TIM11 Remapping input capabilities.}
03144 \textcolor{comment}{  * @param  TIMx: where x can be 2, 5 or 11 to select the TIM peripheral.}
03145 \textcolor{comment}{  * @param  TIM\_Remap: specifies the TIM input remapping source.}
03146 \textcolor{comment}{  *          This parameter can be one of the following values:}
03147 \textcolor{comment}{  *            @arg TIM2\_TIM8\_TRGO: TIM2 ITR1 input is connected to TIM8 Trigger output(default)}
03148 \textcolor{comment}{  *            @arg TIM2\_ETH\_PTP:   TIM2 ITR1 input is connected to ETH PTP trogger output.}
03149 \textcolor{comment}{  *            @arg TIM2\_USBFS\_SOF: TIM2 ITR1 input is connected to USB FS SOF. }
03150 \textcolor{comment}{  *            @arg TIM2\_USBHS\_SOF: TIM2 ITR1 input is connected to USB HS SOF. }
03151 \textcolor{comment}{  *            @arg TIM5\_GPIO:      TIM5 CH4 input is connected to dedicated Timer pin(default)}
03152 \textcolor{comment}{  *            @arg TIM5\_LSI:       TIM5 CH4 input is connected to LSI clock.}
03153 \textcolor{comment}{  *            @arg TIM5\_LSE:       TIM5 CH4 input is connected to LSE clock.}
03154 \textcolor{comment}{  *            @arg TIM5\_RTC:       TIM5 CH4 input is connected to RTC Output event.}
03155 \textcolor{comment}{  *            @arg TIM11\_GPIO:     TIM11 CH4 input is connected to dedicated Timer pin(default) }
03156 \textcolor{comment}{  *            @arg TIM11\_HSE:      TIM11 CH4 input is connected to HSE\_RTC clock}
03157 \textcolor{comment}{  *                                 (HSE divided by a programmable prescaler)  }
03158 \textcolor{comment}{  * @retval None}
03159 \textcolor{comment}{  */}
03160 \textcolor{keywordtype}{void} TIM_RemapConfig(TIM\_TypeDef* TIMx, uint16\_t TIM\_Remap)
03161 \{
03162  \textcolor{comment}{/* Check the parameters */}
03163   assert_param(IS\_TIM\_LIST6\_PERIPH(TIMx));
03164   assert_param(IS\_TIM\_REMAP(TIM\_Remap));
03165 
03166   \textcolor{comment}{/* Set the Timer remapping configuration */}
03167   TIMx->OR =  TIM\_Remap;
03168 \}
03169 \textcolor{comment}{/**}
03170 \textcolor{comment}{  * @\}}
03171 \textcolor{comment}{  */}
03172 
03173 \textcolor{comment}{/**}
03174 \textcolor{comment}{  * @brief  Configure the TI1 as Input.}
03175 \textcolor{comment}{  * @param  TIMx: where x can be 1, 2, 3, 4, 5, 8, 9, 10, 11, 12, 13 or 14 }
03176 \textcolor{comment}{  *         to select the TIM peripheral.}
03177 \textcolor{comment}{  * @param  TIM\_ICPolarity : The Input Polarity.}
03178 \textcolor{comment}{  *          This parameter can be one of the following values:}
03179 \textcolor{comment}{  *            @arg TIM\_ICPolarity\_Rising}
03180 \textcolor{comment}{  *            @arg TIM\_ICPolarity\_Falling}
03181 \textcolor{comment}{  *            @arg TIM\_ICPolarity\_BothEdge  }
03182 \textcolor{comment}{  * @param  TIM\_ICSelection: specifies the input to be used.}
03183 \textcolor{comment}{  *          This parameter can be one of the following values:}
03184 \textcolor{comment}{  *            @arg TIM\_ICSelection\_DirectTI: TIM Input 1 is selected to be connected to IC1.}
03185 \textcolor{comment}{  *            @arg TIM\_ICSelection\_IndirectTI: TIM Input 1 is selected to be connected to IC2.}
03186 \textcolor{comment}{  *            @arg TIM\_ICSelection\_TRC: TIM Input 1 is selected to be connected to TRC.}
03187 \textcolor{comment}{  * @param  TIM\_ICFilter: Specifies the Input Capture Filter.}
03188 \textcolor{comment}{  *          This parameter must be a value between 0x00 and 0x0F.}
03189 \textcolor{comment}{  * @retval None}
03190 \textcolor{comment}{  */}
03191 \textcolor{keyword}{static} \textcolor{keywordtype}{void} TI1_Config(TIM\_TypeDef* TIMx, uint16\_t TIM\_ICPolarity, uint16\_t TIM\_ICSelection,
03192                        uint16\_t TIM\_ICFilter)
03193 \{
03194   uint16\_t tmpccmr1 = 0, tmpccer = 0;
03195 
03196   \textcolor{comment}{/* Disable the Channel 1: Reset the CC1E Bit */}
03197   TIMx->CCER &= (uint16\_t)~TIM_CCER_CC1E;
03198   tmpccmr1 = TIMx->CCMR1;
03199   tmpccer = TIMx->CCER;
03200 
03201   \textcolor{comment}{/* Select the Input and set the filter */}
03202   tmpccmr1 &= ((uint16\_t)~TIM_CCMR1_CC1S) & ((uint16\_t)~TIM_CCMR1_IC1F);
03203   tmpccmr1 |= (uint16\_t)(TIM\_ICSelection | (uint16\_t)(TIM\_ICFilter << (uint16\_t)4));
03204 
03205   \textcolor{comment}{/* Select the Polarity and set the CC1E Bit */}
03206   tmpccer &= (uint16\_t)~(TIM_CCER_CC1P | TIM_CCER_CC1NP);
03207   tmpccer |= (uint16\_t)(TIM\_ICPolarity | (uint16\_t)TIM_CCER_CC1E);
03208 
03209   \textcolor{comment}{/* Write to TIMx CCMR1 and CCER registers */}
03210   TIMx->CCMR1 = tmpccmr1;
03211   TIMx->CCER = tmpccer;
03212 \}
03213 
03214 \textcolor{comment}{/**}
03215 \textcolor{comment}{  * @brief  Configure the TI2 as Input.}
03216 \textcolor{comment}{  * @param  TIMx: where x can be 1, 2, 3, 4, 5, 8, 9 or 12 to select the TIM }
03217 \textcolor{comment}{  *         peripheral.}
03218 \textcolor{comment}{  * @param  TIM\_ICPolarity : The Input Polarity.}
03219 \textcolor{comment}{  *          This parameter can be one of the following values:}
03220 \textcolor{comment}{  *            @arg TIM\_ICPolarity\_Rising}
03221 \textcolor{comment}{  *            @arg TIM\_ICPolarity\_Falling}
03222 \textcolor{comment}{  *            @arg TIM\_ICPolarity\_BothEdge   }
03223 \textcolor{comment}{  * @param  TIM\_ICSelection: specifies the input to be used.}
03224 \textcolor{comment}{  *          This parameter can be one of the following values:}
03225 \textcolor{comment}{  *            @arg TIM\_ICSelection\_DirectTI: TIM Input 2 is selected to be connected to IC2.}
03226 \textcolor{comment}{  *            @arg TIM\_ICSelection\_IndirectTI: TIM Input 2 is selected to be connected to IC1.}
03227 \textcolor{comment}{  *            @arg TIM\_ICSelection\_TRC: TIM Input 2 is selected to be connected to TRC.}
03228 \textcolor{comment}{  * @param  TIM\_ICFilter: Specifies the Input Capture Filter.}
03229 \textcolor{comment}{  *          This parameter must be a value between 0x00 and 0x0F.}
03230 \textcolor{comment}{  * @retval None}
03231 \textcolor{comment}{  */}
03232 \textcolor{keyword}{static} \textcolor{keywordtype}{void} TI2_Config(TIM\_TypeDef* TIMx, uint16\_t TIM\_ICPolarity, uint16\_t TIM\_ICSelection,
03233                        uint16\_t TIM\_ICFilter)
03234 \{
03235   uint16\_t tmpccmr1 = 0, tmpccer = 0, tmp = 0;
03236 
03237   \textcolor{comment}{/* Disable the Channel 2: Reset the CC2E Bit */}
03238   TIMx->CCER &= (uint16\_t)~TIM_CCER_CC2E;
03239   tmpccmr1 = TIMx->CCMR1;
03240   tmpccer = TIMx->CCER;
03241   tmp = (uint16\_t)(TIM\_ICPolarity << 4);
03242 
03243   \textcolor{comment}{/* Select the Input and set the filter */}
03244   tmpccmr1 &= ((uint16\_t)~TIM_CCMR1_CC2S) & ((uint16\_t)~TIM_CCMR1_IC2F);
03245   tmpccmr1 |= (uint16\_t)(TIM\_ICFilter << 12);
03246   tmpccmr1 |= (uint16\_t)(TIM\_ICSelection << 8);
03247 
03248   \textcolor{comment}{/* Select the Polarity and set the CC2E Bit */}
03249   tmpccer &= (uint16\_t)~(TIM_CCER_CC2P | TIM_CCER_CC2NP);
03250   tmpccer |=  (uint16\_t)(tmp | (uint16\_t)TIM_CCER_CC2E);
03251 
03252   \textcolor{comment}{/* Write to TIMx CCMR1 and CCER registers */}
03253   TIMx->CCMR1 = tmpccmr1 ;
03254   TIMx->CCER = tmpccer;
03255 \}
03256 
03257 \textcolor{comment}{/**}
03258 \textcolor{comment}{  * @brief  Configure the TI3 as Input.}
03259 \textcolor{comment}{  * @param  TIMx: where x can be 1, 2, 3, 4, 5 or 8 to select the TIM peripheral.}
03260 \textcolor{comment}{  * @param  TIM\_ICPolarity : The Input Polarity.}
03261 \textcolor{comment}{  *          This parameter can be one of the following values:}
03262 \textcolor{comment}{  *            @arg TIM\_ICPolarity\_Rising}
03263 \textcolor{comment}{  *            @arg TIM\_ICPolarity\_Falling}
03264 \textcolor{comment}{  *            @arg TIM\_ICPolarity\_BothEdge         }
03265 \textcolor{comment}{  * @param  TIM\_ICSelection: specifies the input to be used.}
03266 \textcolor{comment}{  *          This parameter can be one of the following values:}
03267 \textcolor{comment}{  *            @arg TIM\_ICSelection\_DirectTI: TIM Input 3 is selected to be connected to IC3.}
03268 \textcolor{comment}{  *            @arg TIM\_ICSelection\_IndirectTI: TIM Input 3 is selected to be connected to IC4.}
03269 \textcolor{comment}{  *            @arg TIM\_ICSelection\_TRC: TIM Input 3 is selected to be connected to TRC.}
03270 \textcolor{comment}{  * @param  TIM\_ICFilter: Specifies the Input Capture Filter.}
03271 \textcolor{comment}{  *          This parameter must be a value between 0x00 and 0x0F.}
03272 \textcolor{comment}{  * @retval None}
03273 \textcolor{comment}{  */}
03274 \textcolor{keyword}{static} \textcolor{keywordtype}{void} TI3_Config(TIM\_TypeDef* TIMx, uint16\_t TIM\_ICPolarity, uint16\_t TIM\_ICSelection,
03275                        uint16\_t TIM\_ICFilter)
03276 \{
03277   uint16\_t tmpccmr2 = 0, tmpccer = 0, tmp = 0;
03278 
03279   \textcolor{comment}{/* Disable the Channel 3: Reset the CC3E Bit */}
03280   TIMx->CCER &= (uint16\_t)~TIM_CCER_CC3E;
03281   tmpccmr2 = TIMx->CCMR2;
03282   tmpccer = TIMx->CCER;
03283   tmp = (uint16\_t)(TIM\_ICPolarity << 8);
03284 
03285   \textcolor{comment}{/* Select the Input and set the filter */}
03286   tmpccmr2 &= ((uint16\_t)~TIM_CCMR1_CC1S) & ((uint16\_t)~TIM_CCMR2_IC3F);
03287   tmpccmr2 |= (uint16\_t)(TIM\_ICSelection | (uint16\_t)(TIM\_ICFilter << (uint16\_t)4));
03288 
03289   \textcolor{comment}{/* Select the Polarity and set the CC3E Bit */}
03290   tmpccer &= (uint16\_t)~(TIM_CCER_CC3P | TIM_CCER_CC3NP);
03291   tmpccer |= (uint16\_t)(tmp | (uint16\_t)TIM_CCER_CC3E);
03292 
03293   \textcolor{comment}{/* Write to TIMx CCMR2 and CCER registers */}
03294   TIMx->CCMR2 = tmpccmr2;
03295   TIMx->CCER = tmpccer;
03296 \}
03297 
03298 \textcolor{comment}{/**}
03299 \textcolor{comment}{  * @brief  Configure the TI4 as Input.}
03300 \textcolor{comment}{  * @param  TIMx: where x can be 1, 2, 3, 4, 5 or 8 to select the TIM peripheral.}
03301 \textcolor{comment}{  * @param  TIM\_ICPolarity : The Input Polarity.}
03302 \textcolor{comment}{  *          This parameter can be one of the following values:}
03303 \textcolor{comment}{  *            @arg TIM\_ICPolarity\_Rising}
03304 \textcolor{comment}{  *            @arg TIM\_ICPolarity\_Falling}
03305 \textcolor{comment}{  *            @arg TIM\_ICPolarity\_BothEdge     }
03306 \textcolor{comment}{  * @param  TIM\_ICSelection: specifies the input to be used.}
03307 \textcolor{comment}{  *          This parameter can be one of the following values:}
03308 \textcolor{comment}{  *            @arg TIM\_ICSelection\_DirectTI: TIM Input 4 is selected to be connected to IC4.}
03309 \textcolor{comment}{  *            @arg TIM\_ICSelection\_IndirectTI: TIM Input 4 is selected to be connected to IC3.}
03310 \textcolor{comment}{  *            @arg TIM\_ICSelection\_TRC: TIM Input 4 is selected to be connected to TRC.}
03311 \textcolor{comment}{  * @param  TIM\_ICFilter: Specifies the Input Capture Filter.}
03312 \textcolor{comment}{  *          This parameter must be a value between 0x00 and 0x0F.}
03313 \textcolor{comment}{  * @retval None}
03314 \textcolor{comment}{  */}
03315 \textcolor{keyword}{static} \textcolor{keywordtype}{void} TI4_Config(TIM\_TypeDef* TIMx, uint16\_t TIM\_ICPolarity, uint16\_t TIM\_ICSelection,
03316                        uint16\_t TIM\_ICFilter)
03317 \{
03318   uint16\_t tmpccmr2 = 0, tmpccer = 0, tmp = 0;
03319 
03320   \textcolor{comment}{/* Disable the Channel 4: Reset the CC4E Bit */}
03321   TIMx->CCER &= (uint16\_t)~TIM_CCER_CC4E;
03322   tmpccmr2 = TIMx->CCMR2;
03323   tmpccer = TIMx->CCER;
03324   tmp = (uint16\_t)(TIM\_ICPolarity << 12);
03325 
03326   \textcolor{comment}{/* Select the Input and set the filter */}
03327   tmpccmr2 &= ((uint16\_t)~TIM_CCMR1_CC2S) & ((uint16\_t)~TIM_CCMR1_IC2F);
03328   tmpccmr2 |= (uint16\_t)(TIM\_ICSelection << 8);
03329   tmpccmr2 |= (uint16\_t)(TIM\_ICFilter << 12);
03330 
03331   \textcolor{comment}{/* Select the Polarity and set the CC4E Bit */}
03332   tmpccer &= (uint16\_t)~(TIM_CCER_CC4P | TIM_CCER_CC4NP);
03333   tmpccer |= (uint16\_t)(tmp | (uint16\_t)TIM_CCER_CC4E);
03334 
03335   \textcolor{comment}{/* Write to TIMx CCMR2 and CCER registers */}
03336   TIMx->CCMR2 = tmpccmr2;
03337   TIMx->CCER = tmpccer ;
03338 \}
03339 
03340 \textcolor{comment}{/**}
03341 \textcolor{comment}{  * @\}}
03342 \textcolor{comment}{  */}
03343 
03344 \textcolor{comment}{/**}
03345 \textcolor{comment}{  * @\}}
03346 \textcolor{comment}{  */}
03347 
03348 \textcolor{comment}{/**}
03349 \textcolor{comment}{  * @\}}
03350 \textcolor{comment}{  */}
03351 
03352 \textcolor{comment}{/******************* (C) COPYRIGHT 2011 STMicroelectronics *****END OF FILE****/}
\end{DoxyCode}
