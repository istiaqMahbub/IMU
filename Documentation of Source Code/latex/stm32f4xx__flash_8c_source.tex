\section{stm32f4xx\+\_\+flash.\+c}
\label{stm32f4xx__flash_8c_source}\index{C\+:/\+Users/\+Md. Istiaq Mahbub/\+Desktop/\+I\+M\+U/\+M\+P\+U6050\+\_\+\+Motion\+Driver/\+S\+T\+M32\+F4xx\+\_\+\+Std\+Periph\+\_\+\+Driver/src/stm32f4xx\+\_\+flash.\+c@{C\+:/\+Users/\+Md. Istiaq Mahbub/\+Desktop/\+I\+M\+U/\+M\+P\+U6050\+\_\+\+Motion\+Driver/\+S\+T\+M32\+F4xx\+\_\+\+Std\+Periph\+\_\+\+Driver/src/stm32f4xx\+\_\+flash.\+c}}

\begin{DoxyCode}
00001 \textcolor{comment}{/**}
00002 \textcolor{comment}{  ******************************************************************************}
00003 \textcolor{comment}{  * @file    stm32f4xx\_flash.c}
00004 \textcolor{comment}{  * @author  MCD Application Team}
00005 \textcolor{comment}{  * @version V1.0.0}
00006 \textcolor{comment}{  * @date    30-September-2011}
00007 \textcolor{comment}{  * @brief   This file provides firmware functions to manage the following }
00008 \textcolor{comment}{  *          functionalities of the FLASH peripheral:}
00009 \textcolor{comment}{  *            - FLASH Interface configuration}
00010 \textcolor{comment}{  *            - FLASH Memory Programming}
00011 \textcolor{comment}{  *            - Option Bytes Programming}
00012 \textcolor{comment}{  *            - Interrupts and flags management}
00013 \textcolor{comment}{  *  }
00014 \textcolor{comment}{  *  @verbatim}
00015 \textcolor{comment}{  *  }
00016 \textcolor{comment}{  *          ===================================================================}
00017 \textcolor{comment}{  *                                 How to use this driver}
00018 \textcolor{comment}{  *          ===================================================================}
00019 \textcolor{comment}{  *                           }
00020 \textcolor{comment}{  *          This driver provides functions to configure and program the FLASH }
00021 \textcolor{comment}{  *          memory of all STM32F4xx devices.}
00022 \textcolor{comment}{  *          These functions are split in 4 groups:}
00023 \textcolor{comment}{  * }
00024 \textcolor{comment}{  *           1. FLASH Interface configuration functions: this group includes the}
00025 \textcolor{comment}{  *              management of the following features:}
00026 \textcolor{comment}{  *                    - Set the latency}
00027 \textcolor{comment}{  *                    - Enable/Disable the prefetch buffer}
00028 \textcolor{comment}{  *                    - Enable/Disable the Instruction cache and the Data cache}
00029 \textcolor{comment}{  *                    - Reset the Instruction cache and the Data cache}
00030 \textcolor{comment}{  *  }
00031 \textcolor{comment}{  *           2. FLASH Memory Programming functions: this group includes all needed}
00032 \textcolor{comment}{  *              functions to erase and program the main memory:}
00033 \textcolor{comment}{  *                    - Lock and Unlock the FLASH interface}
00034 \textcolor{comment}{  *                    - Erase function: Erase sector, erase all sectors}
00035 \textcolor{comment}{  *                    - Program functions: byte, half word, word and double word}
00036 \textcolor{comment}{  *  }
00037 \textcolor{comment}{  *           3. Option Bytes Programming functions: this group includes all needed}
00038 \textcolor{comment}{  *              functions to manage the Option Bytes:}
00039 \textcolor{comment}{  *                    - Set/Reset the write protection}
00040 \textcolor{comment}{  *                    - Set the Read protection Level}
00041 \textcolor{comment}{  *                    - Set the BOR level}
00042 \textcolor{comment}{  *                    - Program the user Option Bytes}
00043 \textcolor{comment}{  *                    - Launch the Option Bytes loader}
00044 \textcolor{comment}{  *  }
00045 \textcolor{comment}{  *           4. Interrupts and flags management functions: this group }
00046 \textcolor{comment}{  *              includes all needed functions to:}
00047 \textcolor{comment}{  *                    - Enable/Disable the FLASH interrupt sources}
00048 \textcolor{comment}{  *                    - Get flags status}
00049 \textcolor{comment}{  *                    - Clear flags}
00050 \textcolor{comment}{  *                    - Get FLASH operation status}
00051 \textcolor{comment}{  *                    - Wait for last FLASH operation}
00052 \textcolor{comment}{  * }
00053 \textcolor{comment}{  *  @endverbatim}
00054 \textcolor{comment}{  *                      }
00055 \textcolor{comment}{  ******************************************************************************}
00056 \textcolor{comment}{  * @attention}
00057 \textcolor{comment}{  *}
00058 \textcolor{comment}{  * THE PRESENT FIRMWARE WHICH IS FOR GUIDANCE ONLY AIMS AT PROVIDING CUSTOMERS}
00059 \textcolor{comment}{  * WITH CODING INFORMATION REGARDING THEIR PRODUCTS IN ORDER FOR THEM TO SAVE}
00060 \textcolor{comment}{  * TIME. AS A RESULT, STMICROELECTRONICS SHALL NOT BE HELD LIABLE FOR ANY}
00061 \textcolor{comment}{  * DIRECT, INDIRECT OR CONSEQUENTIAL DAMAGES WITH RESPECT TO ANY CLAIMS ARISING}
00062 \textcolor{comment}{  * FROM THE CONTENT OF SUCH FIRMWARE AND/OR THE USE MADE BY CUSTOMERS OF THE}
00063 \textcolor{comment}{  * CODING INFORMATION CONTAINED HEREIN IN CONNECTION WITH THEIR PRODUCTS.}
00064 \textcolor{comment}{  *}
00065 \textcolor{comment}{  * <h2><center>&copy; COPYRIGHT 2011 STMicroelectronics</center></h2>}
00066 \textcolor{comment}{  ******************************************************************************}
00067 \textcolor{comment}{  */}
00068 
00069 \textcolor{comment}{/* Includes ------------------------------------------------------------------*/}
00070 \textcolor{preprocessor}{#}\textcolor{preprocessor}{include} "stm32f4xx_flash.h"
00071 
00072 \textcolor{comment}{/** @addtogroup STM32F4xx\_StdPeriph\_Driver}
00073 \textcolor{comment}{  * @\{}
00074 \textcolor{comment}{  */}
00075 
00076 \textcolor{comment}{/** @defgroup FLASH }
00077 \textcolor{comment}{  * @brief FLASH driver modules}
00078 \textcolor{comment}{  * @\{}
00079 \textcolor{comment}{  */}
00080 
00081 \textcolor{comment}{/* Private typedef -----------------------------------------------------------*/}
00082 \textcolor{comment}{/* Private define ------------------------------------------------------------*/}
00083 \textcolor{preprocessor}{#}\textcolor{preprocessor}{define} \textcolor{preprocessor}{SECTOR\_MASK}               \textcolor{preprocessor}{(}\textcolor{preprocessor}{(}\textcolor{preprocessor}{uint32\_t}\textcolor{preprocessor}{)}0xFFFFFF07\textcolor{preprocessor}{)}
00084 
00085 \textcolor{comment}{/* Private macro -------------------------------------------------------------*/}
00086 \textcolor{comment}{/* Private variables ---------------------------------------------------------*/}
00087 \textcolor{comment}{/* Private function prototypes -----------------------------------------------*/}
00088 \textcolor{comment}{/* Private functions ---------------------------------------------------------*/}
00089 
00090 \textcolor{comment}{/** @defgroup FLASH\_Private\_Functions}
00091 \textcolor{comment}{  * @\{}
00092 \textcolor{comment}{  */}
00093 
00094 \textcolor{comment}{/** @defgroup FLASH\_Group1 FLASH Interface configuration functions}
00095 \textcolor{comment}{  *  @brief   FLASH Interface configuration functions }
00096 \textcolor{comment}{ *}
00097 \textcolor{comment}{}
00098 \textcolor{comment}{@verbatim   }
00099 \textcolor{comment}{ ===============================================================================}
00100 \textcolor{comment}{                       FLASH Interface configuration functions}
00101 \textcolor{comment}{ ===============================================================================}
00102 \textcolor{comment}{}
00103 \textcolor{comment}{   This group includes the following functions:}
00104 \textcolor{comment}{    - void FLASH\_SetLatency(uint32\_t FLASH\_Latency)}
00105 \textcolor{comment}{       To correctly read data from FLASH memory, the number of wait states (LATENCY) }
00106 \textcolor{comment}{       must be correctly programmed according to the frequency of the CPU clock }
00107 \textcolor{comment}{      (HCLK) and the supply voltage of the device.}
00108 \textcolor{comment}{ +-------------------------------------------------------------------------------------+     }
00109 \textcolor{comment}{ | Latency       |                HCLK clock frequency (MHz)                           |}
00110 \textcolor{comment}{ |               |---------------------------------------------------------------------|     }
00111 \textcolor{comment}{ |               | voltage range  | voltage range  | voltage range   | voltage range   |}
00112 \textcolor{comment}{ |               | 2.7 V - 3.6 V  | 2.4 V - 2.7 V  | 2.1 V - 2.4 V   | 1.8 V - 2.1 V   |}
00113 \textcolor{comment}{ |---------------|----------------|----------------|-----------------|-----------------|              }
00114 \textcolor{comment}{ |0WS(1CPU cycle)|0 < HCLK <= 30  |0 < HCLK <= 24  |0 < HCLK <= 18   |0 < HCLK <= 16   |}
00115 \textcolor{comment}{ |---------------|----------------|----------------|-----------------|-----------------|   }
00116 \textcolor{comment}{ |1WS(2CPU cycle)|30 < HCLK <= 60 |24 < HCLK <= 48 |18 < HCLK <= 36  |16 < HCLK <= 32  | }
00117 \textcolor{comment}{ |---------------|----------------|----------------|-----------------|-----------------|   }
00118 \textcolor{comment}{ |2WS(3CPU cycle)|60 < HCLK <= 90 |48 < HCLK <= 72 |36 < HCLK <= 54  |32 < HCLK <= 48  |}
00119 \textcolor{comment}{ |---------------|----------------|----------------|-----------------|-----------------| }
00120 \textcolor{comment}{ |3WS(4CPU cycle)|90 < HCLK <= 120|72 < HCLK <= 96 |54 < HCLK <= 72  |48 < HCLK <= 64  |}
00121 \textcolor{comment}{ |---------------|----------------|----------------|-----------------|-----------------| }
00122 \textcolor{comment}{ |4WS(5CPU cycle)|120< HCLK <= 150|96 < HCLK <= 120|72 < HCLK <= 90  |64 < HCLK <= 80  |}
00123 \textcolor{comment}{ |---------------|----------------|----------------|-----------------|-----------------| }
00124 \textcolor{comment}{ |5WS(6CPU cycle)|120< HCLK <= 168|120< HCLK <= 144|90 < HCLK <= 108 |80 < HCLK <= 96  | }
00125 \textcolor{comment}{ |---------------|----------------|----------------|-----------------|-----------------| }
00126 \textcolor{comment}{ |6WS(7CPU cycle)|      NA        |144< HCLK <= 168|108 < HCLK <= 120|96 < HCLK <= 112 | }
00127 \textcolor{comment}{ |---------------|----------------|----------------|-----------------|-----------------| }
00128 \textcolor{comment}{ |7WS(8CPU cycle)|      NA        |      NA        |120 < HCLK <= 138|112 < HCLK <= 120| }
00129 \textcolor{comment}{
       |***************|****************|****************|*****************|*****************|*****************************+}
00130 \textcolor{comment}{ |               | voltage range  | voltage range  | voltage range   | voltage range   | voltage range
       2.7 V - 3.6 V |}
00131 \textcolor{comment}{ |               | 2.7 V - 3.6 V  | 2.4 V - 2.7 V  | 2.1 V - 2.4 V   | 1.8 V - 2.1 V   | with External
       Vpp = 9V      |}
00132 \textcolor{comment}{
       |---------------|----------------|----------------|-----------------|-----------------|-----------------------------| }
00133 \textcolor{comment}{ |Max Parallelism|      x32       |               x16                |       x8        |          x64 
                     |              }
00134 \textcolor{comment}{
       |---------------|----------------|----------------|-----------------|-----------------|-----------------------------|   }
00135 \textcolor{comment}{ |PSIZE[1:0]     |      10        |               01                 |       00        |           11 
                     |}
00136 \textcolor{comment}{
       +-------------------------------------------------------------------------------------------------------------------+  }
00137 \textcolor{comment}{   @note When VOS bit (in PWR\_CR register) is reset to '0�, the maximum value of HCLK is 144 MHz.}
00138 \textcolor{comment}{         You can use PWR\_MainRegulatorModeConfig() function to set or reset this bit.}
00139 \textcolor{comment}{             }
00140 \textcolor{comment}{    - void FLASH\_PrefetchBufferCmd(FunctionalState NewState)}
00141 \textcolor{comment}{    - void FLASH\_InstructionCacheCmd(FunctionalState NewState)}
00142 \textcolor{comment}{    - void FLASH\_DataCacheCmd(FunctionalState NewState)}
00143 \textcolor{comment}{    - void FLASH\_InstructionCacheReset(void)}
00144 \textcolor{comment}{    - void FLASH\_DataCacheReset(void)}
00145 \textcolor{comment}{   }
00146 \textcolor{comment}{   The unlock sequence is not needed for these functions.}
00147 \textcolor{comment}{ }
00148 \textcolor{comment}{@endverbatim}
00149 \textcolor{comment}{  * @\{}
00150 \textcolor{comment}{  */}
00151 
00152 \textcolor{comment}{/**}
00153 \textcolor{comment}{  * @brief  Sets the code latency value.}
00154 \textcolor{comment}{  * @param  FLASH\_Latency: specifies the FLASH Latency value.}
00155 \textcolor{comment}{  *          This parameter can be one of the following values:}
00156 \textcolor{comment}{  *            @arg FLASH\_Latency\_0: FLASH Zero Latency cycle}
00157 \textcolor{comment}{  *            @arg FLASH\_Latency\_1: FLASH One Latency cycle}
00158 \textcolor{comment}{  *            @arg FLASH\_Latency\_2: FLASH Two Latency cycles}
00159 \textcolor{comment}{  *            @arg FLASH\_Latency\_3: FLASH Three Latency cycles}
00160 \textcolor{comment}{  *            @arg FLASH\_Latency\_4: FLASH Four Latency cycles }
00161 \textcolor{comment}{  *            @arg FLASH\_Latency\_5: FLASH Five Latency cycles }
00162 \textcolor{comment}{  *            @arg FLASH\_Latency\_6: FLASH Six Latency cycles}
00163 \textcolor{comment}{  *            @arg FLASH\_Latency\_7: FLASH Seven Latency cycles      }
00164 \textcolor{comment}{  * @retval None}
00165 \textcolor{comment}{  */}
00166 \textcolor{keywordtype}{void} FLASH_SetLatency(uint32\_t FLASH\_Latency)
00167 \{
00168   \textcolor{comment}{/* Check the parameters */}
00169   assert_param(IS\_FLASH\_LATENCY(FLASH\_Latency));
00170 
00171   \textcolor{comment}{/* Perform Byte access to FLASH\_ACR[8:0] to set the Latency value */}
00172   *(\_\_IO uint8\_t *)ACR_BYTE0_ADDRESS = (uint8\_t)FLASH\_Latency;
00173 \}
00174 
00175 \textcolor{comment}{/**}
00176 \textcolor{comment}{  * @brief  Enables or disables the Prefetch Buffer.}
00177 \textcolor{comment}{  * @param  NewState: new state of the Prefetch Buffer.}
00178 \textcolor{comment}{  *          This parameter  can be: ENABLE or DISABLE.}
00179 \textcolor{comment}{  * @retval None}
00180 \textcolor{comment}{  */}
00181 \textcolor{keywordtype}{void} FLASH_PrefetchBufferCmd(FunctionalState NewState)
00182 \{
00183   \textcolor{comment}{/* Check the parameters */}
00184   assert_param(IS\_FUNCTIONAL\_STATE(NewState));
00185 
00186   \textcolor{comment}{/* Enable or disable the Prefetch Buffer */}
00187   \textcolor{keywordflow}{if}(NewState != DISABLE)
00188   \{
00189     FLASH->ACR |= FLASH_ACR_PRFTEN;
00190   \}
00191   \textcolor{keywordflow}{else}
00192   \{
00193     FLASH->ACR &= (~FLASH_ACR_PRFTEN);
00194   \}
00195 \}
00196 
00197 \textcolor{comment}{/**}
00198 \textcolor{comment}{  * @brief  Enables or disables the Instruction Cache feature.}
00199 \textcolor{comment}{  * @param  NewState: new state of the Instruction Cache.}
00200 \textcolor{comment}{  *          This parameter  can be: ENABLE or DISABLE.}
00201 \textcolor{comment}{  * @retval None}
00202 \textcolor{comment}{  */}
00203 \textcolor{keywordtype}{void} FLASH_InstructionCacheCmd(FunctionalState NewState)
00204 \{
00205   \textcolor{comment}{/* Check the parameters */}
00206   assert_param(IS\_FUNCTIONAL\_STATE(NewState));
00207 
00208   \textcolor{keywordflow}{if}(NewState != DISABLE)
00209   \{
00210     FLASH->ACR |= FLASH_ACR_ICEN;
00211   \}
00212   \textcolor{keywordflow}{else}
00213   \{
00214     FLASH->ACR &= (~FLASH_ACR_ICEN);
00215   \}
00216 \}
00217 
00218 \textcolor{comment}{/**}
00219 \textcolor{comment}{  * @brief  Enables or disables the Data Cache feature.}
00220 \textcolor{comment}{  * @param  NewState: new state of the Data Cache.}
00221 \textcolor{comment}{  *          This parameter  can be: ENABLE or DISABLE.}
00222 \textcolor{comment}{  * @retval None}
00223 \textcolor{comment}{  */}
00224 \textcolor{keywordtype}{void} FLASH_DataCacheCmd(FunctionalState NewState)
00225 \{
00226   \textcolor{comment}{/* Check the parameters */}
00227   assert_param(IS\_FUNCTIONAL\_STATE(NewState));
00228 
00229   \textcolor{keywordflow}{if}(NewState != DISABLE)
00230   \{
00231     FLASH->ACR |= FLASH_ACR_DCEN;
00232   \}
00233   \textcolor{keywordflow}{else}
00234   \{
00235     FLASH->ACR &= (~FLASH_ACR_DCEN);
00236   \}
00237 \}
00238 
00239 \textcolor{comment}{/**}
00240 \textcolor{comment}{  * @brief  Resets the Instruction Cache.}
00241 \textcolor{comment}{  * @note   This function must be used only when the Instruction Cache is disabled.  }
00242 \textcolor{comment}{  * @param  None}
00243 \textcolor{comment}{  * @retval None}
00244 \textcolor{comment}{  */}
00245 \textcolor{keywordtype}{void} FLASH_InstructionCacheReset(\textcolor{keywordtype}{void})
00246 \{
00247   FLASH->ACR |= FLASH_ACR_ICRST;
00248 \}
00249 
00250 \textcolor{comment}{/**}
00251 \textcolor{comment}{  * @brief  Resets the Data Cache.}
00252 \textcolor{comment}{  * @note   This function must be used only when the Data Cache is disabled.  }
00253 \textcolor{comment}{  * @param  None}
00254 \textcolor{comment}{  * @retval None}
00255 \textcolor{comment}{  */}
00256 \textcolor{keywordtype}{void} FLASH_DataCacheReset(\textcolor{keywordtype}{void})
00257 \{
00258   FLASH->ACR |= FLASH_ACR_DCRST;
00259 \}
00260 
00261 \textcolor{comment}{/**}
00262 \textcolor{comment}{  * @\}}
00263 \textcolor{comment}{  */}
00264 
00265 \textcolor{comment}{/** @defgroup FLASH\_Group2 FLASH Memory Programming functions}
00266 \textcolor{comment}{ *  @brief   FLASH Memory Programming functions}
00267 \textcolor{comment}{ *}
00268 \textcolor{comment}{@verbatim   }
00269 \textcolor{comment}{ ===============================================================================}
00270 \textcolor{comment}{                      FLASH Memory Programming functions}
00271 \textcolor{comment}{ ===============================================================================   }
00272 \textcolor{comment}{}
00273 \textcolor{comment}{   This group includes the following functions:}
00274 \textcolor{comment}{    - void FLASH\_Unlock(void)}
00275 \textcolor{comment}{    - void FLASH\_Lock(void)}
00276 \textcolor{comment}{    - FLASH\_Status FLASH\_EraseSector(uint32\_t FLASH\_Sector, uint8\_t VoltageRange)}
00277 \textcolor{comment}{    - FLASH\_Status FLASH\_EraseAllSectors(uint8\_t VoltageRange)}
00278 \textcolor{comment}{    - FLASH\_Status FLASH\_ProgramDoubleWord(uint32\_t Address, uint64\_t Data)}
00279 \textcolor{comment}{    - FLASH\_Status FLASH\_ProgramWord(uint32\_t Address, uint32\_t Data)}
00280 \textcolor{comment}{    - FLASH\_Status FLASH\_ProgramHalfWord(uint32\_t Address, uint16\_t Data)}
00281 \textcolor{comment}{    - FLASH\_Status FLASH\_ProgramByte(uint32\_t Address, uint8\_t Data)}
00282 \textcolor{comment}{   }
00283 \textcolor{comment}{   Any operation of erase or program should follow these steps:}
00284 \textcolor{comment}{   1. Call the FLASH\_Unlock() function to enable the FLASH control register access}
00285 \textcolor{comment}{}
00286 \textcolor{comment}{   2. Call the desired function to erase sector(s) or program data}
00287 \textcolor{comment}{}
00288 \textcolor{comment}{   3. Call the FLASH\_Lock() function to disable the FLASH control register access}
00289 \textcolor{comment}{      (recommended to protect the FLASH memory against possible unwanted operation)}
00290 \textcolor{comment}{    }
00291 \textcolor{comment}{@endverbatim}
00292 \textcolor{comment}{  * @\{}
00293 \textcolor{comment}{  */}
00294 
00295 \textcolor{comment}{/**}
00296 \textcolor{comment}{  * @brief  Unlocks the FLASH control register access}
00297 \textcolor{comment}{  * @param  None}
00298 \textcolor{comment}{  * @retval None}
00299 \textcolor{comment}{  */}
00300 \textcolor{keywordtype}{void} FLASH_Unlock(\textcolor{keywordtype}{void})
00301 \{
00302   \textcolor{keywordflow}{if}((FLASH->CR & FLASH_CR_LOCK) != RESET)
00303   \{
00304     \textcolor{comment}{/* Authorize the FLASH Registers access */}
00305     FLASH->KEYR = FLASH_KEY1;
00306     FLASH->KEYR = FLASH_KEY2;
00307   \}
00308 \}
00309 
00310 \textcolor{comment}{/**}
00311 \textcolor{comment}{  * @brief  Locks the FLASH control register access}
00312 \textcolor{comment}{  * @param  None}
00313 \textcolor{comment}{  * @retval None}
00314 \textcolor{comment}{  */}
00315 \textcolor{keywordtype}{void} FLASH_Lock(\textcolor{keywordtype}{void})
00316 \{
00317   \textcolor{comment}{/* Set the LOCK Bit to lock the FLASH Registers access */}
00318   FLASH->CR |= FLASH_CR_LOCK;
00319 \}
00320 
00321 \textcolor{comment}{/**}
00322 \textcolor{comment}{  * @brief  Erases a specified FLASH Sector.}
00323 \textcolor{comment}{  *   }
00324 \textcolor{comment}{  * @param  FLASH\_Sector: The Sector number to be erased.}
00325 \textcolor{comment}{  *          This parameter can be a value between FLASH\_Sector\_0 and FLASH\_Sector\_11}
00326 \textcolor{comment}{  *    }
00327 \textcolor{comment}{  * @param  VoltageRange: The device voltage range which defines the erase parallelism.  }
00328 \textcolor{comment}{  *          This parameter can be one of the following values:}
00329 \textcolor{comment}{  *            @arg VoltageRange\_1: when the device voltage range is 1.8V to 2.1V, }
00330 \textcolor{comment}{  *                                  the operation will be done by byte (8-bit) }
00331 \textcolor{comment}{  *            @arg VoltageRange\_2: when the device voltage range is 2.1V to 2.7V,}
00332 \textcolor{comment}{  *                                  the operation will be done by half word (16-bit)}
00333 \textcolor{comment}{  *            @arg VoltageRange\_3: when the device voltage range is 2.7V to 3.6V,}
00334 \textcolor{comment}{  *                                  the operation will be done by word (32-bit)}
00335 \textcolor{comment}{  *            @arg VoltageRange\_4: when the device voltage range is 2.7V to 3.6V + External Vpp, }
00336 \textcolor{comment}{  *                                  the operation will be done by double word (64-bit)}
00337 \textcolor{comment}{  *       }
00338 \textcolor{comment}{  * @retval FLASH Status: The returned value can be: FLASH\_BUSY, FLASH\_ERROR\_PROGRAM,}
00339 \textcolor{comment}{  *                       FLASH\_ERROR\_WRP, FLASH\_ERROR\_OPERATION or FLASH\_COMPLETE.}
00340 \textcolor{comment}{  */}
00341 FLASH\_Status FLASH_EraseSector(uint32\_t FLASH\_Sector, uint8\_t VoltageRange)
00342 \{
00343   uint32\_t tmp\_psize = 0x0;
00344   FLASH\_Status status = FLASH_COMPLETE;
00345 
00346   \textcolor{comment}{/* Check the parameters */}
00347   assert_param(IS\_FLASH\_SECTOR(FLASH\_Sector));
00348   assert_param(IS\_VOLTAGERANGE(VoltageRange));
00349 
00350   \textcolor{keywordflow}{if}(VoltageRange == VoltageRange_1)
00351   \{
00352      tmp\_psize = FLASH_PSIZE_BYTE;
00353   \}
00354   \textcolor{keywordflow}{else} \textcolor{keywordflow}{if}(VoltageRange == VoltageRange_2)
00355   \{
00356     tmp\_psize = FLASH_PSIZE_HALF_WORD;
00357   \}
00358   \textcolor{keywordflow}{else} \textcolor{keywordflow}{if}(VoltageRange == VoltageRange_3)
00359   \{
00360     tmp\_psize = FLASH_PSIZE_WORD;
00361   \}
00362   \textcolor{keywordflow}{else}
00363   \{
00364     tmp\_psize = FLASH_PSIZE_DOUBLE_WORD;
00365   \}
00366   \textcolor{comment}{/* Wait for last operation to be completed */}
00367   status = FLASH_WaitForLastOperation();
00368 
00369   \textcolor{keywordflow}{if}(status == FLASH_COMPLETE)
00370   \{
00371     \textcolor{comment}{/* if the previous operation is completed, proceed to erase the sector */}
00372     FLASH->CR &= CR_PSIZE_MASK;
00373     FLASH->CR |= tmp\_psize;
00374     FLASH->CR &= SECTOR_MASK;
00375     FLASH->CR |= FLASH_CR_SER | FLASH\_Sector;
00376     FLASH->CR |= FLASH_CR_STRT;
00377 
00378     \textcolor{comment}{/* Wait for last operation to be completed */}
00379     status = FLASH_WaitForLastOperation();
00380 
00381     \textcolor{comment}{/* if the erase operation is completed, disable the SER Bit */}
00382     FLASH->CR &= (~FLASH_CR_SER);
00383     FLASH->CR &= SECTOR_MASK;
00384   \}
00385   \textcolor{comment}{/* Return the Erase Status */}
00386   \textcolor{keywordflow}{return} status;
00387 \}
00388 
00389 \textcolor{comment}{/**}
00390 \textcolor{comment}{  * @brief  Erases all FLASH Sectors.}
00391 \textcolor{comment}{  *    }
00392 \textcolor{comment}{  * @param  VoltageRange: The device voltage range which defines the erase parallelism.  }
00393 \textcolor{comment}{  *          This parameter can be one of the following values:}
00394 \textcolor{comment}{  *            @arg VoltageRange\_1: when the device voltage range is 1.8V to 2.1V, }
00395 \textcolor{comment}{  *                                  the operation will be done by byte (8-bit) }
00396 \textcolor{comment}{  *            @arg VoltageRange\_2: when the device voltage range is 2.1V to 2.7V,}
00397 \textcolor{comment}{  *                                  the operation will be done by half word (16-bit)}
00398 \textcolor{comment}{  *            @arg VoltageRange\_3: when the device voltage range is 2.7V to 3.6V,}
00399 \textcolor{comment}{  *                                  the operation will be done by word (32-bit)}
00400 \textcolor{comment}{  *            @arg VoltageRange\_4: when the device voltage range is 2.7V to 3.6V + External Vpp, }
00401 \textcolor{comment}{  *                                  the operation will be done by double word (64-bit)}
00402 \textcolor{comment}{  *       }
00403 \textcolor{comment}{  * @retval FLASH Status: The returned value can be: FLASH\_BUSY, FLASH\_ERROR\_PROGRAM,}
00404 \textcolor{comment}{  *                       FLASH\_ERROR\_WRP, FLASH\_ERROR\_OPERATION or FLASH\_COMPLETE.}
00405 \textcolor{comment}{  */}
00406 FLASH\_Status FLASH_EraseAllSectors(uint8\_t VoltageRange)
00407 \{
00408   uint32\_t tmp\_psize = 0x0;
00409   FLASH\_Status status = FLASH_COMPLETE;
00410 
00411   \textcolor{comment}{/* Wait for last operation to be completed */}
00412   status = FLASH_WaitForLastOperation();
00413   assert_param(IS\_VOLTAGERANGE(VoltageRange));
00414 
00415   \textcolor{keywordflow}{if}(VoltageRange == VoltageRange_1)
00416   \{
00417      tmp\_psize = FLASH_PSIZE_BYTE;
00418   \}
00419   \textcolor{keywordflow}{else} \textcolor{keywordflow}{if}(VoltageRange == VoltageRange_2)
00420   \{
00421     tmp\_psize = FLASH_PSIZE_HALF_WORD;
00422   \}
00423   \textcolor{keywordflow}{else} \textcolor{keywordflow}{if}(VoltageRange == VoltageRange_3)
00424   \{
00425     tmp\_psize = FLASH_PSIZE_WORD;
00426   \}
00427   \textcolor{keywordflow}{else}
00428   \{
00429     tmp\_psize = FLASH_PSIZE_DOUBLE_WORD;
00430   \}
00431   \textcolor{keywordflow}{if}(status == FLASH_COMPLETE)
00432   \{
00433     \textcolor{comment}{/* if the previous operation is completed, proceed to erase all sectors */}
00434      FLASH->CR &= CR_PSIZE_MASK;
00435      FLASH->CR |= tmp\_psize;
00436      FLASH->CR |= FLASH_CR_MER;
00437      FLASH->CR |= FLASH_CR_STRT;
00438 
00439     \textcolor{comment}{/* Wait for last operation to be completed */}
00440     status = FLASH_WaitForLastOperation();
00441 
00442     \textcolor{comment}{/* if the erase operation is completed, disable the MER Bit */}
00443     FLASH->CR &= (~FLASH_CR_MER);
00444 
00445   \}
00446   \textcolor{comment}{/* Return the Erase Status */}
00447   \textcolor{keywordflow}{return} status;
00448 \}
00449 
00450 \textcolor{comment}{/**}
00451 \textcolor{comment}{  * @brief  Programs a double word (64-bit) at a specified address.}
00452 \textcolor{comment}{  * @note   This function must be used when the device voltage range is from}
00453 \textcolor{comment}{  *         2.7V to 3.6V and an External Vpp is present.           }
00454 \textcolor{comment}{  * @param  Address: specifies the address to be programmed.}
00455 \textcolor{comment}{  * @param  Data: specifies the data to be programmed.}
00456 \textcolor{comment}{  * @retval FLASH Status: The returned value can be: FLASH\_BUSY, FLASH\_ERROR\_PROGRAM,}
00457 \textcolor{comment}{  *                       FLASH\_ERROR\_WRP, FLASH\_ERROR\_OPERATION or FLASH\_COMPLETE.}
00458 \textcolor{comment}{  */}
00459 FLASH\_Status FLASH_ProgramDoubleWord(uint32\_t Address, uint64\_t Data)
00460 \{
00461   FLASH\_Status status = FLASH_COMPLETE;
00462 
00463   \textcolor{comment}{/* Check the parameters */}
00464   assert_param(IS\_FLASH\_ADDRESS(Address));
00465 
00466   \textcolor{comment}{/* Wait for last operation to be completed */}
00467   status = FLASH_WaitForLastOperation();
00468 
00469   \textcolor{keywordflow}{if}(status == FLASH_COMPLETE)
00470   \{
00471     \textcolor{comment}{/* if the previous operation is completed, proceed to program the new data */}
00472     FLASH->CR &= CR_PSIZE_MASK;
00473     FLASH->CR |= FLASH_PSIZE_DOUBLE_WORD;
00474     FLASH->CR |= FLASH_CR_PG;
00475 
00476     *(\_\_IO uint64\_t*)Address = Data;
00477 
00478     \textcolor{comment}{/* Wait for last operation to be completed */}
00479     status = FLASH_WaitForLastOperation();
00480 
00481     \textcolor{comment}{/* if the program operation is completed, disable the PG Bit */}
00482     FLASH->CR &= (~FLASH_CR_PG);
00483   \}
00484   \textcolor{comment}{/* Return the Program Status */}
00485   \textcolor{keywordflow}{return} status;
00486 \}
00487 
00488 \textcolor{comment}{/**}
00489 \textcolor{comment}{  * @brief  Programs a word (32-bit) at a specified address.}
00490 \textcolor{comment}{  * @param  Address: specifies the address to be programmed.}
00491 \textcolor{comment}{  *         This parameter can be any address in Program memory zone or in OTP zone.  }
00492 \textcolor{comment}{  * @note   This function must be used when the device voltage range is from 2.7V to 3.6V. }
00493 \textcolor{comment}{  * @param  Data: specifies the data to be programmed.}
00494 \textcolor{comment}{  * @retval FLASH Status: The returned value can be: FLASH\_BUSY, FLASH\_ERROR\_PROGRAM,}
00495 \textcolor{comment}{  *                       FLASH\_ERROR\_WRP, FLASH\_ERROR\_OPERATION or FLASH\_COMPLETE.}
00496 \textcolor{comment}{  */}
00497 FLASH\_Status FLASH_ProgramWord(uint32\_t Address, uint32\_t Data)
00498 \{
00499   FLASH\_Status status = FLASH_COMPLETE;
00500 
00501   \textcolor{comment}{/* Check the parameters */}
00502   assert_param(IS\_FLASH\_ADDRESS(Address));
00503 
00504   \textcolor{comment}{/* Wait for last operation to be completed */}
00505   status = FLASH_WaitForLastOperation();
00506 
00507   \textcolor{keywordflow}{if}(status == FLASH_COMPLETE)
00508   \{
00509     \textcolor{comment}{/* if the previous operation is completed, proceed to program the new data */}
00510     FLASH->CR &= CR_PSIZE_MASK;
00511     FLASH->CR |= FLASH_PSIZE_WORD;
00512     FLASH->CR |= FLASH_CR_PG;
00513 
00514     *(\_\_IO uint32\_t*)Address = Data;
00515 
00516     \textcolor{comment}{/* Wait for last operation to be completed */}
00517     status = FLASH_WaitForLastOperation();
00518 
00519     \textcolor{comment}{/* if the program operation is completed, disable the PG Bit */}
00520     FLASH->CR &= (~FLASH_CR_PG);
00521   \}
00522   \textcolor{comment}{/* Return the Program Status */}
00523   \textcolor{keywordflow}{return} status;
00524 \}
00525 
00526 \textcolor{comment}{/**}
00527 \textcolor{comment}{  * @brief  Programs a half word (16-bit) at a specified address. }
00528 \textcolor{comment}{  * @note   This function must be used when the device voltage range is from 2.1V to 3.6V.            
         }
00529 \textcolor{comment}{  * @param  Address: specifies the address to be programmed.}
00530 \textcolor{comment}{  *         This parameter can be any address in Program memory zone or in OTP zone.  }
00531 \textcolor{comment}{  * @param  Data: specifies the data to be programmed.}
00532 \textcolor{comment}{  * @retval FLASH Status: The returned value can be: FLASH\_BUSY, FLASH\_ERROR\_PROGRAM,}
00533 \textcolor{comment}{  *                       FLASH\_ERROR\_WRP, FLASH\_ERROR\_OPERATION or FLASH\_COMPLETE.}
00534 \textcolor{comment}{  */}
00535 FLASH\_Status FLASH_ProgramHalfWord(uint32\_t Address, uint16\_t Data)
00536 \{
00537   FLASH\_Status status = FLASH_COMPLETE;
00538 
00539   \textcolor{comment}{/* Check the parameters */}
00540   assert_param(IS\_FLASH\_ADDRESS(Address));
00541 
00542   \textcolor{comment}{/* Wait for last operation to be completed */}
00543   status = FLASH_WaitForLastOperation();
00544 
00545   \textcolor{keywordflow}{if}(status == FLASH_COMPLETE)
00546   \{
00547     \textcolor{comment}{/* if the previous operation is completed, proceed to program the new data */}
00548     FLASH->CR &= CR_PSIZE_MASK;
00549     FLASH->CR |= FLASH_PSIZE_HALF_WORD;
00550     FLASH->CR |= FLASH_CR_PG;
00551 
00552     *(\_\_IO uint16\_t*)Address = Data;
00553 
00554     \textcolor{comment}{/* Wait for last operation to be completed */}
00555     status = FLASH_WaitForLastOperation();
00556 
00557     \textcolor{comment}{/* if the program operation is completed, disable the PG Bit */}
00558     FLASH->CR &= (~FLASH_CR_PG);
00559   \}
00560   \textcolor{comment}{/* Return the Program Status */}
00561   \textcolor{keywordflow}{return} status;
00562 \}
00563 
00564 \textcolor{comment}{/**}
00565 \textcolor{comment}{  * @brief  Programs a byte (8-bit) at a specified address.}
00566 \textcolor{comment}{  * @note   This function can be used within all the device supply voltage ranges.               }
00567 \textcolor{comment}{  * @param  Address: specifies the address to be programmed.}
00568 \textcolor{comment}{  *         This parameter can be any address in Program memory zone or in OTP zone.  }
00569 \textcolor{comment}{  * @param  Data: specifies the data to be programmed.}
00570 \textcolor{comment}{  * @retval FLASH Status: The returned value can be: FLASH\_BUSY, FLASH\_ERROR\_PROGRAM,}
00571 \textcolor{comment}{  *                       FLASH\_ERROR\_WRP, FLASH\_ERROR\_OPERATION or FLASH\_COMPLETE.}
00572 \textcolor{comment}{  */}
00573 FLASH\_Status FLASH_ProgramByte(uint32\_t Address, uint8\_t Data)
00574 \{
00575   FLASH\_Status status = FLASH_COMPLETE;
00576 
00577   \textcolor{comment}{/* Check the parameters */}
00578   assert_param(IS\_FLASH\_ADDRESS(Address));
00579 
00580   \textcolor{comment}{/* Wait for last operation to be completed */}
00581   status = FLASH_WaitForLastOperation();
00582 
00583   \textcolor{keywordflow}{if}(status == FLASH_COMPLETE)
00584   \{
00585     \textcolor{comment}{/* if the previous operation is completed, proceed to program the new data */}
00586     FLASH->CR &= CR_PSIZE_MASK;
00587     FLASH->CR |= FLASH_PSIZE_BYTE;
00588     FLASH->CR |= FLASH_CR_PG;
00589 
00590     *(\_\_IO uint8\_t*)Address = Data;
00591 
00592     \textcolor{comment}{/* Wait for last operation to be completed */}
00593     status = FLASH_WaitForLastOperation();
00594 
00595     \textcolor{comment}{/* if the program operation is completed, disable the PG Bit */}
00596     FLASH->CR &= (~FLASH_CR_PG);
00597   \}
00598 
00599   \textcolor{comment}{/* Return the Program Status */}
00600   \textcolor{keywordflow}{return} status;
00601 \}
00602 
00603 \textcolor{comment}{/**}
00604 \textcolor{comment}{  * @\}}
00605 \textcolor{comment}{  */}
00606 
00607 \textcolor{comment}{/** @defgroup FLASH\_Group3 Option Bytes Programming functions}
00608 \textcolor{comment}{ *  @brief   Option Bytes Programming functions }
00609 \textcolor{comment}{ *}
00610 \textcolor{comment}{@verbatim   }
00611 \textcolor{comment}{ ===============================================================================}
00612 \textcolor{comment}{                        Option Bytes Programming functions}
00613 \textcolor{comment}{ ===============================================================================  }
00614 \textcolor{comment}{ }
00615 \textcolor{comment}{   This group includes the following functions:}
00616 \textcolor{comment}{   - void FLASH\_OB\_Unlock(void)}
00617 \textcolor{comment}{   - void FLASH\_OB\_Lock(void)}
00618 \textcolor{comment}{   - void FLASH\_OB\_WRPConfig(uint32\_t OB\_WRP, FunctionalState NewState)}
00619 \textcolor{comment}{   - void FLASH\_OB\_RDPConfig(uint8\_t OB\_RDP)}
00620 \textcolor{comment}{   - void FLASH\_OB\_UserConfig(uint8\_t OB\_IWDG, uint8\_t OB\_STOP, uint8\_t OB\_STDBY)}
00621 \textcolor{comment}{   - void FLASH\_OB\_BORConfig(uint8\_t OB\_BOR)}
00622 \textcolor{comment}{   - FLASH\_Status FLASH\_ProgramOTP(uint32\_t Address, uint32\_t Data)                         }
00623 \textcolor{comment}{   - FLASH\_Status FLASH\_OB\_Launch(void)}
00624 \textcolor{comment}{   - uint32\_t FLASH\_OB\_GetUser(void)                        }
00625 \textcolor{comment}{   - uint8\_t FLASH\_OB\_GetWRP(void)                      }
00626 \textcolor{comment}{   - uint8\_t FLASH\_OB\_GetRDP(void)                          }
00627 \textcolor{comment}{   - uint8\_t FLASH\_OB\_GetBOR(void)}
00628 \textcolor{comment}{   }
00629 \textcolor{comment}{   Any operation of erase or program should follow these steps:}
00630 \textcolor{comment}{   1. Call the FLASH\_OB\_Unlock() function to enable the FLASH option control register access}
00631 \textcolor{comment}{}
00632 \textcolor{comment}{   2. Call one or several functions to program the desired Option Bytes:}
00633 \textcolor{comment}{      - void FLASH\_OB\_WRPConfig(uint32\_t OB\_WRP, FunctionalState NewState) => to Enable/Disable }
00634 \textcolor{comment}{        the desired sector write protection}
00635 \textcolor{comment}{      - void FLASH\_OB\_RDPConfig(uint8\_t OB\_RDP) => to set the desired read Protection Level}
00636 \textcolor{comment}{      - void FLASH\_OB\_UserConfig(uint8\_t OB\_IWDG, uint8\_t OB\_STOP, uint8\_t OB\_STDBY) => to configure }
00637 \textcolor{comment}{        the user Option Bytes.}
00638 \textcolor{comment}{      - void FLASH\_OB\_BORConfig(uint8\_t OB\_BOR) => to set the BOR Level              }
00639 \textcolor{comment}{}
00640 \textcolor{comment}{   3. Once all needed Option Bytes to be programmed are correctly written, call the}
00641 \textcolor{comment}{      FLASH\_OB\_Launch() function to launch the Option Bytes programming process.}
00642 \textcolor{comment}{     }
00643 \textcolor{comment}{     @note When changing the IWDG mode from HW to SW or from SW to HW, a system }
00644 \textcolor{comment}{           reset is needed to make the change effective.  }
00645 \textcolor{comment}{}
00646 \textcolor{comment}{   4. Call the FLASH\_OB\_Lock() function to disable the FLASH option control register}
00647 \textcolor{comment}{      access (recommended to protect the Option Bytes against possible unwanted operations)}
00648 \textcolor{comment}{    }
00649 \textcolor{comment}{@endverbatim}
00650 \textcolor{comment}{  * @\{}
00651 \textcolor{comment}{  */}
00652 
00653 \textcolor{comment}{/**}
00654 \textcolor{comment}{  * @brief  Unlocks the FLASH Option Control Registers access.}
00655 \textcolor{comment}{  * @param  None}
00656 \textcolor{comment}{  * @retval None}
00657 \textcolor{comment}{  */}
00658 \textcolor{keywordtype}{void} FLASH_OB_Unlock(\textcolor{keywordtype}{void})
00659 \{
00660   \textcolor{keywordflow}{if}((FLASH->OPTCR & FLASH_OPTCR_OPTLOCK) != RESET)
00661   \{
00662     \textcolor{comment}{/* Authorizes the Option Byte register programming */}
00663     FLASH->OPTKEYR = FLASH_OPT_KEY1;
00664     FLASH->OPTKEYR = FLASH_OPT_KEY2;
00665   \}
00666 \}
00667 
00668 \textcolor{comment}{/**}
00669 \textcolor{comment}{  * @brief  Locks the FLASH Option Control Registers access.}
00670 \textcolor{comment}{  * @param  None}
00671 \textcolor{comment}{  * @retval None}
00672 \textcolor{comment}{  */}
00673 \textcolor{keywordtype}{void} FLASH_OB_Lock(\textcolor{keywordtype}{void})
00674 \{
00675   \textcolor{comment}{/* Set the OPTLOCK Bit to lock the FLASH Option Byte Registers access */}
00676   FLASH->OPTCR |= FLASH_OPTCR_OPTLOCK;
00677 \}
00678 
00679 \textcolor{comment}{/**}
00680 \textcolor{comment}{  * @brief  Enables or disables the write protection of the desired sectors}
00681 \textcolor{comment}{  * @param  OB\_WRP: specifies the sector(s) to be write protected or unprotected.}
00682 \textcolor{comment}{  *          This parameter can be one of the following values:}
00683 \textcolor{comment}{  *            @arg OB\_WRP: A value between OB\_WRP\_Sector0 and OB\_WRP\_Sector11                      }
00684 \textcolor{comment}{  *            @arg OB\_WRP\_Sector\_All}
00685 \textcolor{comment}{  * @param  Newstate: new state of the Write Protection.}
00686 \textcolor{comment}{  *          This parameter can be: ENABLE or DISABLE.}
00687 \textcolor{comment}{  * @retval None  }
00688 \textcolor{comment}{  */}
00689 \textcolor{keywordtype}{void} FLASH_OB_WRPConfig(uint32\_t OB\_WRP, FunctionalState NewState)
00690 \{
00691   FLASH\_Status status = FLASH_COMPLETE;
00692 
00693   \textcolor{comment}{/* Check the parameters */}
00694   assert_param(IS\_OB\_WRP(OB\_WRP));
00695   assert_param(IS\_FUNCTIONAL\_STATE(NewState));
00696 
00697   status = FLASH_WaitForLastOperation();
00698 
00699   \textcolor{keywordflow}{if}(status == FLASH_COMPLETE)
00700   \{
00701     \textcolor{keywordflow}{if}(NewState != DISABLE)
00702     \{
00703       *(\_\_IO uint16\_t*)OPTCR_BYTE2_ADDRESS &= (~OB\_WRP);
00704     \}
00705     \textcolor{keywordflow}{else}
00706     \{
00707       *(\_\_IO uint16\_t*)OPTCR_BYTE2_ADDRESS |= (uint16\_t)OB\_WRP;
00708     \}
00709   \}
00710 \}
00711 
00712 \textcolor{comment}{/**}
00713 \textcolor{comment}{  * @brief  Sets the read protection level.}
00714 \textcolor{comment}{  * @param  OB\_RDP: specifies the read protection level.}
00715 \textcolor{comment}{  *          This parameter can be one of the following values:}
00716 \textcolor{comment}{  *            @arg OB\_RDP\_Level\_0: No protection}
00717 \textcolor{comment}{  *            @arg OB\_RDP\_Level\_1: Read protection of the memory}
00718 \textcolor{comment}{  *            @arg OB\_RDP\_Level\_2: Full chip protection}
00719 \textcolor{comment}{  *   }
00720 \textcolor{comment}{  * !!!Warning!!! When enabling OB\_RDP level 2 it's no more possible to go back to level 1 or 0}
00721 \textcolor{comment}{  *    }
00722 \textcolor{comment}{  * @retval None}
00723 \textcolor{comment}{  */}
00724 \textcolor{keywordtype}{void} FLASH_OB_RDPConfig(uint8\_t OB\_RDP)
00725 \{
00726   FLASH\_Status status = FLASH_COMPLETE;
00727 
00728   \textcolor{comment}{/* Check the parameters */}
00729   assert_param(IS\_OB\_RDP(OB\_RDP));
00730 
00731   status = FLASH_WaitForLastOperation();
00732 
00733   \textcolor{keywordflow}{if}(status == FLASH_COMPLETE)
00734   \{
00735     *(\_\_IO uint8\_t*)OPTCR_BYTE1_ADDRESS = OB\_RDP;
00736 
00737   \}
00738 \}
00739 
00740 \textcolor{comment}{/**}
00741 \textcolor{comment}{  * @brief  Programs the FLASH User Option Byte: IWDG\_SW / RST\_STOP / RST\_STDBY.    }
00742 \textcolor{comment}{  * @param  OB\_IWDG: Selects the IWDG mode}
00743 \textcolor{comment}{  *          This parameter can be one of the following values:}
00744 \textcolor{comment}{  *            @arg OB\_IWDG\_SW: Software IWDG selected}
00745 \textcolor{comment}{  *            @arg OB\_IWDG\_HW: Hardware IWDG selected}
00746 \textcolor{comment}{  * @param  OB\_STOP: Reset event when entering STOP mode.}
00747 \textcolor{comment}{  *          This parameter  can be one of the following values:}
00748 \textcolor{comment}{  *            @arg OB\_STOP\_NoRST: No reset generated when entering in STOP}
00749 \textcolor{comment}{  *            @arg OB\_STOP\_RST: Reset generated when entering in STOP}
00750 \textcolor{comment}{  * @param  OB\_STDBY: Reset event when entering Standby mode.}
00751 \textcolor{comment}{  *          This parameter  can be one of the following values:}
00752 \textcolor{comment}{  *            @arg OB\_STDBY\_NoRST: No reset generated when entering in STANDBY}
00753 \textcolor{comment}{  *            @arg OB\_STDBY\_RST: Reset generated when entering in STANDBY}
00754 \textcolor{comment}{  * @retval None}
00755 \textcolor{comment}{  */}
00756 \textcolor{keywordtype}{void} FLASH_OB_UserConfig(uint8\_t OB\_IWDG, uint8\_t OB\_STOP, uint8\_t OB\_STDBY)
00757 \{
00758   uint8\_t optiontmp = 0xFF;
00759   FLASH\_Status status = FLASH_COMPLETE;
00760 
00761   \textcolor{comment}{/* Check the parameters */}
00762   assert_param(IS\_OB\_IWDG\_SOURCE(OB\_IWDG));
00763   assert_param(IS\_OB\_STOP\_SOURCE(OB\_STOP));
00764   assert_param(IS\_OB\_STDBY\_SOURCE(OB\_STDBY));
00765 
00766   \textcolor{comment}{/* Wait for last operation to be completed */}
00767   status = FLASH_WaitForLastOperation();
00768 
00769   \textcolor{keywordflow}{if}(status == FLASH_COMPLETE)
00770   \{
00771     \textcolor{comment}{/* Mask OPTLOCK, OPTSTRT and BOR\_LEV bits */}
00772     optiontmp =  (uint8\_t)((*(\_\_IO uint8\_t *)OPTCR_BYTE0_ADDRESS) & (uint8\_t)0x0F);
00773 
00774     \textcolor{comment}{/* Update User Option Byte */}
00775     *(\_\_IO uint8\_t *)OPTCR_BYTE0_ADDRESS = OB\_IWDG | (uint8\_t)(OB\_STDBY | (uint8\_t)(OB\_STOP | ((
      uint8\_t)optiontmp)));
00776   \}
00777 \}
00778 
00779 \textcolor{comment}{/**}
00780 \textcolor{comment}{  * @brief  Sets the BOR Level. }
00781 \textcolor{comment}{  * @param  OB\_BOR: specifies the Option Bytes BOR Reset Level.}
00782 \textcolor{comment}{  *          This parameter can be one of the following values:}
00783 \textcolor{comment}{  *            @arg OB\_BOR\_LEVEL3: Supply voltage ranges from 2.7 to 3.6 V}
00784 \textcolor{comment}{  *            @arg OB\_BOR\_LEVEL2: Supply voltage ranges from 2.4 to 2.7 V}
00785 \textcolor{comment}{  *            @arg OB\_BOR\_LEVEL1: Supply voltage ranges from 2.1 to 2.4 V}
00786 \textcolor{comment}{  *            @arg OB\_BOR\_OFF: Supply voltage ranges from 1.62 to 2.1 V}
00787 \textcolor{comment}{  * @retval None}
00788 \textcolor{comment}{  */}
00789 \textcolor{keywordtype}{void} FLASH_OB_BORConfig(uint8\_t OB\_BOR)
00790 \{
00791   \textcolor{comment}{/* Check the parameters */}
00792   assert_param(IS\_OB\_BOR(OB\_BOR));
00793 
00794   \textcolor{comment}{/* Set the BOR Level */}
00795   *(\_\_IO uint8\_t *)OPTCR_BYTE0_ADDRESS &= (~FLASH_OPTCR_BOR_LEV);
00796   *(\_\_IO uint8\_t *)OPTCR_BYTE0_ADDRESS |= OB\_BOR;
00797 
00798 \}
00799 
00800 \textcolor{comment}{/**}
00801 \textcolor{comment}{  * @brief  Launch the option byte loading.}
00802 \textcolor{comment}{  * @param  None}
00803 \textcolor{comment}{  * @retval FLASH Status: The returned value can be: FLASH\_BUSY, FLASH\_ERROR\_PROGRAM,}
00804 \textcolor{comment}{  *                       FLASH\_ERROR\_WRP, FLASH\_ERROR\_OPERATION or FLASH\_COMPLETE.}
00805 \textcolor{comment}{  */}
00806 FLASH\_Status FLASH_OB_Launch(\textcolor{keywordtype}{void})
00807 \{
00808   FLASH\_Status status = FLASH_COMPLETE;
00809 
00810   \textcolor{comment}{/* Set the OPTSTRT bit in OPTCR register */}
00811   *(\_\_IO uint8\_t *)OPTCR_BYTE0_ADDRESS |= FLASH_OPTCR_OPTSTRT;
00812 
00813   \textcolor{comment}{/* Wait for last operation to be completed */}
00814   status = FLASH_WaitForLastOperation();
00815 
00816   \textcolor{keywordflow}{return} status;
00817 \}
00818 
00819 \textcolor{comment}{/**}
00820 \textcolor{comment}{  * @brief  Returns the FLASH User Option Bytes values.}
00821 \textcolor{comment}{  * @param  None}
00822 \textcolor{comment}{  * @retval The FLASH User Option Bytes values: IWDG\_SW(Bit0), RST\_STOP(Bit1)}
00823 \textcolor{comment}{  *         and RST\_STDBY(Bit2).}
00824 \textcolor{comment}{  */}
00825 uint8\_t FLASH_OB_GetUser(\textcolor{keywordtype}{void})
00826 \{
00827   \textcolor{comment}{/* Return the User Option Byte */}
00828   \textcolor{keywordflow}{return} (uint8\_t)(FLASH->OPTCR >> 5);
00829 \}
00830 
00831 \textcolor{comment}{/**}
00832 \textcolor{comment}{  * @brief  Returns the FLASH Write Protection Option Bytes value.}
00833 \textcolor{comment}{  * @param  None}
00834 \textcolor{comment}{  * @retval The FLASH Write Protection  Option Bytes value}
00835 \textcolor{comment}{  */}
00836 uint16\_t FLASH_OB_GetWRP(\textcolor{keywordtype}{void})
00837 \{
00838   \textcolor{comment}{/* Return the FLASH write protection Register value */}
00839   \textcolor{keywordflow}{return} (*(\_\_IO uint16\_t *)(OPTCR_BYTE2_ADDRESS));
00840 \}
00841 
00842 \textcolor{comment}{/**}
00843 \textcolor{comment}{  * @brief  Returns the FLASH Read Protection level.}
00844 \textcolor{comment}{  * @param  None}
00845 \textcolor{comment}{  * @retval FLASH ReadOut Protection Status:}
00846 \textcolor{comment}{  *           - SET, when OB\_RDP\_Level\_1 or OB\_RDP\_Level\_2 is set}
00847 \textcolor{comment}{  *           - RESET, when OB\_RDP\_Level\_0 is set}
00848 \textcolor{comment}{  */}
00849 FlagStatus FLASH_OB_GetRDP(\textcolor{keywordtype}{void})
00850 \{
00851   FlagStatus readstatus = RESET;
00852 
00853   \textcolor{keywordflow}{if} ((*(\_\_IO uint8\_t*)(OPTCR_BYTE1_ADDRESS) != (uint8\_t)OB_RDP_Level_0))
00854   \{
00855     readstatus = SET;
00856   \}
00857   \textcolor{keywordflow}{else}
00858   \{
00859     readstatus = RESET;
00860   \}
00861   \textcolor{keywordflow}{return} readstatus;
00862 \}
00863 
00864 \textcolor{comment}{/**}
00865 \textcolor{comment}{  * @brief  Returns the FLASH BOR level.}
00866 \textcolor{comment}{  * @param  None}
00867 \textcolor{comment}{  * @retval The FLASH BOR level:}
00868 \textcolor{comment}{  *           - OB\_BOR\_LEVEL3: Supply voltage ranges from 2.7 to 3.6 V}
00869 \textcolor{comment}{  *           - OB\_BOR\_LEVEL2: Supply voltage ranges from 2.4 to 2.7 V}
00870 \textcolor{comment}{  *           - OB\_BOR\_LEVEL1: Supply voltage ranges from 2.1 to 2.4 V}
00871 \textcolor{comment}{  *           - OB\_BOR\_OFF   : Supply voltage ranges from 1.62 to 2.1 V  }
00872 \textcolor{comment}{  */}
00873 uint8\_t FLASH_OB_GetBOR(\textcolor{keywordtype}{void})
00874 \{
00875   \textcolor{comment}{/* Return the FLASH BOR level */}
00876   \textcolor{keywordflow}{return} (uint8\_t)(*(\_\_IO uint8\_t *)(OPTCR_BYTE0_ADDRESS) & (uint8\_t)0x0C);
00877 \}
00878 
00879 \textcolor{comment}{/**}
00880 \textcolor{comment}{  * @\}}
00881 \textcolor{comment}{  */}
00882 
00883 \textcolor{comment}{/** @defgroup FLASH\_Group4 Interrupts and flags management functions}
00884 \textcolor{comment}{ *  @brief   Interrupts and flags management functions}
00885 \textcolor{comment}{ *}
00886 \textcolor{comment}{@verbatim   }
00887 \textcolor{comment}{ ===============================================================================}
00888 \textcolor{comment}{                  Interrupts and flags management functions}
00889 \textcolor{comment}{ ===============================================================================  }
00890 \textcolor{comment}{}
00891 \textcolor{comment}{@endverbatim}
00892 \textcolor{comment}{  * @\{}
00893 \textcolor{comment}{  */}
00894 
00895 \textcolor{comment}{/**}
00896 \textcolor{comment}{  * @brief  Enables or disables the specified FLASH interrupts.}
00897 \textcolor{comment}{  * @param  FLASH\_IT: specifies the FLASH interrupt sources to be enabled or disabled.}
00898 \textcolor{comment}{  *          This parameter can be any combination of the following values:}
00899 \textcolor{comment}{  *            @arg FLASH\_IT\_ERR: FLASH Error Interrupt}
00900 \textcolor{comment}{  *            @arg FLASH\_IT\_EOP: FLASH end of operation Interrupt}
00901 \textcolor{comment}{  * @retval None }
00902 \textcolor{comment}{  */}
00903 \textcolor{keywordtype}{void} FLASH_ITConfig(uint32\_t FLASH\_IT, FunctionalState NewState)
00904 \{
00905   \textcolor{comment}{/* Check the parameters */}
00906   assert_param(IS\_FLASH\_IT(FLASH\_IT));
00907   assert_param(IS\_FUNCTIONAL\_STATE(NewState));
00908 
00909   \textcolor{keywordflow}{if}(NewState != DISABLE)
00910   \{
00911     \textcolor{comment}{/* Enable the interrupt sources */}
00912     FLASH->CR |= FLASH\_IT;
00913   \}
00914   \textcolor{keywordflow}{else}
00915   \{
00916     \textcolor{comment}{/* Disable the interrupt sources */}
00917     FLASH->CR &= ~(uint32\_t)FLASH\_IT;
00918   \}
00919 \}
00920 
00921 \textcolor{comment}{/**}
00922 \textcolor{comment}{  * @brief  Checks whether the specified FLASH flag is set or not.}
00923 \textcolor{comment}{  * @param  FLASH\_FLAG: specifies the FLASH flag to check.}
00924 \textcolor{comment}{  *          This parameter can be one of the following values:}
00925 \textcolor{comment}{  *            @arg FLASH\_FLAG\_EOP: FLASH End of Operation flag }
00926 \textcolor{comment}{  *            @arg FLASH\_FLAG\_OPERR: FLASH operation Error flag }
00927 \textcolor{comment}{  *            @arg FLASH\_FLAG\_WRPERR: FLASH Write protected error flag }
00928 \textcolor{comment}{  *            @arg FLASH\_FLAG\_PGAERR: FLASH Programming Alignment error flag}
00929 \textcolor{comment}{  *            @arg FLASH\_FLAG\_PGPERR: FLASH Programming Parallelism error flag}
00930 \textcolor{comment}{  *            @arg FLASH\_FLAG\_PGSERR: FLASH Programming Sequence error flag}
00931 \textcolor{comment}{  *            @arg FLASH\_FLAG\_BSY: FLASH Busy flag}
00932 \textcolor{comment}{  * @retval The new state of FLASH\_FLAG (SET or RESET).}
00933 \textcolor{comment}{  */}
00934 FlagStatus FLASH_GetFlagStatus(uint32\_t FLASH\_FLAG)
00935 \{
00936   FlagStatus bitstatus = RESET;
00937   \textcolor{comment}{/* Check the parameters */}
00938   assert_param(IS\_FLASH\_GET\_FLAG(FLASH\_FLAG));
00939 
00940   \textcolor{keywordflow}{if}((FLASH->SR & FLASH\_FLAG) != (uint32\_t)RESET)
00941   \{
00942     bitstatus = SET;
00943   \}
00944   \textcolor{keywordflow}{else}
00945   \{
00946     bitstatus = RESET;
00947   \}
00948   \textcolor{comment}{/* Return the new state of FLASH\_FLAG (SET or RESET) */}
00949   \textcolor{keywordflow}{return} bitstatus;
00950 \}
00951 
00952 \textcolor{comment}{/**}
00953 \textcolor{comment}{  * @brief  Clears the FLASH's pending flags.}
00954 \textcolor{comment}{  * @param  FLASH\_FLAG: specifies the FLASH flags to clear.}
00955 \textcolor{comment}{  *          This parameter can be any combination of the following values:}
00956 \textcolor{comment}{  *            @arg FLASH\_FLAG\_EOP: FLASH End of Operation flag }
00957 \textcolor{comment}{  *            @arg FLASH\_FLAG\_OPERR: FLASH operation Error flag }
00958 \textcolor{comment}{  *            @arg FLASH\_FLAG\_WRPERR: FLASH Write protected error flag }
00959 \textcolor{comment}{  *            @arg FLASH\_FLAG\_PGAERR: FLASH Programming Alignment error flag }
00960 \textcolor{comment}{  *            @arg FLASH\_FLAG\_PGPERR: FLASH Programming Parallelism error flag}
00961 \textcolor{comment}{  *            @arg FLASH\_FLAG\_PGSERR: FLASH Programming Sequence error flag}
00962 \textcolor{comment}{  * @retval None}
00963 \textcolor{comment}{  */}
00964 \textcolor{keywordtype}{void} FLASH_ClearFlag(uint32\_t FLASH\_FLAG)
00965 \{
00966   \textcolor{comment}{/* Check the parameters */}
00967   assert_param(IS\_FLASH\_CLEAR\_FLAG(FLASH\_FLAG));
00968 
00969   \textcolor{comment}{/* Clear the flags */}
00970   FLASH->SR = FLASH\_FLAG;
00971 \}
00972 
00973 \textcolor{comment}{/**}
00974 \textcolor{comment}{  * @brief  Returns the FLASH Status.}
00975 \textcolor{comment}{  * @param  None}
00976 \textcolor{comment}{  * @retval FLASH Status: The returned value can be: FLASH\_BUSY, FLASH\_ERROR\_PROGRAM,}
00977 \textcolor{comment}{  *                       FLASH\_ERROR\_WRP, FLASH\_ERROR\_OPERATION or FLASH\_COMPLETE.}
00978 \textcolor{comment}{  */}
00979 FLASH\_Status FLASH_GetStatus(\textcolor{keywordtype}{void})
00980 \{
00981   FLASH\_Status flashstatus = FLASH_COMPLETE;
00982 
00983   \textcolor{keywordflow}{if}((FLASH->SR & FLASH_FLAG_BSY) == FLASH_FLAG_BSY)
00984   \{
00985     flashstatus = FLASH\_BUSY;
00986   \}
00987   \textcolor{keywordflow}{else}
00988   \{
00989     \textcolor{keywordflow}{if}((FLASH->SR & FLASH_FLAG_WRPERR) != (uint32\_t)0x00)
00990     \{
00991       flashstatus = FLASH\_ERROR\_WRP;
00992     \}
00993     \textcolor{keywordflow}{else}
00994     \{
00995       \textcolor{keywordflow}{if}((FLASH->SR & (uint32\_t)0xEF) != (uint32\_t)0x00)
00996       \{
00997         flashstatus = FLASH\_ERROR\_PROGRAM;
00998       \}
00999       \textcolor{keywordflow}{else}
01000       \{
01001         \textcolor{keywordflow}{if}((FLASH->SR & FLASH_FLAG_OPERR) != (uint32\_t)0x00)
01002         \{
01003           flashstatus = FLASH\_ERROR\_OPERATION;
01004         \}
01005         \textcolor{keywordflow}{else}
01006         \{
01007           flashstatus = FLASH\_COMPLETE;
01008         \}
01009       \}
01010     \}
01011   \}
01012   \textcolor{comment}{/* Return the FLASH Status */}
01013   \textcolor{keywordflow}{return} flashstatus;
01014 \}
01015 
01016 \textcolor{comment}{/**}
01017 \textcolor{comment}{  * @brief  Waits for a FLASH operation to complete.}
01018 \textcolor{comment}{  * @param  None}
01019 \textcolor{comment}{  * @retval FLASH Status: The returned value can be: FLASH\_BUSY, FLASH\_ERROR\_PROGRAM,}
01020 \textcolor{comment}{  *                       FLASH\_ERROR\_WRP, FLASH\_ERROR\_OPERATION or FLASH\_COMPLETE.}
01021 \textcolor{comment}{  */}
01022 FLASH\_Status FLASH_WaitForLastOperation(\textcolor{keywordtype}{void})
01023 \{
01024   \_\_IO FLASH\_Status status = FLASH\_COMPLETE;
01025 
01026   \textcolor{comment}{/* Check for the FLASH Status */}
01027   status = FLASH\_GetStatus();
01028 
01029   \textcolor{comment}{/* Wait for the FLASH operation to complete by polling on BUSY flag to be reset.}
01030 \textcolor{comment}{     Even if the FLASH operation fails, the BUSY flag will be reset and an error}
01031 \textcolor{comment}{     flag will be set */}
01032   \textcolor{keywordflow}{while}(status == FLASH\_BUSY)
01033   \{
01034     status = FLASH\_GetStatus();
01035   \}
01036   \textcolor{comment}{/* Return the operation status */}
01037   \textcolor{keywordflow}{return} status;
01038 \}
01039 
01040 \textcolor{comment}{/**}
01041 \textcolor{comment}{  * @\}}
01042 \textcolor{comment}{  */}
01043 
01044 \textcolor{comment}{/**}
01045 \textcolor{comment}{  * @\}}
01046 \textcolor{comment}{  */}
01047 
01048 \textcolor{comment}{/**}
01049 \textcolor{comment}{  * @\}}
01050 \textcolor{comment}{  */}
01051 
01052 \textcolor{comment}{/**}
01053 \textcolor{comment}{  * @\}}
01054 \textcolor{comment}{  */}
01055 
01056 \textcolor{comment}{/******************* (C) COPYRIGHT 2011 STMicroelectronics *****END OF FILE****/}
\end{DoxyCode}
