\section{stm32f4xx\+\_\+rcc.\+c}
\label{stm32f4xx__rcc_8c_source}\index{C\+:/\+Users/\+Md. Istiaq Mahbub/\+Desktop/\+I\+M\+U/\+M\+P\+U6050\+\_\+\+Motion\+Driver/\+S\+T\+M32\+F4xx\+\_\+\+Std\+Periph\+\_\+\+Driver/src/stm32f4xx\+\_\+rcc.\+c@{C\+:/\+Users/\+Md. Istiaq Mahbub/\+Desktop/\+I\+M\+U/\+M\+P\+U6050\+\_\+\+Motion\+Driver/\+S\+T\+M32\+F4xx\+\_\+\+Std\+Periph\+\_\+\+Driver/src/stm32f4xx\+\_\+rcc.\+c}}

\begin{DoxyCode}
00001 \textcolor{comment}{/**}
00002 \textcolor{comment}{  ******************************************************************************}
00003 \textcolor{comment}{  * @file    stm32f4xx\_rcc.c}
00004 \textcolor{comment}{  * @author  MCD Application Team}
00005 \textcolor{comment}{  * @version V1.0.0}
00006 \textcolor{comment}{  * @date    30-September-2011}
00007 \textcolor{comment}{  * @brief   This file provides firmware functions to manage the following }
00008 \textcolor{comment}{  *          functionalities of the Reset and clock control (RCC) peripheral:}
00009 \textcolor{comment}{  *           - Internal/external clocks, PLL, CSS and MCO configuration}
00010 \textcolor{comment}{  *           - System, AHB and APB busses clocks configuration}
00011 \textcolor{comment}{  *           - Peripheral clocks configuration}
00012 \textcolor{comment}{  *           - Interrupts and flags management}
00013 \textcolor{comment}{  *}
00014 \textcolor{comment}{  *  @verbatim}
00015 \textcolor{comment}{  *               }
00016 \textcolor{comment}{  *          ===================================================================}
00017 \textcolor{comment}{  *                               RCC specific features}
00018 \textcolor{comment}{  *          ===================================================================}
00019 \textcolor{comment}{  *    }
00020 \textcolor{comment}{  *          After reset the device is running from Internal High Speed oscillator }
00021 \textcolor{comment}{  *          (HSI 16MHz) with Flash 0 wait state, Flash prefetch buffer, D-Cache }
00022 \textcolor{comment}{  *          and I-Cache are disabled, and all peripherals are off except internal}
00023 \textcolor{comment}{  *          SRAM, Flash and JTAG.}
00024 \textcolor{comment}{  *           - There is no prescaler on High speed (AHB) and Low speed (APB) busses;}
00025 \textcolor{comment}{  *             all peripherals mapped on these busses are running at HSI speed.}
00026 \textcolor{comment}{  *           - The clock for all peripherals is switched off, except the SRAM and FLASH.}
00027 \textcolor{comment}{  *           - All GPIOs are in input floating state, except the JTAG pins which}
00028 \textcolor{comment}{  *             are assigned to be used for debug purpose.}
00029 \textcolor{comment}{  *        }
00030 \textcolor{comment}{  *          Once the device started from reset, the user application has to:        }
00031 \textcolor{comment}{  *           - Configure the clock source to be used to drive the System clock}
00032 \textcolor{comment}{  *             (if the application needs higher frequency/performance)}
00033 \textcolor{comment}{  *           - Configure the System clock frequency and Flash settings  }
00034 \textcolor{comment}{  *           - Configure the AHB and APB busses prescalers}
00035 \textcolor{comment}{  *           - Enable the clock for the peripheral(s) to be used}
00036 \textcolor{comment}{  *           - Configure the clock source(s) for peripherals which clocks are not}
00037 \textcolor{comment}{  *             derived from the System clock (I2S, RTC, ADC, USB OTG FS/SDIO/RNG)      }
00038 \textcolor{comment}{  *                        }
00039 \textcolor{comment}{  *  @endverbatim}
00040 \textcolor{comment}{  *    }
00041 \textcolor{comment}{  ******************************************************************************}
00042 \textcolor{comment}{  * @attention}
00043 \textcolor{comment}{  *}
00044 \textcolor{comment}{  * THE PRESENT FIRMWARE WHICH IS FOR GUIDANCE ONLY AIMS AT PROVIDING CUSTOMERS}
00045 \textcolor{comment}{  * WITH CODING INFORMATION REGARDING THEIR PRODUCTS IN ORDER FOR THEM TO SAVE}
00046 \textcolor{comment}{  * TIME. AS A RESULT, STMICROELECTRONICS SHALL NOT BE HELD LIABLE FOR ANY}
00047 \textcolor{comment}{  * DIRECT, INDIRECT OR CONSEQUENTIAL DAMAGES WITH RESPECT TO ANY CLAIMS ARISING}
00048 \textcolor{comment}{  * FROM THE CONTENT OF SUCH FIRMWARE AND/OR THE USE MADE BY CUSTOMERS OF THE}
00049 \textcolor{comment}{  * CODING INFORMATION CONTAINED HEREIN IN CONNECTION WITH THEIR PRODUCTS.}
00050 \textcolor{comment}{  *}
00051 \textcolor{comment}{  * <h2><center>&copy; COPYRIGHT 2011 STMicroelectronics</center></h2>}
00052 \textcolor{comment}{  ******************************************************************************}
00053 \textcolor{comment}{  */}
00054 
00055 \textcolor{comment}{/* Includes ------------------------------------------------------------------*/}
00056 \textcolor{preprocessor}{#}\textcolor{preprocessor}{include} "stm32f4xx_rcc.h"
00057 
00058 \textcolor{comment}{/** @addtogroup STM32F4xx\_StdPeriph\_Driver}
00059 \textcolor{comment}{  * @\{}
00060 \textcolor{comment}{  */}
00061 
00062 \textcolor{comment}{/** @defgroup RCC }
00063 \textcolor{comment}{  * @brief RCC driver modules}
00064 \textcolor{comment}{  * @\{}
00065 \textcolor{comment}{  */}
00066 
00067 \textcolor{comment}{/* Private typedef -----------------------------------------------------------*/}
00068 \textcolor{comment}{/* Private define ------------------------------------------------------------*/}
00069 \textcolor{comment}{/* ------------ RCC registers bit address in the alias region ----------- */}
00070 \textcolor{preprocessor}{#}\textcolor{preprocessor}{define} \textcolor{preprocessor}{RCC\_OFFSET}                \textcolor{preprocessor}{(}RCC_BASE \textcolor{preprocessor}{-} PERIPH_BASE\textcolor{preprocessor}{)}
00071 \textcolor{comment}{/* --- CR Register ---*/}
00072 \textcolor{comment}{/* Alias word address of HSION bit */}
00073 \textcolor{preprocessor}{#}\textcolor{preprocessor}{define} \textcolor{preprocessor}{CR\_OFFSET}                 \textcolor{preprocessor}{(}RCC_OFFSET \textcolor{preprocessor}{+} 0x00\textcolor{preprocessor}{)}
00074 \textcolor{preprocessor}{#}\textcolor{preprocessor}{define} \textcolor{preprocessor}{HSION\_BitNumber}           0x00
00075 \textcolor{preprocessor}{#}\textcolor{preprocessor}{define} \textcolor{preprocessor}{CR\_HSION\_BB}               \textcolor{preprocessor}{(}PERIPH_BB_BASE \textcolor{preprocessor}{+} \textcolor{preprocessor}{(}CR_OFFSET \textcolor{preprocessor}{*} 32\textcolor{preprocessor}{)} \textcolor{preprocessor}{+} \textcolor{preprocessor}{(}
      HSION_BitNumber \textcolor{preprocessor}{*} 4\textcolor{preprocessor}{)}\textcolor{preprocessor}{)}
00076 \textcolor{comment}{/* Alias word address of CSSON bit */}
00077 \textcolor{preprocessor}{#}\textcolor{preprocessor}{define} \textcolor{preprocessor}{CSSON\_BitNumber}           0x13
00078 \textcolor{preprocessor}{#}\textcolor{preprocessor}{define} \textcolor{preprocessor}{CR\_CSSON\_BB}               \textcolor{preprocessor}{(}PERIPH_BB_BASE \textcolor{preprocessor}{+} \textcolor{preprocessor}{(}CR_OFFSET \textcolor{preprocessor}{*} 32\textcolor{preprocessor}{)} \textcolor{preprocessor}{+} \textcolor{preprocessor}{(}
      CSSON_BitNumber \textcolor{preprocessor}{*} 4\textcolor{preprocessor}{)}\textcolor{preprocessor}{)}
00079 \textcolor{comment}{/* Alias word address of PLLON bit */}
00080 \textcolor{preprocessor}{#}\textcolor{preprocessor}{define} \textcolor{preprocessor}{PLLON\_BitNumber}           0x18
00081 \textcolor{preprocessor}{#}\textcolor{preprocessor}{define} \textcolor{preprocessor}{CR\_PLLON\_BB}               \textcolor{preprocessor}{(}PERIPH_BB_BASE \textcolor{preprocessor}{+} \textcolor{preprocessor}{(}CR_OFFSET \textcolor{preprocessor}{*} 32\textcolor{preprocessor}{)} \textcolor{preprocessor}{+} \textcolor{preprocessor}{(}
      PLLON_BitNumber \textcolor{preprocessor}{*} 4\textcolor{preprocessor}{)}\textcolor{preprocessor}{)}
00082 \textcolor{comment}{/* Alias word address of PLLI2SON bit */}
00083 \textcolor{preprocessor}{#}\textcolor{preprocessor}{define} \textcolor{preprocessor}{PLLI2SON\_BitNumber}        0x1A
00084 \textcolor{preprocessor}{#}\textcolor{preprocessor}{define} \textcolor{preprocessor}{CR\_PLLI2SON\_BB}            \textcolor{preprocessor}{(}PERIPH_BB_BASE \textcolor{preprocessor}{+} \textcolor{preprocessor}{(}CR_OFFSET \textcolor{preprocessor}{*} 32\textcolor{preprocessor}{)} \textcolor{preprocessor}{+} \textcolor{preprocessor}{(}
      PLLI2SON_BitNumber \textcolor{preprocessor}{*} 4\textcolor{preprocessor}{)}\textcolor{preprocessor}{)}
00085 
00086 \textcolor{comment}{/* --- CFGR Register ---*/}
00087 \textcolor{comment}{/* Alias word address of I2SSRC bit */}
00088 \textcolor{preprocessor}{#}\textcolor{preprocessor}{define} \textcolor{preprocessor}{CFGR\_OFFSET}               \textcolor{preprocessor}{(}RCC_OFFSET \textcolor{preprocessor}{+} 0x08\textcolor{preprocessor}{)}
00089 \textcolor{preprocessor}{#}\textcolor{preprocessor}{define} \textcolor{preprocessor}{I2SSRC\_BitNumber}          0x17
00090 \textcolor{preprocessor}{#}\textcolor{preprocessor}{define} \textcolor{preprocessor}{CFGR\_I2SSRC\_BB}            \textcolor{preprocessor}{(}PERIPH_BB_BASE \textcolor{preprocessor}{+} \textcolor{preprocessor}{(}CFGR_OFFSET \textcolor{preprocessor}{*} 32\textcolor{preprocessor}{)} \textcolor{preprocessor}{+} \textcolor{preprocessor}{(}
      I2SSRC_BitNumber \textcolor{preprocessor}{*} 4\textcolor{preprocessor}{)}\textcolor{preprocessor}{)}
00091 
00092 \textcolor{comment}{/* --- BDCR Register ---*/}
00093 \textcolor{comment}{/* Alias word address of RTCEN bit */}
00094 \textcolor{preprocessor}{#}\textcolor{preprocessor}{define} \textcolor{preprocessor}{BDCR\_OFFSET}               \textcolor{preprocessor}{(}RCC_OFFSET \textcolor{preprocessor}{+} 0x70\textcolor{preprocessor}{)}
00095 \textcolor{preprocessor}{#}\textcolor{preprocessor}{define} \textcolor{preprocessor}{RTCEN\_BitNumber}           0x0F
00096 \textcolor{preprocessor}{#}\textcolor{preprocessor}{define} \textcolor{preprocessor}{BDCR\_RTCEN\_BB}             \textcolor{preprocessor}{(}PERIPH_BB_BASE \textcolor{preprocessor}{+} \textcolor{preprocessor}{(}BDCR_OFFSET \textcolor{preprocessor}{*} 32\textcolor{preprocessor}{)} \textcolor{preprocessor}{+} \textcolor{preprocessor}{(}
      RTCEN_BitNumber \textcolor{preprocessor}{*} 4\textcolor{preprocessor}{)}\textcolor{preprocessor}{)}
00097 \textcolor{comment}{/* Alias word address of BDRST bit */}
00098 \textcolor{preprocessor}{#}\textcolor{preprocessor}{define} \textcolor{preprocessor}{BDRST\_BitNumber}           0x10
00099 \textcolor{preprocessor}{#}\textcolor{preprocessor}{define} \textcolor{preprocessor}{BDCR\_BDRST\_BB}             \textcolor{preprocessor}{(}PERIPH_BB_BASE \textcolor{preprocessor}{+} \textcolor{preprocessor}{(}BDCR_OFFSET \textcolor{preprocessor}{*} 32\textcolor{preprocessor}{)} \textcolor{preprocessor}{+} \textcolor{preprocessor}{(}
      BDRST_BitNumber \textcolor{preprocessor}{*} 4\textcolor{preprocessor}{)}\textcolor{preprocessor}{)}
00100 \textcolor{comment}{/* --- CSR Register ---*/}
00101 \textcolor{comment}{/* Alias word address of LSION bit */}
00102 \textcolor{preprocessor}{#}\textcolor{preprocessor}{define} \textcolor{preprocessor}{CSR\_OFFSET}                \textcolor{preprocessor}{(}RCC_OFFSET \textcolor{preprocessor}{+} 0x74\textcolor{preprocessor}{)}
00103 \textcolor{preprocessor}{#}\textcolor{preprocessor}{define} \textcolor{preprocessor}{LSION\_BitNumber}           0x00
00104 \textcolor{preprocessor}{#}\textcolor{preprocessor}{define} \textcolor{preprocessor}{CSR\_LSION\_BB}              \textcolor{preprocessor}{(}PERIPH_BB_BASE \textcolor{preprocessor}{+} \textcolor{preprocessor}{(}CSR_OFFSET \textcolor{preprocessor}{*} 32\textcolor{preprocessor}{)} \textcolor{preprocessor}{+} \textcolor{preprocessor}{(}
      LSION_BitNumber \textcolor{preprocessor}{*} 4\textcolor{preprocessor}{)}\textcolor{preprocessor}{)}
00105 \textcolor{comment}{/* ---------------------- RCC registers bit mask ------------------------ */}
00106 \textcolor{comment}{/* CFGR register bit mask */}
00107 \textcolor{preprocessor}{#}\textcolor{preprocessor}{define} \textcolor{preprocessor}{CFGR\_MCO2\_RESET\_MASK}      \textcolor{preprocessor}{(}\textcolor{preprocessor}{(}\textcolor{preprocessor}{uint32\_t}\textcolor{preprocessor}{)}0x07FFFFFF\textcolor{preprocessor}{)}
00108 \textcolor{preprocessor}{#}\textcolor{preprocessor}{define} \textcolor{preprocessor}{CFGR\_MCO1\_RESET\_MASK}      \textcolor{preprocessor}{(}\textcolor{preprocessor}{(}\textcolor{preprocessor}{uint32\_t}\textcolor{preprocessor}{)}0xF89FFFFF\textcolor{preprocessor}{)}
00109 
00110 \textcolor{comment}{/* RCC Flag Mask */}
00111 \textcolor{preprocessor}{#}\textcolor{preprocessor}{define} \textcolor{preprocessor}{FLAG\_MASK}                 \textcolor{preprocessor}{(}\textcolor{preprocessor}{(}\textcolor{preprocessor}{uint8\_t}\textcolor{preprocessor}{)}0x1F\textcolor{preprocessor}{)}
00112 
00113 \textcolor{comment}{/* CR register byte 3 (Bits[23:16]) base address */}
00114 \textcolor{preprocessor}{#}\textcolor{preprocessor}{define} \textcolor{preprocessor}{CR\_BYTE3\_ADDRESS}          \textcolor{preprocessor}{(}\textcolor{preprocessor}{(}\textcolor{preprocessor}{uint32\_t}\textcolor{preprocessor}{)}0x40023802\textcolor{preprocessor}{)}
00115 
00116 \textcolor{comment}{/* CIR register byte 2 (Bits[15:8]) base address */}
00117 \textcolor{preprocessor}{#}\textcolor{preprocessor}{define} \textcolor{preprocessor}{CIR\_BYTE2\_ADDRESS}         \textcolor{preprocessor}{(}\textcolor{preprocessor}{(}\textcolor{preprocessor}{uint32\_t}\textcolor{preprocessor}{)}\textcolor{preprocessor}{(}RCC_BASE \textcolor{preprocessor}{+} 0x0C \textcolor{preprocessor}{+} 0x01\textcolor{preprocessor}{)}\textcolor{preprocessor}{)}
00118 
00119 \textcolor{comment}{/* CIR register byte 3 (Bits[23:16]) base address */}
00120 \textcolor{preprocessor}{#}\textcolor{preprocessor}{define} \textcolor{preprocessor}{CIR\_BYTE3\_ADDRESS}         \textcolor{preprocessor}{(}\textcolor{preprocessor}{(}\textcolor{preprocessor}{uint32\_t}\textcolor{preprocessor}{)}\textcolor{preprocessor}{(}RCC_BASE \textcolor{preprocessor}{+} 0x0C \textcolor{preprocessor}{+} 0x02\textcolor{preprocessor}{)}\textcolor{preprocessor}{)}
00121 
00122 \textcolor{comment}{/* BDCR register base address */}
00123 \textcolor{preprocessor}{#}\textcolor{preprocessor}{define} \textcolor{preprocessor}{BDCR\_ADDRESS}              \textcolor{preprocessor}{(}PERIPH_BASE \textcolor{preprocessor}{+} BDCR_OFFSET\textcolor{preprocessor}{)}
00124 
00125 \textcolor{comment}{/* Private macro -------------------------------------------------------------*/}
00126 \textcolor{comment}{/* Private variables ---------------------------------------------------------*/}
00127 \textcolor{keyword}{static} \_\_I uint8\_t APBAHBPrescTable[16] = \{0, 0, 0, 0, 1, 2, 3, 4, 1, 2, 3, 4, 6, 7, 8, 9\};
00128 
00129 \textcolor{comment}{/* Private function prototypes -----------------------------------------------*/}
00130 \textcolor{comment}{/* Private functions ---------------------------------------------------------*/}
00131 
00132 \textcolor{comment}{/** @defgroup RCC\_Private\_Functions}
00133 \textcolor{comment}{  * @\{}
00134 \textcolor{comment}{  */}
00135 
00136 \textcolor{comment}{/** @defgroup RCC\_Group1 Internal and external clocks, PLL, CSS and MCO configuration functions}
00137 \textcolor{comment}{ *  @brief   Internal and external clocks, PLL, CSS and MCO configuration functions }
00138 \textcolor{comment}{ *}
00139 \textcolor{comment}{@verbatim   }
00140 \textcolor{comment}{ ===============================================================================}
00141 \textcolor{comment}{      Internal/external clocks, PLL, CSS and MCO configuration functions}
00142 \textcolor{comment}{ ===============================================================================  }
00143 \textcolor{comment}{}
00144 \textcolor{comment}{  This section provide functions allowing to configure the internal/external clocks,}
00145 \textcolor{comment}{  PLLs, CSS and MCO pins.}
00146 \textcolor{comment}{  }
00147 \textcolor{comment}{  1. HSI (high-speed internal), 16 MHz factory-trimmed RC used directly or through}
00148 \textcolor{comment}{     the PLL as System clock source.}
00149 \textcolor{comment}{}
00150 \textcolor{comment}{  2. LSI (low-speed internal), 32 KHz low consumption RC used as IWDG and/or RTC}
00151 \textcolor{comment}{     clock source.}
00152 \textcolor{comment}{}
00153 \textcolor{comment}{  3. HSE (high-speed external), 4 to 26 MHz crystal oscillator used directly or}
00154 \textcolor{comment}{     through the PLL as System clock source. Can be used also as RTC clock source.}
00155 \textcolor{comment}{}
00156 \textcolor{comment}{  4. LSE (low-speed external), 32 KHz oscillator used as RTC clock source.   }
00157 \textcolor{comment}{}
00158 \textcolor{comment}{  5. PLL (clocked by HSI or HSE), featuring two different output clocks:}
00159 \textcolor{comment}{      - The first output is used to generate the high speed system clock (up to 168 MHz)}
00160 \textcolor{comment}{      - The second output is used to generate the clock for the USB OTG FS (48 MHz),}
00161 \textcolor{comment}{        the random analog generator (<=48 MHz) and the SDIO (<= 48 MHz).}
00162 \textcolor{comment}{}
00163 \textcolor{comment}{  6. PLLI2S (clocked by HSI or HSE), used to generate an accurate clock to achieve }
00164 \textcolor{comment}{     high-quality audio performance on the I2S interface.}
00165 \textcolor{comment}{  }
00166 \textcolor{comment}{  7. CSS (Clock security system), once enable and if a HSE clock failure occurs }
00167 \textcolor{comment}{     (HSE used directly or through PLL as System clock source), the System clock}
00168 \textcolor{comment}{     is automatically switched to HSI and an interrupt is generated if enabled. }
00169 \textcolor{comment}{     The interrupt is linked to the Cortex-M4 NMI (Non-Maskable Interrupt) }
00170 \textcolor{comment}{     exception vector.   }
00171 \textcolor{comment}{}
00172 \textcolor{comment}{  8. MCO1 (microcontroller clock output), used to output HSI, LSE, HSE or PLL}
00173 \textcolor{comment}{     clock (through a configurable prescaler) on PA8 pin.}
00174 \textcolor{comment}{}
00175 \textcolor{comment}{  9. MCO2 (microcontroller clock output), used to output HSE, PLL, SYSCLK or PLLI2S}
00176 \textcolor{comment}{     clock (through a configurable prescaler) on PC9 pin.}
00177 \textcolor{comment}{}
00178 \textcolor{comment}{@endverbatim}
00179 \textcolor{comment}{  * @\{}
00180 \textcolor{comment}{  */}
00181 
00182 \textcolor{comment}{/**}
00183 \textcolor{comment}{  * @brief  Resets the RCC clock configuration to the default reset state.}
00184 \textcolor{comment}{  * @note   The default reset state of the clock configuration is given below:}
00185 \textcolor{comment}{  *            - HSI ON and used as system clock source}
00186 \textcolor{comment}{  *            - HSE, PLL and PLLI2S OFF}
00187 \textcolor{comment}{  *            - AHB, APB1 and APB2 prescaler set to 1.}
00188 \textcolor{comment}{  *            - CSS, MCO1 and MCO2 OFF}
00189 \textcolor{comment}{  *            - All interrupts disabled}
00190 \textcolor{comment}{  * @note   This function doesn't modify the configuration of the}
00191 \textcolor{comment}{  *            - Peripheral clocks}
00192 \textcolor{comment}{  *            - LSI, LSE and RTC clocks }
00193 \textcolor{comment}{  * @param  None}
00194 \textcolor{comment}{  * @retval None}
00195 \textcolor{comment}{  */}
00196 \textcolor{keywordtype}{void} RCC_DeInit(\textcolor{keywordtype}{void})
00197 \{
00198   \textcolor{comment}{/* Set HSION bit */}
00199   RCC->CR |= (uint32\_t)0x00000001;
00200 
00201   \textcolor{comment}{/* Reset CFGR register */}
00202   RCC->CFGR = 0x00000000;
00203 
00204   \textcolor{comment}{/* Reset HSEON, CSSON and PLLON bits */}
00205   RCC->CR &= (uint32\_t)0xFEF6FFFF;
00206 
00207   \textcolor{comment}{/* Reset PLLCFGR register */}
00208   RCC->PLLCFGR = 0x24003010;
00209 
00210   \textcolor{comment}{/* Reset HSEBYP bit */}
00211   RCC->CR &= (uint32\_t)0xFFFBFFFF;
00212 
00213   \textcolor{comment}{/* Disable all interrupts */}
00214   RCC->CIR = 0x00000000;
00215 \}
00216 
00217 \textcolor{comment}{/**}
00218 \textcolor{comment}{  * @brief  Configures the External High Speed oscillator (HSE).}
00219 \textcolor{comment}{  * @note   After enabling the HSE (RCC\_HSE\_ON or RCC\_HSE\_Bypass), the application}
00220 \textcolor{comment}{  *         software should wait on HSERDY flag to be set indicating that HSE clock}
00221 \textcolor{comment}{  *         is stable and can be used to clock the PLL and/or system clock.}
00222 \textcolor{comment}{  * @note   HSE state can not be changed if it is used directly or through the}
00223 \textcolor{comment}{  *         PLL as system clock. In this case, you have to select another source}
00224 \textcolor{comment}{  *         of the system clock then change the HSE state (ex. disable it).}
00225 \textcolor{comment}{  * @note   The HSE is stopped by hardware when entering STOP and STANDBY modes.  }
00226 \textcolor{comment}{  * @note   This function reset the CSSON bit, so if the Clock security system(CSS)}
00227 \textcolor{comment}{  *         was previously enabled you have to enable it again after calling this}
00228 \textcolor{comment}{  *         function.    }
00229 \textcolor{comment}{  * @param  RCC\_HSE: specifies the new state of the HSE.}
00230 \textcolor{comment}{  *          This parameter can be one of the following values:}
00231 \textcolor{comment}{  *            @arg RCC\_HSE\_OFF: turn OFF the HSE oscillator, HSERDY flag goes low after}
00232 \textcolor{comment}{  *                              6 HSE oscillator clock cycles.}
00233 \textcolor{comment}{  *            @arg RCC\_HSE\_ON: turn ON the HSE oscillator}
00234 \textcolor{comment}{  *            @arg RCC\_HSE\_Bypass: HSE oscillator bypassed with external clock}
00235 \textcolor{comment}{  * @retval None}
00236 \textcolor{comment}{  */}
00237 \textcolor{keywordtype}{void} RCC_HSEConfig(uint8\_t RCC\_HSE)
00238 \{
00239   \textcolor{comment}{/* Check the parameters */}
00240   assert_param(IS\_RCC\_HSE(RCC\_HSE));
00241 
00242   \textcolor{comment}{/* Reset HSEON and HSEBYP bits before configuring the HSE ------------------*/}
00243   *(\_\_IO uint8\_t *) CR_BYTE3_ADDRESS = RCC_HSE_OFF;
00244 
00245   \textcolor{comment}{/* Set the new HSE configuration -------------------------------------------*/}
00246   *(\_\_IO uint8\_t *) CR_BYTE3_ADDRESS = RCC\_HSE;
00247 \}
00248 
00249 \textcolor{comment}{/**}
00250 \textcolor{comment}{  * @brief  Waits for HSE start-up.}
00251 \textcolor{comment}{  * @note   This functions waits on HSERDY flag to be set and return SUCCESS if }
00252 \textcolor{comment}{  *         this flag is set, otherwise returns ERROR if the timeout is reached }
00253 \textcolor{comment}{  *         and this flag is not set. The timeout value is defined by the constant}
00254 \textcolor{comment}{  *         HSE\_STARTUP\_TIMEOUT in stm32f4xx.h file. You can tailor it depending}
00255 \textcolor{comment}{  *         on the HSE crystal used in your application. }
00256 \textcolor{comment}{  * @param  None}
00257 \textcolor{comment}{  * @retval An ErrorStatus enumeration value:}
00258 \textcolor{comment}{  *          - SUCCESS: HSE oscillator is stable and ready to use}
00259 \textcolor{comment}{  *          - ERROR: HSE oscillator not yet ready}
00260 \textcolor{comment}{  */}
00261 ErrorStatus RCC_WaitForHSEStartUp(\textcolor{keywordtype}{void})
00262 \{
00263   \_\_IO uint32\_t startupcounter = 0;
00264   ErrorStatus status = ERROR;
00265   FlagStatus hsestatus = RESET;
00266   \textcolor{comment}{/* Wait till HSE is ready and if Time out is reached exit */}
00267   \textcolor{keywordflow}{do}
00268   \{
00269     hsestatus = RCC\_GetFlagStatus(RCC_FLAG_HSERDY);
00270     startupcounter++;
00271   \} \textcolor{keywordflow}{while}((startupcounter != HSE_STARTUP_TIMEOUT) && (hsestatus == RESET));
00272 
00273   \textcolor{keywordflow}{if} (RCC_GetFlagStatus(RCC_FLAG_HSERDY) != RESET)
00274   \{
00275     status = SUCCESS;
00276   \}
00277   \textcolor{keywordflow}{else}
00278   \{
00279     status = ERROR;
00280   \}
00281   \textcolor{keywordflow}{return} (status);
00282 \}
00283 
00284 \textcolor{comment}{/**}
00285 \textcolor{comment}{  * @brief  Adjusts the Internal High Speed oscillator (HSI) calibration value.}
00286 \textcolor{comment}{  * @note   The calibration is used to compensate for the variations in voltage}
00287 \textcolor{comment}{  *         and temperature that influence the frequency of the internal HSI RC.}
00288 \textcolor{comment}{  * @param  HSICalibrationValue: specifies the calibration trimming value.}
00289 \textcolor{comment}{  *         This parameter must be a number between 0 and 0x1F.}
00290 \textcolor{comment}{  * @retval None}
00291 \textcolor{comment}{  */}
00292 \textcolor{keywordtype}{void} RCC_AdjustHSICalibrationValue(uint8\_t HSICalibrationValue)
00293 \{
00294   uint32\_t tmpreg = 0;
00295   \textcolor{comment}{/* Check the parameters */}
00296   assert_param(IS\_RCC\_CALIBRATION\_VALUE(HSICalibrationValue));
00297 
00298   tmpreg = RCC->CR;
00299 
00300   \textcolor{comment}{/* Clear HSITRIM[4:0] bits */}
00301   tmpreg &= ~RCC_CR_HSITRIM;
00302 
00303   \textcolor{comment}{/* Set the HSITRIM[4:0] bits according to HSICalibrationValue value */}
00304   tmpreg |= (uint32\_t)HSICalibrationValue << 3;
00305 
00306   \textcolor{comment}{/* Store the new value */}
00307   RCC->CR = tmpreg;
00308 \}
00309 
00310 \textcolor{comment}{/**}
00311 \textcolor{comment}{  * @brief  Enables or disables the Internal High Speed oscillator (HSI).}
00312 \textcolor{comment}{  * @note   The HSI is stopped by hardware when entering STOP and STANDBY modes.}
00313 \textcolor{comment}{  *         It is used (enabled by hardware) as system clock source after startup}
00314 \textcolor{comment}{  *         from Reset, wakeup from STOP and STANDBY mode, or in case of failure}
00315 \textcolor{comment}{  *         of the HSE used directly or indirectly as system clock (if the Clock}
00316 \textcolor{comment}{  *         Security System CSS is enabled).             }
00317 \textcolor{comment}{  * @note   HSI can not be stopped if it is used as system clock source. In this case,}
00318 \textcolor{comment}{  *         you have to select another source of the system clock then stop the HSI.  }
00319 \textcolor{comment}{  * @note   After enabling the HSI, the application software should wait on HSIRDY}
00320 \textcolor{comment}{  *         flag to be set indicating that HSI clock is stable and can be used as}
00321 \textcolor{comment}{  *         system clock source.  }
00322 \textcolor{comment}{  * @param  NewState: new state of the HSI.}
00323 \textcolor{comment}{  *          This parameter can be: ENABLE or DISABLE.}
00324 \textcolor{comment}{  * @note   When the HSI is stopped, HSIRDY flag goes low after 6 HSI oscillator}
00325 \textcolor{comment}{  *         clock cycles.  }
00326 \textcolor{comment}{  * @retval None}
00327 \textcolor{comment}{  */}
00328 \textcolor{keywordtype}{void} RCC_HSICmd(FunctionalState NewState)
00329 \{
00330   \textcolor{comment}{/* Check the parameters */}
00331   assert_param(IS\_FUNCTIONAL\_STATE(NewState));
00332 
00333   *(\_\_IO uint32\_t *) CR_HSION_BB = (uint32\_t)NewState;
00334 \}
00335 
00336 \textcolor{comment}{/**}
00337 \textcolor{comment}{  * @brief  Configures the External Low Speed oscillator (LSE).}
00338 \textcolor{comment}{  * @note   As the LSE is in the Backup domain and write access is denied to}
00339 \textcolor{comment}{  *         this domain after reset, you have to enable write access using }
00340 \textcolor{comment}{  *         PWR\_BackupAccessCmd(ENABLE) function before to configure the LSE}
00341 \textcolor{comment}{  *         (to be done once after reset).  }
00342 \textcolor{comment}{  * @note   After enabling the LSE (RCC\_LSE\_ON or RCC\_LSE\_Bypass), the application}
00343 \textcolor{comment}{  *         software should wait on LSERDY flag to be set indicating that LSE clock}
00344 \textcolor{comment}{  *         is stable and can be used to clock the RTC.}
00345 \textcolor{comment}{  * @param  RCC\_LSE: specifies the new state of the LSE.}
00346 \textcolor{comment}{  *          This parameter can be one of the following values:}
00347 \textcolor{comment}{  *            @arg RCC\_LSE\_OFF: turn OFF the LSE oscillator, LSERDY flag goes low after}
00348 \textcolor{comment}{  *                              6 LSE oscillator clock cycles.}
00349 \textcolor{comment}{  *            @arg RCC\_LSE\_ON: turn ON the LSE oscillator}
00350 \textcolor{comment}{  *            @arg RCC\_LSE\_Bypass: LSE oscillator bypassed with external clock}
00351 \textcolor{comment}{  * @retval None}
00352 \textcolor{comment}{  */}
00353 \textcolor{keywordtype}{void} RCC_LSEConfig(uint8\_t RCC\_LSE)
00354 \{
00355   \textcolor{comment}{/* Check the parameters */}
00356   assert_param(IS\_RCC\_LSE(RCC\_LSE));
00357 
00358   \textcolor{comment}{/* Reset LSEON and LSEBYP bits before configuring the LSE ------------------*/}
00359   \textcolor{comment}{/* Reset LSEON bit */}
00360   *(\_\_IO uint8\_t *) BDCR_ADDRESS = RCC_LSE_OFF;
00361 
00362   \textcolor{comment}{/* Reset LSEBYP bit */}
00363   *(\_\_IO uint8\_t *) BDCR_ADDRESS = RCC_LSE_OFF;
00364 
00365   \textcolor{comment}{/* Configure LSE (RCC\_LSE\_OFF is already covered by the code section above) */}
00366   \textcolor{keywordflow}{switch} (RCC\_LSE)
00367   \{
00368     \textcolor{keywordflow}{case} RCC_LSE_ON:
00369       \textcolor{comment}{/* Set LSEON bit */}
00370       *(\_\_IO uint8\_t *) BDCR_ADDRESS = RCC_LSE_ON;
00371       \textcolor{keywordflow}{break};
00372     \textcolor{keywordflow}{case} RCC_LSE_Bypass:
00373       \textcolor{comment}{/* Set LSEBYP and LSEON bits */}
00374       *(\_\_IO uint8\_t *) BDCR_ADDRESS = RCC_LSE_Bypass | RCC_LSE_ON;
00375       \textcolor{keywordflow}{break};
00376     \textcolor{keywordflow}{default}:
00377       \textcolor{keywordflow}{break};
00378   \}
00379 \}
00380 
00381 \textcolor{comment}{/**}
00382 \textcolor{comment}{  * @brief  Enables or disables the Internal Low Speed oscillator (LSI).}
00383 \textcolor{comment}{  * @note   After enabling the LSI, the application software should wait on }
00384 \textcolor{comment}{  *         LSIRDY flag to be set indicating that LSI clock is stable and can}
00385 \textcolor{comment}{  *         be used to clock the IWDG and/or the RTC.}
00386 \textcolor{comment}{  * @note   LSI can not be disabled if the IWDG is running.  }
00387 \textcolor{comment}{  * @param  NewState: new state of the LSI.}
00388 \textcolor{comment}{  *          This parameter can be: ENABLE or DISABLE.}
00389 \textcolor{comment}{  * @note   When the LSI is stopped, LSIRDY flag goes low after 6 LSI oscillator}
00390 \textcolor{comment}{  *         clock cycles. }
00391 \textcolor{comment}{  * @retval None}
00392 \textcolor{comment}{  */}
00393 \textcolor{keywordtype}{void} RCC_LSICmd(FunctionalState NewState)
00394 \{
00395   \textcolor{comment}{/* Check the parameters */}
00396   assert_param(IS\_FUNCTIONAL\_STATE(NewState));
00397 
00398   *(\_\_IO uint32\_t *) CSR_LSION_BB = (uint32\_t)NewState;
00399 \}
00400 
00401 \textcolor{comment}{/**}
00402 \textcolor{comment}{  * @brief  Configures the main PLL clock source, multiplication and division factors.}
00403 \textcolor{comment}{  * @note   This function must be used only when the main PLL is disabled.}
00404 \textcolor{comment}{  *  }
00405 \textcolor{comment}{  * @param  RCC\_PLLSource: specifies the PLL entry clock source.}
00406 \textcolor{comment}{  *          This parameter can be one of the following values:}
00407 \textcolor{comment}{  *            @arg RCC\_PLLSource\_HSI: HSI oscillator clock selected as PLL clock entry}
00408 \textcolor{comment}{  *            @arg RCC\_PLLSource\_HSE: HSE oscillator clock selected as PLL clock entry}
00409 \textcolor{comment}{  * @note   This clock source (RCC\_PLLSource) is common for the main PLL and PLLI2S.  }
00410 \textcolor{comment}{  *  }
00411 \textcolor{comment}{  * @param  PLLM: specifies the division factor for PLL VCO input clock}
00412 \textcolor{comment}{  *          This parameter must be a number between 0 and 63.}
00413 \textcolor{comment}{  * @note   You have to set the PLLM parameter correctly to ensure that the VCO input}
00414 \textcolor{comment}{  *         frequency ranges from 1 to 2 MHz. It is recommended to select a frequency}
00415 \textcolor{comment}{  *         of 2 MHz to limit PLL jitter.}
00416 \textcolor{comment}{  *  }
00417 \textcolor{comment}{  * @param  PLLN: specifies the multiplication factor for PLL VCO output clock}
00418 \textcolor{comment}{  *          This parameter must be a number between 192 and 432.}
00419 \textcolor{comment}{  * @note   You have to set the PLLN parameter correctly to ensure that the VCO}
00420 \textcolor{comment}{  *         output frequency is between 192 and 432 MHz.}
00421 \textcolor{comment}{  *   }
00422 \textcolor{comment}{  * @param  PLLP: specifies the division factor for main system clock (SYSCLK)}
00423 \textcolor{comment}{  *          This parameter must be a number in the range \{2, 4, 6, or 8\}.}
00424 \textcolor{comment}{  * @note   You have to set the PLLP parameter correctly to not exceed 168 MHz on}
00425 \textcolor{comment}{  *         the System clock frequency.}
00426 \textcolor{comment}{  *  }
00427 \textcolor{comment}{  * @param  PLLQ: specifies the division factor for OTG FS, SDIO and RNG clocks}
00428 \textcolor{comment}{  *          This parameter must be a number between 4 and 15.}
00429 \textcolor{comment}{  * @note   If the USB OTG FS is used in your application, you have to set the}
00430 \textcolor{comment}{  *         PLLQ parameter correctly to have 48 MHz clock for the USB. However,}
00431 \textcolor{comment}{  *         the SDIO and RNG need a frequency lower than or equal to 48 MHz to work}
00432 \textcolor{comment}{  *         correctly.}
00433 \textcolor{comment}{  *   }
00434 \textcolor{comment}{  * @retval None}
00435 \textcolor{comment}{  */}
00436 \textcolor{keywordtype}{void} RCC_PLLConfig(uint32\_t RCC\_PLLSource, uint32\_t PLLM, uint32\_t PLLN, uint32\_t PLLP, uint32\_t PLLQ)
00437 \{
00438   \textcolor{comment}{/* Check the parameters */}
00439   assert_param(IS\_RCC\_PLL\_SOURCE(RCC\_PLLSource));
00440   assert_param(IS\_RCC\_PLLM\_VALUE(PLLM));
00441   assert_param(IS\_RCC\_PLLN\_VALUE(PLLN));
00442   assert_param(IS\_RCC\_PLLP\_VALUE(PLLP));
00443   assert_param(IS\_RCC\_PLLQ\_VALUE(PLLQ));
00444 
00445   RCC->PLLCFGR = PLLM | (PLLN << 6) | (((PLLP >> 1) -1) << 16) | (RCC\_PLLSource) |
00446                  (PLLQ << 24);
00447 \}
00448 
00449 \textcolor{comment}{/**}
00450 \textcolor{comment}{  * @brief  Enables or disables the main PLL.}
00451 \textcolor{comment}{  * @note   After enabling the main PLL, the application software should wait on }
00452 \textcolor{comment}{  *         PLLRDY flag to be set indicating that PLL clock is stable and can}
00453 \textcolor{comment}{  *         be used as system clock source.}
00454 \textcolor{comment}{  * @note   The main PLL can not be disabled if it is used as system clock source}
00455 \textcolor{comment}{  * @note   The main PLL is disabled by hardware when entering STOP and STANDBY modes.}
00456 \textcolor{comment}{  * @param  NewState: new state of the main PLL. This parameter can be: ENABLE or DISABLE.}
00457 \textcolor{comment}{  * @retval None}
00458 \textcolor{comment}{  */}
00459 \textcolor{keywordtype}{void} RCC_PLLCmd(FunctionalState NewState)
00460 \{
00461   \textcolor{comment}{/* Check the parameters */}
00462   assert_param(IS\_FUNCTIONAL\_STATE(NewState));
00463   *(\_\_IO uint32\_t *) CR_PLLON_BB = (uint32\_t)NewState;
00464 \}
00465 
00466 \textcolor{comment}{/**}
00467 \textcolor{comment}{  * @brief  Configures the PLLI2S clock multiplication and division factors.}
00468 \textcolor{comment}{  *  }
00469 \textcolor{comment}{  * @note   This function must be used only when the PLLI2S is disabled.}
00470 \textcolor{comment}{  * @note   PLLI2S clock source is common with the main PLL (configured in }
00471 \textcolor{comment}{  *         RCC\_PLLConfig function )  }
00472 \textcolor{comment}{  *             }
00473 \textcolor{comment}{  * @param  PLLI2SN: specifies the multiplication factor for PLLI2S VCO output clock}
00474 \textcolor{comment}{  *          This parameter must be a number between 192 and 432.}
00475 \textcolor{comment}{  * @note   You have to set the PLLI2SN parameter correctly to ensure that the VCO }
00476 \textcolor{comment}{  *         output frequency is between 192 and 432 MHz.}
00477 \textcolor{comment}{  *    }
00478 \textcolor{comment}{  * @param  PLLI2SR: specifies the division factor for I2S clock}
00479 \textcolor{comment}{  *          This parameter must be a number between 2 and 7.}
00480 \textcolor{comment}{  * @note   You have to set the PLLI2SR parameter correctly to not exceed 192 MHz}
00481 \textcolor{comment}{  *         on the I2S clock frequency.}
00482 \textcolor{comment}{  *   }
00483 \textcolor{comment}{  * @retval None}
00484 \textcolor{comment}{  */}
00485 \textcolor{keywordtype}{void} RCC_PLLI2SConfig(uint32\_t PLLI2SN, uint32\_t PLLI2SR)
00486 \{
00487   \textcolor{comment}{/* Check the parameters */}
00488   assert_param(IS\_RCC\_PLLI2SN\_VALUE(PLLI2SN));
00489   assert_param(IS\_RCC\_PLLI2SR\_VALUE(PLLI2SR));
00490 
00491   RCC->PLLI2SCFGR = (PLLI2SN << 6) | (PLLI2SR << 28);
00492 \}
00493 
00494 \textcolor{comment}{/**}
00495 \textcolor{comment}{  * @brief  Enables or disables the PLLI2S. }
00496 \textcolor{comment}{  * @note   The PLLI2S is disabled by hardware when entering STOP and STANDBY modes.  }
00497 \textcolor{comment}{  * @param  NewState: new state of the PLLI2S. This parameter can be: ENABLE or DISABLE.}
00498 \textcolor{comment}{  * @retval None}
00499 \textcolor{comment}{  */}
00500 \textcolor{keywordtype}{void} RCC_PLLI2SCmd(FunctionalState NewState)
00501 \{
00502   \textcolor{comment}{/* Check the parameters */}
00503   assert_param(IS\_FUNCTIONAL\_STATE(NewState));
00504   *(\_\_IO uint32\_t *) CR_PLLI2SON_BB = (uint32\_t)NewState;
00505 \}
00506 
00507 \textcolor{comment}{/**}
00508 \textcolor{comment}{  * @brief  Enables or disables the Clock Security System.}
00509 \textcolor{comment}{  * @note   If a failure is detected on the HSE oscillator clock, this oscillator}
00510 \textcolor{comment}{  *         is automatically disabled and an interrupt is generated to inform the}
00511 \textcolor{comment}{  *         software about the failure (Clock Security System Interrupt, CSSI),}
00512 \textcolor{comment}{  *         allowing the MCU to perform rescue operations. The CSSI is linked to }
00513 \textcolor{comment}{  *         the Cortex-M4 NMI (Non-Maskable Interrupt) exception vector.  }
00514 \textcolor{comment}{  * @param  NewState: new state of the Clock Security System.}
00515 \textcolor{comment}{  *         This parameter can be: ENABLE or DISABLE.}
00516 \textcolor{comment}{  * @retval None}
00517 \textcolor{comment}{  */}
00518 \textcolor{keywordtype}{void} RCC_ClockSecuritySystemCmd(FunctionalState NewState)
00519 \{
00520   \textcolor{comment}{/* Check the parameters */}
00521   assert_param(IS\_FUNCTIONAL\_STATE(NewState));
00522   *(\_\_IO uint32\_t *) CR_CSSON_BB = (uint32\_t)NewState;
00523 \}
00524 
00525 \textcolor{comment}{/**}
00526 \textcolor{comment}{  * @brief  Selects the clock source to output on MCO1 pin(PA8).}
00527 \textcolor{comment}{  * @note   PA8 should be configured in alternate function mode.}
00528 \textcolor{comment}{  * @param  RCC\_MCO1Source: specifies the clock source to output.}
00529 \textcolor{comment}{  *          This parameter can be one of the following values:}
00530 \textcolor{comment}{  *            @arg RCC\_MCO1Source\_HSI: HSI clock selected as MCO1 source}
00531 \textcolor{comment}{  *            @arg RCC\_MCO1Source\_LSE: LSE clock selected as MCO1 source}
00532 \textcolor{comment}{  *            @arg RCC\_MCO1Source\_HSE: HSE clock selected as MCO1 source}
00533 \textcolor{comment}{  *            @arg RCC\_MCO1Source\_PLLCLK: main PLL clock selected as MCO1 source}
00534 \textcolor{comment}{  * @param  RCC\_MCO1Div: specifies the MCO1 prescaler.}
00535 \textcolor{comment}{  *          This parameter can be one of the following values:}
00536 \textcolor{comment}{  *            @arg RCC\_MCO1Div\_1: no division applied to MCO1 clock}
00537 \textcolor{comment}{  *            @arg RCC\_MCO1Div\_2: division by 2 applied to MCO1 clock}
00538 \textcolor{comment}{  *            @arg RCC\_MCO1Div\_3: division by 3 applied to MCO1 clock}
00539 \textcolor{comment}{  *            @arg RCC\_MCO1Div\_4: division by 4 applied to MCO1 clock}
00540 \textcolor{comment}{  *            @arg RCC\_MCO1Div\_5: division by 5 applied to MCO1 clock}
00541 \textcolor{comment}{  * @retval None}
00542 \textcolor{comment}{  */}
00543 \textcolor{keywordtype}{void} RCC_MCO1Config(uint32\_t RCC\_MCO1Source, uint32\_t RCC\_MCO1Div)
00544 \{
00545   uint32\_t tmpreg = 0;
00546 
00547   \textcolor{comment}{/* Check the parameters */}
00548   assert_param(IS\_RCC\_MCO1SOURCE(RCC\_MCO1Source));
00549   assert_param(IS\_RCC\_MCO1DIV(RCC\_MCO1Div));
00550 
00551   tmpreg = RCC->CFGR;
00552 
00553   \textcolor{comment}{/* Clear MCO1[1:0] and MCO1PRE[2:0] bits */}
00554   tmpreg &= CFGR_MCO1_RESET_MASK;
00555 
00556   \textcolor{comment}{/* Select MCO1 clock source and prescaler */}
00557   tmpreg |= RCC\_MCO1Source | RCC\_MCO1Div;
00558 
00559   \textcolor{comment}{/* Store the new value */}
00560   RCC->CFGR = tmpreg;
00561 \}
00562 
00563 \textcolor{comment}{/**}
00564 \textcolor{comment}{  * @brief  Selects the clock source to output on MCO2 pin(PC9).}
00565 \textcolor{comment}{  * @note   PC9 should be configured in alternate function mode.}
00566 \textcolor{comment}{  * @param  RCC\_MCO2Source: specifies the clock source to output.}
00567 \textcolor{comment}{  *          This parameter can be one of the following values:}
00568 \textcolor{comment}{  *            @arg RCC\_MCO2Source\_SYSCLK: System clock (SYSCLK) selected as MCO2 source}
00569 \textcolor{comment}{  *            @arg RCC\_MCO2Source\_PLLI2SCLK: PLLI2S clock selected as MCO2 source}
00570 \textcolor{comment}{  *            @arg RCC\_MCO2Source\_HSE: HSE clock selected as MCO2 source}
00571 \textcolor{comment}{  *            @arg RCC\_MCO2Source\_PLLCLK: main PLL clock selected as MCO2 source}
00572 \textcolor{comment}{  * @param  RCC\_MCO2Div: specifies the MCO2 prescaler.}
00573 \textcolor{comment}{  *          This parameter can be one of the following values:}
00574 \textcolor{comment}{  *            @arg RCC\_MCO2Div\_1: no division applied to MCO2 clock}
00575 \textcolor{comment}{  *            @arg RCC\_MCO2Div\_2: division by 2 applied to MCO2 clock}
00576 \textcolor{comment}{  *            @arg RCC\_MCO2Div\_3: division by 3 applied to MCO2 clock}
00577 \textcolor{comment}{  *            @arg RCC\_MCO2Div\_4: division by 4 applied to MCO2 clock}
00578 \textcolor{comment}{  *            @arg RCC\_MCO2Div\_5: division by 5 applied to MCO2 clock}
00579 \textcolor{comment}{  * @retval None}
00580 \textcolor{comment}{  */}
00581 \textcolor{keywordtype}{void} RCC_MCO2Config(uint32\_t RCC\_MCO2Source, uint32\_t RCC\_MCO2Div)
00582 \{
00583   uint32\_t tmpreg = 0;
00584 
00585   \textcolor{comment}{/* Check the parameters */}
00586   assert_param(IS\_RCC\_MCO2SOURCE(RCC\_MCO2Source));
00587   assert_param(IS\_RCC\_MCO2DIV(RCC\_MCO2Div));
00588 
00589   tmpreg = RCC->CFGR;
00590 
00591   \textcolor{comment}{/* Clear MCO2 and MCO2PRE[2:0] bits */}
00592   tmpreg &= CFGR_MCO2_RESET_MASK;
00593 
00594   \textcolor{comment}{/* Select MCO2 clock source and prescaler */}
00595   tmpreg |= RCC\_MCO2Source | RCC\_MCO2Div;
00596 
00597   \textcolor{comment}{/* Store the new value */}
00598   RCC->CFGR = tmpreg;
00599 \}
00600 
00601 \textcolor{comment}{/**}
00602 \textcolor{comment}{  * @\}}
00603 \textcolor{comment}{  */}
00604 
00605 \textcolor{comment}{/** @defgroup RCC\_Group2 System AHB and APB busses clocks configuration functions}
00606 \textcolor{comment}{ *  @brief   System, AHB and APB busses clocks configuration functions}
00607 \textcolor{comment}{ *}
00608 \textcolor{comment}{@verbatim   }
00609 \textcolor{comment}{ ===============================================================================}
00610 \textcolor{comment}{             System, AHB and APB busses clocks configuration functions}
00611 \textcolor{comment}{ ===============================================================================  }
00612 \textcolor{comment}{}
00613 \textcolor{comment}{  This section provide functions allowing to configure the System, AHB, APB1 and }
00614 \textcolor{comment}{  APB2 busses clocks.}
00615 \textcolor{comment}{  }
00616 \textcolor{comment}{  1. Several clock sources can be used to drive the System clock (SYSCLK): HSI,}
00617 \textcolor{comment}{     HSE and PLL.}
00618 \textcolor{comment}{     The AHB clock (HCLK) is derived from System clock through configurable prescaler}
00619 \textcolor{comment}{     and used to clock the CPU, memory and peripherals mapped on AHB bus (DMA, GPIO...).}
00620 \textcolor{comment}{     APB1 (PCLK1) and APB2 (PCLK2) clocks are derived from AHB clock through }
00621 \textcolor{comment}{     configurable prescalers and used to clock the peripherals mapped on these busses.}
00622 \textcolor{comment}{     You can use "RCC\_GetClocksFreq()" function to retrieve the frequencies of these clocks.  }
00623 \textcolor{comment}{}
00624 \textcolor{comment}{@note All the peripheral clocks are derived from the System clock (SYSCLK) except:}
00625 \textcolor{comment}{       - I2S: the I2S clock can be derived either from a specific PLL (PLLI2S) or}
00626 \textcolor{comment}{          from an external clock mapped on the I2S\_CKIN pin. }
00627 \textcolor{comment}{          You have to use RCC\_I2SCLKConfig() function to configure this clock. }
00628 \textcolor{comment}{       - RTC: the RTC clock can be derived either from the LSI, LSE or HSE clock}
00629 \textcolor{comment}{          divided by 2 to 31. You have to use RCC\_RTCCLKConfig() and RCC\_RTCCLKCmd()}
00630 \textcolor{comment}{          functions to configure this clock. }
00631 \textcolor{comment}{       - USB OTG FS, SDIO and RTC: USB OTG FS require a frequency equal to 48 MHz}
00632 \textcolor{comment}{          to work correctly, while the SDIO require a frequency equal or lower than}
00633 \textcolor{comment}{          to 48. This clock is derived of the main PLL through PLLQ divider.}
00634 \textcolor{comment}{       - IWDG clock which is always the LSI clock.}
00635 \textcolor{comment}{       }
00636 \textcolor{comment}{  2. The maximum frequency of the SYSCLK and HCLK is 168 MHz, PCLK2 82 MHz and PCLK1 42 MHz.}
00637 \textcolor{comment}{     Depending on the device voltage range, the maximum frequency should be }
00638 \textcolor{comment}{     adapted accordingly:}
00639 \textcolor{comment}{ +-------------------------------------------------------------------------------------+     }
00640 \textcolor{comment}{ | Latency       |                HCLK clock frequency (MHz)                           |}
00641 \textcolor{comment}{ |               |---------------------------------------------------------------------|     }
00642 \textcolor{comment}{ |               | voltage range  | voltage range  | voltage range   | voltage range   |}
00643 \textcolor{comment}{ |               | 2.7 V - 3.6 V  | 2.4 V - 2.7 V  | 2.1 V - 2.4 V   | 1.8 V - 2.1 V   |}
00644 \textcolor{comment}{ |---------------|----------------|----------------|-----------------|-----------------|              }
00645 \textcolor{comment}{ |0WS(1CPU cycle)|0 < HCLK <= 30  |0 < HCLK <= 24  |0 < HCLK <= 18   |0 < HCLK <= 16   |}
00646 \textcolor{comment}{ |---------------|----------------|----------------|-----------------|-----------------|   }
00647 \textcolor{comment}{ |1WS(2CPU cycle)|30 < HCLK <= 60 |24 < HCLK <= 48 |18 < HCLK <= 36  |16 < HCLK <= 32  | }
00648 \textcolor{comment}{ |---------------|----------------|----------------|-----------------|-----------------|   }
00649 \textcolor{comment}{ |2WS(3CPU cycle)|60 < HCLK <= 90 |48 < HCLK <= 72 |36 < HCLK <= 54  |32 < HCLK <= 48  |}
00650 \textcolor{comment}{ |---------------|----------------|----------------|-----------------|-----------------| }
00651 \textcolor{comment}{ |3WS(4CPU cycle)|90 < HCLK <= 120|72 < HCLK <= 96 |54 < HCLK <= 72  |48 < HCLK <= 64  |}
00652 \textcolor{comment}{ |---------------|----------------|----------------|-----------------|-----------------| }
00653 \textcolor{comment}{ |4WS(5CPU cycle)|120< HCLK <= 150|96 < HCLK <= 120|72 < HCLK <= 90  |64 < HCLK <= 80  |}
00654 \textcolor{comment}{ |---------------|----------------|----------------|-----------------|-----------------| }
00655 \textcolor{comment}{ |5WS(6CPU cycle)|120< HCLK <= 168|120< HCLK <= 144|90 < HCLK <= 108 |80 < HCLK <= 96  | }
00656 \textcolor{comment}{ |---------------|----------------|----------------|-----------------|-----------------| }
00657 \textcolor{comment}{ |6WS(7CPU cycle)|      NA        |144< HCLK <= 168|108 < HCLK <= 120|96 < HCLK <= 112 | }
00658 \textcolor{comment}{ |---------------|----------------|----------------|-----------------|-----------------| }
00659 \textcolor{comment}{ |7WS(8CPU cycle)|      NA        |      NA        |120 < HCLK <= 138|112 < HCLK <= 120| }
00660 \textcolor{comment}{ +-------------------------------------------------------------------------------------+    }
00661 \textcolor{comment}{   @note When VOS bit (in PWR\_CR register) is reset to '0�, the maximum value of HCLK is 144 MHz.}
00662 \textcolor{comment}{         You can use PWR\_MainRegulatorModeConfig() function to set or reset this bit.}
00663 \textcolor{comment}{}
00664 \textcolor{comment}{@endverbatim}
00665 \textcolor{comment}{  * @\{}
00666 \textcolor{comment}{  */}
00667 
00668 \textcolor{comment}{/**}
00669 \textcolor{comment}{  * @brief  Configures the system clock (SYSCLK).}
00670 \textcolor{comment}{  * @note   The HSI is used (enabled by hardware) as system clock source after}
00671 \textcolor{comment}{  *         startup from Reset, wake-up from STOP and STANDBY mode, or in case}
00672 \textcolor{comment}{  *         of failure of the HSE used directly or indirectly as system clock}
00673 \textcolor{comment}{  *         (if the Clock Security System CSS is enabled).}
00674 \textcolor{comment}{  * @note   A switch from one clock source to another occurs only if the target}
00675 \textcolor{comment}{  *         clock source is ready (clock stable after startup delay or PLL locked). }
00676 \textcolor{comment}{  *         If a clock source which is not yet ready is selected, the switch will}
00677 \textcolor{comment}{  *         occur when the clock source will be ready. }
00678 \textcolor{comment}{  *         You can use RCC\_GetSYSCLKSource() function to know which clock is}
00679 \textcolor{comment}{  *         currently used as system clock source. }
00680 \textcolor{comment}{  * @param  RCC\_SYSCLKSource: specifies the clock source used as system clock.}
00681 \textcolor{comment}{  *          This parameter can be one of the following values:}
00682 \textcolor{comment}{  *            @arg RCC\_SYSCLKSource\_HSI:    HSI selected as system clock source}
00683 \textcolor{comment}{  *            @arg RCC\_SYSCLKSource\_HSE:    HSE selected as system clock source}
00684 \textcolor{comment}{  *            @arg RCC\_SYSCLKSource\_PLLCLK: PLL selected as system clock source}
00685 \textcolor{comment}{  * @retval None}
00686 \textcolor{comment}{  */}
00687 \textcolor{keywordtype}{void} RCC_SYSCLKConfig(uint32\_t RCC\_SYSCLKSource)
00688 \{
00689   uint32\_t tmpreg = 0;
00690 
00691   \textcolor{comment}{/* Check the parameters */}
00692   assert_param(IS\_RCC\_SYSCLK\_SOURCE(RCC\_SYSCLKSource));
00693 
00694   tmpreg = RCC->CFGR;
00695 
00696   \textcolor{comment}{/* Clear SW[1:0] bits */}
00697   tmpreg &= ~RCC_CFGR_SW;
00698 
00699   \textcolor{comment}{/* Set SW[1:0] bits according to RCC\_SYSCLKSource value */}
00700   tmpreg |= RCC\_SYSCLKSource;
00701 
00702   \textcolor{comment}{/* Store the new value */}
00703   RCC->CFGR = tmpreg;
00704 \}
00705 
00706 \textcolor{comment}{/**}
00707 \textcolor{comment}{  * @brief  Returns the clock source used as system clock.}
00708 \textcolor{comment}{  * @param  None}
00709 \textcolor{comment}{  * @retval The clock source used as system clock. The returned value can be one}
00710 \textcolor{comment}{  *         of the following:}
00711 \textcolor{comment}{  *              - 0x00: HSI used as system clock}
00712 \textcolor{comment}{  *              - 0x04: HSE used as system clock}
00713 \textcolor{comment}{  *              - 0x08: PLL used as system clock}
00714 \textcolor{comment}{  */}
00715 uint8\_t RCC_GetSYSCLKSource(\textcolor{keywordtype}{void})
00716 \{
00717   \textcolor{keywordflow}{return} ((uint8\_t)(RCC->CFGR & RCC_CFGR_SWS));
00718 \}
00719 
00720 \textcolor{comment}{/**}
00721 \textcolor{comment}{  * @brief  Configures the AHB clock (HCLK).}
00722 \textcolor{comment}{  * @note   Depending on the device voltage range, the software has to set correctly}
00723 \textcolor{comment}{  *         these bits to ensure that HCLK not exceed the maximum allowed frequency}
00724 \textcolor{comment}{  *         (for more details refer to section above}
00725 \textcolor{comment}{  *           "CPU, AHB and APB busses clocks configuration functions")}
00726 \textcolor{comment}{  * @param  RCC\_SYSCLK: defines the AHB clock divider. This clock is derived from }
00727 \textcolor{comment}{  *         the system clock (SYSCLK).}
00728 \textcolor{comment}{  *          This parameter can be one of the following values:}
00729 \textcolor{comment}{  *            @arg RCC\_SYSCLK\_Div1: AHB clock = SYSCLK}
00730 \textcolor{comment}{  *            @arg RCC\_SYSCLK\_Div2: AHB clock = SYSCLK/2}
00731 \textcolor{comment}{  *            @arg RCC\_SYSCLK\_Div4: AHB clock = SYSCLK/4}
00732 \textcolor{comment}{  *            @arg RCC\_SYSCLK\_Div8: AHB clock = SYSCLK/8}
00733 \textcolor{comment}{  *            @arg RCC\_SYSCLK\_Div16: AHB clock = SYSCLK/16}
00734 \textcolor{comment}{  *            @arg RCC\_SYSCLK\_Div64: AHB clock = SYSCLK/64}
00735 \textcolor{comment}{  *            @arg RCC\_SYSCLK\_Div128: AHB clock = SYSCLK/128}
00736 \textcolor{comment}{  *            @arg RCC\_SYSCLK\_Div256: AHB clock = SYSCLK/256}
00737 \textcolor{comment}{  *            @arg RCC\_SYSCLK\_Div512: AHB clock = SYSCLK/512}
00738 \textcolor{comment}{  * @retval None}
00739 \textcolor{comment}{  */}
00740 \textcolor{keywordtype}{void} RCC_HCLKConfig(uint32\_t RCC\_SYSCLK)
00741 \{
00742   uint32\_t tmpreg = 0;
00743 
00744   \textcolor{comment}{/* Check the parameters */}
00745   assert_param(IS\_RCC\_HCLK(RCC\_SYSCLK));
00746 
00747   tmpreg = RCC->CFGR;
00748 
00749   \textcolor{comment}{/* Clear HPRE[3:0] bits */}
00750   tmpreg &= ~RCC_CFGR_HPRE;
00751 
00752   \textcolor{comment}{/* Set HPRE[3:0] bits according to RCC\_SYSCLK value */}
00753   tmpreg |= RCC\_SYSCLK;
00754 
00755   \textcolor{comment}{/* Store the new value */}
00756   RCC->CFGR = tmpreg;
00757 \}
00758 
00759 
00760 \textcolor{comment}{/**}
00761 \textcolor{comment}{  * @brief  Configures the Low Speed APB clock (PCLK1).}
00762 \textcolor{comment}{  * @param  RCC\_HCLK: defines the APB1 clock divider. This clock is derived from }
00763 \textcolor{comment}{  *         the AHB clock (HCLK).}
00764 \textcolor{comment}{  *          This parameter can be one of the following values:}
00765 \textcolor{comment}{  *            @arg RCC\_HCLK\_Div1:  APB1 clock = HCLK}
00766 \textcolor{comment}{  *            @arg RCC\_HCLK\_Div2:  APB1 clock = HCLK/2}
00767 \textcolor{comment}{  *            @arg RCC\_HCLK\_Div4:  APB1 clock = HCLK/4}
00768 \textcolor{comment}{  *            @arg RCC\_HCLK\_Div8:  APB1 clock = HCLK/8}
00769 \textcolor{comment}{  *            @arg RCC\_HCLK\_Div16: APB1 clock = HCLK/16}
00770 \textcolor{comment}{  * @retval None}
00771 \textcolor{comment}{  */}
00772 \textcolor{keywordtype}{void} RCC_PCLK1Config(uint32\_t RCC\_HCLK)
00773 \{
00774   uint32\_t tmpreg = 0;
00775 
00776   \textcolor{comment}{/* Check the parameters */}
00777   assert_param(IS\_RCC\_PCLK(RCC\_HCLK));
00778 
00779   tmpreg = RCC->CFGR;
00780 
00781   \textcolor{comment}{/* Clear PPRE1[2:0] bits */}
00782   tmpreg &= ~RCC_CFGR_PPRE1;
00783 
00784   \textcolor{comment}{/* Set PPRE1[2:0] bits according to RCC\_HCLK value */}
00785   tmpreg |= RCC\_HCLK;
00786 
00787   \textcolor{comment}{/* Store the new value */}
00788   RCC->CFGR = tmpreg;
00789 \}
00790 
00791 \textcolor{comment}{/**}
00792 \textcolor{comment}{  * @brief  Configures the High Speed APB clock (PCLK2).}
00793 \textcolor{comment}{  * @param  RCC\_HCLK: defines the APB2 clock divider. This clock is derived from }
00794 \textcolor{comment}{  *         the AHB clock (HCLK).}
00795 \textcolor{comment}{  *          This parameter can be one of the following values:}
00796 \textcolor{comment}{  *            @arg RCC\_HCLK\_Div1:  APB2 clock = HCLK}
00797 \textcolor{comment}{  *            @arg RCC\_HCLK\_Div2:  APB2 clock = HCLK/2}
00798 \textcolor{comment}{  *            @arg RCC\_HCLK\_Div4:  APB2 clock = HCLK/4}
00799 \textcolor{comment}{  *            @arg RCC\_HCLK\_Div8:  APB2 clock = HCLK/8}
00800 \textcolor{comment}{  *            @arg RCC\_HCLK\_Div16: APB2 clock = HCLK/16}
00801 \textcolor{comment}{  * @retval None}
00802 \textcolor{comment}{  */}
00803 \textcolor{keywordtype}{void} RCC_PCLK2Config(uint32\_t RCC\_HCLK)
00804 \{
00805   uint32\_t tmpreg = 0;
00806 
00807   \textcolor{comment}{/* Check the parameters */}
00808   assert_param(IS\_RCC\_PCLK(RCC\_HCLK));
00809 
00810   tmpreg = RCC->CFGR;
00811 
00812   \textcolor{comment}{/* Clear PPRE2[2:0] bits */}
00813   tmpreg &= ~RCC_CFGR_PPRE2;
00814 
00815   \textcolor{comment}{/* Set PPRE2[2:0] bits according to RCC\_HCLK value */}
00816   tmpreg |= RCC\_HCLK << 3;
00817 
00818   \textcolor{comment}{/* Store the new value */}
00819   RCC->CFGR = tmpreg;
00820 \}
00821 
00822 \textcolor{comment}{/**}
00823 \textcolor{comment}{  * @brief  Returns the frequencies of different on chip clocks; SYSCLK, HCLK, }
00824 \textcolor{comment}{  *         PCLK1 and PCLK2.       }
00825 \textcolor{comment}{  * }
00826 \textcolor{comment}{  * @note   The system frequency computed by this function is not the real }
00827 \textcolor{comment}{  *         frequency in the chip. It is calculated based on the predefined }
00828 \textcolor{comment}{  *         constant and the selected clock source:}
00829 \textcolor{comment}{  * @note     If SYSCLK source is HSI, function returns values based on HSI\_VALUE(*)}
00830 \textcolor{comment}{  * @note     If SYSCLK source is HSE, function returns values based on HSE\_VALUE(**)}
00831 \textcolor{comment}{  * @note     If SYSCLK source is PLL, function returns values based on HSE\_VALUE(**) }
00832 \textcolor{comment}{  *           or HSI\_VALUE(*) multiplied/divided by the PLL factors.         }
00833 \textcolor{comment}{  * @note     (*) HSI\_VALUE is a constant defined in stm32f4xx.h file (default value}
00834 \textcolor{comment}{  *               16 MHz) but the real value may vary depending on the variations}
00835 \textcolor{comment}{  *               in voltage and temperature.}
00836 \textcolor{comment}{  * @note     (**) HSE\_VALUE is a constant defined in stm32f4xx.h file (default value}
00837 \textcolor{comment}{  *                25 MHz), user has to ensure that HSE\_VALUE is same as the real}
00838 \textcolor{comment}{  *                frequency of the crystal used. Otherwise, this function may}
00839 \textcolor{comment}{  *                have wrong result.}
00840 \textcolor{comment}{  *                }
00841 \textcolor{comment}{  * @note   The result of this function could be not correct when using fractional}
00842 \textcolor{comment}{  *         value for HSE crystal.}
00843 \textcolor{comment}{  *   }
00844 \textcolor{comment}{  * @param  RCC\_Clocks: pointer to a RCC\_ClocksTypeDef structure which will hold}
00845 \textcolor{comment}{  *          the clocks frequencies.}
00846 \textcolor{comment}{  *     }
00847 \textcolor{comment}{  * @note   This function can be used by the user application to compute the }
00848 \textcolor{comment}{  *         baudrate for the communication peripherals or configure other parameters.}
00849 \textcolor{comment}{  * @note   Each time SYSCLK, HCLK, PCLK1 and/or PCLK2 clock changes, this function}
00850 \textcolor{comment}{  *         must be called to update the structure's field. Otherwise, any}
00851 \textcolor{comment}{  *         configuration based on this function will be incorrect.}
00852 \textcolor{comment}{  *    }
00853 \textcolor{comment}{  * @retval None}
00854 \textcolor{comment}{  */}
00855 \textcolor{keywordtype}{void} RCC_GetClocksFreq(RCC\_ClocksTypeDef* RCC\_Clocks)
00856 \{
00857   uint32\_t tmp = 0, presc = 0, pllvco = 0, pllp = 2, pllsource = 0, pllm = 2;
00858 
00859   \textcolor{comment}{/* Get SYSCLK source -------------------------------------------------------*/}
00860   tmp = RCC->CFGR & RCC_CFGR_SWS;
00861 
00862   \textcolor{keywordflow}{switch} (tmp)
00863   \{
00864     \textcolor{keywordflow}{case} 0x00:  \textcolor{comment}{/* HSI used as system clock source */}
00865       RCC\_Clocks->SYSCLK_Frequency = HSI_VALUE;
00866       \textcolor{keywordflow}{break};
00867     \textcolor{keywordflow}{case} 0x04:  \textcolor{comment}{/* HSE used as system clock  source */}
00868       RCC\_Clocks->SYSCLK_Frequency = HSE_VALUE;
00869       \textcolor{keywordflow}{break};
00870     \textcolor{keywordflow}{case} 0x08:  \textcolor{comment}{/* PLL used as system clock  source */}
00871 
00872       \textcolor{comment}{/* PLL\_VCO = (HSE\_VALUE or HSI\_VALUE / PLLM) * PLLN}
00873 \textcolor{comment}{         SYSCLK = PLL\_VCO / PLLP}
00874 \textcolor{comment}{         */}
00875       pllsource = (RCC->PLLCFGR & RCC_PLLCFGR_PLLSRC) >> 22;
00876       pllm = RCC->PLLCFGR & RCC_PLLCFGR_PLLM;
00877 
00878       \textcolor{keywordflow}{if} (pllsource != 0)
00879       \{
00880         \textcolor{comment}{/* HSE used as PLL clock source */}
00881         pllvco = (HSE_VALUE / pllm) * ((RCC->PLLCFGR & RCC_PLLCFGR_PLLN) >> 6);
00882       \}
00883       \textcolor{keywordflow}{else}
00884       \{
00885         \textcolor{comment}{/* HSI used as PLL clock source */}
00886         pllvco = (HSI_VALUE / pllm) * ((RCC->PLLCFGR & RCC_PLLCFGR_PLLN) >> 6);
00887       \}
00888 
00889       pllp = (((RCC->PLLCFGR & RCC_PLLCFGR_PLLP) >>16) + 1 ) *2;
00890       RCC\_Clocks->SYSCLK_Frequency = pllvco/pllp;
00891       \textcolor{keywordflow}{break};
00892     \textcolor{keywordflow}{default}:
00893       RCC\_Clocks->SYSCLK_Frequency = HSI_VALUE;
00894       \textcolor{keywordflow}{break};
00895   \}
00896   \textcolor{comment}{/* Compute HCLK, PCLK1 and PCLK2 clocks frequencies ------------------------*/}
00897 
00898   \textcolor{comment}{/* Get HCLK prescaler */}
00899   tmp = RCC->CFGR & RCC_CFGR_HPRE;
00900   tmp = tmp >> 4;
00901   presc = APBAHBPrescTable[tmp];
00902   \textcolor{comment}{/* HCLK clock frequency */}
00903   RCC\_Clocks->HCLK_Frequency = RCC\_Clocks->SYSCLK_Frequency >> presc;
00904 
00905   \textcolor{comment}{/* Get PCLK1 prescaler */}
00906   tmp = RCC->CFGR & RCC_CFGR_PPRE1;
00907   tmp = tmp >> 10;
00908   presc = APBAHBPrescTable[tmp];
00909   \textcolor{comment}{/* PCLK1 clock frequency */}
00910   RCC\_Clocks->PCLK1_Frequency = RCC\_Clocks->HCLK_Frequency >> presc;
00911 
00912   \textcolor{comment}{/* Get PCLK2 prescaler */}
00913   tmp = RCC->CFGR & RCC_CFGR_PPRE2;
00914   tmp = tmp >> 13;
00915   presc = APBAHBPrescTable[tmp];
00916   \textcolor{comment}{/* PCLK2 clock frequency */}
00917   RCC\_Clocks->PCLK2_Frequency = RCC\_Clocks->HCLK_Frequency >> presc;
00918 \}
00919 
00920 \textcolor{comment}{/**}
00921 \textcolor{comment}{  * @\}}
00922 \textcolor{comment}{  */}
00923 
00924 \textcolor{comment}{/** @defgroup RCC\_Group3 Peripheral clocks configuration functions}
00925 \textcolor{comment}{ *  @brief   Peripheral clocks configuration functions }
00926 \textcolor{comment}{ *}
00927 \textcolor{comment}{@verbatim   }
00928 \textcolor{comment}{ ===============================================================================}
00929 \textcolor{comment}{                   Peripheral clocks configuration functions}
00930 \textcolor{comment}{ ===============================================================================  }
00931 \textcolor{comment}{}
00932 \textcolor{comment}{  This section provide functions allowing to configure the Peripheral clocks. }
00933 \textcolor{comment}{  }
00934 \textcolor{comment}{  1. The RTC clock which is derived from the LSI, LSE or HSE clock divided by 2 to 31.}
00935 \textcolor{comment}{     }
00936 \textcolor{comment}{  2. After restart from Reset or wakeup from STANDBY, all peripherals are off}
00937 \textcolor{comment}{     except internal SRAM, Flash and JTAG. Before to start using a peripheral you}
00938 \textcolor{comment}{     have to enable its interface clock. You can do this using RCC\_AHBPeriphClockCmd()}
00939 \textcolor{comment}{     , RCC\_APB2PeriphClockCmd() and RCC\_APB1PeriphClockCmd() functions.}
00940 \textcolor{comment}{}
00941 \textcolor{comment}{  3. To reset the peripherals configuration (to the default state after device reset)}
00942 \textcolor{comment}{     you can use RCC\_AHBPeriphResetCmd(), RCC\_APB2PeriphResetCmd() and }
00943 \textcolor{comment}{     RCC\_APB1PeriphResetCmd() functions.}
00944 \textcolor{comment}{     }
00945 \textcolor{comment}{  4. To further reduce power consumption in SLEEP mode the peripheral clocks can}
00946 \textcolor{comment}{     be disabled prior to executing the WFI or WFE instructions. You can do this}
00947 \textcolor{comment}{     using RCC\_AHBPeriphClockLPModeCmd(), RCC\_APB2PeriphClockLPModeCmd() and}
00948 \textcolor{comment}{     RCC\_APB1PeriphClockLPModeCmd() functions.  }
00949 \textcolor{comment}{}
00950 \textcolor{comment}{@endverbatim}
00951 \textcolor{comment}{  * @\{}
00952 \textcolor{comment}{  */}
00953 
00954 \textcolor{comment}{/**}
00955 \textcolor{comment}{  * @brief  Configures the RTC clock (RTCCLK).}
00956 \textcolor{comment}{  * @note   As the RTC clock configuration bits are in the Backup domain and write}
00957 \textcolor{comment}{  *         access is denied to this domain after reset, you have to enable write}
00958 \textcolor{comment}{  *         access using PWR\_BackupAccessCmd(ENABLE) function before to configure}
00959 \textcolor{comment}{  *         the RTC clock source (to be done once after reset).    }
00960 \textcolor{comment}{  * @note   Once the RTC clock is configured it can't be changed unless the  }
00961 \textcolor{comment}{  *         Backup domain is reset using RCC\_BackupResetCmd() function, or by}
00962 \textcolor{comment}{  *         a Power On Reset (POR).}
00963 \textcolor{comment}{  *    }
00964 \textcolor{comment}{  * @param  RCC\_RTCCLKSource: specifies the RTC clock source.}
00965 \textcolor{comment}{  *          This parameter can be one of the following values:}
00966 \textcolor{comment}{  *            @arg RCC\_RTCCLKSource\_LSE: LSE selected as RTC clock}
00967 \textcolor{comment}{  *            @arg RCC\_RTCCLKSource\_LSI: LSI selected as RTC clock}
00968 \textcolor{comment}{  *            @arg RCC\_RTCCLKSource\_HSE\_Divx: HSE clock divided by x selected}
00969 \textcolor{comment}{  *                                            as RTC clock, where x:[2,31]}
00970 \textcolor{comment}{  *  }
00971 \textcolor{comment}{  * @note   If the LSE or LSI is used as RTC clock source, the RTC continues to}
00972 \textcolor{comment}{  *         work in STOP and STANDBY modes, and can be used as wakeup source.}
00973 \textcolor{comment}{  *         However, when the HSE clock is used as RTC clock source, the RTC}
00974 \textcolor{comment}{  *         cannot be used in STOP and STANDBY modes.    }
00975 \textcolor{comment}{  * @note   The maximum input clock frequency for RTC is 1MHz (when using HSE as}
00976 \textcolor{comment}{  *         RTC clock source).}
00977 \textcolor{comment}{  *  }
00978 \textcolor{comment}{  * @retval None}
00979 \textcolor{comment}{  */}
00980 \textcolor{keywordtype}{void} RCC_RTCCLKConfig(uint32\_t RCC\_RTCCLKSource)
00981 \{
00982   uint32\_t tmpreg = 0;
00983 
00984   \textcolor{comment}{/* Check the parameters */}
00985   assert_param(IS\_RCC\_RTCCLK\_SOURCE(RCC\_RTCCLKSource));
00986 
00987   \textcolor{keywordflow}{if} ((RCC\_RTCCLKSource & 0x00000300) == 0x00000300)
00988   \{ \textcolor{comment}{/* If HSE is selected as RTC clock source, configure HSE division factor for RTC clock */}
00989     tmpreg = RCC->CFGR;
00990 
00991     \textcolor{comment}{/* Clear RTCPRE[4:0] bits */}
00992     tmpreg &= ~RCC_CFGR_RTCPRE;
00993 
00994     \textcolor{comment}{/* Configure HSE division factor for RTC clock */}
00995     tmpreg |= (RCC\_RTCCLKSource & 0xFFFFCFF);
00996 
00997     \textcolor{comment}{/* Store the new value */}
00998     RCC->CFGR = tmpreg;
00999   \}
01000 
01001   \textcolor{comment}{/* Select the RTC clock source */}
01002   RCC->BDCR |= (RCC\_RTCCLKSource & 0x00000FFF);
01003 \}
01004 
01005 \textcolor{comment}{/**}
01006 \textcolor{comment}{  * @brief  Enables or disables the RTC clock.}
01007 \textcolor{comment}{  * @note   This function must be used only after the RTC clock source was selected}
01008 \textcolor{comment}{  *         using the RCC\_RTCCLKConfig function.}
01009 \textcolor{comment}{  * @param  NewState: new state of the RTC clock. This parameter can be: ENABLE or DISABLE.}
01010 \textcolor{comment}{  * @retval None}
01011 \textcolor{comment}{  */}
01012 \textcolor{keywordtype}{void} RCC_RTCCLKCmd(FunctionalState NewState)
01013 \{
01014   \textcolor{comment}{/* Check the parameters */}
01015   assert_param(IS\_FUNCTIONAL\_STATE(NewState));
01016 
01017   *(\_\_IO uint32\_t *) BDCR_RTCEN_BB = (uint32\_t)NewState;
01018 \}
01019 
01020 \textcolor{comment}{/**}
01021 \textcolor{comment}{  * @brief  Forces or releases the Backup domain reset.}
01022 \textcolor{comment}{  * @note   This function resets the RTC peripheral (including the backup registers)}
01023 \textcolor{comment}{  *         and the RTC clock source selection in RCC\_CSR register.}
01024 \textcolor{comment}{  * @note   The BKPSRAM is not affected by this reset.    }
01025 \textcolor{comment}{  * @param  NewState: new state of the Backup domain reset.}
01026 \textcolor{comment}{  *          This parameter can be: ENABLE or DISABLE.}
01027 \textcolor{comment}{  * @retval None}
01028 \textcolor{comment}{  */}
01029 \textcolor{keywordtype}{void} RCC_BackupResetCmd(FunctionalState NewState)
01030 \{
01031   \textcolor{comment}{/* Check the parameters */}
01032   assert_param(IS\_FUNCTIONAL\_STATE(NewState));
01033   *(\_\_IO uint32\_t *) BDCR_BDRST_BB = (uint32\_t)NewState;
01034 \}
01035 
01036 \textcolor{comment}{/**}
01037 \textcolor{comment}{  * @brief  Configures the I2S clock source (I2SCLK).}
01038 \textcolor{comment}{  * @note   This function must be called before enabling the I2S APB clock.}
01039 \textcolor{comment}{  * @param  RCC\_I2SCLKSource: specifies the I2S clock source.}
01040 \textcolor{comment}{  *          This parameter can be one of the following values:}
01041 \textcolor{comment}{  *            @arg RCC\_I2S2CLKSource\_PLLI2S: PLLI2S clock used as I2S clock source}
01042 \textcolor{comment}{  *            @arg RCC\_I2S2CLKSource\_Ext: External clock mapped on the I2S\_CKIN pin}
01043 \textcolor{comment}{  *                                        used as I2S clock source}
01044 \textcolor{comment}{  * @retval None}
01045 \textcolor{comment}{  */}
01046 \textcolor{keywordtype}{void} RCC_I2SCLKConfig(uint32\_t RCC\_I2SCLKSource)
01047 \{
01048   \textcolor{comment}{/* Check the parameters */}
01049   assert_param(IS\_RCC\_I2SCLK\_SOURCE(RCC\_I2SCLKSource));
01050 
01051   *(\_\_IO uint32\_t *) CFGR_I2SSRC_BB = RCC\_I2SCLKSource;
01052 \}
01053 
01054 \textcolor{comment}{/**}
01055 \textcolor{comment}{  * @brief  Enables or disables the AHB1 peripheral clock.}
01056 \textcolor{comment}{  * @note   After reset, the peripheral clock (used for registers read/write access)}
01057 \textcolor{comment}{  *         is disabled and the application software has to enable this clock before }
01058 \textcolor{comment}{  *         using it.   }
01059 \textcolor{comment}{  * @param  RCC\_AHBPeriph: specifies the AHB1 peripheral to gates its clock.}
01060 \textcolor{comment}{  *          This parameter can be any combination of the following values:}
01061 \textcolor{comment}{  *            @arg RCC\_AHB1Periph\_GPIOA:       GPIOA clock}
01062 \textcolor{comment}{  *            @arg RCC\_AHB1Periph\_GPIOB:       GPIOB clock }
01063 \textcolor{comment}{  *            @arg RCC\_AHB1Periph\_GPIOC:       GPIOC clock}
01064 \textcolor{comment}{  *            @arg RCC\_AHB1Periph\_GPIOD:       GPIOD clock}
01065 \textcolor{comment}{  *            @arg RCC\_AHB1Periph\_GPIOE:       GPIOE clock}
01066 \textcolor{comment}{  *            @arg RCC\_AHB1Periph\_GPIOF:       GPIOF clock}
01067 \textcolor{comment}{  *            @arg RCC\_AHB1Periph\_GPIOG:       GPIOG clock}
01068 \textcolor{comment}{  *            @arg RCC\_AHB1Periph\_GPIOG:       GPIOG clock}
01069 \textcolor{comment}{  *            @arg RCC\_AHB1Periph\_GPIOI:       GPIOI clock}
01070 \textcolor{comment}{  *            @arg RCC\_AHB1Periph\_CRC:         CRC clock}
01071 \textcolor{comment}{  *            @arg RCC\_AHB1Periph\_BKPSRAM:     BKPSRAM interface clock}
01072 \textcolor{comment}{  *            @arg RCC\_AHB1Periph\_CCMDATARAMEN CCM data RAM interface clock}
01073 \textcolor{comment}{  *            @arg RCC\_AHB1Periph\_DMA1:        DMA1 clock}
01074 \textcolor{comment}{  *            @arg RCC\_AHB1Periph\_DMA2:        DMA2 clock}
01075 \textcolor{comment}{  *            @arg RCC\_AHB1Periph\_ETH\_MAC:     Ethernet MAC clock}
01076 \textcolor{comment}{  *            @arg RCC\_AHB1Periph\_ETH\_MAC\_Tx:  Ethernet Transmission clock}
01077 \textcolor{comment}{  *            @arg RCC\_AHB1Periph\_ETH\_MAC\_Rx:  Ethernet Reception clock}
01078 \textcolor{comment}{  *            @arg RCC\_AHB1Periph\_ETH\_MAC\_PTP: Ethernet PTP clock}
01079 \textcolor{comment}{  *            @arg RCC\_AHB1Periph\_OTG\_HS:      USB OTG HS clock}
01080 \textcolor{comment}{  *            @arg RCC\_AHB1Periph\_OTG\_HS\_ULPI: USB OTG HS ULPI clock}
01081 \textcolor{comment}{  * @param  NewState: new state of the specified peripheral clock.}
01082 \textcolor{comment}{  *          This parameter can be: ENABLE or DISABLE.}
01083 \textcolor{comment}{  * @retval None}
01084 \textcolor{comment}{  */}
01085 \textcolor{keywordtype}{void} RCC_AHB1PeriphClockCmd(uint32\_t RCC\_AHB1Periph, FunctionalState NewState)
01086 \{
01087   \textcolor{comment}{/* Check the parameters */}
01088   assert_param(IS\_RCC\_AHB1\_CLOCK\_PERIPH(RCC\_AHB1Periph));
01089 
01090   assert_param(IS\_FUNCTIONAL\_STATE(NewState));
01091   \textcolor{keywordflow}{if} (NewState != DISABLE)
01092   \{
01093     RCC->AHB1ENR |= RCC\_AHB1Periph;
01094   \}
01095   \textcolor{keywordflow}{else}
01096   \{
01097     RCC->AHB1ENR &= ~RCC\_AHB1Periph;
01098   \}
01099 \}
01100 
01101 \textcolor{comment}{/**}
01102 \textcolor{comment}{  * @brief  Enables or disables the AHB2 peripheral clock.}
01103 \textcolor{comment}{  * @note   After reset, the peripheral clock (used for registers read/write access)}
01104 \textcolor{comment}{  *         is disabled and the application software has to enable this clock before }
01105 \textcolor{comment}{  *         using it. }
01106 \textcolor{comment}{  * @param  RCC\_AHBPeriph: specifies the AHB2 peripheral to gates its clock.}
01107 \textcolor{comment}{  *          This parameter can be any combination of the following values:}
01108 \textcolor{comment}{  *            @arg RCC\_AHB2Periph\_DCMI:   DCMI clock}
01109 \textcolor{comment}{  *            @arg RCC\_AHB2Periph\_CRYP:   CRYP clock}
01110 \textcolor{comment}{  *            @arg RCC\_AHB2Periph\_HASH:   HASH clock}
01111 \textcolor{comment}{  *            @arg RCC\_AHB2Periph\_RNG:    RNG clock}
01112 \textcolor{comment}{  *            @arg RCC\_AHB2Periph\_OTG\_FS: USB OTG FS clock}
01113 \textcolor{comment}{  * @param  NewState: new state of the specified peripheral clock.}
01114 \textcolor{comment}{  *          This parameter can be: ENABLE or DISABLE.}
01115 \textcolor{comment}{  * @retval None}
01116 \textcolor{comment}{  */}
01117 \textcolor{keywordtype}{void} RCC_AHB2PeriphClockCmd(uint32\_t RCC\_AHB2Periph, FunctionalState NewState)
01118 \{
01119   \textcolor{comment}{/* Check the parameters */}
01120   assert_param(IS\_RCC\_AHB2\_PERIPH(RCC\_AHB2Periph));
01121   assert_param(IS\_FUNCTIONAL\_STATE(NewState));
01122 
01123   \textcolor{keywordflow}{if} (NewState != DISABLE)
01124   \{
01125     RCC->AHB2ENR |= RCC\_AHB2Periph;
01126   \}
01127   \textcolor{keywordflow}{else}
01128   \{
01129     RCC->AHB2ENR &= ~RCC\_AHB2Periph;
01130   \}
01131 \}
01132 
01133 \textcolor{comment}{/**}
01134 \textcolor{comment}{  * @brief  Enables or disables the AHB3 peripheral clock.}
01135 \textcolor{comment}{  * @note   After reset, the peripheral clock (used for registers read/write access)}
01136 \textcolor{comment}{  *         is disabled and the application software has to enable this clock before }
01137 \textcolor{comment}{  *         using it. }
01138 \textcolor{comment}{  * @param  RCC\_AHBPeriph: specifies the AHB3 peripheral to gates its clock.}
01139 \textcolor{comment}{  *          This parameter must be: RCC\_AHB3Periph\_FSMC}
01140 \textcolor{comment}{  * @param  NewState: new state of the specified peripheral clock.}
01141 \textcolor{comment}{  *          This parameter can be: ENABLE or DISABLE.}
01142 \textcolor{comment}{  * @retval None}
01143 \textcolor{comment}{  */}
01144 \textcolor{keywordtype}{void} RCC_AHB3PeriphClockCmd(uint32\_t RCC\_AHB3Periph, FunctionalState NewState)
01145 \{
01146   \textcolor{comment}{/* Check the parameters */}
01147   assert_param(IS\_RCC\_AHB3\_PERIPH(RCC\_AHB3Periph));
01148   assert_param(IS\_FUNCTIONAL\_STATE(NewState));
01149 
01150   \textcolor{keywordflow}{if} (NewState != DISABLE)
01151   \{
01152     RCC->AHB3ENR |= RCC\_AHB3Periph;
01153   \}
01154   \textcolor{keywordflow}{else}
01155   \{
01156     RCC->AHB3ENR &= ~RCC\_AHB3Periph;
01157   \}
01158 \}
01159 
01160 \textcolor{comment}{/**}
01161 \textcolor{comment}{  * @brief  Enables or disables the Low Speed APB (APB1) peripheral clock.}
01162 \textcolor{comment}{  * @note   After reset, the peripheral clock (used for registers read/write access)}
01163 \textcolor{comment}{  *         is disabled and the application software has to enable this clock before }
01164 \textcolor{comment}{  *         using it. }
01165 \textcolor{comment}{  * @param  RCC\_APB1Periph: specifies the APB1 peripheral to gates its clock.}
01166 \textcolor{comment}{  *          This parameter can be any combination of the following values:}
01167 \textcolor{comment}{  *            @arg RCC\_APB1Periph\_TIM2:   TIM2 clock}
01168 \textcolor{comment}{  *            @arg RCC\_APB1Periph\_TIM3:   TIM3 clock}
01169 \textcolor{comment}{  *            @arg RCC\_APB1Periph\_TIM4:   TIM4 clock}
01170 \textcolor{comment}{  *            @arg RCC\_APB1Periph\_TIM5:   TIM5 clock}
01171 \textcolor{comment}{  *            @arg RCC\_APB1Periph\_TIM6:   TIM6 clock}
01172 \textcolor{comment}{  *            @arg RCC\_APB1Periph\_TIM7:   TIM7 clock}
01173 \textcolor{comment}{  *            @arg RCC\_APB1Periph\_TIM12:  TIM12 clock}
01174 \textcolor{comment}{  *            @arg RCC\_APB1Periph\_TIM13:  TIM13 clock}
01175 \textcolor{comment}{  *            @arg RCC\_APB1Periph\_TIM14:  TIM14 clock}
01176 \textcolor{comment}{  *            @arg RCC\_APB1Periph\_WWDG:   WWDG clock}
01177 \textcolor{comment}{  *            @arg RCC\_APB1Periph\_SPI2:   SPI2 clock}
01178 \textcolor{comment}{  *            @arg RCC\_APB1Periph\_SPI3:   SPI3 clock}
01179 \textcolor{comment}{  *            @arg RCC\_APB1Periph\_USART2: USART2 clock}
01180 \textcolor{comment}{  *            @arg RCC\_APB1Periph\_USART3: USART3 clock}
01181 \textcolor{comment}{  *            @arg RCC\_APB1Periph\_UART4:  UART4 clock}
01182 \textcolor{comment}{  *            @arg RCC\_APB1Periph\_UART5:  UART5 clock}
01183 \textcolor{comment}{  *            @arg RCC\_APB1Periph\_I2C1:   I2C1 clock}
01184 \textcolor{comment}{  *            @arg RCC\_APB1Periph\_I2C2:   I2C2 clock}
01185 \textcolor{comment}{  *            @arg RCC\_APB1Periph\_I2C3:   I2C3 clock}
01186 \textcolor{comment}{  *            @arg RCC\_APB1Periph\_CAN1:   CAN1 clock}
01187 \textcolor{comment}{  *            @arg RCC\_APB1Periph\_CAN2:   CAN2 clock}
01188 \textcolor{comment}{  *            @arg RCC\_APB1Periph\_PWR:    PWR clock}
01189 \textcolor{comment}{  *            @arg RCC\_APB1Periph\_DAC:    DAC clock}
01190 \textcolor{comment}{  * @param  NewState: new state of the specified peripheral clock.}
01191 \textcolor{comment}{  *          This parameter can be: ENABLE or DISABLE.}
01192 \textcolor{comment}{  * @retval None}
01193 \textcolor{comment}{  */}
01194 \textcolor{keywordtype}{void} RCC_APB1PeriphClockCmd(uint32\_t RCC\_APB1Periph, FunctionalState NewState)
01195 \{
01196   \textcolor{comment}{/* Check the parameters */}
01197   assert_param(IS\_RCC\_APB1\_PERIPH(RCC\_APB1Periph));
01198   assert_param(IS\_FUNCTIONAL\_STATE(NewState));
01199 
01200   \textcolor{keywordflow}{if} (NewState != DISABLE)
01201   \{
01202     RCC->APB1ENR |= RCC\_APB1Periph;
01203   \}
01204   \textcolor{keywordflow}{else}
01205   \{
01206     RCC->APB1ENR &= ~RCC\_APB1Periph;
01207   \}
01208 \}
01209 
01210 \textcolor{comment}{/**}
01211 \textcolor{comment}{  * @brief  Enables or disables the High Speed APB (APB2) peripheral clock.}
01212 \textcolor{comment}{  * @note   After reset, the peripheral clock (used for registers read/write access)}
01213 \textcolor{comment}{  *         is disabled and the application software has to enable this clock before }
01214 \textcolor{comment}{  *         using it.}
01215 \textcolor{comment}{  * @param  RCC\_APB2Periph: specifies the APB2 peripheral to gates its clock.}
01216 \textcolor{comment}{  *          This parameter can be any combination of the following values:}
01217 \textcolor{comment}{  *            @arg RCC\_APB2Periph\_TIM1:   TIM1 clock}
01218 \textcolor{comment}{  *            @arg RCC\_APB2Periph\_TIM8:   TIM8 clock}
01219 \textcolor{comment}{  *            @arg RCC\_APB2Periph\_USART1: USART1 clock}
01220 \textcolor{comment}{  *            @arg RCC\_APB2Periph\_USART6: USART6 clock}
01221 \textcolor{comment}{  *            @arg RCC\_APB2Periph\_ADC1:   ADC1 clock}
01222 \textcolor{comment}{  *            @arg RCC\_APB2Periph\_ADC2:   ADC2 clock}
01223 \textcolor{comment}{  *            @arg RCC\_APB2Periph\_ADC3:   ADC3 clock}
01224 \textcolor{comment}{  *            @arg RCC\_APB2Periph\_SDIO:   SDIO clock}
01225 \textcolor{comment}{  *            @arg RCC\_APB2Periph\_SPI1:   SPI1 clock}
01226 \textcolor{comment}{  *            @arg RCC\_APB2Periph\_SYSCFG: SYSCFG clock}
01227 \textcolor{comment}{  *            @arg RCC\_APB2Periph\_TIM9:   TIM9 clock}
01228 \textcolor{comment}{  *            @arg RCC\_APB2Periph\_TIM10:  TIM10 clock}
01229 \textcolor{comment}{  *            @arg RCC\_APB2Periph\_TIM11:  TIM11 clock}
01230 \textcolor{comment}{  * @param  NewState: new state of the specified peripheral clock.}
01231 \textcolor{comment}{  *          This parameter can be: ENABLE or DISABLE.}
01232 \textcolor{comment}{  * @retval None}
01233 \textcolor{comment}{  */}
01234 \textcolor{keywordtype}{void} RCC_APB2PeriphClockCmd(uint32\_t RCC\_APB2Periph, FunctionalState NewState)
01235 \{
01236   \textcolor{comment}{/* Check the parameters */}
01237   assert_param(IS\_RCC\_APB2\_PERIPH(RCC\_APB2Periph));
01238   assert_param(IS\_FUNCTIONAL\_STATE(NewState));
01239 
01240   \textcolor{keywordflow}{if} (NewState != DISABLE)
01241   \{
01242     RCC->APB2ENR |= RCC\_APB2Periph;
01243   \}
01244   \textcolor{keywordflow}{else}
01245   \{
01246     RCC->APB2ENR &= ~RCC\_APB2Periph;
01247   \}
01248 \}
01249 
01250 \textcolor{comment}{/**}
01251 \textcolor{comment}{  * @brief  Forces or releases AHB1 peripheral reset.}
01252 \textcolor{comment}{  * @param  RCC\_AHB1Periph: specifies the AHB1 peripheral to reset.}
01253 \textcolor{comment}{  *          This parameter can be any combination of the following values:}
01254 \textcolor{comment}{  *            @arg RCC\_AHB1Periph\_GPIOA:   GPIOA clock}
01255 \textcolor{comment}{  *            @arg RCC\_AHB1Periph\_GPIOB:   GPIOB clock }
01256 \textcolor{comment}{  *            @arg RCC\_AHB1Periph\_GPIOC:   GPIOC clock}
01257 \textcolor{comment}{  *            @arg RCC\_AHB1Periph\_GPIOD:   GPIOD clock}
01258 \textcolor{comment}{  *            @arg RCC\_AHB1Periph\_GPIOE:   GPIOE clock}
01259 \textcolor{comment}{  *            @arg RCC\_AHB1Periph\_GPIOF:   GPIOF clock}
01260 \textcolor{comment}{  *            @arg RCC\_AHB1Periph\_GPIOG:   GPIOG clock}
01261 \textcolor{comment}{  *            @arg RCC\_AHB1Periph\_GPIOG:   GPIOG clock}
01262 \textcolor{comment}{  *            @arg RCC\_AHB1Periph\_GPIOI:   GPIOI clock}
01263 \textcolor{comment}{  *            @arg RCC\_AHB1Periph\_CRC:     CRC clock}
01264 \textcolor{comment}{  *            @arg RCC\_AHB1Periph\_DMA1:    DMA1 clock}
01265 \textcolor{comment}{  *            @arg RCC\_AHB1Periph\_DMA2:    DMA2 clock}
01266 \textcolor{comment}{  *            @arg RCC\_AHB1Periph\_ETH\_MAC: Ethernet MAC clock}
01267 \textcolor{comment}{  *            @arg RCC\_AHB1Periph\_OTG\_HS:  USB OTG HS clock}
01268 \textcolor{comment}{  *                  }
01269 \textcolor{comment}{  * @param  NewState: new state of the specified peripheral reset.}
01270 \textcolor{comment}{  *          This parameter can be: ENABLE or DISABLE.}
01271 \textcolor{comment}{  * @retval None}
01272 \textcolor{comment}{  */}
01273 \textcolor{keywordtype}{void} RCC_AHB1PeriphResetCmd(uint32\_t RCC\_AHB1Periph, FunctionalState NewState)
01274 \{
01275   \textcolor{comment}{/* Check the parameters */}
01276   assert_param(IS\_RCC\_AHB1\_RESET\_PERIPH(RCC\_AHB1Periph));
01277   assert_param(IS\_FUNCTIONAL\_STATE(NewState));
01278 
01279   \textcolor{keywordflow}{if} (NewState != DISABLE)
01280   \{
01281     RCC->AHB1RSTR |= RCC\_AHB1Periph;
01282   \}
01283   \textcolor{keywordflow}{else}
01284   \{
01285     RCC->AHB1RSTR &= ~RCC\_AHB1Periph;
01286   \}
01287 \}
01288 
01289 \textcolor{comment}{/**}
01290 \textcolor{comment}{  * @brief  Forces or releases AHB2 peripheral reset.}
01291 \textcolor{comment}{  * @param  RCC\_AHB2Periph: specifies the AHB2 peripheral to reset.}
01292 \textcolor{comment}{  *          This parameter can be any combination of the following values:}
01293 \textcolor{comment}{  *            @arg RCC\_AHB2Periph\_DCMI:   DCMI clock}
01294 \textcolor{comment}{  *            @arg RCC\_AHB2Periph\_CRYP:   CRYP clock}
01295 \textcolor{comment}{  *            @arg RCC\_AHB2Periph\_HASH:   HASH clock}
01296 \textcolor{comment}{  *            @arg RCC\_AHB2Periph\_RNG:    RNG clock}
01297 \textcolor{comment}{  *            @arg RCC\_AHB2Periph\_OTG\_FS: USB OTG FS clock}
01298 \textcolor{comment}{  * @param  NewState: new state of the specified peripheral reset.}
01299 \textcolor{comment}{  *          This parameter can be: ENABLE or DISABLE.}
01300 \textcolor{comment}{  * @retval None}
01301 \textcolor{comment}{  */}
01302 \textcolor{keywordtype}{void} RCC_AHB2PeriphResetCmd(uint32\_t RCC\_AHB2Periph, FunctionalState NewState)
01303 \{
01304   \textcolor{comment}{/* Check the parameters */}
01305   assert_param(IS\_RCC\_AHB2\_PERIPH(RCC\_AHB2Periph));
01306   assert_param(IS\_FUNCTIONAL\_STATE(NewState));
01307 
01308   \textcolor{keywordflow}{if} (NewState != DISABLE)
01309   \{
01310     RCC->AHB2RSTR |= RCC\_AHB2Periph;
01311   \}
01312   \textcolor{keywordflow}{else}
01313   \{
01314     RCC->AHB2RSTR &= ~RCC\_AHB2Periph;
01315   \}
01316 \}
01317 
01318 \textcolor{comment}{/**}
01319 \textcolor{comment}{  * @brief  Forces or releases AHB3 peripheral reset.}
01320 \textcolor{comment}{  * @param  RCC\_AHB3Periph: specifies the AHB3 peripheral to reset.}
01321 \textcolor{comment}{  *          This parameter must be: RCC\_AHB3Periph\_FSMC}
01322 \textcolor{comment}{  * @param  NewState: new state of the specified peripheral reset.}
01323 \textcolor{comment}{  *          This parameter can be: ENABLE or DISABLE.}
01324 \textcolor{comment}{  * @retval None}
01325 \textcolor{comment}{  */}
01326 \textcolor{keywordtype}{void} RCC_AHB3PeriphResetCmd(uint32\_t RCC\_AHB3Periph, FunctionalState NewState)
01327 \{
01328   \textcolor{comment}{/* Check the parameters */}
01329   assert_param(IS\_RCC\_AHB3\_PERIPH(RCC\_AHB3Periph));
01330   assert_param(IS\_FUNCTIONAL\_STATE(NewState));
01331 
01332   \textcolor{keywordflow}{if} (NewState != DISABLE)
01333   \{
01334     RCC->AHB3RSTR |= RCC\_AHB3Periph;
01335   \}
01336   \textcolor{keywordflow}{else}
01337   \{
01338     RCC->AHB3RSTR &= ~RCC\_AHB3Periph;
01339   \}
01340 \}
01341 
01342 \textcolor{comment}{/**}
01343 \textcolor{comment}{  * @brief  Forces or releases Low Speed APB (APB1) peripheral reset.}
01344 \textcolor{comment}{  * @param  RCC\_APB1Periph: specifies the APB1 peripheral to reset.}
01345 \textcolor{comment}{  *          This parameter can be any combination of the following values:}
01346 \textcolor{comment}{  *            @arg RCC\_APB1Periph\_TIM2:   TIM2 clock}
01347 \textcolor{comment}{  *            @arg RCC\_APB1Periph\_TIM3:   TIM3 clock}
01348 \textcolor{comment}{  *            @arg RCC\_APB1Periph\_TIM4:   TIM4 clock}
01349 \textcolor{comment}{  *            @arg RCC\_APB1Periph\_TIM5:   TIM5 clock}
01350 \textcolor{comment}{  *            @arg RCC\_APB1Periph\_TIM6:   TIM6 clock}
01351 \textcolor{comment}{  *            @arg RCC\_APB1Periph\_TIM7:   TIM7 clock}
01352 \textcolor{comment}{  *            @arg RCC\_APB1Periph\_TIM12:  TIM12 clock}
01353 \textcolor{comment}{  *            @arg RCC\_APB1Periph\_TIM13:  TIM13 clock}
01354 \textcolor{comment}{  *            @arg RCC\_APB1Periph\_TIM14:  TIM14 clock}
01355 \textcolor{comment}{  *            @arg RCC\_APB1Periph\_WWDG:   WWDG clock}
01356 \textcolor{comment}{  *            @arg RCC\_APB1Periph\_SPI2:   SPI2 clock}
01357 \textcolor{comment}{  *            @arg RCC\_APB1Periph\_SPI3:   SPI3 clock}
01358 \textcolor{comment}{  *            @arg RCC\_APB1Periph\_USART2: USART2 clock}
01359 \textcolor{comment}{  *            @arg RCC\_APB1Periph\_USART3: USART3 clock}
01360 \textcolor{comment}{  *            @arg RCC\_APB1Periph\_UART4:  UART4 clock}
01361 \textcolor{comment}{  *            @arg RCC\_APB1Periph\_UART5:  UART5 clock}
01362 \textcolor{comment}{  *            @arg RCC\_APB1Periph\_I2C1:   I2C1 clock}
01363 \textcolor{comment}{  *            @arg RCC\_APB1Periph\_I2C2:   I2C2 clock}
01364 \textcolor{comment}{  *            @arg RCC\_APB1Periph\_I2C3:   I2C3 clock}
01365 \textcolor{comment}{  *            @arg RCC\_APB1Periph\_CAN1:   CAN1 clock}
01366 \textcolor{comment}{  *            @arg RCC\_APB1Periph\_CAN2:   CAN2 clock}
01367 \textcolor{comment}{  *            @arg RCC\_APB1Periph\_PWR:    PWR clock}
01368 \textcolor{comment}{  *            @arg RCC\_APB1Periph\_DAC:    DAC clock}
01369 \textcolor{comment}{  * @param  NewState: new state of the specified peripheral reset.}
01370 \textcolor{comment}{  *          This parameter can be: ENABLE or DISABLE.}
01371 \textcolor{comment}{  * @retval None}
01372 \textcolor{comment}{  */}
01373 \textcolor{keywordtype}{void} RCC_APB1PeriphResetCmd(uint32\_t RCC\_APB1Periph, FunctionalState NewState)
01374 \{
01375   \textcolor{comment}{/* Check the parameters */}
01376   assert_param(IS\_RCC\_APB1\_PERIPH(RCC\_APB1Periph));
01377   assert_param(IS\_FUNCTIONAL\_STATE(NewState));
01378   \textcolor{keywordflow}{if} (NewState != DISABLE)
01379   \{
01380     RCC->APB1RSTR |= RCC\_APB1Periph;
01381   \}
01382   \textcolor{keywordflow}{else}
01383   \{
01384     RCC->APB1RSTR &= ~RCC\_APB1Periph;
01385   \}
01386 \}
01387 
01388 \textcolor{comment}{/**}
01389 \textcolor{comment}{  * @brief  Forces or releases High Speed APB (APB2) peripheral reset.}
01390 \textcolor{comment}{  * @param  RCC\_APB2Periph: specifies the APB2 peripheral to reset.}
01391 \textcolor{comment}{  *          This parameter can be any combination of the following values:}
01392 \textcolor{comment}{  *            @arg RCC\_APB2Periph\_TIM1:   TIM1 clock}
01393 \textcolor{comment}{  *            @arg RCC\_APB2Periph\_TIM8:   TIM8 clock}
01394 \textcolor{comment}{  *            @arg RCC\_APB2Periph\_USART1: USART1 clock}
01395 \textcolor{comment}{  *            @arg RCC\_APB2Periph\_USART6: USART6 clock}
01396 \textcolor{comment}{  *            @arg RCC\_APB2Periph\_ADC1:   ADC1 clock}
01397 \textcolor{comment}{  *            @arg RCC\_APB2Periph\_ADC2:   ADC2 clock}
01398 \textcolor{comment}{  *            @arg RCC\_APB2Periph\_ADC3:   ADC3 clock}
01399 \textcolor{comment}{  *            @arg RCC\_APB2Periph\_SDIO:   SDIO clock}
01400 \textcolor{comment}{  *            @arg RCC\_APB2Periph\_SPI1:   SPI1 clock}
01401 \textcolor{comment}{  *            @arg RCC\_APB2Periph\_SYSCFG: SYSCFG clock}
01402 \textcolor{comment}{  *            @arg RCC\_APB2Periph\_TIM9:   TIM9 clock}
01403 \textcolor{comment}{  *            @arg RCC\_APB2Periph\_TIM10:  TIM10 clock}
01404 \textcolor{comment}{  *            @arg RCC\_APB2Periph\_TIM11:  TIM11 clock}
01405 \textcolor{comment}{  * @param  NewState: new state of the specified peripheral reset.}
01406 \textcolor{comment}{  *          This parameter can be: ENABLE or DISABLE.}
01407 \textcolor{comment}{  * @retval None}
01408 \textcolor{comment}{  */}
01409 \textcolor{keywordtype}{void} RCC_APB2PeriphResetCmd(uint32\_t RCC\_APB2Periph, FunctionalState NewState)
01410 \{
01411   \textcolor{comment}{/* Check the parameters */}
01412   assert_param(IS\_RCC\_APB2\_RESET\_PERIPH(RCC\_APB2Periph));
01413   assert_param(IS\_FUNCTIONAL\_STATE(NewState));
01414   \textcolor{keywordflow}{if} (NewState != DISABLE)
01415   \{
01416     RCC->APB2RSTR |= RCC\_APB2Periph;
01417   \}
01418   \textcolor{keywordflow}{else}
01419   \{
01420     RCC->APB2RSTR &= ~RCC\_APB2Periph;
01421   \}
01422 \}
01423 
01424 \textcolor{comment}{/**}
01425 \textcolor{comment}{  * @brief  Enables or disables the AHB1 peripheral clock during Low Power (Sleep) mode.}
01426 \textcolor{comment}{  * @note   Peripheral clock gating in SLEEP mode can be used to further reduce}
01427 \textcolor{comment}{  *         power consumption.}
01428 \textcolor{comment}{  * @note   After wakeup from SLEEP mode, the peripheral clock is enabled again.}
01429 \textcolor{comment}{  * @note   By default, all peripheral clocks are enabled during SLEEP mode.}
01430 \textcolor{comment}{  * @param  RCC\_AHBPeriph: specifies the AHB1 peripheral to gates its clock.}
01431 \textcolor{comment}{  *          This parameter can be any combination of the following values:}
01432 \textcolor{comment}{  *            @arg RCC\_AHB1Periph\_GPIOA:       GPIOA clock}
01433 \textcolor{comment}{  *            @arg RCC\_AHB1Periph\_GPIOB:       GPIOB clock }
01434 \textcolor{comment}{  *            @arg RCC\_AHB1Periph\_GPIOC:       GPIOC clock}
01435 \textcolor{comment}{  *            @arg RCC\_AHB1Periph\_GPIOD:       GPIOD clock}
01436 \textcolor{comment}{  *            @arg RCC\_AHB1Periph\_GPIOE:       GPIOE clock}
01437 \textcolor{comment}{  *            @arg RCC\_AHB1Periph\_GPIOF:       GPIOF clock}
01438 \textcolor{comment}{  *            @arg RCC\_AHB1Periph\_GPIOG:       GPIOG clock}
01439 \textcolor{comment}{  *            @arg RCC\_AHB1Periph\_GPIOG:       GPIOG clock}
01440 \textcolor{comment}{  *            @arg RCC\_AHB1Periph\_GPIOI:       GPIOI clock}
01441 \textcolor{comment}{  *            @arg RCC\_AHB1Periph\_CRC:         CRC clock}
01442 \textcolor{comment}{  *            @arg RCC\_AHB1Periph\_BKPSRAM:     BKPSRAM interface clock}
01443 \textcolor{comment}{  *            @arg RCC\_AHB1Periph\_DMA1:        DMA1 clock}
01444 \textcolor{comment}{  *            @arg RCC\_AHB1Periph\_DMA2:        DMA2 clock}
01445 \textcolor{comment}{  *            @arg RCC\_AHB1Periph\_ETH\_MAC:     Ethernet MAC clock}
01446 \textcolor{comment}{  *            @arg RCC\_AHB1Periph\_ETH\_MAC\_Tx:  Ethernet Transmission clock}
01447 \textcolor{comment}{  *            @arg RCC\_AHB1Periph\_ETH\_MAC\_Rx:  Ethernet Reception clock}
01448 \textcolor{comment}{  *            @arg RCC\_AHB1Periph\_ETH\_MAC\_PTP: Ethernet PTP clock}
01449 \textcolor{comment}{  *            @arg RCC\_AHB1Periph\_OTG\_HS:      USB OTG HS clock}
01450 \textcolor{comment}{  *            @arg RCC\_AHB1Periph\_OTG\_HS\_ULPI: USB OTG HS ULPI clock}
01451 \textcolor{comment}{  * @param  NewState: new state of the specified peripheral clock.}
01452 \textcolor{comment}{  *          This parameter can be: ENABLE or DISABLE.}
01453 \textcolor{comment}{  * @retval None}
01454 \textcolor{comment}{  */}
01455 \textcolor{keywordtype}{void} RCC_AHB1PeriphClockLPModeCmd(uint32\_t RCC\_AHB1Periph, FunctionalState NewState)
01456 \{
01457   \textcolor{comment}{/* Check the parameters */}
01458   assert_param(IS\_RCC\_AHB1\_LPMODE\_PERIPH(RCC\_AHB1Periph));
01459   assert_param(IS\_FUNCTIONAL\_STATE(NewState));
01460   \textcolor{keywordflow}{if} (NewState != DISABLE)
01461   \{
01462     RCC->AHB1LPENR |= RCC\_AHB1Periph;
01463   \}
01464   \textcolor{keywordflow}{else}
01465   \{
01466     RCC->AHB1LPENR &= ~RCC\_AHB1Periph;
01467   \}
01468 \}
01469 
01470 \textcolor{comment}{/**}
01471 \textcolor{comment}{  * @brief  Enables or disables the AHB2 peripheral clock during Low Power (Sleep) mode.}
01472 \textcolor{comment}{  * @note   Peripheral clock gating in SLEEP mode can be used to further reduce}
01473 \textcolor{comment}{  *           power consumption.}
01474 \textcolor{comment}{  * @note   After wakeup from SLEEP mode, the peripheral clock is enabled again.}
01475 \textcolor{comment}{  * @note   By default, all peripheral clocks are enabled during SLEEP mode.}
01476 \textcolor{comment}{  * @param  RCC\_AHBPeriph: specifies the AHB2 peripheral to gates its clock.}
01477 \textcolor{comment}{  *          This parameter can be any combination of the following values:}
01478 \textcolor{comment}{  *            @arg RCC\_AHB2Periph\_DCMI:   DCMI clock}
01479 \textcolor{comment}{  *            @arg RCC\_AHB2Periph\_CRYP:   CRYP clock}
01480 \textcolor{comment}{  *            @arg RCC\_AHB2Periph\_HASH:   HASH clock}
01481 \textcolor{comment}{  *            @arg RCC\_AHB2Periph\_RNG:    RNG clock}
01482 \textcolor{comment}{  *            @arg RCC\_AHB2Periph\_OTG\_FS: USB OTG FS clock  }
01483 \textcolor{comment}{  * @param  NewState: new state of the specified peripheral clock.}
01484 \textcolor{comment}{  *          This parameter can be: ENABLE or DISABLE.}
01485 \textcolor{comment}{  * @retval None}
01486 \textcolor{comment}{  */}
01487 \textcolor{keywordtype}{void} RCC_AHB2PeriphClockLPModeCmd(uint32\_t RCC\_AHB2Periph, FunctionalState NewState)
01488 \{
01489   \textcolor{comment}{/* Check the parameters */}
01490   assert_param(IS\_RCC\_AHB2\_PERIPH(RCC\_AHB2Periph));
01491   assert_param(IS\_FUNCTIONAL\_STATE(NewState));
01492   \textcolor{keywordflow}{if} (NewState != DISABLE)
01493   \{
01494     RCC->AHB2LPENR |= RCC\_AHB2Periph;
01495   \}
01496   \textcolor{keywordflow}{else}
01497   \{
01498     RCC->AHB2LPENR &= ~RCC\_AHB2Periph;
01499   \}
01500 \}
01501 
01502 \textcolor{comment}{/**}
01503 \textcolor{comment}{  * @brief  Enables or disables the AHB3 peripheral clock during Low Power (Sleep) mode.}
01504 \textcolor{comment}{  * @note   Peripheral clock gating in SLEEP mode can be used to further reduce}
01505 \textcolor{comment}{  *         power consumption.}
01506 \textcolor{comment}{  * @note   After wakeup from SLEEP mode, the peripheral clock is enabled again.}
01507 \textcolor{comment}{  * @note   By default, all peripheral clocks are enabled during SLEEP mode.}
01508 \textcolor{comment}{  * @param  RCC\_AHBPeriph: specifies the AHB3 peripheral to gates its clock.}
01509 \textcolor{comment}{  *          This parameter must be: RCC\_AHB3Periph\_FSMC}
01510 \textcolor{comment}{  * @param  NewState: new state of the specified peripheral clock.}
01511 \textcolor{comment}{  *          This parameter can be: ENABLE or DISABLE.}
01512 \textcolor{comment}{  * @retval None}
01513 \textcolor{comment}{  */}
01514 \textcolor{keywordtype}{void} RCC_AHB3PeriphClockLPModeCmd(uint32\_t RCC\_AHB3Periph, FunctionalState NewState)
01515 \{
01516   \textcolor{comment}{/* Check the parameters */}
01517   assert_param(IS\_RCC\_AHB3\_PERIPH(RCC\_AHB3Periph));
01518   assert_param(IS\_FUNCTIONAL\_STATE(NewState));
01519   \textcolor{keywordflow}{if} (NewState != DISABLE)
01520   \{
01521     RCC->AHB3LPENR |= RCC\_AHB3Periph;
01522   \}
01523   \textcolor{keywordflow}{else}
01524   \{
01525     RCC->AHB3LPENR &= ~RCC\_AHB3Periph;
01526   \}
01527 \}
01528 
01529 \textcolor{comment}{/**}
01530 \textcolor{comment}{  * @brief  Enables or disables the APB1 peripheral clock during Low Power (Sleep) mode.}
01531 \textcolor{comment}{  * @note   Peripheral clock gating in SLEEP mode can be used to further reduce}
01532 \textcolor{comment}{  *         power consumption.}
01533 \textcolor{comment}{  * @note   After wakeup from SLEEP mode, the peripheral clock is enabled again.}
01534 \textcolor{comment}{  * @note   By default, all peripheral clocks are enabled during SLEEP mode.}
01535 \textcolor{comment}{  * @param  RCC\_APB1Periph: specifies the APB1 peripheral to gates its clock.}
01536 \textcolor{comment}{  *          This parameter can be any combination of the following values:}
01537 \textcolor{comment}{  *            @arg RCC\_APB1Periph\_TIM2:   TIM2 clock}
01538 \textcolor{comment}{  *            @arg RCC\_APB1Periph\_TIM3:   TIM3 clock}
01539 \textcolor{comment}{  *            @arg RCC\_APB1Periph\_TIM4:   TIM4 clock}
01540 \textcolor{comment}{  *            @arg RCC\_APB1Periph\_TIM5:   TIM5 clock}
01541 \textcolor{comment}{  *            @arg RCC\_APB1Periph\_TIM6:   TIM6 clock}
01542 \textcolor{comment}{  *            @arg RCC\_APB1Periph\_TIM7:   TIM7 clock}
01543 \textcolor{comment}{  *            @arg RCC\_APB1Periph\_TIM12:  TIM12 clock}
01544 \textcolor{comment}{  *            @arg RCC\_APB1Periph\_TIM13:  TIM13 clock}
01545 \textcolor{comment}{  *            @arg RCC\_APB1Periph\_TIM14:  TIM14 clock}
01546 \textcolor{comment}{  *            @arg RCC\_APB1Periph\_WWDG:   WWDG clock}
01547 \textcolor{comment}{  *            @arg RCC\_APB1Periph\_SPI2:   SPI2 clock}
01548 \textcolor{comment}{  *            @arg RCC\_APB1Periph\_SPI3:   SPI3 clock}
01549 \textcolor{comment}{  *            @arg RCC\_APB1Periph\_USART2: USART2 clock}
01550 \textcolor{comment}{  *            @arg RCC\_APB1Periph\_USART3: USART3 clock}
01551 \textcolor{comment}{  *            @arg RCC\_APB1Periph\_UART4:  UART4 clock}
01552 \textcolor{comment}{  *            @arg RCC\_APB1Periph\_UART5:  UART5 clock}
01553 \textcolor{comment}{  *            @arg RCC\_APB1Periph\_I2C1:   I2C1 clock}
01554 \textcolor{comment}{  *            @arg RCC\_APB1Periph\_I2C2:   I2C2 clock}
01555 \textcolor{comment}{  *            @arg RCC\_APB1Periph\_I2C3:   I2C3 clock}
01556 \textcolor{comment}{  *            @arg RCC\_APB1Periph\_CAN1:   CAN1 clock}
01557 \textcolor{comment}{  *            @arg RCC\_APB1Periph\_CAN2:   CAN2 clock}
01558 \textcolor{comment}{  *            @arg RCC\_APB1Periph\_PWR:    PWR clock}
01559 \textcolor{comment}{  *            @arg RCC\_APB1Periph\_DAC:    DAC clock}
01560 \textcolor{comment}{  * @param  NewState: new state of the specified peripheral clock.}
01561 \textcolor{comment}{  *          This parameter can be: ENABLE or DISABLE.}
01562 \textcolor{comment}{  * @retval None}
01563 \textcolor{comment}{  */}
01564 \textcolor{keywordtype}{void} RCC_APB1PeriphClockLPModeCmd(uint32\_t RCC\_APB1Periph, FunctionalState NewState)
01565 \{
01566   \textcolor{comment}{/* Check the parameters */}
01567   assert_param(IS\_RCC\_APB1\_PERIPH(RCC\_APB1Periph));
01568   assert_param(IS\_FUNCTIONAL\_STATE(NewState));
01569   \textcolor{keywordflow}{if} (NewState != DISABLE)
01570   \{
01571     RCC->APB1LPENR |= RCC\_APB1Periph;
01572   \}
01573   \textcolor{keywordflow}{else}
01574   \{
01575     RCC->APB1LPENR &= ~RCC\_APB1Periph;
01576   \}
01577 \}
01578 
01579 \textcolor{comment}{/**}
01580 \textcolor{comment}{  * @brief  Enables or disables the APB2 peripheral clock during Low Power (Sleep) mode.}
01581 \textcolor{comment}{  * @note   Peripheral clock gating in SLEEP mode can be used to further reduce}
01582 \textcolor{comment}{  *         power consumption.}
01583 \textcolor{comment}{  * @note   After wakeup from SLEEP mode, the peripheral clock is enabled again.}
01584 \textcolor{comment}{  * @note   By default, all peripheral clocks are enabled during SLEEP mode.}
01585 \textcolor{comment}{  * @param  RCC\_APB2Periph: specifies the APB2 peripheral to gates its clock.}
01586 \textcolor{comment}{  *          This parameter can be any combination of the following values:}
01587 \textcolor{comment}{  *            @arg RCC\_APB2Periph\_TIM1:   TIM1 clock}
01588 \textcolor{comment}{  *            @arg RCC\_APB2Periph\_TIM8:   TIM8 clock}
01589 \textcolor{comment}{  *            @arg RCC\_APB2Periph\_USART1: USART1 clock}
01590 \textcolor{comment}{  *            @arg RCC\_APB2Periph\_USART6: USART6 clock}
01591 \textcolor{comment}{  *            @arg RCC\_APB2Periph\_ADC1:   ADC1 clock}
01592 \textcolor{comment}{  *            @arg RCC\_APB2Periph\_ADC2:   ADC2 clock}
01593 \textcolor{comment}{  *            @arg RCC\_APB2Periph\_ADC3:   ADC3 clock}
01594 \textcolor{comment}{  *            @arg RCC\_APB2Periph\_SDIO:   SDIO clock}
01595 \textcolor{comment}{  *            @arg RCC\_APB2Periph\_SPI1:   SPI1 clock}
01596 \textcolor{comment}{  *            @arg RCC\_APB2Periph\_SYSCFG: SYSCFG clock}
01597 \textcolor{comment}{  *            @arg RCC\_APB2Periph\_TIM9:   TIM9 clock}
01598 \textcolor{comment}{  *            @arg RCC\_APB2Periph\_TIM10:  TIM10 clock}
01599 \textcolor{comment}{  *            @arg RCC\_APB2Periph\_TIM11:  TIM11 clock}
01600 \textcolor{comment}{  * @param  NewState: new state of the specified peripheral clock.}
01601 \textcolor{comment}{  *          This parameter can be: ENABLE or DISABLE.}
01602 \textcolor{comment}{  * @retval None}
01603 \textcolor{comment}{  */}
01604 \textcolor{keywordtype}{void} RCC_APB2PeriphClockLPModeCmd(uint32\_t RCC\_APB2Periph, FunctionalState NewState)
01605 \{
01606   \textcolor{comment}{/* Check the parameters */}
01607   assert_param(IS\_RCC\_APB2\_PERIPH(RCC\_APB2Periph));
01608   assert_param(IS\_FUNCTIONAL\_STATE(NewState));
01609   \textcolor{keywordflow}{if} (NewState != DISABLE)
01610   \{
01611     RCC->APB2LPENR |= RCC\_APB2Periph;
01612   \}
01613   \textcolor{keywordflow}{else}
01614   \{
01615     RCC->APB2LPENR &= ~RCC\_APB2Periph;
01616   \}
01617 \}
01618 
01619 \textcolor{comment}{/**}
01620 \textcolor{comment}{  * @\}}
01621 \textcolor{comment}{  */}
01622 
01623 \textcolor{comment}{/** @defgroup RCC\_Group4 Interrupts and flags management functions}
01624 \textcolor{comment}{ *  @brief   Interrupts and flags management functions }
01625 \textcolor{comment}{ *}
01626 \textcolor{comment}{@verbatim   }
01627 \textcolor{comment}{ ===============================================================================}
01628 \textcolor{comment}{                   Interrupts and flags management functions}
01629 \textcolor{comment}{ ===============================================================================  }
01630 \textcolor{comment}{}
01631 \textcolor{comment}{@endverbatim}
01632 \textcolor{comment}{  * @\{}
01633 \textcolor{comment}{  */}
01634 
01635 \textcolor{comment}{/**}
01636 \textcolor{comment}{  * @brief  Enables or disables the specified RCC interrupts.}
01637 \textcolor{comment}{  * @param  RCC\_IT: specifies the RCC interrupt sources to be enabled or disabled.}
01638 \textcolor{comment}{  *          This parameter can be any combination of the following values:}
01639 \textcolor{comment}{  *            @arg RCC\_IT\_LSIRDY: LSI ready interrupt}
01640 \textcolor{comment}{  *            @arg RCC\_IT\_LSERDY: LSE ready interrupt}
01641 \textcolor{comment}{  *            @arg RCC\_IT\_HSIRDY: HSI ready interrupt}
01642 \textcolor{comment}{  *            @arg RCC\_IT\_HSERDY: HSE ready interrupt}
01643 \textcolor{comment}{  *            @arg RCC\_IT\_PLLRDY: main PLL ready interrupt}
01644 \textcolor{comment}{  *            @arg RCC\_IT\_PLLI2SRDY: PLLI2S ready interrupt  }
01645 \textcolor{comment}{  * @param  NewState: new state of the specified RCC interrupts.}
01646 \textcolor{comment}{  *          This parameter can be: ENABLE or DISABLE.}
01647 \textcolor{comment}{  * @retval None}
01648 \textcolor{comment}{  */}
01649 \textcolor{keywordtype}{void} RCC_ITConfig(uint8\_t RCC\_IT, FunctionalState NewState)
01650 \{
01651   \textcolor{comment}{/* Check the parameters */}
01652   assert_param(IS\_RCC\_IT(RCC\_IT));
01653   assert_param(IS\_FUNCTIONAL\_STATE(NewState));
01654   \textcolor{keywordflow}{if} (NewState != DISABLE)
01655   \{
01656     \textcolor{comment}{/* Perform Byte access to RCC\_CIR[14:8] bits to enable the selected interrupts */}
01657     *(\_\_IO uint8\_t *) CIR_BYTE2_ADDRESS |= RCC\_IT;
01658   \}
01659   \textcolor{keywordflow}{else}
01660   \{
01661     \textcolor{comment}{/* Perform Byte access to RCC\_CIR[14:8] bits to disable the selected interrupts */}
01662     *(\_\_IO uint8\_t *) CIR_BYTE2_ADDRESS &= (uint8\_t)~RCC\_IT;
01663   \}
01664 \}
01665 
01666 \textcolor{comment}{/**}
01667 \textcolor{comment}{  * @brief  Checks whether the specified RCC flag is set or not.}
01668 \textcolor{comment}{  * @param  RCC\_FLAG: specifies the flag to check.}
01669 \textcolor{comment}{  *          This parameter can be one of the following values:}
01670 \textcolor{comment}{  *            @arg RCC\_FLAG\_HSIRDY: HSI oscillator clock ready}
01671 \textcolor{comment}{  *            @arg RCC\_FLAG\_HSERDY: HSE oscillator clock ready}
01672 \textcolor{comment}{  *            @arg RCC\_FLAG\_PLLRDY: main PLL clock ready}
01673 \textcolor{comment}{  *            @arg RCC\_FLAG\_PLLI2SRDY: PLLI2S clock ready}
01674 \textcolor{comment}{  *            @arg RCC\_FLAG\_LSERDY: LSE oscillator clock ready}
01675 \textcolor{comment}{  *            @arg RCC\_FLAG\_LSIRDY: LSI oscillator clock ready}
01676 \textcolor{comment}{  *            @arg RCC\_FLAG\_BORRST: POR/PDR or BOR reset}
01677 \textcolor{comment}{  *            @arg RCC\_FLAG\_PINRST: Pin reset}
01678 \textcolor{comment}{  *            @arg RCC\_FLAG\_PORRST: POR/PDR reset}
01679 \textcolor{comment}{  *            @arg RCC\_FLAG\_SFTRST: Software reset}
01680 \textcolor{comment}{  *            @arg RCC\_FLAG\_IWDGRST: Independent Watchdog reset}
01681 \textcolor{comment}{  *            @arg RCC\_FLAG\_WWDGRST: Window Watchdog reset}
01682 \textcolor{comment}{  *            @arg RCC\_FLAG\_LPWRRST: Low Power reset}
01683 \textcolor{comment}{  * @retval The new state of RCC\_FLAG (SET or RESET).}
01684 \textcolor{comment}{  */}
01685 FlagStatus RCC_GetFlagStatus(uint8\_t RCC\_FLAG)
01686 \{
01687   uint32\_t tmp = 0;
01688   uint32\_t statusreg = 0;
01689   FlagStatus bitstatus = RESET;
01690 
01691   \textcolor{comment}{/* Check the parameters */}
01692   assert_param(IS\_RCC\_FLAG(RCC\_FLAG));
01693 
01694   \textcolor{comment}{/* Get the RCC register index */}
01695   tmp = RCC\_FLAG >> 5;
01696   \textcolor{keywordflow}{if} (tmp == 1)               \textcolor{comment}{/* The flag to check is in CR register */}
01697   \{
01698     statusreg = RCC->CR;
01699   \}
01700   \textcolor{keywordflow}{else} \textcolor{keywordflow}{if} (tmp == 2)          \textcolor{comment}{/* The flag to check is in BDCR register */}
01701   \{
01702     statusreg = RCC->BDCR;
01703   \}
01704   \textcolor{keywordflow}{else}                       \textcolor{comment}{/* The flag to check is in CSR register */}
01705   \{
01706     statusreg = RCC->CSR;
01707   \}
01708 
01709   \textcolor{comment}{/* Get the flag position */}
01710   tmp = RCC\_FLAG & FLAG_MASK;
01711   \textcolor{keywordflow}{if} ((statusreg & ((uint32\_t)1 << tmp)) != (uint32\_t)RESET)
01712   \{
01713     bitstatus = SET;
01714   \}
01715   \textcolor{keywordflow}{else}
01716   \{
01717     bitstatus = RESET;
01718   \}
01719   \textcolor{comment}{/* Return the flag status */}
01720   \textcolor{keywordflow}{return} bitstatus;
01721 \}
01722 
01723 \textcolor{comment}{/**}
01724 \textcolor{comment}{  * @brief  Clears the RCC reset flags.}
01725 \textcolor{comment}{  *         The reset flags are: RCC\_FLAG\_PINRST, RCC\_FLAG\_PORRST,  RCC\_FLAG\_SFTRST,}
01726 \textcolor{comment}{  *         RCC\_FLAG\_IWDGRST, RCC\_FLAG\_WWDGRST, RCC\_FLAG\_LPWRRST}
01727 \textcolor{comment}{  * @param  None}
01728 \textcolor{comment}{  * @retval None}
01729 \textcolor{comment}{  */}
01730 \textcolor{keywordtype}{void} RCC_ClearFlag(\textcolor{keywordtype}{void})
01731 \{
01732   \textcolor{comment}{/* Set RMVF bit to clear the reset flags */}
01733   RCC->CSR |= RCC_CSR_RMVF;
01734 \}
01735 
01736 \textcolor{comment}{/**}
01737 \textcolor{comment}{  * @brief  Checks whether the specified RCC interrupt has occurred or not.}
01738 \textcolor{comment}{  * @param  RCC\_IT: specifies the RCC interrupt source to check.}
01739 \textcolor{comment}{  *          This parameter can be one of the following values:}
01740 \textcolor{comment}{  *            @arg RCC\_IT\_LSIRDY: LSI ready interrupt}
01741 \textcolor{comment}{  *            @arg RCC\_IT\_LSERDY: LSE ready interrupt}
01742 \textcolor{comment}{  *            @arg RCC\_IT\_HSIRDY: HSI ready interrupt}
01743 \textcolor{comment}{  *            @arg RCC\_IT\_HSERDY: HSE ready interrupt}
01744 \textcolor{comment}{  *            @arg RCC\_IT\_PLLRDY: main PLL ready interrupt}
01745 \textcolor{comment}{  *            @arg RCC\_IT\_PLLI2SRDY: PLLI2S ready interrupt  }
01746 \textcolor{comment}{  *            @arg RCC\_IT\_CSS: Clock Security System interrupt}
01747 \textcolor{comment}{  * @retval The new state of RCC\_IT (SET or RESET).}
01748 \textcolor{comment}{  */}
01749 ITStatus RCC_GetITStatus(uint8\_t RCC\_IT)
01750 \{
01751   ITStatus bitstatus = RESET;
01752 
01753   \textcolor{comment}{/* Check the parameters */}
01754   assert_param(IS\_RCC\_GET\_IT(RCC\_IT));
01755 
01756   \textcolor{comment}{/* Check the status of the specified RCC interrupt */}
01757   \textcolor{keywordflow}{if} ((RCC->CIR & RCC\_IT) != (uint32\_t)RESET)
01758   \{
01759     bitstatus = SET;
01760   \}
01761   \textcolor{keywordflow}{else}
01762   \{
01763     bitstatus = RESET;
01764   \}
01765   \textcolor{comment}{/* Return the RCC\_IT status */}
01766   \textcolor{keywordflow}{return}  bitstatus;
01767 \}
01768 
01769 \textcolor{comment}{/**}
01770 \textcolor{comment}{  * @brief  Clears the RCC's interrupt pending bits.}
01771 \textcolor{comment}{  * @param  RCC\_IT: specifies the interrupt pending bit to clear.}
01772 \textcolor{comment}{  *          This parameter can be any combination of the following values:}
01773 \textcolor{comment}{  *            @arg RCC\_IT\_LSIRDY: LSI ready interrupt}
01774 \textcolor{comment}{  *            @arg RCC\_IT\_LSERDY: LSE ready interrupt}
01775 \textcolor{comment}{  *            @arg RCC\_IT\_HSIRDY: HSI ready interrupt}
01776 \textcolor{comment}{  *            @arg RCC\_IT\_HSERDY: HSE ready interrupt}
01777 \textcolor{comment}{  *            @arg RCC\_IT\_PLLRDY: main PLL ready interrupt}
01778 \textcolor{comment}{  *            @arg RCC\_IT\_PLLI2SRDY: PLLI2S ready interrupt  }
01779 \textcolor{comment}{  *            @arg RCC\_IT\_CSS: Clock Security System interrupt}
01780 \textcolor{comment}{  * @retval None}
01781 \textcolor{comment}{  */}
01782 \textcolor{keywordtype}{void} RCC_ClearITPendingBit(uint8\_t RCC\_IT)
01783 \{
01784   \textcolor{comment}{/* Check the parameters */}
01785   assert_param(IS\_RCC\_CLEAR\_IT(RCC\_IT));
01786 
01787   \textcolor{comment}{/* Perform Byte access to RCC\_CIR[23:16] bits to clear the selected interrupt}
01788 \textcolor{comment}{     pending bits */}
01789   *(\_\_IO uint8\_t *) CIR_BYTE3_ADDRESS = RCC\_IT;
01790 \}
01791 
01792 \textcolor{comment}{/**}
01793 \textcolor{comment}{  * @\}}
01794 \textcolor{comment}{  */}
01795 
01796 \textcolor{comment}{/**}
01797 \textcolor{comment}{  * @\}}
01798 \textcolor{comment}{  */}
01799 
01800 \textcolor{comment}{/**}
01801 \textcolor{comment}{  * @\}}
01802 \textcolor{comment}{  */}
01803 
01804 \textcolor{comment}{/**}
01805 \textcolor{comment}{  * @\}}
01806 \textcolor{comment}{  */}
01807 
01808 \textcolor{comment}{/******************* (C) COPYRIGHT 2011 STMicroelectronics *****END OF FILE****/}
\end{DoxyCode}
